\documentclass[12pt, a4paper]{article}                      % use "amsart" instead of "article" for AMSLaTeX format
\usepackage[a4paper,margin=1in]{geometry}                   % Adjust margins
\usepackage{amsmath,soul,mathtools, amsthm, amssymb,listings,color,textcomp,adjustbox,tikz,pgfplots}        % Math packages: amsmath, amssymb, listings, color
\usepackage[hidelinks]{hyperref}
\usepgfplotslibrary{external,fillbetween}
\pgfplotsset{compat=1.14}
\usepackage[makeroom]{cancel}
\title{\bf{Homework \textnumero 9}}
\author{Author: David Oniani
\\
\ \ \ Instructor: Tommy Occhipinti}
\date{October 25, 2018}

\usepackage{listings}
\usepackage{color}

%%%%%%%%%%%%%%% S E T S %%%%%%%%%%%%%%%
\newcommand{\nats}{\mathbb{N}}
\newcommand{\ints}{\mathbb{Z}}
\newcommand{\rats}{\mathbb{Q}}
\newcommand{\reals}{\mathbb{R}}
\newcommand{\irrats}{\mathbb{I}}

\newcommand{\pnats}{\mathbb{N}^+}
\newcommand{\pints}{\mathbb{Z}^+}
\newcommand{\prats}{\mathbb{Q}^+}
\newcommand{\preals}{\mathbb{R}^+}
\newcommand{\nreals}{\mathbb{R}^-}

\newcommand{\nints}{\mathbb{Z}^-}
\newcommand{\nrats}{\mathbb{Q}^-}
%%%%%%%%%%%%%%%%%%%%%%%%%%%%%%%%%%%%%%%

% Calligraphy
\newcommand\und[1]{\underline{\smash{#1}}}

% Operators
\DeclarePairedDelimiter\abs{\lvert}{\rvert}



\definecolor{dkgreen}{rgb}{0,0.6,0}
\definecolor{gray}{rgb}{0.5,0.5,0.5}
\definecolor{mauve}{rgb}{0.58,0,0.82}
\definecolor{backcolour}{rgb}{0.95,0.95,0.92}

\lstset{
backgroundcolor=\color{backcolour},
aboveskip=3mm,
belowskip=3mm,
showstringspaces=false,
columns=flexible,
basicstyle={\small\ttfamily},
numbers=left,
numberstyle=\normalsize\color{gray},
keywordstyle=\color{blue},
commentstyle=\color{dkgreen},
stringstyle=\color{mauve},
breaklines=true,
breakatwhitespace=true,
tabsize=4
}


\begin{document}
\maketitle
\begin{itemize}
\item[64.]
List three elements of each of the following sets:
\begin{itemize}
\item[(a)]
$\mathcal{P} (\{2, 7\})$
\begin{quote}
$\emptyset, \{2\}, \{7\}$
\end{quote}

\item[]

\item[(b)]
$\mathbb{Q} \times (\mathbb{R} - \mathbb{Q})$
\begin{quote}
$(-\dfrac{1}{2}, \sqrt{2}), (\dfrac{2}{5}, \sqrt{5}), (-\dfrac{3}{7}, \sqrt{7})$
\end{quote}

\item[]

\item[(c)]
$U_{i = 3}^5\{2^i, 2^i + 1\}$
\begin{quote}
$\{8, 9\}, \{16, 17\}, \{32, 33\}$
\end{quote}

\item[]

\item[(d)]
$\mathcal{P}(\mathcal{P}(\{1, 2\}))$
\begin{quote}
$\{1\}, \{2\}, \emptyset$
\end{quote}

\item[]

\item[(e)]
$\mathcal{P}([0, 1]) \times \mathcal{P}([0, 1])$
\begin{quote}
$(\{0\}, \{0\}), (\{0.37\}, \{0.86512\}), (\{1\}, \{1\})$
\end{quote}

\item[]

\item[(f)]
$\mathcal{P}([0, 1] \times [0, 1])$
\begin{quote}
$\{(0, 0)\}, \{(0.1, 0.2), (0.5, 0.6)\}, \{(0.892, 0.912), (0.942, 0.999)\}$
\end{quote}
\end{itemize}

\item[]
\item[]

\item[65.]
Prove or Disprove: If $A$ and $B$ are sets, then $\mathcal{P}(A - B) = \mathcal{P}(A) - \mathcal{P}(B)$.
\begin{quote}
It's false. Counterexample: Let $A = \{1, 2\}$ and $B = \{1\}$. Then we have
$A - B = \{2\}$ and $\mathcal{P}(A - B) = \{\emptyset, \{2\}\}$. On the other hand, $\mathcal{P}(A) = \{\emptyset, \{1\}, \{2\}, \{1, 2\}\}$ and $\mathcal{P}(B) = \{\emptyset, \{1\}\}$.
Thus, $\mathcal{P}(A) - \mathcal{P}(B) = \{\{2\}, \{1, 2\}\}$. Hence, we got that:
$$\mathcal{P}(A - B) = \{\emptyset, \{2\}\} \mbox{ and } \mathcal{P}(A) - \mathcal{P}(B) = \{\{2\}, \{1, 2\}\}$$
which means that $\mathcal{P}(A - B) \neq \mathcal{P}(A) - \mathcal{P}(B)$ and the initial claim is false.
\end{quote}

\item[]

\item[66.]
If for all $n \in \pints$ we have $A_n = \{n, n + 1, n + 2, ..., 2n\}$, find $\bigcap_{i = 10}^{20} A_i$. Can you generalize this result?
\begin{quote}
$\bigcap_{i = 10}^{20} A_i = \{20\}$ because the last element of the set when $i = 10$ is 20 (and all other elements of the set are less than 20),
the first element of the set when $i = 20$ is $20$ (and all the elements of the set are greater than 20), and all the sets in-between contain the element
20 meaning that the only element that is shared in-between is $\{20\}$.\\\\\\
Generalization: $\bigcap_{i=k}^{2k}A_i = 2k$ where $1 \leq k \leq 2k$.\\\\
Proof: Since $A_n = \{n, n + 1, n + 2, ..., 2n\}$, we know that the biggest element of $A_k$
is $2k$. On the other hand, we also know that the smallest element of $A_{2k}$
is $2k$. Besides, all the sets in-between ($A_k, A_{k + 1}, ... A_{2k - 1}$) will have $2k$ in them.
Thus, $\bigcap_{i=k}^{2k}A_i = 2k$.\\\\
Note: This is nearly the best generalization we can do. The generalization will only work with the
intersections of the type $\bigcap_{i=k}^{2k}A_i$. It won't work with some random intersection like $\bigcap_{i=1}^{3}A_i$.
In this case, we have $A_1 = \{1, 2\}, A_2 = \{2, 3, 4\}$, and $A_3 = \{3, 4, 5, 6\}$ thus, their intersection is the empty set
and not $2 \times 3 = 6$.
\end{quote}

\item[]
\item[]

\item[67.]
If $S$ is a set, define $\Xi(S)$ to be the set $\{(x, y) \in \mathcal{P}(S) \times \mathcal{P}(S) \ | \ x \cap y \neq \emptyset\}$.
\begin{itemize}
\item[(a)]
Explain why $(\{1, 3\}, \{2, 3\})$ is an element of $\Xi(\{1, 2, 3\})$.
\begin{quote}
Let's first notice that $\{1, 3\} \cap \{2, 3\} \neq \emptyset$ thus, we took that possibility out of way.
Next, sets $\{1, 3\}$ and $\{2, 3\}$ are in $\mathcal{P}(\{1, 2, 3\})$ (the powerset of $\{1, 2, 3\}$).
At last, the elements of the set $\Xi(\{1, 2, 3\})$ are the elements of the cartesian product of $\mathcal{P}(\{1, 2, 3\})$ and $\mathcal{P}(\{1, 2, 3\})$.
Then, since $\mathcal{P}(\{1, 2, 3\})$ itself contains $\{1, 3\}$ and $\{2, 3\}$, we know that the cartesian product of $\{1, 3\}$ and $\{2, 3\}$ on itself will
contain a tuple $(\{1, 3\}, \{2, 3\})$. Hence, $(\{1, 3\}, \{2, 3\}) \in \Xi(\{1, 2, 3\})$.
\end{quote}

\item[]

\item[(b)]
If $S = \{1, 2, 3, 4\}$, give 3 examples of $\mathcal{P} \times \mathcal{P}$
that are not in $\Xi(\{1, 2, 3\})$.
\begin{quote}
Note: In \underline{tuples}, the order MATTERS!\\\\
1. $(\{1\}, \{2, 3\})$ because $\{1\} \cap \{2, 3\} = \emptyset$.\\
2. $(\{\emptyset\}, \{\emptyset\})$ because $\emptyset \cap \emptyset = \emptyset$.\\
3. $(\{2, 3\}, \{1\})$ because $\{2, 3\} \cap \{1\} = \emptyset$.\\
\end{quote}

\item[(c)]
Explain carefully in words what $\Xi(S)$ is.\\
\begin{quote}
It is the set of ordered pairs (tuples) of sets such that the sets in the tuples
have at least one common element.\ Besides, the sets in the tuples are also the subsets
of the cartesian product of the powerset of the same set.
\end{quote}

\item[]
\item[]

\item[(d)]
Prove that if $S$ is a subset of $T$ then $\Xi(S)$ is a subset of $\Xi(T)$.
\begin{quote}
Suppose, we have two sets $S$ and $T$ such that $S \subseteq T$. If $S = T$, then it is clear that $\Xi(S)$ equals and thus, is a subset of $\Xi(T)$.
Now then, let's consider the case where $S \subset T$. If $S \subset T$, then $T = S + U$ where $U$ is some set of elements which $T$ has but $S$ does not.
Because $T = S + U$, $\mathcal{P}(T)$ or the powerset of the set $T$ will have all the subsets of $S$. On the other hand,
$\Xi(T)$ is concerned about the cartesian of the powerset of $T$ on itself. Thus, since $T$ has all the subsets of $S$,
$\Xi(T)$ will contain all the elements that $\Xi(S)$ does.
\end{quote}

\item[]

\item[(e)]
Prove that if $S$ is finite, then $\mid\Xi(S)\mid \ \geq \ \mid S\mid$. (Hint: To do this, explain how to find
a different element of $\Xi(S)$ for each $s \in S$. How do you know they are different?)
\begin{quote}
For each element $s \in S$, we can find a tuple $(\{s\}, \{S\})$ which is not in $\Xi(S)$.
We know that it is not in $S$ because there is no way for the set to contain itself unless it
is the infinite set. Thus, we have: for each $s \in S$, we construct a tuple $(\{s\}, \{S\})$
which is not in $S$. This will give us that the cardinality of $\Xi{(S)}$ equals the cardinality
of $S$. But then there is a tuple $(\{S\}, \{S\})$ which is in $\Xi{(S)}$, but not in $S$. Thus,
if $S$ is not empty $\mid\Xi(S)\mid \geq \mid S\mid$. If $S$ is empty, then $\mid\Xi(S)\mid \ = \ \mid S\mid$
and finally, if $S$ is finite, then $\mid\Xi(S)\mid \ \geq \ \mid S\mid$.
\end{quote}
\end{itemize}

\item[]
\item[]

\item[68.]
Here are five similar looking definitions that apply to subsets $S$ of $\mathbb{R}$.
\item[$\bullet$]
$S$ is \textbf{spread out} if for every $x, y \in S$ if $x \neq y$ then $\abs{x - y} \geq 1$.
\item[$\bullet$]
$S$ is called \textbf{broad} if for every $x \in \mathbb{R}$, there exists $y \in S$ such that $\abs{x - y} \leq 1$.
\item[$\bullet$]
$S$ is called \textbf{clumped} if for every $x, y \in S$ one has $\abs{x - y} \leq 1$.
\item[$\bullet$]
$S$ is called \textbf{bunched} if there exists $x \in S$ such that for every $y \in S$ one has $\abs{x - y} \leq 1$.
\item[$\bullet$]
$S$ is called \textbf{clustered} if for every $x \in S$ there exists $y \in S - \{x\}$ such that $\abs{x - y} \leq 1$.
\begin{itemize}
\item[]
\item[]
IMPORTANT NOTE: When talking about \und{broad, clumped, and bunched}\\
sets I will not consider the cases $x = y$. The reason
is that if $x = y$, then $\abs{x - y} = 0 \leq 1$ and it does not give us any valuable
info about the set.
\item[]
\item[]
\item[(a)]
Decide which of the above five definitions apply to each of the following sets.

\begin{itemize}
\item[i.]
$\{1, 2, 3\}$

\item[]

\begin{quote}
Set $\{1, 2, 3\}$ is \textbf{spread out} because for $1, 2 \in \{1, 2, 3\}$, $\abs{1 - 2} = \abs{2 - 1} = 1 \geq 1$;
for $2, 3 \in \{1, 2, 3\}$, we have $\abs{2 - 3} = \abs{2 - 3} = 1 \geq 1$; and finally, for $1, 3 \in \{1, 2, 3\}$, we have $\abs{1 - 3} = \abs{3 - 1} = 2 \geq 1$.
\\\\
Set $\{1, 2, 3\}$ is \und{NOT} \textbf{broad} because for $5 \in \mathbb{R}$, we have $\abs{5 - 1} = \abs{1 - 5} = 4 > 1$,
$\abs{5 - 2} = \abs{2 - 5} = 3 > 1$, and $\abs{5 - 3} = \abs{3 - 5} = 2 > 1$.\\
\\
Set $\{1, 2, 3\}$ is \textbf{clumped} because for $1, 2 \in \{1, 2, 3\}$, $\abs{1 - 2} = \abs{2 - 1} = 1 \leq 1$;
for $2, 3 \in \{1, 2, 3\}$, we have $\abs{2 - 3} = \abs{2 - 3} = 1 \leq 1$; and for $1, 3 \in \{1, 2, 3\}$, we have $\abs{1 - 3} = \abs{3 - 1} = 2 \leq 1$.
Besides, for $1, 2, 3 \in \{1, 2, 3\}$ we have $\abs{1 - 1} = \abs{2 - 2} = \abs{3 - 3} = 0 \leq 1$.
\\\\
Set $\{1, 2, 3\}$ is \textbf{bunched} because for $2 \in S$, we have $\abs{2 - 1} = 1 \leq 1$, $\abs{2 - 2} = 0 \leq 1$, and $\abs{2 - 3} = 1 \leq 1$.
\\\\
Set $\{1, 2, 3\}$ is \textbf{clustered} because for $1 \in S$, $\abs{1 - 2} = 1 \leq 1$; for $2 \in S$, $\abs{2 - 3} = 1 \leq 1$;
for $3 \in S$, $\abs{3 - 2} = 1 \leq 1$.
\end{quote}

\item[]

\item[ii.]
$\mathbb{R}$
\begin{quote}
Set $\mathbb{R}$ is not \und{NOT} \textbf{spread out} because for $0.1, 0.2 \in \mathbb{R}$, we have $0.1 \neq 0.2$ but $\abs{0.1 + 0.2} = 0.3 < 1$.
\\\\
Set $\mathbb{R}$ is \textbf{broad} because for every $x \in \mathbb{R}$, we can find $y \in \mathbb{R}$ such that $y = x + 1$ and then $\abs{x - y} = \abs{x - (x + 1)} = \abs{x - x - 1} = \abs{-1} = 1 \leq 1$.
\\\\
Set $\mathbb{R}$ is \und{NOT} \textbf{clumped} because for $5, 7 \in \mathbb{R}$, we have $\abs{5 - 7} = \abs{7 - 5} = 2 > 1$.
\\\\
$\mathbb{R}$ is \und{NOT} \textbf{bunched}. Proof: Suppose, for the sake of contradiction, that $x \in \mathbb{R}$ and also suppose that for every $y \in \mathbb{R}$, we have $\abs{x - y} \leq 1$ (thus, suppose that $\mathbb{R}$ is bunched).
Then we can take $y = x + 2$ (note that $y \in \mathbb{R}$) and we have $\abs{x - y} = \abs{x - x + 2} = 2 > 1$. Thus, we have reached the contradiction and $\mathbb{R}$ is not bunched.
\\\\
$\mathbb{R}$ is \textbf{clustered}. Let $x \in \mathbb{R}$, then we can have $y = x + 1$. Then since $x \neq y$ we know that $y \in \mathbb{R} - \{x\}$
and finally, $\abs{x - y} = \abs{x - (x + 1)} = \abs{x - x - 1} = \abs{-1} = 1 \leq 1$.
\end{quote}

\item[]

\item[iii.]
$\mathbb{Z}$
\begin{quote}
Set $\mathbb{Z}$ is \textbf{spread out}. Suppose $x, y \in \mathbb{Z}$ and $x \neq y$, then we have two cases:\\\\
Case I: $x > y$\\
Case II: $x < y$\\
\\
Proof of Case I: If $x > y$, we know that $x - y \geq 1$ and then $\abs{x - y} = \abs{y - x} \geq 1$.\\
Proof of Case II: If $x < y$, we know that $y - x \geq 1$ and then $\abs{x - y} = \abs{y - x} \geq = 1$.
\\\\
Set $\mathbb{Z}$ is \textbf{broad} because for every $x \in \mathbb{Z}$, we can find $y \in \mathbb{Z}$ such that $y = x + 1$ and then $\abs{x - y} = \abs{x - (x + 1)} = \abs{x - x - 1} = \abs{-1} = 1 \leq 1$.
\\\\
Set $\mathbb{Z}$ is \und{NOT} \textbf{clumped} because for $5, 7 \in \mathbb{Z}$, we have $\abs{5 - 7} = \abs{7 - 5} = 2 > 1$.
\\\\
$\mathbb{Z}$ is \und{NOT} \textbf{bunched}. Proof: Suppose, for the sake of contradiction, that $x \in \mathbb{Z}$ and also suppose that for every $y \in \mathbb{Z}$, we have $\abs{x - y} \leq 1$ (thus, suppose that $\mathbb{Z}$ is bunched).
Then we can take $y = x + 2$ (note that $y \in \mathbb{Z}$) and we have $\abs{x - y} = \abs{x - x + 2} = 2 > 1$. Thus, we have reached the contradiction and $\mathbb{Z}$ is not bunched.
\\\\
$\mathbb{Z}$ is \textbf{clustered}. Let $x \in \mathbb{Z}$, then we can have $y = x + 1$. Then since $x \neq y$ we know that $y \in \mathbb{Z} - \{x\}$
and finally, $\abs{x - y} = \abs{x - (x + 1)} = \abs{x - x - 1} = \abs{-1} = 1 \leq 1$.
\end{quote}

\item[]

\item[iv.]
$[2, 4]$
\begin{quote}
Interval $[2, 4]$ is \und{NOT} \textbf{spread out} because for $2.25, 2.5 \in [2, 4]$ we have that $2.25 \neq 2.5$ but $\abs{2.25 - 2.5} = 0.25 < 1$.
\\\\
Interval $[2, 4]$ is \und{NOT} \textbf{broad} because for $8 \in \mathbb{R}$, the differences between 8 and the members of $[2, 4]$ will range from 4 to 6.
Thus, if we take $x = 8 \in \mathbb{R}$, for all $y \in [2, 4]$, $\abs{x - y} = [4, 6]$ and it is clear that for all $y \in [2, 4]$, $\abs{x - y} \geq 4 > 1$.
\\\\
Interval $[2, 4]$ is \und{NOT} \textbf{clumped} because for $2.2, 3.4 \in S$, we have $\abs{2.2 - 3.4} = 1.2 > 1$.
\\\\
Interval $[2, 4]$ is \textbf{bunched}. Proof: Let $x = 3 \in [2, 4]$. Then consider the following three cases:\\\\
Case I: $2 \leq y < 3$\\
Case II: $y = 3$\\
Case III: $3 < y \leq 4$\\\\
If $2 \leq y < 3$, then $x - y \in [0, 1)$ and thus $\abs{x - y} \leq 1$.\\
If $y = 3$, then $x - y = 0$ and thus $\abs{x - y} = 0 \leq 1$.\\
If $3 < y \leq 4$, then $x - y \in [-1, 0)$ and thus $\abs{x - y} \leq 1$.\\
Thus, we have exhaustively shown that the interval $[2, 4]$ is bunched.
\\\\
Interval $[2, 4]$ is \textbf{clustered}. Proof: Suppose $x \in [2, 4]$. Then consider the following three cases:\\\\
Case I: $2 \leq x < 3$\\
Case II: $x = 3$\\
Case III: $3 < x \leq 4$\\\\
If $2 \leq x < 3$, then we can take $y = 3$ and we have that $x \neq y$ while $\abs{x - y} \leq 1$.\\
If $x = 3$, then we can take $y = 2.8$ and we have that $x \neq y$ while $\abs{x - y} = 0.2 \leq 1$.\\
If $3 < x \leq 4$, then we can take $y = 3$ and we have that $x \neq y$ while $\abs{x - y} \leq 1$.\\
Thus, we have exhaustively shown that the interval $[2, 4]$ is clustered.
\end{quote}

\item[]

\item[v.]
The set of even integers.\\\\
\und{\textbf{Note: We denote the set of even integers as the set $E = \{... -2, 0, 2 ...\}$.}}
\item[]
\item[]
\begin{quote}
Set $E$ is \textbf{spread out}. Proof: Suppose $x, y \in S$ and $x \neq y$. Then, since $E$ is the set of even integers,
we know that $x - y \geq 2$ and finally $\abs{x - y} \geq 2 \geq 1$.\\\\
Set $E$ is \und{NOT}  \textbf{broad}. Proof: Suppose, for the sake of contradiction, that $E$ is broad and let $x = 0$.
Then we know that $x \in E$, but for all $y \in E$, $\abs{x - y} \geq 2$ which contradicts the fact that $E$ is broad (for every $x \in E$ there exists $y$ such that $x \neq y$ and $\abs{x - y} \leq 1$).\\\\
Set $E$ is \und{NOT} \textbf{clumped}. Proof: Suppose, for the sake of contradiction, that $E$ is clumped. Then for all $x, y \in S$, we must have $\abs{x - y} \leq 1$. However, if $x = 0, y = 2$, we have $\abs{x - y} = 2 > 1$.\\\\
Set $E$ is \und{NOT} \textbf{bunched}. Proof: Suppose, for the sake of contradiction, that $E$ is bunched. Then, since $E$ is the set of even integers, if $x \in E$, then for all $y \in E - \{x\}$, $\abs{x - y} \geq 2$ and we have reached the contradiction.\\\\
Set $E$ is \und{NOT} \textbf{clustered}. Proof: Suppose, for the sake of contradiction, that $E$ is clustered. Then, since $E$ is the set of even integers, if $x \in E$, then for all $y \in E - \{x\}$, $\abs{x - y} \geq 2$ and we have reached the contradiction.
\end{quote}

\item[]

\item[vi.]
$\bigcup_{i\in\mathbb{Z}}[2i, 2i + 1]$
\begin{quote}
$\bigcup_{i\in\mathbb{Z}}[2i, 2i + 1]$ is \textbf{spread out}. Proof: Suppose $x, y \in \bigcup_{i\in\mathbb{Z}}[2i, 2i + 1]$ and $x \neq y$.
Then we know that the difference between any two different elements of the set is either greater than or equal than 1 or less than or equal than -1 and thus, $\abs{x - y} \geq 1$.\\\\
$\bigcup_{i\in\mathbb{Z}}[2i, 2i + 1]$ is \und{NOT} \textbf{broad}. Proof: Let $i = 1$, then $\bigcup_{i\in\mathbb{Z}}[2i, 2i + 1] = \{2, 3\}$. Now, let's take $10 \in \mathbb{R}$. Then
$\abs{10 - 2} = \abs{2 - 10} = 8 > 1$ and $\abs{10 - 3} = \abs{3 - 10} = 7 > 1$.\\\\
$\bigcup_{i\in\mathbb{Z}}[2i, 2i + 1]$ is \textbf{clumped}. Proof: Suppose $x, y \in \bigcup_{i\in\mathbb{Z}}[2i, 2i + 1]$. Then we know that the difference between any two elements are greater than
or equal than 1. Thus, if $x \neq y$, $\abs{x - y} \leq 1$ and if $x = y$, $\abs{x - y} \leq 1$.\\\\
$\bigcup_{i\in\mathbb{Z}}[2i, 2i + 1]$ is \textbf{bunched}. Proof: Consider $\bigcup_{i\in\mathbb{Z}}[2i, 2i + 1]$. We know that $\bigcup_{i\in\mathbb{Z}}[2i, 2i + 1]$ always has 2 elements
$2i$ and $2i + 1$. Then if we take $2i$, we can find $2i + 1$ and $\abs{(2i + 1) - (2i)} = 1 \leq 1$ and if we take $2i + 1$, we can find $2i + 1$ and $\abs{(2i) - (2i + 1)} = 1 \leq 1$.
Now, if we take $3, $\\\\
$\bigcup_{i\in\mathbb{Z}}[2i, 2i + 1]$ is \textbf{clustered}. Proof: Suppose $x \in \bigcup_{i\in\mathbb{Z}}[2i, 2i + 1]$. Then we know that $x = 2k$ or $x = 2k + 1$.\\
Case I: If $x = 2k$, we can take $y = 2k - 1$ and if $2k - 1$ does not exist, take $y = 2k + 1$ and then $\abs{x - y} = 1 \leq 1$.\\
Case II: If $x = 2k + 1$, we can take $y = 2k$ and if $2k$ does not exist, take $y = 2k + 2$ and then $\abs{x - y} = 1 \leq 1$.\\
\end{quote}
\end{itemize}
\item[]
\item[(b)]
Give examples of the following, or explain why none exists.
\item[]
\begin{itemize}
\item[i.]
A broad, spread out set
\begin{quote}
There exists such set. Example: The set of positive integers.\\\\
Proof: Consider the set of integers $\mathbb{Z}$. Then we know that for every $x, y \in \mathbb{Z}$, if $x \neq y$, then $\abs{x - y} \geq 1$. Thus, \und{it is spread out}.
On the other hand, suppose $y \in \mathbb{R}$. Then, let's define the $round$ of $y$ as the closest integer to $y$ rounded down (e.g., $round(1.2) = 1$ and $round(1.9) = 1$).
We can then take $x = round(y)$ and this way, we have $\abs{x - y} \leq 1$. Thus, for each $y$, we can follow the process above and we get that \und{the set $\mathbb{Z}$ is broad}. $\qedsymbol$
\end{quote}

\item[]

\item[ii.]
A spread out, clumped set
\begin{quote}
There exists. Example: $S = \{1, 2\}$.\\\\
Proof: $S$ is spread out because $\abs{1 - 2} = \abs{2 - 1} = 1 \geq 1$ and $S$ is clumped because
$\abs{1 - 1} = \abs{2 - 2} = 0 \leq 1$ and $\abs{1 - 2} = \abs{2 - 1} = 1 \leq 1$. $\qedsymbol$
\end{quote}

\item[]

\item[iii.]
A subset of $\mathbb{R}$ that is not broad, not clumped, and not spread out.
\begin{quote}
Such set exists. Example: $S = \{1, 1.2, 3\}$.\\\\
Proof:\\
$S$ \und{is not} broad because for $x = 25$, $\abs{25 - 1} = 24 > 1$, $\abs{25 - 1.2} = 23.8 > 1$ and $\abs{25 - 3} = 22 > 1$.\\
$S$ \und{is not} clumped because $\abs{3 - 1.2} = 1.8 > 1$.\\
$S$ \und{is not} spread out because $\abs{1 - 1.2} = 0.8 < 1$. $\qedsymbol$
\end{quote}
\item[]
\item[]
\item[iv.]
An infinite clumped set
\begin{quote}
There exists. Example $S = [1, 2]$ (a closed interval from 1 to 2).\\\\
Proof: Suppose $x, y \in S$. Then, $1 \leq x \leq 2$ and $1 \leq y \leq 2$ and
finally, $0 \leq \abs{x - y} \leq 1$. $\qedsymbol$
\end{quote}
\end{itemize}

\item[]
\item[(c)]
For each of the five properties above, prove or disprove the statement: A subset of a set with PROPERTY is also PROPERTY.
\\\\
Note: \und{The empty set is spread out, broad, clumped, bunched, and clustered}. Thus, I am not going to consider it.
\begin{itemize}
\item[i.]
A subset of the spread out set is also spread out.
\begin{quote}
It's true.\\\\
Proof: Suppose $S$ is a spread-out set and $B \subseteq S$. Now, since $S$ is spread out, we know that for all $x, y \in S$, if $x \neq y$, then $\abs{x - y} \geq 1$.
Let $x, y \in B$. Then, since $B \subseteq S$, we know that $x, y \in S$ and thus, $\abs{x - y} \geq 1$.
$\qedsymbol$
\end{quote}
\item[]
\item[ii.]
A subset of the broad set is also broad.
\begin{quote}
It is false.\\\\
Proof: Suppose $S = \reals$. Then $S$ is broad because for every $y \in \reals$, we can take the same element $y \in S$ (since $S = \reals$)
and then $\abs{y - y} = 0 \leq 1$. Now, let's take a look at the subset $A$ of $S$ such that $A = \{0\}$. Then for $y = 12$, $\abs{12 - 0} = 12 > 1$.
Thus, the subset of the broad set is not necessarily broad.
$\qedsymbol$
\end{quote}
\item[]
\item[iii.]
A subset of the clumped set is also clumped.
\begin{quote}
It is true.\\\\
Proof: Suppose $S$ is a clumped set and $A$ is subset of $S$. Let $x, y \in A$.
Then, since $A \subseteq S$, we have $x, y \in S$ and $\abs{x - y} \leq 1$. $\qedsymbol$
\end{quote}
\item[]
\item[iv.]
A subset of the bunched set is also bunched.
\begin{quote}
It is false.\\\\
Proof: Suppose we have a set $S = \{1, 1.5, 2.2\}$. Then $S$ is broad since $\abs{1.5 - 1} = 0.5 \leq 1$, $\abs{1.5 - 2.2} = 0.7 \leq 1$,
and $\abs{1.5 - 1.5} = 0 \leq 1$. However, the subset of $S$, $A = \{1, 2.2\}$ is not bunched because $\abs{1 - 2.2} = \abs{2.2 - 1} = 1.2 > 1. \qed$
\item[]
\item[v.]
A subset of the clustered set is also clustered.
\begin{quote}
It is false.\\\\
Proof: Let $S = \{1.2, 2, 3\}$. Then $S$ is clustered because $\abs{3 - 2} = 1 \leq 1$, $\abs{2 - 3} = 1 \leq 1$, and $\abs{1.2 - 2} = 0.8 \leq 1$.
However, if we now take a look at the subset $A$ of $S$, such that $A = \{1.2, 3\}$, then $\abs{3 - 1.2} = 1.8 > 1. \qed$
\end{quote}
\end{quote}
\end{itemize}

\item[]

\item[(d)]
For each of the five properties above, prove or disprove the statement: If $S$ and
$T$ are sets with PROPERTY than $S \cup T$ is a set with PROPERTY.
\begin{itemize}
\item[i.]
If $S$ and $T$ are spread out, then $S \cup T$ is spread out.
\begin{quote}
It is false. Proof: Let $S = \{1, 2\}$ then $\abs{1 - 2} = \abs{2 - 1} = 1 \geq 1$ and thus, $S$ is spread out.\\
Now, let $T = \{2.2, 3.2\}$. Then $\abs{2.2 - 3.2} = \abs{3.2 - 2.2} = 1 \geq 1$ and hence, $T$ is spread out. On the other
hand, $S \cup T = \{1, 2, 2.2, 3.2\}$ and $\abs{2.2 - 2} = 0.2 < 1$. Thus, while $S$ and $T$ are
spread out, $S \cup T$ is not spread out.
\end{quote}

\item[]

\item[ii.]
If $S$ and $T$ are broad, then $S \cup T$ is broad.
\begin{quote}
It is true.\\
Proof: Suppose $S$ and $T$ are the broad sets, let $y \in \reals$.
Then we know that $S \cup T$ contains the elements of $S$ and $T$.
Besides, for $y$, we could find $s \in S$ or $t \in T$ such that $\abs{y - s} \leq 1$
and $\abs{y - t} \leq 1$. Therefore, for $y \in \reals$ we can just take $s \in S$ (or $t \in T$) which will
also be an element of the set $S \cup T$ and we will have $\abs{y - s} \leq 1$.
\qed
\end{quote}

\item[]

\item[iii.]
If $S$ and $T$ are clumped, then $S \cup T$ is clumped.
\begin{quote}
It is false.\\
Proof: Let $S = \{1, 2\}$ and $T = \{3, 4\}$. Then $S$ is clumped because $\abs{1 - 2} = \abs{2 - 1} = 1 > 1$
and $T$ is clumped because $\abs{3 - 4} = \abs{3 - 4} = 1 \leq 1$. Now, consider the set $S \cup T$.
$S \cup T = \{1, 2, 3, 4\}$ and $\abs{1 - 4} = 3 > 1$ and the set $S \cup T$ is not clumped. $\qed$
\end{quote}
\item[]
\item[iv.]
If $S$ and $T$ are bunched, then $S \cup T$ is bunched.
\begin{quote}
It is false.\\
Proof: Let $S = \{1\}$ and $T = \{3\}$. Then $S$ is bunched because $\abs{1 - 1} = 0 \leq 1$
and $T$ is bunched as well since $\abs{3 - 3} = 0 \leq 1$. Consider the set $S \cup T$.
$S \cup T = \{1, 3\}$ and then $\abs{1 - 3} = \abs{3 - 1} = 2 > 1$. Thus, the union of two bunched
sets is not necessarily bunched. $\qed$
\end{quote}
\item[]
\item[v.]
If $S$ and $T$ are clustered, then $S \cup T$ is clustered.
\begin{quote}
It is true.\\
Proof: Suppose $S$ and $T$ are clustered. Then we know that for every $s \in S$, there exists $x \in S - \{s\}$
such that $\abs{s - x} \leq 1$. Also, we know that for every $t \in T$, there exists $y \in T - \{t\}$ such that $\abs{t - y} \leq 1$.
Now, consider the set $S \cup T$ and let $z \in S \cup T$. Since $z \in S \cup T$, it means that $z \in S$ or $x \in T$.
If $z \in S$, we know that there exists an element $f \in S$ such that $\abs{z - f} \leq 1$ and since $f \in S$, $f \in S \cup T$.
And if $z \in T$, we know that there exists an element $g \in T$ such that $\abs{z - g} \leq 1$ and since $g \in T$, $g \in S \cup T$.
\qed
\end{quote}
\end{itemize}

\item[]
\item[(e)]
For each of the five properties above, prove or disprove the statement:
If $S$ and $T$ are sets with PROPERTY than $S \cap T$ is a set with PROPERTY.
\begin{itemize}
\item[i.]
If $S$ and $T$ are spread out, then $S \cap T$ is spread out.
\begin{quote}
It is true.\\
Proof: Suppose $S$ and $T$ are spread out sets. Then we know that $S \cap T$
is a subset of $S$. Then since we already proved that the subset of the spread
out set is spread out, $S \cap T$ is also spread out. $\qed$
\end{quote}
\item[]
\item[ii.]
If $S$ and $T$ are broad, then $S \cap T$ is broad.
\begin{quote}
It is false.\\
Proof: Let $S = \{-2.4, -1,2, 0, 1.2, 2.4 ...\}$ and $T = \{-3, -1.5, 0, 1.5, 3, \}$.
Then $S$ and $T$ are both broad, however, $S \cap T = \{-12, -6, 6, 12\}$ is not as if we take
$2 \in \reals$, the best we can get is $\abs{2 - 6} = 4 > 1$. $\qed$
\end{quote}
\item[]
\item[iii.]
If $S$ and $T$ are clumped, then $S \cap T$ is clamped.
\begin{quote}
Since we have already proved that any subset of the clumped set is clumped.
Then $S \cap T$ is just a subset of $S$ (and $T$) and it is clumped as well. $\qed$
\end{quote}
\item[]
\item[iv.]
If $S$ and $T$ are bunched, then $S \cap T$ is bunched.
\begin{quote}
It is false.\\
Proof: Let $S = \{1, 2, 2.8\}$ and $T = \{1, 1.9, 2.8\}$. Then $S$ is bunched
because $\abs{2 - 1} = 1 \leq 1$ and $\abs{2 - 2.8} = 0.8 \leq 1$.
$T$ is bunched too as $\abs{1.9 - 1} = 0.9 \leq 1$ and $\abs{1.9 - 2.8} = 0.9 \leq 1$.
On the other hand, set $S \cap T = \{1, 2.8\}$ which is not bunched as $\abs{1 - 2.8} = \abs{2.8 - 1} = 1.8 > 1$.
$\qed$
\end{quote}
\item[]
\item[v.]
If $S$ and $T$ are clustered, then $S \cap T$ is clustered.
\begin{quote}
It is false.\\
Proof: Consider two clustered sets $S = \{1, 2, 2.7\}$ and $T = \{1, 1.8, 2.7\}$.
Then $S \cap T = \{1, 2.7\}$ and $\abs{1 - 2.7} = 1.7 > 1$.
$\qed$
\end{quote}
\end{itemize}
\item[]
\item[]
\item[]
{\Large Video: Ramanujan, 1729, and Fermat’s Last Theorem}
\\\\
Description: Matt Parker discusses Hardy (author of A Mathematician’s Apology)’s relationship with Ramanujan,
and discusses a famous story of the taxi cab number 1729, as well as surprising connections between this story and an attempted proof of Fermat’s Last Theorem.
Link: \url{https://www.youtube.com/watch?v=_o0cIpLQApk}
\item[]
\item[69.]
How did Ramanujan and Hardy meet?
\begin{quote}
Hardy wanted to learn more about Ramanujan since he heard that every number was a Ramanujan's friend.
Since Ramanujan was ill, Hardy personally visited Ramanujan who was residing Putney.
\end{quote}
\item[]
\item[70.]
In the video Parker discusses the near miss “solution” $65601^3 + 67402^3 = 83802^3$
to Fermat’s Last Theorem. Explain how you could know this equation was not correct
without doing any calculation at all.
\begin{quote}
65601 is odd thus, $65601^3$ is odd as well. 67402 is even, thus $67402^3$ is even as well.
83802 is even and thus, $83802^3$ is even too. Hence, we have that the sum of the odd and
the even integer is even which is not true.
\end{quote}
\item[]
\item[71.]
What is the connection between 1729 and Fermat’s Last Theorem?
\begin{quote}
There was a reason why Ramanujan knew that $1729 = 9^3 + 10^3 = 12^3 + 1^3$ (1729 is the smallest number
to be written in as the sum of two cubes in two different ways). In 1976, a researched discovered that Ramanujan
was working on the Fermat's Last Theorem. And the fermat last theorem says $x^n + y^n = z^n$ has no solutions
when $a, b, c \in \pnats$ and $n > 2$ (it is easy to notice that the pattern in 1729 is, at some extent, similar to that in the Fermat’s Last Theorem).
\end{quote}
\newpage

{\Large \und{Bookwork 4.1}}
\item[]
\item[1.]
Let $U = \{1, 2, 3, 4, 5, 6\}, A = \{1, 2, 3\}, B = \{2, 3, 4\}$, and $C = \{4, 5, 6\}$.
Find each of the following sets.\\\\
Note: I am assuming that $U$ is the universal set.
\\
\begin{itemize}
\item[(a)]
$A \cup B = \{1, 2, 3, 4\}$
\item[(b)]
$A \cap B = \{2, 3\}$
\item[(c)]
$A^c = U - \{1, 2, 3\} = \{4, 5, 6\}$
\item[(d)]
$B^c = U - \{2, 3, 4\} = \{1, 5, 6\}$
\item[(e)]
$B \cap C = \{4\}$
\item[(f)]
$A \cap (B \cup C) = A \cap \{2, 3, 4, 5, 6\} = \{2, 3\}$
\item[(g)]
$(A \cap B) \cup (A \cap C) = \{2, 3\} \cup \emptyset = \emptyset$
\item[(h)]
$A \cap C^c = \{1, 2, 3\} \cap \{1, 2, 3\} = \{1, 2, 3\}$
\item[(i)]
$A - C = \{1, 2, 3\}$
\item[(j)]
$\mathcal{P}(A) \cap \mathcal{P}(B) = \{\emptyset, \{1\}, \{2\}, \{3\}, \{1, 2\}, \{2, 3\}, \{1, 3\}, \{1, 2, 3\}\} \ \cap$\\
$\{\emptyset, \{4\}, \{5\}, \{6\}, \{4, 5\}, \{5, 6\}, \{4, 6\}, \{4, 5, 6\}\} = \emptyset$
\end{itemize}

\item[]
\item[]

\item[3.]
\begin{itemize}
\item[(b)]
Prove: $A \subseteq B$ if and only if $A \cap B = A$.
\begin{proof}
Since this is the if and only if proof, we first have to prove the
statement and then the converse of it.\\\\
I. Suppose, for the sake of contradiction, that $A \subseteq B$ and
$A \cap B \neq A$. Then it must be the case that $A$ has some elements which
$B$ does not and $A \nsubseteq B$.\\\\
II. Suppose that $A \cap B = A$, then we have that $B$ has all the elements of $A$
which makes $A$ the subset of $B$ and thus $A \subseteq B$.
\end{proof}
\item[]
\item[(c)]
Prove: If $A \subseteq B$ and $A \subseteq C$, then $A \subseteq B \cap C$.
\begin{proof}
Suppose $A, B$, and $C$ are the sets such that $A \subseteq B$ and $A \subseteq C$.
Then, we know that both $B$ and $C$ have all the elements of $A$ and thus, $A \subseteq B \cap C$.
\end{proof}
\item[]
\item[(e)]
Prove: $(A - B) \cap (A \cap B) = \emptyset$.
\begin{proof}
Suppose $A$ and $B$ are the sets. Then $(A - B) \cap B = \emptyset$.
Now, consider $A \cap B$ which is the set of elements which are both in $A$ and $B$.
Since $(A - B) \cap B = \emptyset$, it means that $(A - B) \cap (A \cap B) = \emptyset$.
\end{proof}
\item[]
\item[(f)]
Prove: $(A - B) \cap C = (A \cap C) - B$
\begin{proof}
To show the equality of two sets, we must show that the first one is the subset
of the second one and that the second one is the subset of the first one.
\\\\
Prove that $(A - B) \cap C \subseteq (A \cap C) - B$:\\
Let $x \in (A - B) \cap C$. Then $x \in A - B$, $x \notin B$, and $x \in C$. Since $x \in A - B$,
$x \in A$ and as $x \in C$, we have $x \in A \cap C$. Finally, since $x \notin B$ and $x \in A \cap C$,
we have $x \in (A \cap C) - B$.
\\\\
Prove that $(A \cap C) - B \subseteq (A - B) \cap C$\\
Let $x \in (A \cap C) - B$. then $x \in A$, $x \in C$, and $x \notin B$.
Since $x \in A$ and $x \notin B$, $x \in A - B$. At last, since $x \in A - B$
and $x \in C$, $x \in (A - B) \cap C$.
\end{proof}
\end{itemize}

\item[]
\item[]

\item[7.]
\begin{itemize}
\item[(a)]
For any set $A$, is $\{\emptyset\} \subseteq \mathcal{P}(A)$.
\begin{quote}
No. Counterexample: Let $A = \{1\}$. Then $\mathcal{P}(A) = \{\emptyset, \{1\}\}$
and $\{\emptyset\} \nsubseteq \mathcal{P}(A)$.
\end{quote}
\item[]
\item[(b)]
Prove: If $A \cap B = \emptyset$, then $\mathcal{P}(A) \cap \mathcal{P}(B) = \emptyset$. Is the converse true?
\begin{proof}
Suppose $A$ and $B$ are the sets such that $A \cap B = \emptyset$. Then let's take a look
at $\mathcal{P}(A)$ and $\mathcal{P}(B)$. Since the $\emptyset$ is the subset of every set,
$\emptyset \in \mathcal{P}(A)$ and $\emptyset \in \mathcal{P}(B)$. On the other hand, $\mathcal{P}(A)$
is the powerset of $A$ which is the set of all subsets of $A$ as well as $\mathcal{P}(B)$ is the set
of all subsets of $B$ and since $A \cap B = \emptyset$, $\mathcal{P}(A) \cap \mathcal{P}(B) = \emptyset$.
\\\\
Is the converse true?
The converse will be true as well. If $\mathcal{P}(A) \cap \mathcal{P}(B) = \emptyset$, it means
that there were no shared elements in $\mathcal{P}(A)$ and $\mathcal{P}(B)$. Therefore, since
powersets are the sets of all subsets of $A$ and $B$, it must be the case that they have no elements
in common and $A \cap B = \emptyset$.
\end{proof}
\end{itemize}

\item[]
\item[]

\item[9.]
Suppose $A$ is a set with $n$ elements where $n$ is a positive integer.
\begin{itemize}
\item[(a)]
How many subsets of $A$ contain exactly one element?
\begin{quote}
Since there are $n$ elements in the set $A$, there will be $n$ subsets with one element.
\end{quote}
\item[]
\item[(b)]
How many subsets of $A$ contain exactly $n - 1$ elements.
\begin{quote}
Since there are $n$ elements in $A$, the number of subsets
with $n - 1$ elements will be $A_{n - 1}^n = \dfrac{n!}{n - (n - 1)!} = n$ (a.k.a, the number of permutations).
(the question does not ask for the proof, but if one wanted to, it could be proved
using mathematical induction).
\end{quote}
\end{itemize}
\end{itemize}

\end{itemize}

\end{document}
