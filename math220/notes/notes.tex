\documentclass[12pt, a4paper]{article}             % use "amsart" instead of "article" for AMSLaTeX format
\usepackage[a4paper,margin=1in]{geometry}          % Adjust margins
\usepackage{amsmath,amssymb,listings,color}        % Math packages: amsmath, amssymb, listings, color

\title{\bf{Notes for MATH 220}}
\author{Author: David Oniani
\\
\ \ \ Instructor: Tommy Occhipinti}
\date{August 29, 2018}

\usepackage{listings}
\usepackage{color}

\definecolor{dkgreen}{rgb}{0,0.6,0}
\definecolor{gray}{rgb}{0.5,0.5,0.5}
\definecolor{mauve}{rgb}{0.58,0,0.82}
\definecolor{backcolour}{rgb}{0.95,0.95,0.92}

\lstset{
backgroundcolor=\color{backcolour},
aboveskip=3mm,
belowskip=3mm,
showstringspaces=false,
columns=flexible,
basicstyle={\small\ttfamily},
numbers=left,
numberstyle=\normalsize\color{gray},
keywordstyle=\color{blue},
commentstyle=\color{dkgreen},
stringstyle=\color{mauve},
breaklines=true,
breakatwhitespace=true,
tabsize=4
}


\begin{document}
\maketitle


\begin{enumerate}
\item
\text{Definition: A statement is a sentence that is either true or false.}

\begin{quote}
\begin{itemize}
\item
3 is prime $=>$ true
\item
5 is even $=>$ false
\end{itemize}
\
For something to be a statement, it should be completely precise and unambiguous.
For instance, "million is a big number" is not a statement as "big" is not really defined.
$\pi \approx 3.1415$ does not have a true value by itself.
We usually represent the sentences by the letters $P$ and $Q$ (capitalized).
Why $P$ and $Q$? These letters are called propositions (sort of tradition).
Proposition means things that we proved to be true.
$P : 11 > 4$, we can write $P$ is true.
\
Goal is to combine easy statements to get complex ones and then deal with them.
\end{quote}

\item
If $P$ is a statement, the negation of $P$ is false.
\begin{quote}
Example: $P$ : 91 is prime.
$~P$ : It is not the case that 91 is prime.
\end{quote}

A truth table
\\
$P  ~P$ \\
$T   F$
\

\item
$P \wedge Q$ is called the converse of $P \vee Q$ (the implication that is flipped backwards).
Converse is logically different from the original implication(-s).

\item
$P or ~P$ is called a tautology because it does not.

\item
Statement $(~Q) => (~P)$ is called the contrapositive of $P => Q$.
PAY ATTENTION THE THE FACT THAT $P$ AND $Q$ ARE FLIPPED.
You can replace implication with its contrapositive. On the other hand,
if you replace the implication with the converse, then you have not
basically done anything at this point.

\item
What is the negation of each of the following?

\begin{itemize}
\item
$~(~P) => P$
\item
The negtaion of $~(P \ and \ Q)$ is 3 true values and one false value \
$~(P \ and \ Q) => (~P) \ or \ (~Q)$
\end{itemize}

\end{enumerate}


\end{document}
