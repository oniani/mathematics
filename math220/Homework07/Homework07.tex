\documentclass[12pt, a4paper]{article}                      % use "amsart" instead of "article" for AMSLaTeX format
\usepackage[a4paper,margin=1in]{geometry}                   % Adjust margins
\usepackage{amsmath,amssymb,hyperref,mathtools,listings,color,textcomp,adjustbox,tikz,pgfplots}        % Math packages: amsmath, amssymb, listings, color
\usepgfplotslibrary{external,fillbetween}
\pgfplotsset{compat=1.14}
\usepackage[makeroom]{cancel}
\title{\bf{Homework \textnumero 7}}
\author{Author: David Oniani
\\
\ \ \ Instructor: Tommy Occhipinti}
\date{October 12, 2018}
\hypersetup{colorlinks=false, pdfborder={0 0 0}}
\usepackage{listings}
\usepackage{color}

\newcommand{\natn}{\mathbb{N}}
\newcommand{\intz}{\mathbb{Z}}
\newcommand{\intzp}{\mathbb{Z^+}}
\newcommand{\intzn}{\mathbb{Z^-}}

\definecolor{dkgreen}{rgb}{0,0.6,0}
\definecolor{gray}{rgb}{0.5,0.5,0.5}
\definecolor{mauve}{rgb}{0.58,0,0.82}
\definecolor{backcolour}{rgb}{0.95,0.95,0.92}

\lstset{
backgroundcolor=\color{backcolour},
aboveskip=3mm,
belowskip=3mm,
showstringspaces=false,
columns=flexible,
basicstyle={\small\ttfamily},
numbers=left,
numberstyle=\normalsize\color{gray},
keywordstyle=\color{blue},
commentstyle=\color{dkgreen},
stringstyle=\color{mauve},
breaklines=true,
breakatwhitespace=true,
tabsize=4
}


\begin{document}
\maketitle


\begin{itemize}
\item[43.]
Prove that for all $n \in \intzp$, $11^n - 6$ is divisible by 5.
\begin{quote}
Let's prove this by induction. For this we have to have a base case
and an inductive hypothesis.\\\\
Base case: if $n = 1$, $11^n - 6 = 11 - 6 = 5$ and thus, since 5 is divisible by 5, $11^n - 6$ is divisible by 5. Hence, the base case check is done.\\\\
Inductive case: suppose for all $k \in \intzp$, $11^k - 6$ is divisible by 5 and prove that $11^{k + 1} - 6$ is divisible by 5.
Notice that
$$11^{k + 1} - 6 = 11 \times 11^k - 6 = 10 \times 11^k + (11^k + 6)$$
Now, once again, notice that $10 \times 11^k$ is divisible by 5 since 10 is a multiple of 5. According to our assumption,
$11^k - 6$ is also divisible by 5.
Finally, we get that $11^{k + 1} - 6 = 10\times 11^k + (11^k + 6)$ which means that $11^{k + 1} - 6$ is the sum of two numbers which
are both divisible by 5.
\begin{flushright}
\textit{Q.E.D.}
\end{flushright}
\end{quote}

\item[]

\item[44.]
Prove that for $n \in \intzp$, we have $1^2 + 2^2 + 3^2 + ... + n^2 = \dfrac{n(n + 1)(2n + 1)}{6}$.
\begin{quote}
Let's prove this by taking an inductive approach.\\\\
Base case: if $n = 1$, $1^2 = \dfrac{1(1 + 1)(2 + 1)}{6} = \dfrac{1 \times 2 \times 3}{6} = 1$ and thus, the base case check is done.\\\\
Inductive hypothesis: suppose for $k \in \intzp$, $1^2 + 2^2 + 3^2 + ... + k^2 = \dfrac{k(k + 1)(2k + 1)}{6}$ and prove that $1^2 + 2^2 + 3^2 + ... + k^2 + (k + 1)^2 = \dfrac{(k + 1)(k + 2)(2k + 3)}{6}$.
According to our initial assumption, we have:\\
\begin{align*}
1^2 + 2^2 + 3^2 + ... + k^2 + (k + 1)^2 = (1^2 + 2^2 + 3^2 + ... + k^2) + (k + 1)^2 =\\
1^2 + 2^2 + 3^2 + ... + k^2 + (k + 1)^2 = \dfrac{k(k + 1)(2k + 1)}{6} + (k + 1)^2 =\\
\dfrac{k(k + 1)(2k + 1)}{6} + \dfrac{6(k + 1)^2}{6} =\\
\dfrac{k(k + 1)(2k + 1) + 6(k + 1)^2}{6} =\\
\dfrac{(k + 1)(k(2k + 1) + 6(k + 1))}{6} =\\
\dfrac{(k + 1)(2k^2 + k + 6k + 6)}{6} =\\
\dfrac{(k + 1)(2k^2 + 7k + 6)}{6} =\\
\dfrac{(k + 1)((k + 2)(2k + 3))}{6}
\end{align*}
Thus, we proved that $1^2 + 2^2 + 3^2 + ... + k^2 + (k + 1)^2 = \dfrac{(k + 1)(k + 2)(2k + 3)}{6}$ which finishes
our proof.
\begin{flushright}
\textit{Q.E.D.}
\end{flushright}
\end{quote}

\item[]

\item[45.]
Prove that for all $n \in \intzp$, $2^{n + 2} + 3^{2n + 1}$ is divisible by 7.
\begin{quote}
To prove it using induction, we'll need a base case and an inductive hypothesis.\\\\
Base case: let $n = 1$, then $2^{1 + 2} + 3^{2 \times 1 + 1} = 2^3 + 3^3 = 8 + 27 = 35 = 7 \times 5$ thus is divisible
by 7. And we've checked the base case.\\\\
Inductive hypothesis: now, we have to prove the inductive hypothesis. Here is the hypothesis:
suppose that for all $k \in \intzp$, $2^{k + 2} + 3^{2k + 1}$ is divisible by 7 and show that
$2^{k + 3} + 3^{2k + 3}$ is also divisible by 7.\\\\
Notice that
\begin{align*}
2^{k + 3} + 3^{2k + 3} = \underbrace{2^{k + 2} + 2^{k + 2}}_{2^{k + 3}} + \underbrace{3^{2k + 1} + 3^{2k + 1} + 7 \times 3^{2k + 1}}_{3^{2k + 3}} =\\
2 \times (2^{k + 2} + 3 ^ {2k + 1}) + 7 \times 3^{2k + 1}
\end{align*}
And now it's clear that since $2^{k + 2} + 3 ^ {2k + 1}$ and $7 \times 3^{2k + 1}$ are the multiples of 7,
$2 \times (2^{k + 2} + 3 ^ {2k + 1}) + 7 \times 3^{2k + 1}$ is also a multiple of 7 and thus is divisible by 7.
Hence, we proved the inductive hypothesis and it concludes our proof.
\begin{flushright}
\textit{Q.E.D.}
\end{flushright}
\end{quote}

\item[]

\item[46.]
Let $F_n$ denote the $n^{th}$ Fibonacci number. Prove that $\displaystyle\sum_{k=1}^{n} F^2_k = F_nF_{n + 1}$.
\begin{quote}
Let's prove this using induction. For this we have to have a base case and inductive hypothesis.\\\\
Let's first look at the Fibonacci sequence. It is 1, 1, 2, 3, 5, 8, 13, 21 ... where each element is the
sum of two previous elements.\\\\
Base case: let $k = 1$, then $\displaystyle\sum_{k=1}^{1} F^2_k = 1 \times 1 = 1$ and indeed the second
element of the sequence is 1.\\\\
Inductive hypothesis: let $F_m$ denote the $m^{th}$ Fibonacci number and also suppose that $\displaystyle\sum_{j=1}^{m} F^2_j = F_mF_{m + 1}$.\\
Then, we have to prove that $\displaystyle\sum_{j=1}^{m + 1} F^2_j = F_{m + 1}F_{m + 2}$.\\\\
Notice that $\displaystyle\sum_{j=1}^{m + 1} F^2_j = F_1^2 + F_2^2 + F_3^2 + ... + F_{m + 1}^2$ and using our assumption, we get:
$$F_1^2 + F_2^2 + F_3^2 + ... + F_{m + 1}^2 = F_mF_{m + 1} + F_{m + 1}^2 = F_{m + 1}(F_m + F_{m + 1})$$.
Now, let's remember the way Fibonacci numbers are created. To get the next Fibonacci number, one must sum up the previous two
thus, $F_m + F_{m + 1} = F_{m + 2}$ and we finally get that $\displaystyle\sum_{j=1}^{m + 1} F^2_j = F_1^2 + F_2^2 + F_3^2 + ... + F_{m + 1}^2 = F_mF_{m + 1} + F_{m + 1}^2 = F_{m + 1}(F_m + F_{m + 1}) = F_{m + 1}F_{m + 2}$.
Hence, we proved that $\displaystyle\sum_{j=1}^{m + 1} F^2_j = F_{m + 1}F_{m + 2}$ and this concludes our inductive reasoning as well as the proof.\\
Finally, if $F_n$ denotes the $n^{th}$ Fibonacci number, $\displaystyle\sum_{k=1}^{n} F^2_k = F_nF_{n + 1}$.
\begin{flushright}
\textit{Q.E.D.}
\end{flushright}
\end{quote}

\item[]

\item[47.]
Prove that for all $n \in \intzp_{\geq 12}$ we have $n! \geq 5^n$.
\begin{quote}
To prove the statement above, we can use induction. We need to have a base case as well as an inductive hypothesis.\\\\
Base case: let $n = 12$, then we have $12! > 5^{12}$ which is true since $12! = 479001600$ and $5^{12} = 244140625$ (thus $12! > 5^{12}$).\\\\
Inductive hypothesis: suppose that for every $k \in \intzp_{\geq 12}$ we have $k! \geq 5^k$ and prove that $(k + 1)! > 5^{k + 1}$.\\\\
Since we assumed that $k! \geq 5^k$, by multiplying both sides on $k + 1$, we have $(k + 1)! \geq 5^k (k + 1)$.
Now, since $k \geq 12$, we know that $k + 1 \geq 13 > 5$ and thus, since $(k + 1)! \geq 5^k (k + 1)$, we also know that
$(k + 1)! \geq 5^{k + 1}$.
\begin{flushright}
\textit{Q.E.D.}
\end{flushright}
\end{quote}

\item[]

\item[48.]
Suppose that $x \neq 0$ is a real number and $x + \dfrac{1}{x} \in \intz$.
Prove that $x^n + 1/x^n \in \intz$ for all $n \in \intzp$. (Hint: $(x + \dfrac{1}{x})(x^n + \dfrac{1}{x^n})$ is probably nicer than you think!).
\begin{quote}
Let's use induction to prove this. Hence, we need the base case and an inductive hypothesis.\\\\
Base case: if $x = 1$, $x + \dfrac{1}{x} = 1 + \dfrac{1}{1} = 2$ thus, we have that if $x = 1$
$x + \dfrac{1}{x}$ is also in $\intz$. Now, we have to check $x^n + \dfrac{1}{x^n}$.
$x^n + \dfrac{1}{x^n} = 1^n + \dfrac{1}{1^n} = 1 + \dfrac{1}{1} = 2$ and the base case is checked.\\\\
Now, we have to come up with an inductive hypothesis.\\
Inductive hypothesis is that if $y \neq 0$ and $y + \dfrac{1}{y} \in \intz$, we have $y^m + \dfrac{1}{y^m} \in \intz$ and we have to prove that for all $m \in \intzp$,
$y^{m + 1} + \dfrac{1}{y^{m + 1}} \in \intz$.\\\\
Notice that
$$(y + 1/y)(y^m + 1/y^m) =  y^{m + 1} + \dfrac{1}{y^{m + 1}} + y^{m - 1} + \dfrac{1}{y^{m - 1}}$$
and
$$y^{m + 1} + \dfrac{1}{y^{m + 1}} = \underbrace{(y + \dfrac{1}{y})(y^m + \dfrac{1}{y^m})}_{\mbox{we know it's an integer}} - \underbrace{(y^{m - 1} + \dfrac{1}{y^{m - 1}})}_{\mbox{it's an integer too}}$$
By the assumption we made in the inductive hypothesis, we know that $(y + \dfrac{1}{y})(y^m + \dfrac{1}{y^m})$ is an integer.
According to our assumption, $y^{m - 1} + \dfrac{1}{y^{m - 1}}$ is also an integer and if we subtract one integer from the other, we always end up with an integer.
Thus, $y^{m + 1} + \dfrac{1}{y^{m + 1}} = (y + \dfrac{1}{y})(y^m + \dfrac{1}{y^m}) - (y^{m - 1} + \dfrac{1}{y^{m - 1}})$ which means that $y^{m + 1} + \dfrac{1}{y^{m + 1}}$
is an integer and this concludes our proof.
\begin{flushright}
\textit{Q.E.D.}
\end{flushright}
\end{quote}

\item[49.]
At about 2:00 he discusses the fact that $1 + 3 + 5 + 7 + ... + (2n - 1) = n^2$. That is,
the sum of the first n odd numbers is $n^2$. Through triangular squares, he gives a
striking visual proof of this fact. Give a proof of this fact using mathematical induction
(without any need to reference triangular squares).
\begin{quote}
Base case: let $n = 1$, then $1 = 1^2$ and the base case check is done.\\\\
Inductive hypothesis: suppose that for $k \geq 1$, $1 + 3 + 5 + 7 + ... + (2k - 1) = k^2$
and prove that $1 + 3 + 5 + 7 + ... + (2k - 1) + (2k + 1) = (k + 1)^2$.\\\\
We have:
\begin{align*}
1 + 3 + 5 + 7 + ... + (2k - 1)^2 + (2k + 1)^2 =\\
(1 + 3 + 5 + 7 + ... + (2k - 1)^2) + (2k + 1)^2 =\\
k^2 + (2k + 1) =\\
k^2 + 2k + 1 =\\
(k + 1)^2
\end{align*}
Thus, we proved that $1 + 3 + 5 + 7 + ... + (2k - 1) + (2k + 1) = (k + 1)^2$ which
concludes our proof by induction.
\begin{flushright}
\textit{Q.E.D.}
\end{flushright}
\end{quote}

\item[]
\item[]
\item[]
{\large Mathologer video: \url{https://www.youtube.com/watch?v=yk6wbvNPZW0}}

\item[50.]
At about 6:35 he shows a diagram and says “at this point, the dark green overlap will
be exactly as large as the white empty area.” Why is that?
\begin{quote}
Well, it is not that they actually are "exactly" the same. If we look closer we see that
that white empty area is $5^2 + 5^2 = 50$ while the area of the dark green triangle is $7^2 = 49$.
He said that they must be equal because he is trying to prove that $\sqrt{2}$ is rational. In other words,
since two triangles overlap on that dark green triangle, we basically have 2 layers of the same triangle
there so we can just take it off and hope that it fits into the white area. If it fits, it means that $sqrt{2}$
is rational and otherwise, if it does not, it means that $\sqrt{2}$ is irrational. And obviously it does not fit
because the areas are different (green triangular area is 49 and the white area is 50).
\end{quote}

\item[]

\item[51.]
At 7:20, he suggests that there is an even easier way to reach a contradiction. What
is it?
\begin{quote}
It's what I mentioned above. That rhombus-looking white area is basically two triangles.
Both of the triangles have 5 rows and we know that their areas are $5^2$, then $2 \times 5^2 =50$.
On the other hand, the area of the dark green triangle is $7^2 = 49$. Thus, we cannot perform
any geometric action (rotation, flipping, etc.) to somehow make the dark-green-area triangle fit into
the white space.
\end{quote}

\item[]
\item[52.]
At the very end of the video, he gives a demonstration that $T_{20} + T_{20} + T_{20} = T_{35}$, and
uses that identity to produce the smaller identity $T_5 + T_5 + T_5 = T_9$, much like the key step in
the proofs earlier in the video that $\sqrt{3}$ and $\sqrt{2}$ are irrational. Why is it not a contradiction
here?
\begin{quote}
That's because $T_{20} + T_{20} + T_{20} = T_{35}$ and there are no leftover triangles.
Then, we can just use the fact that $T_{20} + T_{20} + T_{20} = T_{35}$ as a base case and prove that $T_5 + T_5 + T_5 = T_9$.
NOTE: Induction is not necessarily about increasing numbers. We can just set something to $1000 - x$ and increase $x$ each time
while $1000 - x$ decreases. So that's why we can use $T_{20} + T_{20} + T_{20} = T_{35}$ as a base case and prove it
for smaller triangles.
\end{quote}

\newpage
{\Large Bookwork}
\item[]
\item[5.]
Show: $n^2 \leq 2^n$ for each integer $n \geq 4$.
\begin{quote}
Let's use induction.\\\\
Base case: let $n = 4$, then $2^4 \leq 2^4$ which is equivalent to $16 \leq 16$ which is true
thus, the base case check is done.\\\\
Inductive hypothesis: suppose that for $k \geq 4$, $k^2 \leq 2^k$ and prove that $(k + 1)^2 \leq 2^{k + 1}$.\\\\
Notice that since we assumed that $k^2 \leq 2^k$ for $k \geq 4$, we have effectively assumed that $2k^2 \leq 2^{k + 1}$.
Thus, if we now prove that for $k \geq 4$, $2k^2 \geq (k + 1)^2$, the proof is done. So, let's just solve the inequality $k \geq 4$, $2k^2 \geq (k + 1)^2$
and see if it is true for $k \geq 4$. We have:
\begin{align}
2k^2 \geq (k + 1)^2\\
2k^2 \geq k^2 + 2k + 1\\
k^2 \geq 2k + 1\\
k^2 - 2k - 1 \geq 0\\
(k - (1 + \sqrt{2}))(k - (1 - \sqrt{2})) \geq 0
\end{align}
And finally, we get that $k \in (-\infty, 1 - \sqrt{2}) \cup (1 + \sqrt{2}, +\infty)$.
Now, notice that $4 > 1 + \sqrt{2}$ and thus, for all $k \geq 4$, this inequality holds
and this concludes our proof.
\end{quote}

\item[]
\item[12.]
Let $b_n = 1/3 + 1/15 + ... + 1/(4n^2 - 1)$.
\begin{itemize}
\item[(a)]
Compute $b_1, b_2, b_3, b_4, \mbox{ and } b_5$.
\begin{quote}
$b_1 = \dfrac{1}{4 \times 1^2 - 1} = 1/3$\\
$b_2 = b_1 + \dfrac{1}{4 \times 2^2 - 1} = 1/3 + 1/15 = 2/5$\\
$b_3 = b_1 + b_2 + \dfrac{1}{4 \times 3^2 - 1} = 1/3 + 1/15 + 1/35 = 3/7$\\
$b_4 = b_1 + b_2 + b_3 + b_4 = 3/7 + \dfrac{1}{4 \times 4^2 - 1} = 3/7 + 1/63 = 4/9$\\
$b_5 = b_1 + b_2 + b_3 + b_4 + \dfrac{1}{4 \times 5^2 - 1} = 3/9 + 1/99 = 5/11$.
\end{quote}

\item[]
\item[]
\item[]

\item[(b)]
Conjecture a closed formula for $b_n$.
\begin{quote}
I conjectured that $b_n = \dfrac{n}{2n + 1}$.\\
Here is the explanation:
\begin{align*}
b_{\underbar 1} = 1/3 = \underbar 1 / (2 \times \underbar 1 + 1)\\
b_{\underbar 2} = 2/5 = \underbar 2 / (2 \times \underbar 2 + 1)\\
b_{\underbar 3} = 3/7 = \underbar 3 / (2 \times \underbar 3 + 1)\\
b_{\underbar 4} = 4/9 = \underbar 4 /(2 \times \underbar 4 + 1)\\
b_{\underbar 5} = 5/11 =  \underbar 5 /(2 \times \underbar 5 + 1)\\
.................................................
\end{align*}
After what we have jotted down, I think it's very likely that the pattern is $b_n = \dfrac{n}{2n + 1}$.
Thus, my conjecture is that $b_n = \dfrac{n}{2n + 1}$ and all we have left is to prove it!
\end{quote}

\item[]

\item[(c)]
Prove your conjecture by mathematical induction.
\begin{quote}
To prove that $b_n = \dfrac{n}{2n + 1}$, we'll use induction. Thus, we need a base case check and a proof of inductive hypothesis.\\\\
Base case: let $n = 1$, then $b_1 = \dfrac{1}{2 \times 1 + 1} = 1/3$. Thus the base case check is complete.\\\\

Inductive hypothesis: suppose that for $k \geq 1$, $b_k = \dfrac{k}{2k + 1}$ and prove that $b_{k + 1} = \dfrac{k + 1}{2k + 3}$.\\\\
Notice that $b_{k + 1} = b_k + \dfrac{1}{4(k + 1)^2 - 1}$.\\
Thus, according to our assumption in the inductive hypothesis, we have:
\begin{align*}
b_{k + 1} = b_k + \dfrac{1}{4(k + 1)^2 - 1} = \dfrac{k}{2k + 1} + \dfrac{1}{4(k + 1)^2 - 1} =\\
\dfrac{k}{2k + 1} + \dfrac{1}{(2k + 1)(2k + 3)} =\\
\dfrac{k (2k + 3)}{(2k + 1)(2k + 3)} + \dfrac{1}{(2k + 1)(2k + 3)} =\\
\dfrac{k (2k + 3) + 1}{(2k + 1)(2k + 3)} =\\
\dfrac{2k^2 + 3k + 1}{(2k + 1)(2k + 3)} =\\
\dfrac{(2k + 1)(k + 1)}{(2k + 1)(2k + 3)} =\\
\dfrac{k + 1}{2k + 3}
\end{align*}
Thus, we got that $b_{k + 1} = \dfrac{k + 1}{2k + 3}$ which concludes our proof.
\end{quote}
\begin{flushright}
\textit{Q.E.D.}
\end{flushright}

\item[(d)]
Show that the infinite series $\displaystyle\sum_{n=1}^{\infty}\dfrac{1}{4n^2 - 1}$ converges and determine its sum.
\begin{quote}
Since we now know that $b_n = \dfrac{n}{2n + 1}$, then we have:
\begin{align*}
\lim_{n \rightarrow \infty}\displaystyle\sum_{n=1}^{\infty} \dfrac{1}{4n^2 - 1} = \lim_{n \rightarrow \infty}\dfrac{n}{2n + 1}\\
\lim_{n \rightarrow \infty}\dfrac{n}{2n + 1} = \lim_{n \rightarrow \infty}\dfrac{1}{2 + \dfrac{1}{n}} = \dfrac{1}{2}
\end{align*}
Hence, we found that the sum has a limit when $n \rightarrow \infty$ which is $\dfrac{1}{2}$ and thus, the series converges.
Finally, it converges and the sum's it's limit when $n$ approaches infinity is $\dfrac{1}{2}$.
\end{quote}
\end{itemize}

\item[]

\item[13.]
Let $c_n = 1^3 + ... + n^3$.
\begin{itemize}
\item[(a)]
\begin{align*}
c_1 = 1^3 = 1\\
c_2 = 1^3 + 2^3 = 1 + 8 = 9\\
c_3 = 1^3 + 2^3 + 3^3 = 1 + 8 + 27 = 36\\
c_4 = 1^3 + 2^3 + 3^3 + 4^3 = 1 + 8 + 27 + 64 = 100\\
c_5 = 1^3 + 2^3 + 3^3 + 4^3 5^3 = 1 + 8 + 27 + 64 + 125 = 225
\end{align*}
\begin{quote}

\item[]

\item[(b)]
Conjecture a closed formula for $c_n$.
\begin{quote}
I conjectured that $c_n = \dfrac{n^2(n + 1)^2}{4}$.\\
Here is why:\\\\
\begin{align*}
c_{\underbar 1} = 1 = \dfrac{\underbar 1^2(\underbar 1 + 1)^2}{4}\\
c_{\underbar 2} = 9 = \dfrac{\underbar 2^2(\underbar 2 + 1)^2}{4}\\
c_{\underbar 3} = 36 = \dfrac{\underbar 3^2(\underbar 3 + 1)^2}{4}\\
c_{\underbar 4} = 100 = \dfrac{\underbar 4^2(\underbar 4 + 1)^2}{4}\\
c_{\underbar 5} = 225 = \dfrac{\underbar 5^2(\underbar 5 + 1)^2}{4}
\end{align*}
\end{quote}

\item[]

\item[(c)]
Prove your conjecture.
\begin{quote}
Let's prove this conjecture using mathematical induction. Thus, we first need to check the base case
and the proceed by proving an inductive hypothesis.\\\\
Base case: let $n = 1$, then $c_1 = \dfrac{\underbar 1^2(\underbar 1 + 1)^2}{4} = 1$ which is correct
thus, this finishes the base case check.\\\\
Inductive hypothesis: suppose that $c_k = \dfrac{k^2(k + 1)^2}{4}$ where $k \geq 1$ and prove that
$c_{k + 1} = \dfrac{(k + 1)^2(k + 2)^2}{4}$.\\\\
According to our assumption in the inductive hypothesis, we have:
\begin{align*}
c_{k + 1} = c_k + (k + 1)^3 = \dfrac{k^2(k + 1)^2}{4} + (k + 1)^3 =\\
\dfrac{k^2(k + 1)^2 + 4(k + 1)^3}{4} =\\
\dfrac{(k + 1)^2(k^2 + 4(k + 1))}{4} =\\
\dfrac{(k + 1)^2(k^2 + 4k + 4)}{4} =\\
\dfrac{(k + 1)^2(k + 2)^12}{4}
\end{align*}
Thus, we got that $c_{k + 1} = \dfrac{(k + 1)^2(k + 2)^2}{4}$ which concludes our proof.
\begin{flushright}
\textit{Q.E.D.}
\end{flushright}
\end{quote}
\end{quote}
\end{itemize}

\item[]

\item[16.]
Conjecture and prove a closed formula for the product $(1 - 1/4) \times (1 - 1/9) \times ... \times (1 - 1/n^2)$.
\begin{quote}
I conjectured that $P_n = \dfrac{n + 1}{2n}$ for $n \geq 2$.\\
Here is why:\\
\begin{align*}
P_{\underbar 1} = 1 - 1/4 = 3/4 = \dfrac{\underbar 2 + 1}{2 \times \underbar 2}\\
P_{\underbar 2} = (1 - 1/4) \times (1 - 1/9) = 3/4 \times 8/9 = 2/3 = \dfrac{\underbar 3 + 1}{2 \times \underbar 3}\\
P_{\underbar 3} = (1 - 1/4) \times (1 - 1/9) \times (1 - 1/16) = 3/4 \times 8/9 \times 15/16 = 5/8 = \dfrac{\underbar 4 + 1}{2 \times \underbar 4}\\
\end{align*}

Now, let's prove it using mathematical induction.\\\\
Base case: let $n = 2$, then $\dfrac{\underbar 2 + 1}{2 \times \underbar 2} = 1/4$ which is true.
Hence the base case is checked.\\\\
Inductive hypothesis: suppose that for $k \geq 1$, $P_k = \dfrac{k + 1}{2k}$. Then we have
to show that $P_{k + 1} = \dfrac{k + 2}{2(k + 1)}$.\\\\
We get:
\begin{align*}
P_{k + 1} = P_k \times (1 - \dfrac{1}{(k + 1)^2}) =\\
\dfrac{k + 1}{2k} \times (1 - \dfrac{1}{(k + 1)^2}) =\\
\dfrac{k + 1}{2k} \times \dfrac{k^2 + 2k}{(k + 1)^2} =\\
\dfrac{k + 2}{2(k + 1)}
\end{align*}
Thus, we proved that $P_{k + 1} = \dfrac{k + 2}{2(k + 1)}$ which concludes our proof by indunction.
\end{quote}

\item[18.]
Prove: For every positive integer $n$, 5 divides $n^5 - n$.
\begin{itemize}
\item[(b)]
\begin{quote}
Let's prove it using induction. Then we need a base case and an inductive hypothesis.\\\\
Base case: let $n = 1$, then $1^5 - 1 = 0$ and 0 is indeed divisible by 5 thus the base case is checked.\\\\
Inductive hypothesis: suppose that $k^5 - k$, where $k \geq 1$, is divisible by 5 and prove that $(k + 1)^5 - (k + 1)$
is also divisible by 5.\\\\
We have:
\begin{align*}
(k + 1)^5 - (k + 1) = (k + 1)((k + 1)^4 - 1) =\\
(k + 1)(((k + 1)^2)^2 - 1) =\\
(k + 1)((k^2 + 2k + 1)^2 - 1) =\\
(k + 1)(k^4 + 4k^2 + 1 + 4k^3 + 4k + 2k^2 - 1) =\\
(k + 1)(k^4 + 4k^3 + 6k^2 + 4k) =\\
k^5 + 4k^4 + 6k^3 + 4k^2 + k^4 + 4k^3 + 6k^2 + 4k =\\
k^5 + 5k^4 + 10k^3 + 10k^2 + 4k
\end{align*}
Thus, we got that $(k + 1)^5 - (k + 1) = k^5 + 5k^4 + 10k^3 + 10k^2 + 4k$. Since $5k^4 + 10k^3 + 10k^2$
is divisible by 5 we only have to prove that $k^5 + 4k$ is divisible by 5 so that then their sum: $5k^4 + 10k^3 + 10k^2 + k^5 + 4$
is equal to $(k + 1)^5 - (k + 1)$ and is divisible by 5.\\\\
Now, notice that $k^5 + 4k = k^5 - k + 5k$. According to our assumption in the inductive hypothesis, $k^5 - k$ is divisible
by 5 and $5k$ is five times integer positive $k$ thus, is also a multiple of 5. Then, we have that $k^5 + 4k$ is the sum of
two numbers which are divisible by 5 and $k^5 + 4k$ is itself divisible by 5.\\\\
Thus, we got that $(k + 1)^5 - (k + 1)$ is divisible by 5 which concludes our proof.
\begin{flushright}
\textit{Q.E.D.}
\end{flushright}
\end{quote}
\end{itemize}
\end{itemize}
\end{document}
