\documentclass[12pt, a4paper]{article}                      % use "amsart" instead of "article" for AMSLaTeX format
\usepackage[a4paper,margin=1in]{geometry}                   % Adjust margins
\usepackage{amsmath,amssymb,listings,color,textcomp}        % Math packages: amsmath, amssymb, listings, color
\usepackage[makeroom]{cancel}

\title{\bf{Homework \textnumero 5}}
\author{Author: David Oniani
\\
\ \ \ Instructor: Tommy Occhipinti}
\date{September 23, 2018}

\usepackage{listings}
\usepackage{color}

\definecolor{dkgreen}{rgb}{0,0.6,0}
\definecolor{gray}{rgb}{0.5,0.5,0.5}
\definecolor{mauve}{rgb}{0.58,0,0.82}
\definecolor{backcolour}{rgb}{0.95,0.95,0.92}

\lstset{
backgroundcolor=\color{backcolour},
aboveskip=3mm,
belowskip=3mm,
showstringspaces=false,
columns=flexible,
basicstyle={\small\ttfamily},
numbers=left,
numberstyle=\normalsize\color{gray},
keywordstyle=\color{blue},
commentstyle=\color{dkgreen},
stringstyle=\color{mauve},
breaklines=true,
breakatwhitespace=true,
tabsize=4
}


\begin{document}
\maketitle

{\large Section 3.2}
\\

\begin{enumerate}
\item[23.]
Prove that if $x$ and $y$ are integers and $xy - 1$ is even then $x$ and $y$ are odd.
\begin{quote}
Let's prove it by contrapositive. Contrapositive of the initial statement (which is equivalent to the initial statement) is:
\begin{center}
If $x$ is even or $y$ is even, then $xy - 1$ is odd.
\end{center}
If $x$ is even or $y$ is even, $xy$ is even. Then we can write that $xy = 2k$ where $k \in \mathbb{Z}$.
Then, $xy - 1 = 2k - 1 = 2(k - 1) + 1$ where $k \in \mathbb{Z}$. Now, let $t = l - 1$ where $l \in \mathbb{Z}$
and we get $xy - 1 = 2t + 1$. Thus, $xy - 1$ is odd.
\begin{flushright}
\textit{Q.E.D.}
\end{flushright}
\end{quote}

\item[24.]
Prove that if $x$ and $y$ are real numbers whose mean is $m$ then at least one of\\
$x$ and $y$ is $\geq m$.
\begin{quote}
Suppose, for the sake of contradiction, that $x$ and $y$ are both $< m$.
Then,
$$x < m$$
$$y < m$$
By adding the inequalities, we get:
$$
x + y < 2m
$$
And finally,
$$
\dfrac{x + y}{2} < m
$$
which contradicts the initial statement that the mean of $x$ and $y$ is $m$.
\begin{flushright}
\textit{Q.E.D.}
\end{flushright}
\end{quote}

\item[25.]
Suppose $S$ is a set of 250 distinct real numbers whose mean is 4. Must there exists
$x \in S$ such that $x > 4$? Be sure to prove your answer.
\begin{quote}
Yes. Let's prove it!\\
Suppose, for the sake of contradiction, that all elements of $S$ are $\leq 4$.
Then the sum of all the elements will be less $\leq 1000$ with equality happening only
when all the members of the set are equal to $4$ which contradict the initial statement that
$S$ is a set of 250 \underline{distinct} elements. Thus, only one of the elements of $S$ is allowed
to be equal to 4. Finally, we get two cases:
\begin{center}
\begin{itemize}
\item[1.]
All 250 elements of $S$ are less than 4.
\item[2.]
249 elements of $S$ are less than 4 and one is equal to 4.
\end{itemize}
\end{center}
If all 250 elements of $S$ are less than 4, then their sum is less than $4 \times 250 = 1000$ and their mean
is less than $1000 / 250 = 4$ which contradicts the initial statement that the mean of all elements of $S$ is 4.

If 249 elements of $S$ are less than 4 and one is equal to 4, then the sum of 249 elements which are less than 4 is less than
$249 \times 4 = 996$. Then let this sum of 249 numbers be equal to $996 - k$ where $k > 0$. Then the sum of all the elements including
the one which equals 4 is:
$$
996 - k + 4 = 1000 - k \mbox{ where } k > 0
$$
Using the fact above, we get that the mean of all the elements of $S$ is $(1000 - k)/250$ where $k > 0$.
And finally, we get:
$$
\dfrac{1000 - k}{250} = 4 - \dfrac{k}{250} \mbox{ where } k > 0
$$
And $4 - \dfrac{k}{250} \mbox{ where } k > 0$ is clearly less than 4 which contradicts
the initial claim that the mean of all elements of $S$ is 4.
\begin{flushright}
\textit{Q.E.D.}
\end{flushright}
\end{quote}

\item[26.]
Suppose $a, b, c \in \mathbb{Z}$ and $a^2 + b^2 = c^2$.
Prove that at least one of a and b is even.
\begin{quote}
Suppose, for the sake of contradiction, that both $a$ and $b$ are odd.
Then, we can write $a = 2k - 1$ and $b = 2l - 1$ where $k,l \in \mathbb{Z}$.
Then, we have:
$$
a^2 + b^2 = 4k^2 - 4k + 1 + 4l^2 - 4l + 1 = 4k^2 + 4l^2 - 4l - 4k + 2 = \\
$$
$$
= 2 \times (2k^2 + 2l^2 - 2l - 2k + 1)
$$

Now, it's easy to see that $a^2 + b^2$ is the multiplication of an even and odd
integers ($2$ is even and $2k^2 + 2l^2 - 2l - 2k + 1$ is odd). $2k^2 + 2l^2 - 2l - 2k + 1$ is
odd since $2k^2 + 2l^2 - 2l - 2k + 1 = 2 \times (k^2 + l^2 - l - k) + 1$ and if we let $t = k^2 + l^2 - l - k$
where $t \in \mathbb{Z}$ (since $k^2 + l^2 - l - k \in \mathbb{Z}$), then we have that $2k^2 + 2l^2 - 2l - 2k + 1 = 2t + 1$
which is an even number plus one which is always odd. Finally, we conclude that $2$ is only once in the number that is supposed to be a perfect square as
$2k^2 + 2l^2 - 2l - 2k + 1$ is odd and is not a multiple of 2 which means
that $a^2 + b^2$ is not a perfect square which contradicts the initial claim that the sum
$a^2 + b^2$ is the perfect square.
\begin{flushright}
\textit{Q.E.D.}
\end{flushright}
\end{quote}

\item[27.]
Prove that if $x,y \in \mathbb{R^+}$, then $x + y \geq 2\sqrt{xy}$.
\begin{quote}
Suppose, for the sake of contradiction, that $x + y < 2\sqrt{xy}$.
Then, since $x,y \in \mathbb{R^+}$, we have:
\begin{align}
x + y < 2\sqrt{xy}\\
x^2 + y^2 + 2xy < 4xy\\
x^2 + y^2 + 2xy -4xy < 0\\
x^2 + y^2 - 2xy < 0\\
(x - y)^2 < 0
\end{align}
Thus, we got that $(x-y)^2 < 0$ which is false since the square of a number
is always $\geq 0$. Finally, since by assuming that $x + y < 2 \sqrt{xy} \mbox{ where } x, y\in \mathbb{R^+}$, we basically
got the nonsensical inequality $(x-y)^2 < 0$, something has to be wrong with this assumption and we got that
if $x,y \in \mathbb{R^+}$, then $x + y \geq 2\sqrt{xy}$
\begin{flushright}
\textit{Q.E.D.}
\end{flushright}
\end{quote}

\item[28.]
Prove that if $n$ is an integer, there exist three consecutive integers that sum\\
to $n$ if and only if $n$ is a multiple of 3.
\begin{quote}
Let's first prove that if $n$ is not a multiple of 3, one cannot find
three consecutive integers with the property that they sum to $n$.
\begin{itemize}
\item[(a)]
Suppose, for the sake of contradiction, that $n$ is not a multiple of 3.
Then let's define three consecutive integers, $m, m + 1$ and $m + 2$, where $m \in \mathbb{Z}$.
Then we have:
$$
m + m + 1 + m + 2 = 3m + 3 = 3 \times (m + 1)
$$
Thus, we got that the sum of three consecutive integers is a multiple of 3
which contradicts the statement that $n$ is not a multiple of 3.
\end{itemize}
Now, lets prove the second half of the problem. Let's show that
if three consecutive integers sum to $n$, then $n$ is a multiple
of 3.
\begin{itemize}
\item[(b)]
Let $m, m+1, m+2$ where $m \in \mathbb{Z}$ be three consecutive integers.
We have:
$$
n = m + m + 1 + m + 2 = 3m + 3 = 3\times(m+1)
$$
Thus, we got that $n$ is a multiple of 3 which proves the iff.
\end{itemize}
\begin{flushright}
\textit{Q.E.D.}
\end{flushright}
\end{quote}


\item[29.]
A subset $S$ of $\mathbb{R}$ has the property that for all $x \in \mathbb{R}$
there exists $y \in S$ such that $|x - y| < 1$. Prove that $S$ is infinite.

\begin{quote}
Suppose, for the sake of contradiction, that $S$ is finite.
Inequality, $|x - y| < 1$ can be transformed into the following system:
$$
\begin{cases}
x - y < 1\\
x - y > -1
\end{cases}
$$
And from the system above, we get the following system:
$$
\begin{cases}
y > x - 1\\
y < x + 1
\end{cases}
$$
Hence, we know that $y$ is in the open interval $(x-1, x+1)$.
Now, since we also know that $x \in \mathbb{R}$, interval $(x-1, x+1)$
has infinitely many elements in it which contradicts our assumption
that $S$ is finite.
\begin{flushright}
\textit{Q.E.D.}
\end{flushright}
\end{quote}

\item[30.]
\begin{quote}
A subset $S$ of $\mathbb{Z}$ is called \textbf{non-differential} if for every $x, y \in S$ we have
$x - y \notin S$. Here are some statements about non-differential sets. Decide which statements are
true and which are false, and provide a proof or counterexample for each as appropriate.
\begin{itemize}
\item[]
Before going right into the proof, note that in any set we can do self-subtration
(e.g, if the set is $\{1, 3\}$), we can write $1 - 1 = 0$. HOWEVER, we are not going
to consider those trivial cases. We are going to consider it if and only if element
0 is in the set (because all the self-subtractions are zero and it is only the case when
the element 0 is in the set when such subtractions make sense in proving or disproving something).
\item[(a)]
Every non-differential set is finite.
\begin{quote}
This is false. Counterexample:\\
Let $S = \{1,3,5,7,9,11...\}$ thus, $S$ is a set of all positive odd integers.
Then we know that for every $x, y$, $x - y$ is even. But all the members of $S$
are odd. Thus, For every $x,y \in S$, $x - y \notin S$ and $S$ is an infinite set
which also turns out to be non-differential and the initial statement is false.
\end{quote}
\item[(b)]
The intersection of two non-differential sets is non-differential.
\begin{quote}
This is true. Let's prove it.\\
Let $x, y \in A \cap B$. Then, since $x, y \in A$, $x - y \notin A$ as well as $x - y \notin A \cap B$. 
\begin{flushright}
\textit{Q.E.D.}
\end{flushright}
\end{quote}
\item[(c)]
The union of two non-differential sets is non-differential.
\begin{quote}
This is false. Counterexample:\\
Let $A = \{1, 3\}$ then $A$ is non-differential since $1 - 3  \notin A$
and $3 - 1 \notin A$. Now, let $B = \{1, 4\}$, then $B$ is non-differential too
as $1 - 4 \notin B$ and $4 - 1 \notin B$. Finally, we get $A \cup B = \{1, 3, 4\}$
which is \underline{NOT} non-differential because $4 - 3 = 1 \in A \cup B$.
\end{quote}
\item[(d)]
No non-differential set contains the element 0.
\begin{quote}
It's true.\\
For a set to be non-differential there should be no $x, y$ such that $x - y \in S$.
For the sake of contradiction, suppose that we have a non-differential set $A$ such that
$0 \in A$. If $A$ has more than one elements, let the other element (any element which is
not 0) be $k$. Then we get $k - 0 = k \in A$ which contradicts the initial statement that
$A$ is non-differential as we found two elements $x = 0$ and $y = k$ such that $x - y \in A$.
If $A$ has only one element which is $0$, then it is \underline{NOT} non-differential anyway,
because $0 - 0 = 0 \in A$. Hence, no non-differential set contains element 0.
\end{quote}
\item[(e)]
Every subset of a non-differential set is non-differential.
\begin{quote}
It's true.\\
Suppose we have two sets, $A$ and $B$ such that $B \subseteq A$ and $A$ is non-differential.
Let $x \in B$, then we know that there exists no $y$ in $A$ such that $x - y \in A$.
Since such $y$ does not exist in $A$, it does not exist in $B$ is as well since it is the subset
of $A$.
\begin{flushright}
\textit{Q.E.D.}
\end{flushright}
\end{quote}
\item[(f)]
There is no non-differential set with exactly 5 elements.
\begin{quote}
It's false. Counterexample:\\
Let $A = \{1, 3, 8, 19, 50\}$, then we have:
\begin{center}
$1 - 3 = -2 \notin A$\\
$3 - 1 = 2 \notin A$\\
$3 - 8 = -5 \notin A$\\
$8 - 3 = 5 \notin A$\\
$8 - 19 = -11 \notin A$\\
$19 - 8 = 11 \notin A$\\
$19 - 50 = -31 \notin A$\\
$50 - 19 = 31 \notin A$\\
\end{center}
\end{quote}
\item[(g)]
If $S$ is non-differential, so is $\mathbb{Z} - S$.
\begin{quote}
It's false. Counterexample:\\
Let $A = \{1, 3\}$. $A$ is non-differential since $1 - 3 = -2 \notin A$ and $3 - 1 = 2 \notin A$.
Then we know that $Z - A$ would include numbers $7,8,15$. But $15 - 8 = 7 \in Z - A$ which is not
non-differential.
\end{quote}
\item[(h)]
If $S$ is a non-differential set, then so is the $S_{+3} = \{x + 3 \ | \ x \in S\}$.
\begin{quote}
It's false. Counterexample:\\
Let $A = \{1, 3, 8, 19, 50\}$, then $A$ is non-differential since:
\begin{center}
$1 - 3 = -2 \notin A$\\
$3 - 1 = 2 \notin A$\\
$3 - 8 = -5 \notin A$\\
$8 - 3 = 5 \notin A$\\
$8 - 19 = -11 \notin A$\\
$19 - 8 = 11 \notin A$\\
$19 - 50 = -31 \notin A$\\
$50 - 19 = 31 \notin A$\\
\end{center}
$A_{+3} = \{4, 7, 11, 22, 53\}$. Now, notice that $22 - 11 = 11 \in A_{+3}$ thus, we found
two elements $x = 22$ and $y = 11$ such that $x - y \in A$ and so $A$ is \underline{NOT} non-differential.
Thus, the initial statement is false.
\end{quote}
\end{itemize}
\item[31.]
A subset $A$ of $\mathbb{R}$ is called \textbf{cofinite} if $\mathbb{R} - A$ is finite. Here are some statements about
cofinite sets. Decide which statements are true and which are false, and provide a proof
or counterexample for each as appropriate
\begin{itemize}
\item[]
Before jumping in the proofs, let's make a little note.
If $A$ is cofinite, it is some subset of $\mathbb{R}$.
Then, we can represent it as $A = \mathbb{R} - F$
where $F$ is some finite set.\\
Why finite?\\
If $F$ be infinite, then $\mathbb{R} - (\mathbb{R} - F) = F$ is also
infinite and this contradicts the fact that $A$ is cofinite.
Thus, we know (and will use) the fact that any \underline{cofinite} set $A$
can be represented as $\mathbb{R} - F$ where $F$ is a finite set.
\item[(a)]
If $A \subseteq B$ and $B$ is cofinite then $A$ is cofinite.
\begin{quote}
It's false. Counterexample:\\
Let $B = \mathbb{R} - \{0, 1\}$. Then $B$ is cofinite as $\mathbb{R} - B = \{0, 1\}$.
Now let $A = \{-1, -2\}$, then $A \subseteq B$, however, $\mathbb{R} - A$ is not finite
as $R$ has an infinite number of elements and subtracting only a finite number of elements
(2 elements) still leaves it will infinitely many.
\end{quote}
\item[(b)]
There exist two cofinite sets $A$ and $B$ with the property that\\
$A \cap B = \emptyset$.
\begin{quote}
It's false. Suppose, for the sake of contradiction, that $A$ and $B$ are cofinite
sets. Then we know that both $A$ and $B$ are of the form $\mathbb{R} - F$ where $F$
is some finite set (if $F$ is infinite, $\mathbb{R} - (\mathbb{R} - F) = F$ and the set
is \underline{NOT} cofinite). Let $A = \mathbb{R} - C$ and $B = \mathbb{R} - D$. Then we know that
both $A$ and $B$ contain sets $\mathbb{R} - C - D$. Thus $\mathbb{R} - C - D \subseteq A \cap B$
which is never an $\emptyset$ since sets $C$ and $D$ are finite and $\mathbb{R} - C - D$ is infinite.\\
\end{quote}
\item[(c)]
If $A$ is cofinite, then $A$ contains a positive integer.
\begin{quote}
It's true.\\
We know that $\mathbb{R}$ contains all the positive integers. For $A$
to be cofinite it $\mathbb{R} - A$ should be finite thus, it should
have a finite number of elements. If $A$ has no positive integers,
it means that $\mathbb{R} - A$ is infinite since it contain
at least all the positive integers. Thus, $A$ is not cofinite and
we've encountered a contradiction. And finally, the statement if
$A$ is cofinite, then $A$ contains a positive integer is true.
\end{quote}
\item[(d)]
The intersection of two cofinite sets is cofinite.
\begin{quote}
It's true.\\
Suppose $A = \mathbb{R} - F$ and $B = \mathbb{R} - G$ are cofinite sets ($F$ and $G$ are finite).
Then, their intersection will be:
$$A \cap B = \mathbb{R} - F - G = \mathbb{R} - (F \cup G)$$
And we get:
$$\mathbb{R}  - (\mathbb{R} - (F \cup G)) = F \cup G$$
Now, since $F$ and $G$ are finite, $F \cup G$ is also finite
and we proved that the intersection of the two cofinite sets is cofinite.
\begin{flushright}
\textit{Q.E.D.}
\end{flushright}
\end{quote}
\item[(e)]
The union of two cofinite sets is cofinite.
\begin{quote}
It's true.
Suppose $A = \mathbb{R} - F$ and $B = \mathbb{R} - G$ are cofinite sets ($F$ and $G$ are finite).
Then, their union will be:
$$A \cup B = (\mathbb{R} - F) \cup (\mathbb{R} - G) = \mathbb{R} - (F \cap G)$$
And then we get:
$$\mathbb{R} - (\mathbb{R} - (F \cap G)) = F \cap G$$
Now, since $F$ and $G$ are finite, so is $F \cap G$ and the union of two cofinite
sets is cofinite.
\begin{flushright}
\textit{Q.E.D.}
\end{flushright}
\end{quote}
\item[(f)]
If $A$ and $B$ are cofinite then $A - B$ is finite.
\begin{quote}
It's true.\\
Suppose $A = \mathbb{R} - F$ and $B = \mathbb{R} - G$ are cofinite sets ($F$ and $G$ are finite).
Then, $A - B = (\mathbb{R} - F) - (\mathbb{R} - G) = G - F$.
Now, since $F$ and $G$ are finite, $G - F$ is finite (even if $F = G$, empty set is considered finite
with the cardinality zero).
\begin{flushright}
\textit{Q.E.D.}
\end{flushright}
\end{quote}
\item[(e)]
Every cofinite set is infinite.
\begin{quote}
It's true.
Let $A$ be a cofinite set. Then, we know that it is some subset of $\mathbb{R}$
and we can write it as $A = \mathbb{R} - F$ where $F$ is some set.
Then, we have:
$$\mathbb{R} - (\mathbb{R} - F) = F$$
Now, since $A$ is cofinite, then $F$ has to be finite by the definition of the cofinite set.
Then we get that $\mathbb{R} - F$ is infinite since $F$ is finite and $\mathbb{R}$ minus
any finite set is always infinite.
\begin{flushright}
\textit{Q.E.D.}
\end{flushright}
\end{quote}
\end{itemize}
\item[32.]
We say that a subset $S$ of $\mathbb{Z}$ is angled if for every $x, y, z \in S$ we have $x + y > z$.
\begin{itemize}
\item[(a)]
\begin{quote}
\begin{gather*}
S = \{3\} \ \mbox{since 3 + 3 $>$ 3}\\
S = \{3, 4\} \ \mbox{since 3 + 4 $>$ 3, 3 + 4 $>$ 4, 3 + 3 $>$ 3, and 4 + 4 $>$ 4}\\\\
S = \{3, 4, 5\}\\
\mbox{since 4 + 5 $>$ 3, 3 + 5 $>$ 4, $3 + 4 > 5$, 3 + 3 $>$ 3, 4 + 4 $>$ 4,}\\
\mbox{and 5 + 5 $>$ 5}
\end{gather*}
\end{quote}
\item[(b)]
\begin{quote}
\begin{gather*}
S = \{3, 2, 7\} \ \mbox{as 3 + 2 $<$ 7}\\
S = \{12, -13, 29, 47\} \ \mbox{as 12 - 13 $<$ 29}\\
S = \{1, 2, 3, 4, 5\} \ \mbox{as 1 + 2 $<$ 4}
\end{gather*}
\end{quote}
\item[(c)]
Can 0 be an element of an angled set?
\begin{quote}
No.\\
If the set contains element 0, then $0 + 0 = 0$ thus, $0 + 0 \not> 0$
and the set is not angled.
\end{quote}
\item[(d)]
Prove or disprove: If $S$ is angled and $x \in S$ then $x > 0$.
\begin{quote}
It's true so let's prove it.\\
Suppose, for the sake of contradiction, that $x \leq 0 \in S$ where $S$ is angled.
Then, we must have that $x + x > 2x$. But if $x \leq 0$, $x + x$ is always less than or equal to $2x$.
Thus, we've encountered a contradiction and if $S$ is angled and $x \in S$ then $x > 0$.
\begin{flushright}
\textit{Q.E.D.}
\end{flushright}
\end{quote}
\item[(e)]
Prove or disprove: If $S$ is angled then there exists $c \in \mathbb{Z}$ such that for every $x \in S$
we have $x < c$.
\begin{quote}
It's true.\\
Suppose, for the sake of contradiction, that we cannot find $c \in \mathbb{Z}$ such that
for every $x \in S$ we have $x < c$. Then, it is clearly the case that $S$ contains
the biggest element of $\mathbb{Z}$. BUT, unfortunately, there is no "BIGGEST" element
in $\mathbb{Z}$ as if we pick some element $x$ to be the biggest, we can always take $x + 1$
which will be bigger than $x$. Thus, we've encountered a contradiction and if $S$ is angled then there exists $c \in \mathbb{Z}$ such that for every $x \in S$
we have $x < c$.
\begin{flushright}
\textit{Q.E.D.}
\end{flushright}
\end{quote}
\item[(f)]
Prove or disprove: There exists $c \in \mathbb{Z}^+$ such that
if $S$ is angled and $x \in S$ then we have $x < c$.
\begin{quote}
It's true.\\
In (d), we proved that if $S$ in angled and $x \in S$, then $x > 0$.
Let $x$ be an element of $S$, then we can set $c = x + 1$ ($c$ is positive
since $x > 0$) which will be greater than $x$.
\end{quote}
\item[(g)]
Prove or disprove: Every angled set is finite.
\begin{quote}
It's true.\\
Suppose, for the sake of contradiction, that $S$ is an infinite angled set.
Then, for every $x, y, z, x + y > z$. We also know, from the previous proofs,
that all elements of $S$ are positive. Now, since all elements of $S$ have to
be positive and it's infinite, we know that there is no biggest element in the
set. Then, if we take two fixed elements $x$ and $y$, we can certainly find $z$
(we can change $z$ until we get the value bigger than $x + y$).
such that $z \geq x + y$, since the set is infinite and thus, we've encountered a
contradiction. Hence, every angled set is finite.
\end{quote}
\item[(h)]
Prove or disprove: For every $n \in \mathbb{Z}^+$ there exists an angled set $S$ such that $|S| = n$.
\begin{quote}
It's true.\\
Let's construct a set in the following way:\\
\begin{align*}
A = \{a_1, a_2, a_3, ... a_n\} \mbox{ where } a_1 \leq a_2 \leq a_3 \ ... \leq a_n \mbox{ and }\\
a_1, a_2, a_3 ... \mbox{ are consecutive positive integers}.
\end{align*}
Then, our primary concern is that $2a_1 > a_n$. If $a_1, a_2, a_3...$ are conse
positive integers, then $a_n = a_1 + n - 1$ and we have:
$$2a_1 > a_1 + n - 1$$
and finally, we get:
$$a_1 > n - 1$$
Thus, the set $A = \{n, n + 1, n + 2, n + 3, n + 4, ... \ 2n\}$ is now angled because
for every $x,y,z \in \mathbb{Z}, x + y > z$.
\begin{flushright}
\textit{Q.E.D.}
\end{flushright}
\end{quote}
\end{itemize}
\item[33.]
Use quantifiers to precisely write down, in mathematical language, the definition given
for $\sum^{\infty}_{n=0}a_n = X$ outlined in the video at the 4:00 minute mark.
\begin{quote}
This sum means that when we generate a list of numbers by cutting off the sums at finite points:\\
\begin{align*}
s_0 = a_0\\
s_1 = a_0 + a_1\\
s_2 = a_0 + a_1 + a_2\\
s_3 = a_0 + a_1 + a_2 + a_3\\
s_4 = a_0 + a_1 + a_2 + a_3 + a_4\\
s_5 = a_0 + a_1 + a_2 + a_3 + a_4 + a_5\\
s_6 = a_0 + a_1 + a_2 + a_3 + a_4 + a_5 + a_6\\
s_7 = a_0 + a_1 + a_2 + a_3 + a_4 + a_5 + a_6 + a_7\\
s_8 = a_0 + a_1 + a_2 + a_3 + a_4 + a_5 + a_6 + a_7 + a_8\\
s_9 = a_0 + a_1 + a_2 + a_3 + a_4 + a_5 + a_6 + a_7 + a_8 + a_9\\
...
\end{align*}
Then, these sums approach $X$ in the sense that no matter how small is the distance,
at some point down list, all the numbers start falling within this distance of $X$.
Thus, the further we proceed with the list $s_1, s_2, s_3, s_4, s_5, s_6, s_7, s_8, s_9, s_10, s_11 ...$
the more these numbers approach $X$ and the smaller the distance between $X$ and the sum.
\end{quote}
\item[34.]
Explain why it makes sense, in a way, for $1 - 1 + 1 - 1 + 1 - 1 + ...$ to equal $1/2$,
as is suggested in the video at about 6:45. Is this what you learned in Calculus II?
\begin{quote}
We can cut the line of the length 1 in two pieces with proportions $(1 - p)$ and $p$.
Then, we can cut $p$ in two with same proportions ($(1 - p) / p$) and continue doing it
infinite. Finally, we can sum up the pieces to get the equation:
$$(1-p) + p(1 - p) + p^2(1 - p) + p^3(1 - p) + ... = 1$$
Now, we can divide both sides by $1 - p$ and we get:
$$1 + p + p^2 + p^3 + ... = \dfrac{1}{1 - p}$$
If we plug in, $p = -1$, we get:
$$1 - 1 + 1 - 1 + 1 - 1 + 1 ... = \dfrac{1}{2}$$
which seems to be true. Unfortunately, it is not true. One can calculate
the sum this way if and only if $-1 < p < 1$ meaning that one cannot
simply plug $p = -1$ or $p = 12$ and get the sum for the infinite geometric
series. This sum is sometimes 1 and sometimes -1. That's what Calc II says.
\end{quote}
\item[35.]
In the sense of distance discussed at 12:45, how far apart are 5 and 13? How about
-1 and -15?
\begin{quote}
5 and 13 are 1/8 apart from each other.\\
-1 and -15 are 1/16 apart from each other.
\end{quote}
\newpage
{\Large Bookwork\\}
\item[1.]
Let $a$ be an integer. Prove: If $a^2$ is even, then $a$ is even.
\begin{quote}
Let's prove the contrapositive. The contrapositive of the initial
statement is: If $a$ is not even, $a^2$ is not even (where $a$ is an integer).
Then suppose an integer $a$ is not even, thus is odd. Since $a$ is odd, we can
write $a = 2k + 1$ where $k \in \mathbb{Z}$ and we have:
$$a^2 = (2k + 1)^2 = 4k^2 + 4k + 1 = 2 (2k^2 + 2k) + 1$$
Now, let $l = 2k^2 + 2k$ where $l \in \mathbb{Z}$ (since $2k^2 + 2k \in \mathbb{Z}$).
And finally, we have:
$$a^2 = (2k + 1)^2 = 4k^2 + 4k + 1 = 2 (2k^2 + 2k) + 1 = 2l + 1 \ \mbox{where} \ l \in \mathbb{Z}$$
Thus, we got that $a^2$ is of the form $2l + 1$ which means that $a^2$ is odd thus, not even.
\begin{flushright}
\textit{Q.E.D.}
\end{flushright}
\end{quote}
\item[5(a)]
Prove that $\sqrt{3} \notin Q$.
\begin{quote}
Suppose, for the sake of contradiction, that $\sqrt{3} \in Q$. Then, we can
represent $\sqrt{3}$ as $\dfrac{a}{b}$ where $a,b \in \mathbb{Z}$. Let's assume
that $a$ and $b$ do not have any common factors and if they do,
let's cancel them out and write already cancelled-out form. Thus, assume that
the fraction $\dfrac{a}{b}$ is already cancelled out and $a$ and $b$ do not have
common factors. Then we have:
\begin{align*}
\dfrac{a}{b} = \sqrt{3}\\
\dfrac{a^2}{b^2} = 3\\
a^2 = 3b^2
\end{align*}
Hence, we got that $a^2$ is divisible by 3 which means that $a$ is also divisible by 3.
Now, let $a = 3k$ where $k \in \mathbb{Z}$. After substituting $a$, we get:
$$3b^2 = (3k)^2 = 9k^2$$
$$b^2 = 3k^2$$
Thus, we got that $b^2$ is divisble by 3 which means that $b$ is also divisble by 3.
However, we assumed that $a$ and $b$ had no common factors and now, we encounter the
contradiction.
\begin{flushright}
\textit{Q.E.D.}
\end{flushright}
\end{quote}
\item[6(d)]
Prove that $r + \sqrt{2} \notin \mathbb{Q}$ where $r \in \mathbb{Q}$.
\begin{quote}
Suppose, for the sake of contradiction, that $r + \sqrt{2} \in \mathbb{Q}$.
Now, since $r, r + \sqrt{2} \in \mathbb{Q}$, we can write $r = \dfrac{a}{b}$ and $r + \sqrt{2} = \dfrac{c}{d}$
where $a,b,c,d \in \mathbb{Z}$. Then we have:
$$\dfrac{a}{b} + \sqrt{2} = \dfrac{c}{d}$$
$$\dfrac{bc - ad}{bd} = \sqrt{2}$$
Now, let $x = bc - ad$ and $y = bd$ (where $x,y \in \mathbb{Z}$ as $bc - ad, bd \in \mathbb{Z}$).
And finally, we got that $\dfrac{x}{y} = \sqrt{2}$ which is false since $\sqrt{2}$ is irrational
and cannot be represented as a fraction of two even integers.
\begin{flushright}
\textit{Q.E.D.}
\end{flushright}
Proof that $\sqrt{2}$ is irrational:\\\\
Suppose, for the sake of contradiction, that $\sqrt{2} \in Q$. Then, we can
represent $\sqrt{2}$ as $\dfrac{a}{b}$ where $a,b \in \mathbb{Z}$. Let's assume
that $a$ and $b$ do not have any common factors and if they do,
let's cancel them out and write already cancelled-out form. Thus, assume that
the fraction $\dfrac{a}{b}$ is already cancelled out and $a$ and $b$ do not have
common factors. Then we have:
\begin{align*}
\dfrac{a}{b} = \sqrt{2}\\
\dfrac{a^2}{b^2} = 2\\
a^2 = 2b^2
\end{align*}
Hence, we got that $a^2$ is divisible by 3 which means that $a$ is also divisible by 2.
Now, let $a = 2k$ where $k \in \mathbb{Z}$. After substituting $a$, we get:
$$2b^2 = (2k)^2 = 4k^2$$
$$b^2 = 2k^2$$
Thus, we got that $b^2$ is divisble by 2 which means that $b$ is also divisble by 2.
However, we assumed that $a$ and $b$ had no common factors and now, we encounter the
contradiction.
\begin{flushright}
\textit{Q.E.D.}
\end{flushright}
\end{quote}
\item[9.]
Prove: If $x$ is irrational, then $\sqrt{x}$ is irrational.
\begin{quote}
Suppose, for the sake of contradiction, that $x$ is irrational and $\sqrt{x}$
is rational. Then we can represent $\sqrt{x}$ as $\dfrac{a}{b}$ where $a,b \in \mathbb{Z}$
and have no common factors. Then, by squaring both sides, we get:
$$\dfrac{a^2}{b^2} = x$$
This, now means that $x$ can be represented as the fraction of two integers
as if $a, b$ are integers, so are $a^2, b^2$ and we've encountered a contradiction.
Thus, if $x$ is irrational, then $\sqrt{x}$ is irrational.
\begin{flushright}
\textit{Q.E.D.}
\end{flushright}
\end{quote}
\newpage
\item[11.]
Prove: $\sqrt[^4]{2} \notin \mathbb{Q}$.
\begin{quote}
Suppose, for the sake of contradiction, that $\sqrt[^4]{2} \in \mathbb{Q}$.
Then we can represent $\sqrt[^4]{2}$ as $\dfrac{a}{b}$ where $a,b \in \mathbb{Z}$
and have no common factors. Then, by squaring both sides, we get:
$$\dfrac{a^2}{b^2} = \sqrt{2}$$
This, however, means that $\sqrt{2}$ can be represented as a fraction of two integers
(because $a, b$ are integers, so are $a^2, b^2$) which is clearly impossible since
we've already proven that $\sqrt{2}$ is irrational, thus cannot be represented as a fraction
of two integers. Hence, we've encountered a contradiction and $\sqrt[^4]{2} \notin \mathbb{Q}$.
\begin{flushright}
\textit{Q.E.D.}
\end{flushright}
\end{quote}
\end{quote}

\end{enumerate}
\end{document}
