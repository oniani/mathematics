\documentclass[12pt, a4paper]{article}
\usepackage[margin=0.7in]{geometry}

\usepackage[none]{hyphenat}

\usepackage{bookmark}
\usepackage{adjustbox}
\usepackage{mathtools}
\usepackage{amsmath}
\usepackage{amssymb}
\usepackage{amsthm}
\usepackage{amsfonts}
\usepackage{algorithmic}

\usepackage{pgfplots}
\pgfplotsset{compat=1.16}
\usepackage{tikz}
\usetikzlibrary{arrows}
\usepackage{graphicx}
\usepackage{pst-solides3d}
\usepackage{xcolor}
\usepackage{hyperref}
\usepackage[cache=false]{minted}

\usepackage{soul}
\usepackage{cite}
\usepackage{textcomp}

\usepackage{wasysym}

% Sets and related operations
\newcommand{\nats}{\mathbb{N}}         % Natural numbers
\newcommand{\pnats}{\mathbb{N}^+}      % Positive natural numbers

\newcommand{\ints}{\mathbb{Z}}         % Integers
\newcommand{\pints}{\mathbb{Z}^+}      % Positive integers
\newcommand{\nints}{\mathbb{Z}^-}      % Negative integers

\newcommand{\rats}{\mathbb{Q}}         % Rational numbers
\newcommand{\prats}{\mathbb{Q}^+}      % Positive rational numbers
\newcommand{\nrats}{\mathbb{Q}^-}      % Negative rational numbers

\newcommand{\reals}{\mathbb{R}}        % Real numbers
\newcommand{\preals}{\mathbb{R}^+}     % Positive real numbers
\newcommand{\nreals}{\mathbb{R}^-}     % Negative real numbers

\newcommand{\irrats}{\mathbb{I}}       % Irrational numbers


\newcommand{\pset}{\mathcal{P}}        % Powerset
\newcommand{\card}{\abs}               % Cardinality
\newcommand{\topology}{\mathcal{T}}    % Topology
\newcommand{\basis}{\mathcal{B}}       % Basis
\newcommand{\oldemptyset}{\emptyset}   % Old empty set
\renewcommand{\emptyset}{\varnothing}  % New and nice empty set

% Calligraphy
\newcommand\und[1]{\underline{\smash{#1}}}

% Operators
\DeclarePairedDelimiter\abs{\lvert}{\rvert}
\DeclarePairedDelimiter\ceil{\lceil}{\rceil}
\DeclarePairedDelimiter\floor{\lfloor}{\rfloor}

% Other
\newcommand{\rarr}{\rightarrow}
\newcommand{\larr}{\leftarrow}

% Setting stuff
\setlength{\parindent}{0pt}


% Title
\title{\rule{\paperwidth - 150pt}{1pt}\textbf{\\\textit{Topology}\\}\rule{\paperwidth - 150pt}{1pt}}

\author
{
Author: David Oniani\\
Instructor: Dr. Eric Westlund
}

\date{January 20, 2019}


\begin{document}
\maketitle

\center{\Large Assignment \textnumero{4}}

\begin{itemize}
\item[]
\item[]
{\large \textbf{Section 23}}
\vspace{0.3cm}

% Begin here!
\item[6.]
Let $A \subset X$. Show that if $C$ is a connected subspace of $X$
that intersects both $A$ and $X - A$, then $C$ intersects $\mbox{Bd} \ A$.
\begin{quote}
At first, recall that $\mbox{Bd} \ A = \overline{A} \cap \overline{X - A}$.
Now, suppose, for the sake of contradiction, that $C$ is connected and $C \cap \mbox{Bd} \ A = \emptyset$.
Consider two sets $U_1 = C \cap \overline{A}$ and $U_2 = C \cap \overline{X - A}$.
Now, since $C \cap A \subset U_1$ and $C \cap \overline{X - A} \subset U_2$, it follows
from our assumptions that $U_1$ and $U_2$ are two nonempty subsets of $C$.
Notice that $C = U_1 \cup U_2$ with $U_1, U_2$ being both open and closed subsets of $C$.
However, $U_1 \cap U_2 = C \cap \overline{A} \cap \overline{X - A} = C \cap \mbox{Bd} \ A = \emptyset$
which means that $C$ is disconnected and contradicts the fact that $C$ is connected.\
Finally, we have reached the contradiction and $C$ intersects $\mbox{Bd} \ A$.$\qed$
\end{quote}

\item[]
\item[]
\item[]
{\large \textbf{Section 24}}
\vspace{0.3cm}

\item[1.]
\begin{itemize}
\item[(c)]
Show that $\reals^n$ and $\reals$ are not homeomorphic if $n > 1$.
\begin{quote}
Suppose, for the sake of contradiction, that for $n > 1$, $\reals^n$ and $\reals$
are homeomorphic. Then, by the definition of homeomorphism, there exists a function
$f : \reals \to \reals^n$. Consider $\left.f \right|_{\reals^n-\{0\}} : \reals^n-\{0\} \to \reals - \{f(0)\}$,
$\left.f \right|_{\reals^n-\{0\}}$ is a restriction of $f$ and hence is a homeomorphism.
Now, notice that $\reals^n-\{0\}$ is a connected space, however, $\reals-\{f(0)\}$ is not a connected
space and we have reached the contradiction since $\left.f \right|_{\reals^n-\{0\}}$ is a homeomorphism.
Finally, we have that for $n > 1$, $\reals^n$ and $\reals$ are not homeomorphic.
In short, by taking away $0$, we make $\reals$ disconnected, but taking away any point from $\reals^n$
leaves it connected.$\qed$
\end{quote}
\end{itemize}

\item[]
\item[]

\item[3.]
Let $f : X \to X$ be continuous. Show that if $X = [0, 1]$,
there is a point $x$ such that $f(x) = x$. The point $x$
is called a \textbf{\textit{fixed point}} of $f$. What happens if $X$
equals $[0, 1)$ or $(0, 1)$?
\begin{quote}
In the order topology, $X$ is an ordered set and connected space.
Let $a, b \in X$. Let's now pick a midpoint $x_1$ between $f(a)$ and $f(b)$.
Then, according to \textbf{Theorem 24.3}, $\exists c_1 \in [a, b]$ such that
$f(c_1) = x_1$. Now, if $c_1 = x_1$, we found the fixed point of $f$ and if $c_1 \neq x_1$, we pick the midpoint $x_2$ between $c_1$ and $x_1$.
Then $\exists c_2 \in [c_1, x_1]$ or $c_2 \in [x_1, c_1]$ (depending on whether $x_1 > c_1$ or $c_1 > x_1$) such that 
$f(c_2) = x_2$. Now, if $c_2 = x_2$ then we have found the point and if not we countinue
this way. Thinking of computer science, this is a recursive approach to the problem (though, I think recursion comes from math anyway, right?).
The simply outline of the algorithm would look something like this:
\begin{algorithmic}
    \IF{$x_n = c_n$}
        \STATE hooray! we have found a point! we are done, return the point!
    \ELSE
        \STATE Consider the midpoint between $x_n$ and $c_n$
    \ENDIF
\end{algorithmic}

After repeating this process, we will end up with two convergent series: $c_1, c_2, c_3, ...$ and $x_1, x_2, x_3, ...$ with the property that $\abs{c_n - x_n} \to 0$ as $n \to \infty$. In others words, we have $\displaystyle\lim_{n \to \infty}\abs{c_n - x_n} \to 0$. This is due to the continuity
of $f$ on $X$. Therefore, we have $\abs{x_n - f(x_n)} = \abs{f(c_n) - f(x_n)} \to 0$ from which we get that $f(x_n) = (x_n)$.$\qed$
\newline
\newline
This fact/theorem does not hold for intervals $[0, 1)$ and $(0, 1)$. This is due to\\
\vspace{0.25cm}
$f$ not being uniformly continuous on these intervals. For instance, a function\\
\vspace{0.25cm}
$f(x) = \dfrac{x + 2}{3}$ has a fixed point $x = 1$, but
\vspace{0.25cm}
it has no fixed points on intervals $[0, 1)$\\
 or $(0, 1)$.
\end{quote}

\item[]
\item[]

\item[8.]
\begin{itemize}
\item[(a)]
Is a product of path-connected spaces necessarily path-connected?
\begin{quote}
Yes. Suppose that $X$ and $Y$ are path-connected.
Let $x_1 \times x_2, y_1 \times y_2 \in X \times Y$.
Notice that $X \times y_1$ is homeomorphic to $X$
and thus, is path-connected. Therefore, there exists
a continuous function $f : [0, 1] \to X \times y_1$
with $f(0) = x_1 \times y_1$ and $f(1) = x_2 \times y_2$.
Besides, $x_2 \times Y$ is homeomorphic to $Y$ and thus,
is path-connected. Therefore, there exists a continuous
function $g$ such that $g : [0, 1] \to x_2 \times Y$
with $g(0) = x_2 \times y_1$ and $g(1) = x_2 \times y_2$.
Let's now define a function $h$ in the following way:
$$
h(x) = 
\begin{cases}
f(\frac{x}{2}) \mbox{ if } 0 \leq x \leq \frac{1}{2}\\
f(\frac{x}{2} + \frac{1}{2}) \mbox{ if } \frac{1}{2} \leq x \leq 1
\end{cases}
$$
The according to the \textbf{Theorem 18.3 (The pasting lemma)}, $h$ is continuous. Besides,
$h(0) = f(\frac{0}{2}) = f(0) = x_1 \times y_1$ and $h(1) = g(\frac{1}{2} + \frac{1}{2}) = g(1) = x_2 \times y_2$.
Therefore, $h$ is a path from $x_1 \times y_1$ to $x_2 \times y_2$
and because $x_1 \times y_1$ and $x_2 \times y_2$ are arbitrary, we have that
$X \times Y$ is path-connected.$\qed$
\end{quote}

\item[]
\item[]

\item[(b)]
If $A \subset X$ and $A$ is path-connected, is $\bar{A}$
necessarily path-connected?
\begin{quote}
No. For instance, \textbf{\textit{topologist's sine curve}}.
\end{quote}

\item[]

\item[(c)]
If $f: X \to Y$ is continuous and $X$ is path-connected, is $f(X)$ necessarily path-connected?
\begin{quote}
Yes, this is due to the fact that the composition of continuous functions is always continuous.
\end{quote}

\item[]

\item[(d)]
If $\{A_\alpha\}$ is a collection of path-connected subspaces of $X$ and if $\bigcap A_\alpha \neq \emptyset$, is $\bigcup A_\alpha$ necessarily path-connected?
\begin{quote}
Yes. Let $x, y \in \bigcup X_\alpha$ and let $z \in \bigcap A_\alpha$.
Then for some $b$ and $c$, $x \in A_b$ and $y \in A_c$. Besides, $z \in A_b$
and $z \in A_c$. Now, because $A_b$ is path-connected, there is a path $f$
from $x$ to $z$. On the other hand, because $A_c$ is path-connected,
there is a path $g$ from $z$ to $y$. Now, according to the \textbf{Theorem 18.3 (The pasting lemma)},
we can glue these two paths together and make a path $h$ from $x$ to $y$.$\qed$
\end{quote}
\end{itemize}

% If add another '\item[]' as the convention, section title has one blank line... Have to come up with better writing conventions!
\item[]
\item[]
{\large \textbf{Section 26}}
\vspace{0.3cm}

\item[5.]
Let $A$ and $B$ be disjoint compact subspaces of the Hausdorff space $X$. Show that there exist disjoint open sets $U$ and $V$ containing $A$ and $B$, respectively.
\begin{quote}
Note that since $A$ and $B$ are compact subspaces of a Hausdorff space, they are closed.
Then $X - A$ and $X - B$ are open. Since $A$ and $B$ are disjoint, $U = X - B$ contains $A$ and $V = X - A$ contains $B$.$\qed$
\end{quote}

\item[]
\item[]
\item[]
{\large \textbf{Section 27}}
\vspace{0.3cm}

\item[2.]
Let $X$ be a metric space with metric $d$; let $A \subset X$ be nonempty.
\begin{itemize}
\item[(a)]
Show that $d(x, A) = 0$ if and only if $x \in \bar{A}$.
\begin{quote}
The function of $x$ described in the problem is continuous, so its set of zeros is a closed set.
This closed set contains $A$ therefore, it also contains $\bar{A}$. On the other hand, if $x \notin \bar{A}$ then $\exists \epsilon > 0$ with $U_\epsilon(x) \subset X - \bar{A}$, and in this case it follows that $d(x, A) \geq \epsilon > 0$.$\qed$
\end{quote}

\item[]

\item[(b)]
Show that if $A$ is compact, $d(x, A) = d(x, a)$ for some $a \in A$.
\begin{quote}
The function $f(a) = d(x, a)$ is continuous and $d(x, A)$ is the greatest lower bound for its set of
values. Now, because of the compactness of $A$, this greatest lower bound is a minimum value that is realized at some
point of $A$ (\textbf{Theorem 27.4 (Extreme value theorem)}).$\qed$
\end{quote}

\item[]

\item[(c)]
Define the $\epsilon$-neighborhood of $A$ in $X$ to be the set
$$U(A, \epsilon) = \{x \mid d(x, A) < \epsilon\}.$$
Show that $U(A, \epsilon)$ equals to the union of the open balls $B_d(a, \epsilon)$ for $a \in A$.
\begin{quote}
Note that the $\epsilon$-neighborhood of $A$ in $X$
corresponds to all points in $X$ that are within a distance
$\epsilon$ of some point in $A$. It includes all of $A$.
Then $x \in U(A, k)$ if and only if $d(x, a) < k$ for some $a \in A$. It follows that $x \in \bigcup_{a \in A}B(a, k)$.$\qed$
\end{quote}
\end{itemize}

\item[]
\item[]

\item[6.]
Let $A_0$ be the closed interval $[0, 1]$ in $\reals$.
Let $A_1$ be the set obtained from $A_0$ by deleting its
"middle third`` $(\frac{1}{3}, \frac{2}{3}$. Let $A_2$
be the set obtained from $A_1$ by deleting "middle thirds``
$(\frac{1}{9}, \frac{2}{9})$ and $(\frac{7}{9}, \frac{8}{9})$.
In general, define $A_n$ by the equation
$$A_n = A_{n - 1} - \displaystyle\bigcup_{k = 0}^{\infty}\Bigg(\dfrac{1 + 3k}{3^n}, \dfrac{2 + 3k}{3^n}\Bigg).$$
The intersection
$$C = \bigcap_{n \in \pints}A_n$$
is called the \textbf{\textit{Cantor set}}; it is a subspace of $[0, 1]$.
\begin{itemize}
\item[(a)]
Show that $C$ is totally disconnected.
\begin{quote}
Suppose, for the sake of contradiction, that $C$
is not totally disconnected. Then $\exists [x, y] \subset C$.
Let $K \in \pints$ with
$$K > \log_3\Bigg(\dfrac{1}{y - x}\Bigg).$$
Then for $k > K$, $\frac{1}{3^k} < y - x$. But now since $C = \bigcap A_j$, $C$ must contain intervals (if it contains any intervals at all) with length less  than $\dfrac{1}{3^k}$. Therefore, we have reached the contradiction and $C$ contains no intervals. Hence, $C$ it is totally disconnected.$\qed$
\end{quote}

\item[]

\item[(b)]
Show that $C$ is compact.
\begin{quote}
Notice that $C$ is closed and bounded. Therefore, according to the $\textbf{Theorem 27.3}$, $C$ is compact.$\qed$
\end{quote}

\item[]

\item[(c)]
Show that each set $A_n$ is a union of finitely many disjoint closed intervals of length $1/3^n$; and show that the end points of these intervals lie in $C$.
\begin{quote}
Notice that $A_n = \displaystyle\bigcup_{k = 0}^{\frac{3^n - 1}{2}}\bigg(\dfrac{2k}{3^n}, \dfrac{2k + 1}{3^n}\bigg)$.
Then $\forall k$, $\dfrac{2k + 1}{3^n} - \dfrac{2k}{3^n} = \dfrac{1}{3^n}$. Therefore, each interval in the union
has the length $1/3^n$.$\qed$
\end{quote}

\item[]
\item[]

\item[(d)]
Show that $C$ has no isolated points.
\begin{quote}
Observe that every point of the Cantor set is a limit point of itself. Therefore, it has no isolated points.$\qed$
\end{quote}

\item[]

\item[(e)]
Conclude that $C$ is uncountable.
\begin{quote}
$C$ is nonempty since it is the intersection of a nested sequence of closed intervals. Besides, it is Hausdorff, has no isolated points as well as is compact. Now, according to the $\textbf{Theorem 27.7}$, $C$ is uncountable.$\qed$
\end{quote}
\end{itemize}

\end{itemize}
\end{document}
