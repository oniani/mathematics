\documentclass[12pt, a4paper]{article}                  % use "amsart" instead of "article" for AMSLaTeX format
\usepackage[a4paper,margin=1in]{geometry}               % Adjust margins
\usepackage{amsmath,amssymb,textcomp,listings,graphicx,adjustbox}    % Math packages: amsmath, amssymb, listings, color
\usepackage[makeroom]{cancel}

\title{\bf{Homework \textnumero 2}}
\author{Author: David Oniani
\\
\ \ \ Instructor: Tommy Occhipinti}
\date{September 3, 2018}

\usepackage{listings}
\usepackage{color}

\definecolor{dkgreen}{rgb}{0,0.6,0}
\definecolor{gray}{rgb}{0.5,0.5,0.5}
\definecolor{mauve}{rgb}{0.58,0,0.82}
\definecolor{backcolour}{rgb}{0.95,0.95,0.92}

\lstset{
backgroundcolor=\color{backcolour},
aboveskip=3mm,
belowskip=3mm,
showstringspaces=false,
columns=flexible,
basicstyle={\small\ttfamily},
numbers=left,
numberstyle=\normalsize\color{gray},
keywordstyle=\color{blue},
commentstyle=\color{dkgreen},
stringstyle=\color{mauve},
breaklines=true,
breakatwhitespace=true,
tabsize=4
}


\begin{document}
\maketitle

\begin{enumerate}
\item[6.]
Write the negation of the following statements as naturally as possible. Do not simply
write “it is not the case that...” For some statements you are given the context, but
you should not negate that part.
\begin{enumerate}
\item[(a)]
About an integer $n \geq 2$:\\
$n$ is prime or $n+1$ is prime.\\
\textbf{Negation}: $n$ and $n+1$ are composite.

\item[(b)]
About matrices $A$ and $B$:\\
The determinant of $A$ is negative and the determinant of $B$ is positive.\\
\textbf{Negation}: The determinant of $A$ is not negative and the determinant of $B$ is not positive.

\item[(c)]
About matrices $f(x)$ and $g(x)$:\\
If $f(x)$ and $g(x)$ are differentiable, then so is $f(x) + g(x)$.\\
\textbf{Negation}: $f(x)$ and $g(x)$ are differentiable, but there exists $f(x) + g(x)$ that is not.

\item[(d)]
About an integer $n$ in $\mathbb{Z}^+$:\\
$n+1!$ is a perfect square if and only if $n=4$ or $n=5$.\\
\textbf{Negation}: There exists $n \notin \{4,5\}$ where $n+1!$ is a perfect square.

\item[(e)]
The populations of Decorah, IA and Minneapolis, MN are the same.\\
\textbf{Negation}:The populations of Decorah, IA and Minneapolis, MN are different.
\end{enumerate}

\item[2.]
Suppose $P$ and $Q$ are propositions for which "$P$ and $Q$" is false and "$P$ or $Q$" is true.
What can be said about the truth values of $P$ and $Q$.\\\\
The fact that "$P$ and $Q$" is false implies that at least one of the propositions is false.
On the other hand, since "$P$ or $Q$" is true, at least one of the propositions is true.
Thus, conclude that if $P$ is true, then $Q$ is false and if $P$ is false, then $Q$ is true.
\item[4.]
If $P \iff Q$ is true, what is the truth value of $P \iff (\text{not}-Q)$?\\\\
Since $P \iff Q$ is true, we can conclude that either both of the propositions are true
or both of the propositions are false. Either way, $P \iff (\text{not}-Q)$ will be false
as the propositions $P$ and $\text{not}-Q$ should have the same values which will be impossible.
\\
\item[6 (b).]
Construct a truth table to show that $P \Rightarrow (Q \wedge R)$ is logically equivalent to $(P \Rightarrow Q) \wedge (P \rightarrow R)$.
\textit{Lets's take a look at the truth-table below.}

\begin{table}[h!]
	\centering
	\begin{adjustbox}{max width=\textwidth}
	\resizebox{0.95\linewidth}{!}
	{
		\begin{tabular}{*{8}{|c}|}
		\hline
		$P$ & $Q$ & $R$ & $Q \wedge R$ & $P \Rightarrow (Q \wedge R)$ & $P \Rightarrow Q$ & $P \Rightarrow R$ & $(P \Rightarrow Q) \wedge (P \Rightarrow R)$\\ \hline
		F & F & F & F & T & T & T & T\\ \hline
		F & F & T & F & T & T & T & T\\ \hline
		F & T & F & F & T & T & T & T\\ \hline
		F & T & T & T & T & T & T & T\\ \hline
        T & F & F & F & F & F & F & F\\ \hline
        T & F & T & F & F & F & T & F\\ \hline
        T & T & F & F & F & T & F & F\\ \hline
        T & T & T & T & T & T & T & T\\
        \hline
	\end{tabular}
    }
	\end{adjustbox}
\end{table}
According to the truth-table, column $P \Rightarrow (Q \wedge R)$ matches column $(P \Rightarrow Q) \wedge (P \Rightarrow R)$ with the values (T, T, T, T, F, F, F, T).
Thus, these two statements are indeed equivalent.
\begin{flushright}
\textit{Q.E.D.}
\end{flushright}

\item[7 (a).]
Construct a truth-table for the proposition $(P \vee Q) \Rightarrow (P \ \ \wedge \sim Q)$.
\begin{table}[h!]
	\centering
	\begin{adjustbox}{max width=\textwidth}
	\resizebox{0.8\linewidth}{!}
	{
		\begin{tabular}{*{5}{|c}|}
		\hline
		$P$ & $Q$ & $P \vee Q$ & $(P \ \ \wedge \sim Q)$ & $(P \vee Q) \Rightarrow (P \ \ \wedge \sim Q)$\\ \hline
		F & F & F & F & T\\ \hline
		F & T & T & F & F\\ \hline
		T & F & T & T & T\\ \hline
        T & T & T & F & T\\
        \hline
	\end{tabular}
    }
	\end{adjustbox}
\end{table}
\\This will be the truth-table for the proposition $(P \vee Q) \Rightarrow (P \ \ \wedge \sim Q)$.

\item[7 (b).]
Find the simpler proposition that is logically equivalent to the proposition in (a).
\begin{table}[h!]
	\centering
	\begin{adjustbox}{max width=\textwidth}
	\resizebox{0.8\linewidth}{!}
	{
		\begin{tabular}{*{6}{|c}|}
		\hline
		$P$ & $Q$ & $P \vee Q$ & $(P \ \ \wedge \sim Q)$ & $(P \vee Q) \Rightarrow (P \ \ \wedge \sim Q)$ & $(Q \Rightarrow P)$\\ \hline
		F & F & F & F & T & T\\ \hline
		F & T & T & F & F & F\\ \hline
		T & F & T & T & T & T\\ \hline
        T & T & T & F & T & T\\
        \hline
	\end{tabular}
    }
	\end{adjustbox}
\end{table}
\\ Such proposition is $P \Rightarrow Q$. To verify the equivalence, let's look at the truth-table above where we see that
$(Q \Rightarrow P) \iff (P \vee Q) \Rightarrow (P \ \ \wedge \sim Q)$ as columns for these exactly match each other with the
values (T, F, T, T).
\end{enumerate}

\newpage
\begin{center}
\LARGE{\textbf{A Page for Feedback}}
\end{center}

\end{document}
