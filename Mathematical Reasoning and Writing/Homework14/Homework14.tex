\documentclass[12pt, a4paper]{article}
\usepackage[a4paper, margin=1in]{geometry}


\usepackage{adjustbox}
\usepackage{mathtools}
\usepackage{amsmath}
\usepackage{amssymb}
\usepackage{amsthm}

\usepackage{pgfplots}
\usepackage{listings}
\usepackage{color}
\usepackage{tikz}

\usepackage{textcomp}
\usepackage{soul}

\usepackage[hidelinks]{hyperref}
\usepgfplotslibrary{external,fillbetween}
\pgfplotsset{compat=1.14}
\usepackage[makeroom]{cancel}
\title{\bf{Homework \textnumero 14}}
\author{Author: David Oniani
\\
\ \ \ Instructor: Tommy Occhipinti}
\date{December 1, 2018}

\usepackage{listings}
\usepackage{color}

%%%%%%%%%%%%%%% S E T S %%%%%%%%%%%%%%%
\newcommand{\nats}{\mathbb{N}}
\newcommand{\ints}{\mathbb{Z}}
\newcommand{\rats}{\mathbb{Q}}
\newcommand{\reals}{\mathbb{R}}
\newcommand{\irrats}{\mathbb{I}}

\newcommand{\pnats}{\mathbb{N}^+}
\newcommand{\pints}{\mathbb{Z}^+}
\newcommand{\prats}{\mathbb{Q}^+}
\newcommand{\preals}{\mathbb{R}^+}
\newcommand{\nreals}{\mathbb{R}^-}

\newcommand{\nints}{\mathbb{Z}^-}
\newcommand{\nrats}{\mathbb{Q}^-}
%%%%%%%%%%%%%%%%%%%%%%%%%%%%%%%%%%%%%%%

% Calligraphy
\newcommand\und[1]{\underline{\smash{#1}}}

% Operators
\DeclarePairedDelimiter\abs{\lvert}{\rvert}
\DeclarePairedDelimiter\ceil{\lceil}{\rceil}
\DeclarePairedDelimiter\floor{\lfloor}{\rfloor}

% Other
\newcommand{\rarr}{\rightarrow}

\definecolor{dkgreen}{rgb}{0,0.6,0}
\definecolor{gray}{rgb}{0.5,0.5,0.5}
\definecolor{mauve}{rgb}{0.58,0,0.82}
\definecolor{backcolour}{rgb}{0.95,0.95,0.92}

\lstset{
backgroundcolor=\color{backcolour},
aboveskip=3mm,
belowskip=3mm,
showstringspaces=false,
columns=flexible,
basicstyle={\small\ttfamily},
numbers=left,
numberstyle=\normalsize\color{gray},
keywordstyle=\color{blue},
commentstyle=\color{dkgreen},
stringstyle=\color{mauve},
breaklines=true,
breakatwhitespace=true,
tabsize=4
}


\begin{document}
\maketitle
{\Large 5.3/5.4 - Cardinalities of Infinite Sets}
\begin{itemize}
\item[]

\item[98.]
To prove if and only if statement, we should prove the statement directly and then prove the converse of it.\\\\
\textbf{Proving that if $\abs{A} = \abs{A - \{a\}}$, then $A$ is infinite where $A$ is a set and $a \in A$.}
\begin{quote}
Suppose, for the sake of contradiction, that $A$ is finite and $\abs{A} = \abs{A - \{a\}}$. Then it has some fixed size $k$ where $k \in \pints$. Thus, $\abs{A} = k$.
If we remove one element from $A$, its size will be $k - 1$. Hence, $\abs{A - \{a\}} = k - 1$. Finally, $\abs{A} \neq \abs{A - \{a\}}$
since $k \neq k - 1$ and we have reached the contradiction. Therefore, $A$ is infinite.
$\qed$
\end{quote}
\item[]
\textbf{Proving that if $A$ is infinite, then $\abs{A} = \abs{A - \{a\}}$ where $A$ is a set and $a \in A$.}
\begin{quote}
If $A$ is infinite, we can find a bijection betweewn $\abs{A}$ and $\abs{A - \{a\}}$ where $a \in A$.
To make the bijection easier to understand, let's have the set $A$ as the set of elements $a_1, a_2, a_3 ...$.
Thus, $A = \{a_1, a_2, a_3, ...\}$. Let's suppose that the element which we took out of $A$ is some $a_k$ (instead of calling it $a$, we call it $a_k$).
Now we want to produce a bijection from $A$ to $A - \{a_k\}$.
Here is the bijection:
$$f : A \mapsto A - \{a_k\} : a_x \mapsto \begin{cases} a_x \mbox{ if $x < k$} \\ a_{x + 1} \mbox{ if $x \geq k$}\end{cases}$$
If you look at the function, all we first took the mapping $A \rarr A$ (such that $a_1 \mapsto a_1, a_2 \mapsto a_2$ etc.) and then removed one element from the right side of the mapping,
and shifted the set up. Hence, now we have a bijection (since $f : A \mapsto A$ is always a bijection if we map the first element of $A$
to the first element of $A$, then the second element of $A$ to the second element of $A$ and proceed this way).
$\qed$
\end{quote}
Finally, we have proved the if and only statment in both ways first directly and then conversely and thus, we have proven
that the set $A$ is infinite if and only if $\abs{A} = \abs{A - \{a\}}$ where $a \in A$.
$\qed$

\item[]

\item[99.]
The set is countable if it has the same cardinality as $\pints$.
Hence, all we have to do is to show that $\mathcal{P}(\pints)$ does
not have the same cardinality as $\pints$. According to the
theorem that we covered in class, for an arbitrary set $S$, $\abs{S} < \abs{\mathcal{P}(S)}$.
Hence, we know that $\mathcal{P}(S)$ has a different cardinality from $\pints$ (more precisely, greater cardinality)
and thus, $\mathcal{P}(S)$ is uncountable.
$\qed$

\item[]
\item[]

{\Large \textbf{Video: Who Cares about Topology?}}
\item[100.]
He means that any point $(x, y)$ where $0 \leq x, y \leq 1$ will be a part
of the square (he mentions loop since these points were initially on the closed loop but he then opened
and straightened out the loop and made a copy of that line to create a mini coordinate system).
\item[101.]
That's because if he allowed $(x, y)$ and $(y, x)$, since the points $(x, y)$ and $(y, x)$
represent the same line segment on a closed loop, he would always the the trivial solution to the problem (would make him think that he solved the problem where in reality, he just showed that the line segment shares a midpoint with itself which is vacuously true).
This would not be helpful since he would get an automatic trivial
solution to all the points. He needs to have distinct segments which intersect and share a midpoint.
Hence, to make the problem-solving process more comfortable, he got rid of the annoying case.

\item[102.]
The theorem that says that one cannot glue the edge of a mobius strip to a plane without forcing it to intersect itself.

\item[]
\item[]

{\und{\large Section 5.3}}
\item[2.]
\begin{itemize}
\item[(a)]
If $A$ is finite, then $A - \{a_0\}$ is automatically denumerable (since $A$ is finite).\\
If $A$ is infinite, we can show that $f : A \rarr A - \{a_0\} : x \mapsto x$ is a bijection and thus, since
$A$ is countable, $A - \{a_0\}$ is countable as well.
\item[(b)]
If $A$ is finite, then $A - \{a\}$ is automatically denumerable (since $A$ is finite).\\
If $A$ is infinite, we can show that $f : A \rarr A - \{a\} : x \mapsto x$ is a bijection and thus, since
$A$ is countable, $A - \{a\}$ is countable as well.
\end{itemize}

\item[]

\item[8.]
Suppose, for the sake of contradiction, $B - A$ is countable, $A$ is also countable, and $B$ is uncountable,.
Notice that $B = (B \cap A) \cup (B - A)$ is countable since it is the union of two countable sets.
Hence, we have reached the contradiction since we assumed that $B$ is uncountable and thus, $B - A$ is countable.
$\qed$

\item[]

\item[15.]
Since $A \times A$ is countable, then its subset $B = \{(x, x) \ | \ x \in A\}$ is also countable.
Then, $f : A \rarr B : x \mapsto (x, x)$ is a bijection and thus, $\abs{A} = \abs{B}$ which means that
$A$ is countable as well (since $B$ is countable).\\\\
\und{Proof that $f : A \rarr B : x \mapsto (x, x)$ is a bijection.}\\
If $f(x_1) = f(x_2)$, it means that $(x_1, x_1) = (x_2, x_2)$
and thus, $x_1 = x_2$. Hence, $f$ is injective.\\
For every $(x_i, x_i) \in B$, we can let $x = x_i$ and we get $f(x_i) = (x_i, x_i)$ and thus $f$ is surjective.\\\\
Since we have proven that $f$ is both injective and surjective, it means that $f$ is a bijection.

\item[]

\item[16.]
Two such sets are $(0, 1)$ and $\reals$. We know that $(0, 1) \neq \reals$. We also know that $(0, 1) \approx \reals$ (we have proved it multiple times at this point, but here
is a bijection: $f : (0, 1) \rarr \reals : x \mapsto \mbox{tan}(\pi x - \dfrac{\pi}{2})$). Besides, $(0, 1) \cup \reals = \reals$ and $(0, 1) \cap \reals = (0, 1)$.
Thus, we have found two sets $(0, 1)$ and $\reals$, such that:
$$(0, 1) \approx \reals \approx (0, 1) \cup \reals \approx (0, 1) \cap \reals$$

\item[]
\item[]

{\und{\large Section 5.4}}

\item[6.]
Suppose, for the sake of contradiction, that the infinite set $I' = \{x \in (0, 1) \ | x = .a_1a_2...a_n... \mbox{ where each $a_i = 3$ or 8}\}$ is countable.
Being countable means having a bijection with $\pints$. Hence, we have essentially assumed that there exists a function $f$ such that
$f : \pints \rarr I'$ is a bijection. Let's visualize it.
\begin{align*}
1 \mapsto .a_{1,1}a_{1,2}a_{1,3}...\\
2 \mapsto .a_{2,1}a_{2,2}a_{2,3}...\\
3 \mapsto .a_{3,1}a_{3,2}a_{3,3}...\\
...........................
\end{align*}
Now, let's define a function $transform$ in the following way:
$$transform(x) = \begin{cases} 8 & \mbox{if } n\mbox{ is 3} \\ 3 & \mbox{if } n\mbox{ is 8} \end{cases}$$
Simply put, this function is defined for only two inputs -- 3 and 8 -- and if the input is 3, it returns 8 while
if the input is 8, it returns 3.\\\\
Now, consider the number $.transform(a_{1,1})transform(a_{2,2})transform(a_{3,3})...$. We know that it is not equal
to the first number ($.a_{1,1}a_{1,2}a_{1,3}...$) since $transform(a_{1,1}) \neq a_{1,1}$. We also know that it is not equal
to the second number in the mapping as $transform(a_{2,2}) \neq a_{2,2}$. It does not equal to the third either because $transform(a_{3,3}) \neq a_{3,3}$.
As a result, we've got that:
\begin{align*}
.transform(a_{1,1})transform(a_{2,2})transform(a_{3,3})... \neq .a_{1,1}a_{1,2}a_{1,3}...\\
.transform(a_{1,1})transform(a_{2,2})transform(a_{3,3})... \neq .a_{2,1}a_{2,2}a_{2,3}...\\
.transform(a_{1,1})transform(a_{2,2})transform(a_{3,3})... \neq .a_{3,1}a_{3,2}a_{3,3}...\\
............................................................................................................
\end{align*}
Hence, we have constructed an element of the set $I'$ such that there is no corresponding element in $\ints$ that maps to it and the function $f : \pints \rarr I'$ is not onto hence, is not a bijection.
Finally, we have reached the contradiction and the set $I'$ is not countable.
\end{itemize}
\end{document}
