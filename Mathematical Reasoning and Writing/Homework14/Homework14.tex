\documentclass[12pt, a4paper]{article}
\usepackage[a4paper, margin=1in]{geometry}


\usepackage{adjustbox}
\usepackage{mathtools}
\usepackage{amsmath}
\usepackage{amssymb}
\usepackage{amsthm}

\usepackage{pgfplots}
\usepackage{listings}
\usepackage{color}
\usepackage{tikz}

\usepackage{textcomp}
\usepackage{soul}

\usepackage[hidelinks]{hyperref}
\usepgfplotslibrary{external,fillbetween}
\pgfplotsset{compat=1.14}
\usepackage[makeroom]{cancel}
\title{\bf{Homework \textnumero 14}}
\author{Author: David Oniani
\\
\ \ \ Instructor: Tommy Occhipinti}
\date{December 1, 2018}

\usepackage{listings}
\usepackage{color}

%%%%%%%%%%%%%%% S E T S %%%%%%%%%%%%%%%
\newcommand{\nats}{\mathbb{N}}
\newcommand{\ints}{\mathbb{Z}}
\newcommand{\rats}{\mathbb{Q}}
\newcommand{\reals}{\mathbb{R}}
\newcommand{\irrats}{\mathbb{I}}

\newcommand{\pnats}{\mathbb{N}^+}
\newcommand{\pints}{\mathbb{Z}^+}
\newcommand{\prats}{\mathbb{Q}^+}
\newcommand{\preals}{\mathbb{R}^+}
\newcommand{\nreals}{\mathbb{R}^-}

\newcommand{\nints}{\mathbb{Z}^-}
\newcommand{\nrats}{\mathbb{Q}^-}
%%%%%%%%%%%%%%%%%%%%%%%%%%%%%%%%%%%%%%%

% Calligraphy
\newcommand\und[1]{\underline{\smash{#1}}}

% Operators
\DeclarePairedDelimiter\abs{\lvert}{\rvert}
\DeclarePairedDelimiter\ceil{\lceil}{\rceil}
\DeclarePairedDelimiter\floor{\lfloor}{\rfloor}

% Other
\newcommand{\rarr}{\rightarrow}

\definecolor{dkgreen}{rgb}{0,0.6,0}
\definecolor{gray}{rgb}{0.5,0.5,0.5}
\definecolor{mauve}{rgb}{0.58,0,0.82}
\definecolor{backcolour}{rgb}{0.95,0.95,0.92}

\lstset{
backgroundcolor=\color{backcolour},
aboveskip=3mm,
belowskip=3mm,
showstringspaces=false,
columns=flexible,
basicstyle={\small\ttfamily},
numbers=left,
numberstyle=\normalsize\color{gray},
keywordstyle=\color{blue},
commentstyle=\color{dkgreen},
stringstyle=\color{mauve},
breaklines=true,
breakatwhitespace=true,
tabsize=4
}


\begin{document}
\maketitle
{\Large 5.3/5.4 - Cardinalities of Infinite Sets}
\begin{itemize}
\item[]
\item[]
{\und{\large Section 5.3}}
\item[1.]
The bijection is $f : N \rarr A_k : x \mapsto x + k + 1$ \\\\
It is injection since if $f(x_1) = f(x_2)$, it means that $x_1 + k + 1 = x_2 + k + 1$ and thus, $x_1 = x_2$.
It is surjection since if we have some $y \in A_k$, then we let $x = y - k - 1$ and we get that $f(x) = y - k - 1 + k + 1 = y$.
Hence, the function is both injective and surjective and thus, it is a bijection.

\item[2.]
\begin{itemize}
\item[(a)]
If $A$ is finite, then $A - \{a_0\}$ is automatically denumerable (since $A$ is finite).\\
If $A$ is infinite, we can show that $f : A \rarr A - \{a_0\} : x \mapsto x$ is a bijection and thus, since
$A$ is countable, $A - \{a_0\}$ is countable as well.
\item[(b)]
If $A$ is finite, then $A - \{a\}$ is automatically denumerable (since $A$ is finite).\\
If $A$ is infinite, we can show that $f : A \rarr A - \{a\} : x \mapsto x$ is a bijection and thus, since
$A$ is countable, $A - \{a\}$ is countable as well.
\end{itemize}

\item[]

\item[8.]
Suppose, for the sake of contradiction, $B - A$ is countable, $A$ is also countable, and $B$ is uncountable,.
Notice that $B = (B \cap A) \cup (B - A)$ is countable since it is the union of two countable sets.
Hence, we have reached the contradiction since we assumed that $B$ is uncountable and thus, $B - A$ is countable.
$\qed$

\item[]

\item[15.]
Since $A \times A$ is countable, then its subset $B = \{(x, x) \ | \ x \in A\}$ is also countable.
Then, $f : A \rarr B : x \mapsto (x, x)$ is a bijection and thus, $\abs{A} = \abs{B}$ which means that
$A$ is countable as well (since $B$ is countable).\\\\
\und{Proof that $f : A \rarr B : x \mapsto (x, x)$ is a bijection.}\\
If $f(x_1) = f(x_2)$, it means that $(x_1, x_1) = (x_2, x_2)$
and thus, $x_1 = x_2$. Hence, $f$ is injective.\\
For every $(x_i, x_i) \in B$, we can let $x = x_i$ and we get $f(x_i) = (x_i, x_i)$ and thus $f$ is surjective.\\\\
Since we have proven that $f$ is both injective and surjective, it means that $f$ is a bijection.

\item[]

\item[16.]
NOT DONE YET.

\item[]
\item[]

{\und{\large Section 5.4}}

\item[6.]
Suppose, for the sake of contradiction, that the infinite set $I' = \{x \in (0, 1) \ | x = .a_1a_2...a_n... \mbox{ where each $a_i = 3$ or 8}\}$ is countable.
Being countable means having a bijection with $\pints$. Hence, we have essentially assumed that there exists a function $f$ such that
$f : \pints \rarr I'$ is a bijection. Let's visualize it.
\begin{align*}
1 \mapsto .a_{1,1}a_{1,2}a_{1,3}...\\
2 \mapsto .a_{2,1}a_{2,2}a_{2,3}...\\
3 \mapsto .a_{3,1}a_{3,2}a_{3,3}...\\
...........................
\end{align*}
Now, let's define a function $transform$ in the following way:
$$transform(x) = \begin{cases} 8 & \mbox{if } n\mbox{ is 3} \\ 3 & \mbox{if } n\mbox{ is 8} \end{cases}$$
Simply put, this function is defined for only two inputs - 3 and 8 - and if the input is 3, it returns 8 while
if the input is 8, it returns 3.\\\\
Now, consider the number $.transform(a_{1,1})transform(a_{2,2})transform(a_{3,3})...$. We know that it is not equal
to the first number ($.a_{1,1}a_{1,2}a_{1,3}...$) since $transform(a_{1,1}) \neq a_{1,1}$. We also know that it is not equal
to the second number in the mapping as $transform(a_{2,2}) \neq a_{2,2}$. It does not equal to the third either because $transform(a_{3,3}) \neq a_{3,3}$.
As a result, we've got that:
\begin{align*}
.transform(a_{1,1})transform(a_{2,2})transform(a_{3,3})... \neq .a_{1,1}a_{1,2}a_{1,3}...\\
.transform(a_{1,1})transform(a_{2,2})transform(a_{3,3})... \neq .a_{2,1}a_{2,2}a_{2,3}...\\
.transform(a_{1,1})transform(a_{2,2})transform(a_{3,3})... \neq .a_{3,1}a_{3,2}a_{3,3}...\\
............................................................................................................
\end{align*}
Hence, we have constructed an element of the set $I'$ such that there is no corresponding element in $\ints$ that maps to it and the function $f : \pints \rarr I'$ is not onto hence, is not a bijection.
Finally, we have reached the contradiction and the set $I'$ is not countable.
\end{itemize}
\end{document}
