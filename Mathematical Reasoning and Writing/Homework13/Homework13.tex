\documentclass[12pt, a4paper]{article}
\usepackage[a4paper, margin=1in]{geometry}


\usepackage{adjustbox}
\usepackage{mathtools}
\usepackage{amsmath}
\usepackage{amssymb}
\usepackage{amsthm}

\usepackage{pgfplots}
\usepackage{listings}
\usepackage{color}
\usepackage{tikz}

\usepackage{textcomp}
\usepackage{soul}

\usepackage[hidelinks]{hyperref}
\usepgfplotslibrary{external,fillbetween}
\pgfplotsset{compat=1.14}
\usepackage[makeroom]{cancel}
\title{\bf{Homework \textnumero 13}}
\author{Author: David Oniani
\\
\ \ \ Instructor: Tommy Occhipinti}
\date{November 28, 2018}

\usepackage{listings}
\usepackage{color}

%%%%%%%%%%%%%%% S E T S %%%%%%%%%%%%%%%
\newcommand{\nats}{\mathbb{N}}
\newcommand{\ints}{\mathbb{Z}}
\newcommand{\rats}{\mathbb{Q}}
\newcommand{\reals}{\mathbb{R}}
\newcommand{\irrats}{\mathbb{I}}

\newcommand{\pnats}{\mathbb{N}^+}
\newcommand{\pints}{\mathbb{Z}^+}
\newcommand{\prats}{\mathbb{Q}^+}
\newcommand{\preals}{\mathbb{R}^+}
\newcommand{\nreals}{\mathbb{R}^-}

\newcommand{\nints}{\mathbb{Z}^-}
\newcommand{\nrats}{\mathbb{Q}^-}
%%%%%%%%%%%%%%%%%%%%%%%%%%%%%%%%%%%%%%%

% Calligraphy
\newcommand\und[1]{\underline{\smash{#1}}}

% Operators
\DeclarePairedDelimiter\abs{\lvert}{\rvert}
\DeclarePairedDelimiter\ceil{\lceil}{\rceil}
\DeclarePairedDelimiter\floor{\lfloor}{\rfloor}

% Other
\newcommand{\rarr}{\rightarrow}

\definecolor{dkgreen}{rgb}{0,0.6,0}
\definecolor{gray}{rgb}{0.5,0.5,0.5}
\definecolor{mauve}{rgb}{0.58,0,0.82}
\definecolor{backcolour}{rgb}{0.95,0.95,0.92}

\lstset{
backgroundcolor=\color{backcolour},
aboveskip=3mm,
belowskip=3mm,
showstringspaces=false,
columns=flexible,
basicstyle={\small\ttfamily},
numbers=left,
numberstyle=\normalsize\color{gray},
keywordstyle=\color{blue},
commentstyle=\color{dkgreen},
stringstyle=\color{mauve},
breaklines=true,
breakatwhitespace=true,
tabsize=4
}


\begin{document}
\maketitle
\begin{itemize}

\item[89.]
\begin{itemize}
\item[(a)]
There is such function. Consider the following function:
$$f : S \rarr \mathcal{P}(S) : x \mapsto \{x\}$$
It is injective since if $f(x_1) = f(x_2)$, it means that $\{x_1\} = \{x_2\}$
and thus, $x_1 = x_2$. In other words, $x_1$ and $x_2$ are the same since sets do not
allow duplicates. Hence, we got that different inputs give us different outputs and the
function is injective.

\item[]

\item[(b)]
Unfortunately (or fortunately), there is none. For a function $f : S \rarr \mathcal{P}(S)$
to be onto, we need to have something mapped to all of the elements of $\mathcal{P}(S)$.
However, since $S$ is finite (actually, it would also work with infinite sets), we know that $\abs{\mathcal{P}(S)} > \abs{S}$ and even if
we map all of the elements of $S$ to some elements of $\mathcal{P}(S)$, we will still be left
with at least one element. Thus, there is no such function.

\item[]

\item[(c)]
There is such function. Consider the following function:
$$f : \mathcal{P}(S) \rarr S : \{x\} \mapsto x \mbox{ such that } \abs{\{x\}} = 1$$
Then we know that all of the elements of $S$ will have the set of the element
mapped to it and the function is onto.

\item[]

\item[(d)]
There is such. Consider the following function:
$$f : S \rarr S \times S : x \mapsto (x, x)$$
Now, to make it crystal clear that it is injective, suppose, for the sake
of contradiction that $f(x_1) = f(x_2)$ and $x_1 \neq x_2$. Then we have that
$(x_1, x_1) = (x_2, x_2)$ and thus $x_1 = x_2$. However, we assumed that $x_1 \neq x_2$
and we have reached the contradiction. Hence, if $f(x_1) = f(x_2)$, then $x_1 = x_2$
and the function is injective.

\item[]

\item[(e)]
There is such. Consider the following function:
$$f : S \times S \rarr S : (x, x) \mapsto x$$

Notice that it is surjective since for any $s \in S$, we have
$(s, s)$ that maps to it.
\end{itemize}

\newpage

\item[90.]
Suppose, for the sake of contradiction, that we have twelve people $p_1, p_2, p_3, ... p_{11}, p_{12}$ and
there are no disjoint groups such that the sums of their ages are the same. 
We have that $p_1, p_2, p_3, ... p_{11}, p_{12} \leq 122$. We also know that
$p_1 \neq p_2 \neq p_3 ... \neq p_{11} \neq p_{12}$ since if $p_i = p_j$ for some $i \neq j \leq 12$,
then we can take two disjoint groups/sets $\{p_i\}$ and $\{p_j\}$ and their sums would be the same and we reach the contradiction.
Therefore, we have that $1 \leq p_1 \neq p_2 \neq p_3 ... p_{11} \neq p_{12} \leq 122$. Now let's figure
how many ways are there to decompose 12 people into proper subsets/subgroups. There will be $2^{12} - 2$ ways of doing it.
This is because for a person $p_k \in \{p_1, p_2, p_3, ... p_12\}$, $p_k$ is either in a subgroup or not and since there are 12 people,
we get $2^{12}$ subgroups and we exclude the group with no people in it and the group with all 12 people in it (so that the subgroup/subset is proper).
Hence, we can form $2^{12} - 2 = 4094$ subgroups. The sum of the arbitrarily taken subgroup will be at least 1 and at most $112 + 113 + 114 + ... + 122 = 1287$.
Now, since $4096 > 2 \times 1287 = 2807$, according to the pigeonhole principle, there will be at least 3 different groups with the same sum. Thus, there will be
two different groups with the same sum. Now, to ensure that they are disjoint, let's take out all the elements they share and we are done.$\qed$


\item[]
\item[]

\item[91.]
Zhang has proven that there are infinitely many consecutive prime pairs that can be separated by number $N$ which is less than or equal than $7 \times 10^7$.

\item[92.]
If there is no integer $N$ such that there are infinitely many consecutive primes with the property $p_2 - p_1 = N$,
it means that there are finitely many consecutive consecutive primes such that $\abs{p_1 - p_2} < 70, 000, 000$ and
we would face the contradiction.
\end{itemize}
\end{document}
