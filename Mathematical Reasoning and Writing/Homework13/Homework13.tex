\documentclass[12pt, a4paper]{article}
\usepackage[a4paper, margin=1in]{geometry}


\usepackage{adjustbox}
\usepackage{mathtools}
\usepackage{amsmath}
\usepackage{amssymb}
\usepackage{amsthm}

\usepackage{pgfplots}
\usepackage{listings}
\usepackage{color}
\usepackage{tikz}

\usepackage{textcomp}
\usepackage{soul}

\usepackage[hidelinks]{hyperref}
\usepgfplotslibrary{external,fillbetween}
\pgfplotsset{compat=1.14}
\usepackage[makeroom]{cancel}
\title{\bf{Homework \textnumero 13}}
\author{Author: David Oniani
\\
\ \ \ Instructor: Tommy Occhipinti}
\date{November 30, 2018}

\usepackage{listings}
\usepackage{color}

%%%%%%%%%%%%%%% S E T S %%%%%%%%%%%%%%%
\newcommand{\nats}{\mathbb{N}}
\newcommand{\ints}{\mathbb{Z}}
\newcommand{\rats}{\mathbb{Q}}
\newcommand{\reals}{\mathbb{R}}
\newcommand{\irrats}{\mathbb{I}}

\newcommand{\pnats}{\mathbb{N}^+}
\newcommand{\pints}{\mathbb{Z}^+}
\newcommand{\prats}{\mathbb{Q}^+}
\newcommand{\preals}{\mathbb{R}^+}
\newcommand{\nreals}{\mathbb{R}^-}

\newcommand{\nints}{\mathbb{Z}^-}
\newcommand{\nrats}{\mathbb{Q}^-}
%%%%%%%%%%%%%%%%%%%%%%%%%%%%%%%%%%%%%%%

% Calligraphy
\newcommand\und[1]{\underline{\smash{#1}}}

% Operators
\DeclarePairedDelimiter\abs{\lvert}{\rvert}
\DeclarePairedDelimiter\ceil{\lceil}{\rceil}
\DeclarePairedDelimiter\floor{\lfloor}{\rfloor}

% Other
\newcommand{\rarr}{\rightarrow}

\definecolor{dkgreen}{rgb}{0,0.6,0}
\definecolor{gray}{rgb}{0.5,0.5,0.5}
\definecolor{mauve}{rgb}{0.58,0,0.82}
\definecolor{backcolour}{rgb}{0.95,0.95,0.92}

\lstset{
backgroundcolor=\color{backcolour},
aboveskip=3mm,
belowskip=3mm,
showstringspaces=false,
columns=flexible,
basicstyle={\small\ttfamily},
numbers=left,
numberstyle=\normalsize\color{gray},
keywordstyle=\color{blue},
commentstyle=\color{dkgreen},
stringstyle=\color{mauve},
breaklines=true,
breakatwhitespace=true,
tabsize=4
}


\begin{document}
\maketitle
{\Large 5.2 - Combinatorics of Finite Sets}
\begin{itemize}
\item[89.]
\begin{itemize}
\item[(a)]
There is such function. Consider the following function:
$$f : S \rarr \mathcal{P}(S) : x \mapsto \{x\}$$
It is injective since if $f(x_1) = f(x_2)$, it means that $\{x_1\} = \{x_2\}$
and thus, $x_1 = x_2$. In other words, $x_1$ and $x_2$ are the same since sets do not
allow duplicates. Hence, we got that different inputs give us different outputs and the
function is injective.

\item[]

\item[(b)]
Unfortunately (or fortunately), there is none. For a function $f : S \rarr \mathcal{P}(S)$
to be onto, we need to have something mapped to all of the elements of $\mathcal{P}(S)$.
However, since $S$ is finite (actually, it would also work with infinite sets), we know that $\abs{\mathcal{P}(S)} > \abs{S}$ and even if
we map all of the elements of $S$ to some elements of $\mathcal{P}(S)$, we will still be left
with at least one element. Thus, there is no such function.

\item[]

\item[(c)]
There is such function. Consider the following function:
$$f : S \rarr \mathcal{P}(S) : x \mapsto \{x\}$$
Then we know that all of the elements of $\mathcal{P}(S)$ will have the element
mapped to it and the function is onto. It is one-to-one as well seince the set does not
allow duplicates. Then, $f^{-1}$ or the inverse of $f$ will be a bijection from $\mathcal{P}(S)$
to $S$.
\item[]

\item[(d)]
There is such. Consider the following function:
$$f : S \rarr S \times S : x \mapsto (x, x)$$
Now, to make it crystal clear that it is injective, suppose, for the sake
of contradiction that $f(x_1) = f(x_2)$ and $x_1 \neq x_2$. Then we have that
$(x_1, x_1) = (x_2, x_2)$ and thus $x_1 = x_2$. However, we assumed that $x_1 \neq x_2$
and we have reached the contradiction. Hence, if $f(x_1) = f(x_2)$, then $x_1 = x_2$
and the function is injective.

\item[]

\item[(e)]
There is such. Consider the following function:
$$f : S \times S \rarr S : (x, x) \mapsto x$$

Notice that it is surjective since for any $s \in S$, we have
$(s, s)$ that maps to it.
\end{itemize}

\newpage

\item[90.]
Suppose, for the sake of contradiction, that we have twelve people $p_1, p_2, p_3, ... p_{11}, p_{12}$ and
there are no disjoint groups such that the sums of their ages are the same. 
We have that $p_1, p_2, p_3, ... p_{11}, p_{12} \leq 122$. We also know that
$p_1 \neq p_2 \neq p_3 ... \neq p_{11} \neq p_{12}$ since if $p_i = p_j$ for some $i \neq j \leq 12$,
then we can take two disjoint groups/sets $\{p_i\}$ and $\{p_j\}$ and their sums would be the same and we reach the contradiction.
Therefore, we have that $1 \leq p_1 \neq p_2 \neq p_3 ... p_{11} \neq p_{12} \leq 122$. Now let's figure
how many ways are there to decompose 12 people into proper subsets/subgroups. There will be $2^{12} - 2$ ways of doing it.
This is because for a person $p_k \in \{p_1, p_2, p_3, ... p_12\}$, $p_k$ is either in a subgroup or not and since there are 12 people,
we get $2^{12}$ subgroups and we exclude the group with no people in it and the group with all 12 people in it (so that the subgroup/subset is proper).
Hence, we can form $2^{12} - 2 = 4094$ subgroups. The sum of the arbitrarily taken subgroup will be at least 1 and at most $112 + 113 + 114 + ... + 122 = 1287$.
Now, since $4096 > 2 \times 1287 = 2807$, according to the pigeonhole principle, there will be at least 3 different groups with the same sum. Thus, there will be
two different groups with the same sum. Now, to ensure that they are disjoint, let's take out all the elements they share and we are done.$\qed$


\item[]
\item[]

{\Large \textbf{Video: Gaps Between Primes}}

\item[91.]
Zhang proved that there are infinitely many consecutive prime pairs that can be separated by number $N$ which is less than or equal to $7 \times 10^7$.

\item[92.]
If there is no integer $N$ such that there are infinitely many consecutive primes with the property $p_2 - p_1 = N$,
it means that there are finitely many consecutive primes such that $\abs{p_1 - p_2} < 7 \times 10^7$ and
we would face the contradiction (contradicts what Zhang has proven).

\item[]
\item[]

{\Large Bookwork}
\item[1.]
Since $\abs{A^c} = 7$ and $A$ is a subset of the set $X$ which has the cardinality of 16, $\abs{A} = 16 - 7 = 9$.
Now, since $A \cap B = 5$ and $A \cup B = X$, we have:
\begin{align}
\abs{X}= \abs{A} + \abs{B} - \abs{A \cap B}\\
16 = 9 + \abs{B} - 5\\
\abs{B} = 12
\end{align}
Hence, we've got that $\abs{B} = 12$.

\item[]

\item[2.]
\begin{itemize}
\item[(a)]
Suppose $B$ is a finite set and $A$ is an arbitrary set.
Then, $\abs{A \cap B} \leq \abs{B}$ and hence, $A \cap B$ is finite.
$\qed$
\item[(b)]
Suppose, for the sake of contradiction, that $B$ is infinite, $A$ is finite, and $B - A$ is finite.
From the previous question, we know that $A \cap B$ is finite.
Notice that $(B - A) \cup (A \cap B) = B$. Now, if $B - A$ is finite,
then $A \cap B$ is finite and $(B - A) \cup (A \cap B)$ which means that
$B$ is finite and we face the contradiction. Thus, $B - A$ is infinite.
$\qed$
\end{itemize}

\item[]

\item[11.]
\begin{itemize}
\item[(a)]
The maximum number of days in a year is 366.
If some of the people share the birth date, it automatically shows that at least two of the people have a birthday on the same date.
Then the worst case scenario for this problem would be that all the people were born on different days. That will give us $100 \times 36600$
different days. But since there are 36601 people, according to the pigeonhole principle,
two of the people will have to be born on the same date. Hence, we've proven
that in all cases, at least two people are born on the same date (the same year and the same date).
$\qed$

\item[(b)]
Again, if two people share the same birth date (in this case, the same year, the same day, and the same hour of birth) then we have automatically shown that at least two of the people are born on the same date.
The worst case scenario would be that every person is born on a different date. There are $366 \times 100 \times 24 = 36600 \times 24$ hours in 100 years and since we have $36600 \times 24 + 1$ people, according
to the pigeonhole principle, at least two will share the same birth date.
$\qed$
\end{itemize}

\item[]

\item[14.]
Consider the following three cases:\\\\
Case I: There is a person who knows everyone.\\
Case II: There is a person who knows nobody.\\
Case III: There is no person who knows everyone and no person who knows nobody.
\\\\
Case I proof: If there is a person who knows everyone, it means that he knows $n - 1$ people. Hence, all people know at least 1 person. Now since there are $n$ people in total, by the pigeonhole principle, it must be the case that two people have the same number of friends. \\\\
Case II proof: If there is a person who does not know anyone, it means that a maximum number of people one can know is $n - 2$. However, now we got that for any person, the number of people he/she knows is between
0 and $n - 2$. And now, since there are $n$ party members and $n - 2 + 1 = n - 1$ different possibilities, according to the pigeonhole principle, it must be the case that at least two know the same number of people.
\\\\
Case III proof: If there are no person who knows everyone and no person who knows nobody, then the number of people a given person knows will be between $1$ and $n - 2$ and there will be $n - 2 - 1 + 1 = n - 2$ possibilities in total
resulting in at least 3 people knowing the same number of people which automatically means that at least two people know the same number of people.
$\qed$

\item[]

\item[17.]
\begin{itemize}
\item[(a)]
For every point $(x_i, y_i) \in A$, let's look at $(mod(x_i, 2), mod(y_i, 2))$
where $mod(m,n)$ is defined as a remainder that is left after dividing $m$ by $n$.
For, $(mod(x_i, 2), mod(y_i, 2))$ there are only for possibilities: $(0, 0), (0, 1), (1, 0), (1, 1)$.
Now, since we have 5 points, it must be the case that for at least two points (according to the pigeonhole principle)
$P = (x_j, y_j)$ and $Q = (x_k, y_k)$, $(mod(x_j, 2), mod(y_j, 2)) = (mod(x_k, 2), mod(y_k, 2))$. Then, the midpoint
of $PQ$ will have integer coordinates as
$$MQ = (\dfrac{x_j + x_k}{2}, \dfrac{y_j + y_k}{2}) \mbox{ where $M$ is the midpoint}$$
And $x_j + x_k$ as well as $y_j + y_k$ are both divisible by 2.
$\qed$
\end{itemize}

\item[]
\item[]
\item[]

{\Large 5.1 - Cardinality}
\item[93.]
We have proven (in the class) that if $\abs{A} = \abs{B}$ and $\abs{B} = \abs{C}$, then $\abs{A} = \abs{C}$ and thus, $\abs{A} = \abs{B} = \abs{C}$.
Hence, let's first show that $\abs{A} = \abs{B}$. To show this, we must show that there exists a bijection from $A$ to $B$.
And there exists such, it is $f : A \rarr B : x \mapsto x + 1$. Likewise, $g : B \rarr C : x \mapsto x + 1$ is also a bijection.
Hence, we've got that $\abs{A} = \abs{B}$ and $\abs{B} = \abs{C}$. Thus, $\abs{A} = \abs{B} = \abs{C}$.

\item[]

\item[94.]
Yes (we could prove it using the Cantor-Bernstein theorem).
We define the bijection by the following function construction:
$$f : [0, 1] \rarr [0, 1)=  \begin{cases} x \mbox{ if $0 \leq x < 1$, and $x \neq \dfrac{1}{2^n}$ (where $n \in \pints$)} \\ \dfrac{1}{2^{n + 1}} \mbox{ if $x = \dfrac{1}{2^n}$ (where $n \in \pints$)} \end{cases}$$

\item[]

\item[95.]
These two sets are not equinumerous hence, there is no bijection between them.
We have proven in class that for any set $S$, $\abs{S} \leq \abs{\mathcal{P}(S)}$. Hence, we can set $S = \pints$
and we have $\abs{\pints} \leq \abs{\mathcal{P}(\pints)}$ and the function $f : \ints \mapsto \mathcal{P}$ is not onto thus, is not a bijection.\\\\
P.S. I could repeat the proof that for any set $S$, $\abs{S} \leq \abs{\mathcal{P}(S)}$, but we have proven it so
I just used it.

\item[]
\item[]

{\Large Bookwork}
\item[1.]
NOTE: I am going to use the fact that it is sufficient to show that two sets $A$ and $B$ are equal
if there exists a bijection $f : A \rarr B$. Also, I am going to use the theorem that if $\abs{A} = \abs{B}$
then $\abs{B} = \abs{A}$ (it is not necessary but still mentioning it).
\begin{itemize}
\item[(a)]
\und{Showing that $C \approx \nats$.}\\
Here is a bijection:
$$f : \nats \rarr C : x \mapsto x^3$$
\\\\
\und{Showing that $C \approx \ints$.}\\
Here is a bijection:
$$f : \ints \rarr C : x \mapsto x^3$$
\item[(b)]
Here is a bijection:
$$f : \pints \rarr \nints : x \mapsto -x$$
\end{itemize}

\item[]

\item[2.]
\begin{itemize}
\item[(a)]
To show that $(0,1) \approx (0, 3)$ we must show that there exists a bijection from $(0, 1)$ to $(0, 3)$.\\\\
Here is a bijection:
$$f : (0, 1) \rarr (0, 3) : x \mapsto 3x$$

\item[(e)]
Here is a bijection:
$$f : (0, 1) \rarr (a, a + 3) = \begin{cases} x + a \mbox{ if $0 < x < \dfrac{1}{3}$}\\ x + a + 1 \mbox{ if $\dfrac{1}{3} \leq x \leq \dfrac{2}{3}$ }\\x + a + 2 \mbox{ if $\dfrac{2}{3} < x < 1$}\end{cases}$$
\end{itemize}

\item[]

\item[4]
\begin{itemize}
\item[(a)]
The sentence is equivalent to saying that the set of all 3-dimensional points with real coordinates is equinumerous
to the cartesian product of the cartesian product of the sets $(0, 1)$ with $(0, 1)$ with the set $(0, 1)$ (sort of cartesian product taken thrice).
\\\\
Proof: We have proven in the class that $(0, 1) \approx \reals$ (theorem). Then we know that $\abs{(0, 1) \times (0, 1)} = \abs{\reals^2}$.
Then, we have $\abs{I \times I \times I} = \abs{(0, 1) \times (0, 1) \times (0, 1)} = \abs{((0, 1) \times (0, 1)) \times (0, 1)} = \abs{\reals^2 \times \reals = \reals^3}$.
$\qed$
\item[(b)]
The generalization would be that for any interval $P = (m, n)$ where $x, y \in \reals$,
$P \times P \times P = \reals^3$. To prove this, it is sufficient to prove that $P \approx \reals$.
Thus, we have to show that there exists a bijection between $P$ and $\reals$.
Here is the bijection:
$$f : P \rarr \reals : x = \dfrac{n}{x} + m$$
\end{itemize}

\item[]

\item[13.]
\begin{itemize}
\item[(b)]
Interval $(0, 1)$ is a proper subset of $\reals$. Here is a bijection between $(0, 1)$ and $\reals$:
$$f : (0, 1) \rarr \reals : x = \mbox{tan}(\pi x - \dfrac{\pi}{2})$$
\end{itemize}

\item[15.]
\begin{itemize}
\item[(a)]
Let's prove the contrapositive. Thus, let's assume that $A \not \approx B$ and prove that
$\abs{A - B} \neq \abs{B - A}$. Notice that:
$$\abs{A} = \abs{A \cap B} + \abs{A \cap B^c}$$
$$\mbox{and}$$
$$\abs{B} = \abs{B \cap A}+\abs{B \cap A^c} = \abs{A \cap B}+\abs{B \cap A^c}$$
Hence, we have that $\abs{A \cap B} + \abs{A \cap B^c} \neq \abs{A \cap B}+\abs{B \cap A^c}$
and thus $\abs{A \cap B^c} \neq \abs{B \cap A^c}$ and finally since $A \cap B^c = A - B$ and $B \cap A^c = B - A$,
we have that $\abs{A - B} \neq \abs{B - A}$.
$\qed$
\item[(b)]
Yes. If $A \approx B$, then we have:
$$\abs{A} = \abs{A \cap B} + \abs{A \cap B^c} = \abs{B \cap A}+\abs{B \cap A^c} = \abs{B}$$
Now, notice that $A \cap B^c = A - B$ and from $\abs{A \cap B} + \abs{A \cap B^c} = \abs{B \cap A}+\abs{B \cap A^c}$,
we get that $\abs{A - B} = \abs{B - A}$.
$\qed$
\end{itemize}

\item[]
\item[]

{\Large Video: Pi Is (still) Wrong.}
\item[96.]
$\tau = 2\pi$. Thus, $\mbox{sin}(\tau / 8) = \mbox{sin}(2\pi / 8) = \mbox{sin}(\pi / 4) = \dfrac{\sqrt{2}}{2}$.
\item[97.]
$C = \tau r$, \ $A = \dfrac{\tau r^2}{2}$, \ $V = \dfrac{2}{3}\tau r^3$.\\\\
I actually do not think so. The primary reason is a bit complicated and is connected to how I remember formulas in general.
Another reason is that now we have two fractions for the formulas of $A$ and $V$ whereas we only had one for $V$ when used $\pi$
instead of $\tau$. Also, generally speaking, I have used the area of the circle a lot more often than the length, hence having a nice
$A = \pi r^2$ is a lot more convenient than $\dfrac{\tau r^2}{2}$ and dealing with fractions.
\end{itemize}
\end{document}
