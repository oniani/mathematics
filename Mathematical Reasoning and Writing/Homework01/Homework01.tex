\documentclass[12pt, a4paper]{article}             % use "amsart" instead of "article" for AMSLaTeX format
\usepackage[a4paper,margin=1in]{geometry}          % Adjust margins
\usepackage{amsmath,amssymb,listings,color}        % Math packages: amsmath, amssymb, listings, color
\usepackage[makeroom]{cancel}

\title{\bf{Getting Started with \LaTeX}}
\author{Author: David Oniani
\\
\ \ \ Instructor: Tommy Occhipinti}
\date{August 29, 2018}

\usepackage{listings}
\usepackage{color}

\definecolor{dkgreen}{rgb}{0,0.6,0}
\definecolor{gray}{rgb}{0.5,0.5,0.5}
\definecolor{mauve}{rgb}{0.58,0,0.82}
\definecolor{backcolour}{rgb}{0.95,0.95,0.92}

\lstset{
backgroundcolor=\color{backcolour},
aboveskip=3mm,
belowskip=3mm,
showstringspaces=false,
columns=flexible,
basicstyle={\small\ttfamily},
numbers=left,
numberstyle=\normalsize\color{gray},
keywordstyle=\color{blue},
commentstyle=\color{dkgreen},
stringstyle=\color{mauve},
breaklines=true,
breakatwhitespace=true,
tabsize=4
}


\begin{document}
\maketitle


\begin{enumerate}
\item[1.]
\text{Render the following text in \LaTeX \ as exactly as you can:}

\begin{quote}

\text{The quadratic formula asserts that if $a, b, c \in \mathbb{N}$ and $a \neq 0$ then the roots of}
\text{the polynomial $ax^2 + bx + c$ are given by}

\begin{center}
\text{$x = \dfrac{-b \pm \sqrt{b^2-4ac}}{2a}.$}
\end{center}

\text{The case of the cubic $ax^3 + bx^2 + cx + d$ is more complicated, and one must}
\text{first compute $\Delta_0 = b^2 - 3ac, \Delta_1 = 2b^3 - 9abc + 27ad^2$ and}

\begin{center}
\text{$C = \sqrt[3]{\dfrac{\Delta_1 \pm \sqrt{\Delta_1^2 - 4\Delta_0^3}}{2}}.$}
\end{center}

\text{In this case, one finds that one root of the cubic is given by}

\begin{center}
\text{$x = \dfrac{1}{3a}(b + C + \dfrac{\Delta_0}{C}).$}
\end{center}

\text{A similar formula exists for the roots of the quartic $ax^4 + bx^3 + cx^2 + dx + e,$}
\text{however it is even more complicated. Remarkably, Abel and Ruffini proved}
\text{in 1824 that no similar formula exists for the roots of the quintic $ax^5 + bx^4 +$}
\text{$cx^3 + dx^2 + ex + f$, or indeed for any polynomial of degree $\geq 5.$}
\end{quote}

\item[2.]
\text{How convinced are you by the evidence and argument presented that there are more}
\text{odd numbered houses than even numbered houses? Which do you find more convincing,}
\text{the argument or the numerical evidence? [Your answer here should probably be about}
\text{4 sentences.]}

\begin{quote}
I think both proofs were reasonable. I would say that the argument was stronger here.
This is not because the bell curve analysis was irrelevant, but rather because tbe argument was a lot simpler (subjective opinion)
to explain in human words without using math, statistics, bell curves or deviations. Thus, simpler is better and I go with the argument here.
I was pretty convinced as I thought of the same as soon as James said that odd-numbered houses were a bit more than the even-numbered ones.
\end{quote}

\item[3.]
\text{Suppose you looked at all numbers from 1 to 1000000. What percentage of them do}
\text{you think \underbar{start} with an odd number?}

\begin{quote}
It will be 55.5556\%. To see why it's true, let's jot down
the list of odd-digit-starting numbers for different ranges.

\begin{center}
$5$ for numbers in the range $1-9$ \\
$50$ for numbers in the range $10-99$ \\
$500$ for numbers in the range $100-999$ \\
$5000$ for numbers in the range $1000-9999$ \\
$50000$ for numbers in the range $10000-99999$ \\
$500000$ for numbers in the range $100000-999999$ \\
$1$ for $1000000$
\end{center}

We get, $5 + 50 + 500 + 5000 + 50000 + 500000 + 1 = 555556$ and finally, $555556 / 1000000 \times 100\% = 55.5556\%.$
\\\\
Here is a program to confirm our proof. It is written in \textbf{Haskell},
a purely functional programming language which is based upon \textbf{Lambda calculus}
and is used by a lot of mathematicians worldwide.

\begin{lstlisting}[language=haskell]
-- Importing digitToInt function from Data.Char module
import Data.Char(digitToInt)


-- | Function to check the oddity of the first digit
isOddFirst :: Int -> Bool
isOddFirst x
    | firstDigit `mod` 2 == 1 = True
    | otherwise = False
    where
        firstDigit = digitToInt $ head $ show x

-- Function to count the numbers that start with an odd digit
oddStartingNums :: [Int] -> Int
oddStartingNums xs = length $ filter isOddFirst xs


main = do
    let bigInt = oddStartingNums [1..1000000] :: Int
    let pctg =  show (fromIntegral bigInt * 100 / 1000000) ++ "%"

    print bigInt -- 555556
    putStr pctg -- 55.5556% (indeed, it's 55.5556% !!)
\end{lstlisting}
\end{quote}

\item[4.]
Various theorems and laws. Apart from that, I learned that something that seems true might not be true, vice versa.

\item[5.]
Conjecture 17 (otherwise known as the examples of Moessner's process). 
I found it interesting how these square numbers were generated. It
should also be noted that the same idea (with a bit of modification)
could generate cubic and quartic power numbers and even factorials!

Here is the explanation for the conjecture.

Conjecture 17 states: Write down the positive integers, delete every second, and form the
partial sums of those remaining.\\\\
Let's list the natural numbers up to 20 (we could continue indefinitely).\\

1 2 3 4 5 6 7 8 9 10 11 12 13 14 15 16 17 18 19 20\\

Now, let's cross out every second number.
We get:\\

1 \cancel{2} 3 \cancel{4} 5 \cancel{6} 7 \cancel{8} 9 \cancel{10} 11 \cancel{12} 13 \cancel{14} 15 \cancel{16} 17 \cancel{18} 19 \cancel{20}\\

If we list the partial sums, we get:\\

1 4 9 16 25 36 49 64 81\\

\textbf{Here is why it happens:}\\

It's clear that numbers 1, 3, 5, 7 ... etc. form the arithmetic progression where the common difference is 2.
Then we can write the following:\\
$$\sum_{i=1}^n 2n-1 = \dfrac{1 + 2n - 1}{2} \times n = n \times n = n^2$$

Thus, this proves that all the partial sums are the squares.
\end{enumerate}

\newpage
\begin{center}
\LARGE{\textbf{A Page for Feedback}}
\end{center}


\end{document}
