\documentclass[12pt, a4paper]{article}
\usepackage[a4paper, margin=1in]{geometry}


\usepackage{adjustbox}
\usepackage{mathtools}
\usepackage{amsmath}
\usepackage{amssymb}
\usepackage{amsthm}
\usepackage{systeme}

\usepackage{pgfplots}
\usepackage{listings}
\usepackage{color}
\usepackage{tikz}

\usepackage{textcomp}
\usepackage{soul}

\usepackage[hidelinks]{hyperref}
\usepgfplotslibrary{external,fillbetween}
\pgfplotsset{compat=1.14}
\usepackage[makeroom]{cancel}
\title{\bf{Homework \textnumero 10}}
\author{Author: David Oniani
\\
\ \ \ Instructor: Tommy Occhipinti}
\date{October 31, 2018}

\usepackage{listings}
\usepackage{color}

%%%%%%%%%%%%%%% S E T S %%%%%%%%%%%%%%%
\newcommand{\nats}{\mathbb{N}}
\newcommand{\ints}{\mathbb{Z}}
\newcommand{\rats}{\mathbb{Q}}
\newcommand{\reals}{\mathbb{R}}
\newcommand{\irrats}{\mathbb{I}}

\newcommand{\pnats}{\mathbb{N}^+}
\newcommand{\pints}{\mathbb{Z}^+}
\newcommand{\prats}{\mathbb{Q}^+}
\newcommand{\preals}{\mathbb{R}^+}
\newcommand{\nreals}{\mathbb{R}^-}

\newcommand{\nints}{\mathbb{Z}^-}
\newcommand{\nrats}{\mathbb{Q}^-}
%%%%%%%%%%%%%%%%%%%%%%%%%%%%%%%%%%%%%%%

% Calligraphy
\newcommand\und[1]{\underline{\smash{#1}}}

% Operators
\DeclarePairedDelimiter\abs{\lvert}{\rvert}
\DeclarePairedDelimiter\ceil{\lceil}{\rceil}
\DeclarePairedDelimiter\floor{\lfloor}{\rfloor}


\definecolor{dkgreen}{rgb}{0,0.6,0}
\definecolor{gray}{rgb}{0.5,0.5,0.5}
\definecolor{mauve}{rgb}{0.58,0,0.82}
\definecolor{backcolour}{rgb}{0.95,0.95,0.92}

\lstset{
backgroundcolor=\color{backcolour},
aboveskip=3mm,
belowskip=3mm,
showstringspaces=false,
columns=flexible,
basicstyle={\small\ttfamily},
numbers=left,
numberstyle=\normalsize\color{gray},
keywordstyle=\color{blue},
commentstyle=\color{dkgreen},
stringstyle=\color{mauve},
breaklines=true,
breakatwhitespace=true,
tabsize=4
}


\begin{document}
\maketitle

\begin{itemize}
\item[72.]
For each $A$ and $B$ below, give an example of a function from $A$ to $B$ that is non-constant and not given piecewise.
\begin{itemize}
\item[(a)]
$A = (0, 1), \ B = [0, 1]$
\begin{quote}
$f : A \rightarrow B : x \mapsto \dfrac{\cos{x} + 1}{2}$
\end{quote}

\item[]

\item[(b)]
$A = [0, 1], \ B = (0, 1)$
\begin{quote}
$f : A \rightarrow B : x \mapsto \dfrac{x + 1}{x + 2}$
\end{quote}

\item[]

\item[(c)]
$A = \reals, \ B = 2 \ints$ (the set of even integers)
\begin{quote}
$f : A \rightarrow B : x \mapsto \floor{x} \times 2$
\end{quote}

\item[]

\item[(d)]
$A$ is the set of finite subsets of $\pints$, $B = \pints$
\begin{quote}
$f : A \rightarrow B : x \mapsto \bigcup_{X \in A}$
\end{quote}
\end{itemize}

\item[]
\item[]

\item[73.]
Consider the function: $f : \ints \rightarrow \ints : n \mapsto 2n + 1$. Is it one-to-one? Is it onto?
Be sure to prove your answers.
\begin{quote}
The function is one-to-one but \und{NOT} onto.
\\\\
To prove that $f$ is one-to-one, suppose, for the sake of contradiction, that $x_1, x_2 \in \ints$ such that $x_1 \neq x_2$
and $2x_1 + 1 = 2x_2 + 1$. Then, if $2x_1 + 1 = 2x_2 + 1$, it means that $2x_1 = 2x_2$ and $x_1 = x_2$. Hence, we have reached
the contradiction and $f$ is one-to-one.
\\\\
To prove that $f$ is not onto, let the output for some input of the function be $4$ (since $4 \in \ints$, it is perfectly valid assumption).
Then we have that for some $n$, $2n + 1 = 4$
and $n = \dfrac{3}{2}$. Hence we got that $n \neq \mathbb{Z}$ and thus the function $f$ is not onto.
$\qed$
\end{quote}

\item[74.]
\begin{itemize}
\item[(a)]
Prove the function $f : \reals \times \reals \rightarrow \reals \times \reals : (x, y) \mapsto (x + y, x - y)$
is a bijection.
\begin{quote}
If we want to prove that the function is bijection, we must prove that the function is both one-to-one and onto.
\\\\
\und{One-to-one proof}\\\\
Suppose, for the sake of contradiction, that $(x_1, y_1), (x_2, y_2) \in \reals \times \reals$
and $f(x_1, y_1) = f(x_2, y_2)$. Then it must be the case that $x_1 + y_1 = x_2 + y_2$ and $x_1 - y_1 = x_2 - y_2$.
Thus, we have the following system of two equations:\\
\begin{equation*}
    \systeme{
    x_1 + y_1 = x_2 + y_2,
    x_1 - y_1 = x_2 - y_2
    }
\end{equation*}
Now, if we sum up the equations, we get $2x_1 = 2x_2$ which means that $x_1 = x_2$. On the other hand,
if we substitute $x_2$ with $x_1$ in the second equation of the system, we get $-y_1 = -y_2$ from which
we also get that $y_1 = y_2$. Hence, we got that if $f(x_1, y_1) = f(x_2, y_2)$, it must be the case
that $x_1 = x_2$ and $y_1 = y_2$. In other words, the inputs map the same outputs only if they are the same.
Hence, the different inputs map to the different outputs and the function is one-to-one.
\\\\
\und{Onto proof}\\\\
Suppose $(x_1, y_1)$ is in the codomain of the function (where $x_1, y_1 \in \reals$). Then we can let $x = \dfrac{x_1 + y_1}{2}$
and $y = \dfrac{x_1 - y_2}{2}$. We get $x + y = \dfrac{x_1 + y_1}{2} + \dfrac{x_1 - y_2}{2} = \dfrac{2x_1}{2} = x_1$
and $x - y = \dfrac{x_1 + y_1}{2} - \dfrac{x_1 - y_2}{2} = \dfrac{2y_1}{2} = y_1$.
Hence, we have shown that for every element in the codomain, we can find the element in the domain
that maps to that element. Thus, the function is onto.
\\\\
Now, since we proved that the function is both one-to-one and onto, we have effectively proven that the function is bijection.
$\qed$
\end{quote}
\end{itemize}

\item[]

\item[(b)]
Prove that function $g : \ints \times \ints \rightarrow \ints \times \ints : (x, y) \mapsto (x + y, x - y)$ is NOT a bijection.
\begin{quote}
For the function to be bijection, it should be both one-to-one and onto. This function is not onto. Here is the proof by counterexample:\\\\
Let the output of the function be $(1, 2)$. Then $(1, 2) \in \ints \times \ints$ thus, is in the codomain. Then, we have a system:\\\\
\begin{equation*}
    \systeme{
    x + y = 1,
    x - y = 2
    }
\end{equation*}
If we add the two equations together, we get $2x = 3$ and $x = \dfrac{3}{2}$. Now, if we substitute $x$
in the first equation, we get $y = -\dfrac{1}{2}$. Thus, we got the input pair $(\dfrac{3}{2}, \dfrac{1}{2})$
where both of the elements are non-integers and hence, $(\dfrac{3}{2}, \dfrac{1}{2}) \notin \ints \times \ints$
and the function is not onto which also proves that the function is not bijection.
$\qed$
\end{quote}

\item[]
\item[]

\item[75.]
Consider the function $f : \ints \times \ints \rightarrow \rats$ given by $f(a, b) = \dfrac{a}{\abs{b} + 1}$.
Is $f$ one-to-one. Is $f$ onto? Be sure to prove your answers.
\begin{quote}
$f$ is onto but \und{NOT} one-to-one.\\\\
To prove that $f$ is not one-to-one consider the following pairs: $a = 1, b = 2$ and $a = 1, b = -2$.
If $a = 1$ and $b = 2$, $f(1, 2) = \dfrac{1}{\abs{2} + 1} = \dfrac{1}{3}$. On the other hand, if $a = 1$ and $b = -2$,
$f(1, -2) = \dfrac{1}{\abs{-2} + 1} = \dfrac{1}{3}$. Thus, we have that $f(1, 2) = f(1, -2)$ and different inputs
give us the same outputs from which we deduce that $f$ is not one-to-one.
\\\\
To prove that $f$ is onto, suppose $x \in \rats$. Then $x = \dfrac{m}{n}$ where $m, n \in \ints$.
Now, consider the following two cases:\\\\
Case I: $x$ is negative.\\
Case II: $x$ is zero\\
Case III: $x$ is positive.\\\\
Case I Proof: If $x < 0$, then $\dfrac{m}{n} < 0$. If $\dfrac{m}{n} < 0$, $-\dfrac{m}{n} > 0$.
Then, let $a = -m$ and $\abs{b} + 1 = n$ and we get two solution pairs $(-m, n - 1)$ and $(-m, 1 - n)$.
\\\\
Case II: If $x = 0$, let $a = 0$ and we will have infinitely many solutions.
\\\\
Case III: If $x > 0$, $\dfrac{m}{n} > 0$ and we let $a = m$ and $\abs{b} + 1 = n$. Finally,
we get two solution pairs $(m, n - 1)$ and $(-m, 1 - n)$.
$\qed$
\end{quote}

\item[]
\item[]

\item[76.]
Consider the function $g : \mathcal{P}(\{1, 2, 3, 4, 5\}) \rightarrow \mathcal{P}(\{1, 2, 3, 4\}) : A \mapsto A - \{5\}$.
Is $g$ one-to-one? Is it onto? Be sure to prove your answers.
\begin{quote}
Function $g$ is both onto but \und{NOT} one-to-one.\\\\
\und{Onto proof}\\\\
Let $S$ be a set in the codomain. Now, let's explore the connections between $\mathcal{P}(\{1, 2, 3, 4, 5\})$ and $\mathcal{P}(\{1, 2, 3, 4\})$.
It is clear that $\mathcal{P}(\{1, 2, 3, 4\}) \subset \mathcal{P}(\{1, 2, 3, 4, 5\})$.\ In other words, $\mathcal{P}(\{1, 2, 3, 4, 5\})$ is $\mathcal{P}(\{1, 2, 3, 4\})$ plus
the sets generated by all the combinations of 5 with the elements of $\mathcal{P}(\{1, 2, 3, 4, 5\})$. Then, since $S$ is in the codomain,
$S - \{5\}$ is guaranteed to be in the domain since if we subtract 5 from all the element sets of $\mathcal{P}(\{1, 2, 3, 4, 5\})$, we get the set $\mathcal{P}(\{1, 2, 3, 4\})$.
Hence, for every output, we have an input and the set is onto.
\\\\
\und{Proving that $g$ is not one-to-one}\\\\
Let $A_1 = \{1, 2, 3\}$ and $A_2 = \{2, 3, 4\}$, then $A_1 \neq A_2$, but $g(A_1) = g(A_2)$
since $A_1 - \{5\} = A_2 - \{5\} = \{2, 3, 4\}$. Hence, for different inputs, we got the same output
and thus, the function $g$ is not one-to-one.
$\qed$
\end{quote}
\end{itemize}
\end{document}
