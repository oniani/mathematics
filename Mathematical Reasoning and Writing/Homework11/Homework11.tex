\documentclass[12pt, a4paper]{article}
\usepackage[a4paper, margin=1in]{geometry}


\usepackage{adjustbox}
\usepackage{mathtools}
\usepackage{amsmath}
\usepackage{amssymb}
\usepackage{amsthm}

\usepackage{pgfplots}
\usepackage{listings}
\usepackage{color}
\usepackage{tikz}

\usepackage{textcomp}
\usepackage{soul}

\usepackage[hidelinks]{hyperref}
\usepgfplotslibrary{external,fillbetween}
\pgfplotsset{compat=1.14}
\usepackage[makeroom]{cancel}
\title{\bf{Homework \textnumero 11}}
\author{Author: David Oniani
\\
\ \ \ Instructor: Tommy Occhipinti}
\date{November 7, 2018}

\usepackage{listings}
\usepackage{color}

%%%%%%%%%%%%%%% S E T S %%%%%%%%%%%%%%%
\newcommand{\nats}{\mathbb{N}}
\newcommand{\ints}{\mathbb{Z}}
\newcommand{\rats}{\mathbb{Q}}
\newcommand{\reals}{\mathbb{R}}
\newcommand{\irrats}{\mathbb{I}}

\newcommand{\pnats}{\mathbb{N}^+}
\newcommand{\pints}{\mathbb{Z}^+}
\newcommand{\prats}{\mathbb{Q}^+}
\newcommand{\preals}{\mathbb{R}^+}
\newcommand{\nreals}{\mathbb{R}^-}

\newcommand{\nints}{\mathbb{Z}^-}
\newcommand{\nrats}{\mathbb{Q}^-}
%%%%%%%%%%%%%%%%%%%%%%%%%%%%%%%%%%%%%%%

% Calligraphy
\newcommand\und[1]{\underline{\smash{#1}}}

% Operators
\DeclarePairedDelimiter\abs{\lvert}{\rvert}
\DeclarePairedDelimiter\ceil{\lceil}{\rceil}
\DeclarePairedDelimiter\floor{\lfloor}{\rfloor}

% Other
\newcommand{\rarr}{\rightarrow}

\definecolor{dkgreen}{rgb}{0,0.6,0}
\definecolor{gray}{rgb}{0.5,0.5,0.5}
\definecolor{mauve}{rgb}{0.58,0,0.82}
\definecolor{backcolour}{rgb}{0.95,0.95,0.92}

\lstset{
backgroundcolor=\color{backcolour},
aboveskip=3mm,
belowskip=3mm,
showstringspaces=false,
columns=flexible,
basicstyle={\small\ttfamily},
numbers=left,
numberstyle=\normalsize\color{gray},
keywordstyle=\color{blue},
commentstyle=\color{dkgreen},
stringstyle=\color{mauve},
breaklines=true,
breakatwhitespace=true,
tabsize=4
}


\begin{document}
\maketitle

\begin{itemize}
\item[77.]
\begin{itemize}
\item[(a)]
\begin{quote}
It's true.\\\\
Proof: Suppose that $f : A \rightarrow B$ is an onto function.
Then, we know that every element of $B$ has at least one element
of $A$ mapped to it. Now, consider $\mathcal{F} : A \times A \rightarrow B \times B : (a_1, a_2) \mapsto (f(a_1), f(a_2))$.
To prove that $\mathcal{F}$ is onto, we have to show that for every $x \in B \times B$, we can find $y \in A \times A$ that maps
to it. Every $x \in B \times B$ will have a type $(f(a_1), f(a_2))$. Let's take a look at $(f(a_1), f(a_2))$.
We have $f(a_1) \in B$ and $f(a_2) \in B$ because $f : A \rightarrow B$ is onto. Finally, since $B \times B$
is a set of all possible pairs of $B$, we get that for arbitrary $(a_1, a_2) \in A \times A$, $(f(a_1), f(a_2)) \in B \times B$.
And we showed that $\mathcal{F}$ is onto.
$\qed$
\end{quote}

\item[]

\item[(b)]
\begin{quote}
It is true.\\\\
Suppose $f : A \rightarrow B$ is a one-to-one function. Then we know that every $a \in A$,
maps to a different $b \in B$. Consider $\mathcal{F} : A \times A \rightarrow B \times B : (a_1, a_2) \mapsto (f(a_1), f(a_2))$.
Since we know that all the ordered pairs in the set $A \times A$ are different (fortunately, sets do not allow duplicates),
we can say that every $(a_1, a_2)$ will map to a different $(b_1, b_2)$.\\\\
Extended proof:\\\\
Suppose, for the sake of contradiction, that $(a_1, a_2), (a_3, a_4) \in A \times A$, $(b_1, b_2) \in B \times B$ where $(a_1, a_2) \neq (a_3, a_4)$, and
that $\mathcal{F}((a_1, a_2)) =(b_1, b_2)$ and $\mathcal{F}((a_3, a_4)) =(b_1, b_2)$. Our assumption is equivalent to assuming
that $(f(a_1), f(a_2)) =(b_1, b_2)$ and $(f(a_3), f(a_4)) = (b_1, b_2)$ where $f$ is one-to-one (according to the problem description).
Since, by our assumption, $(a_1, a_2) \neq (a_3, a_4)$, it means that either $a_1 \neq a_3$ or $a_2 \neq a_4$ or both). In any case, since $f$ is one-to-one,
tuples $(f(a_1), f(a_2))$ and $(f(a_2), f(a_3))$ will be different and thus $(f(a_1), f(a_2)) \neq (f(a_2), f(a_3))$.
Hence, we've reached the contradiction and $\mathcal{F}$ is one-to-one.
\end{quote}

\end{itemize}

\item[]
\item[]

\item[78.]
\begin{itemize}
\item[(a)]
\begin{quote}
Every function $g$ is a subset of the cartesian product of its domain and codomain.
Thus, $f$ is a subset of $S \times \emptyset = \emptyset$. Then the only
to have a function being a subset of $\emptyset$ is to make its domain
an empty set thus $S = \emptyset$ (it is vacuously true). Thus, if $f : S \rightarrow \emptyset$ is a function,
it tells me that $S = \emptyset$.
\end{quote}

\item[]

\item[(b)]
\begin{quote}
It tells me that $S = \emptyset$. It is vacuously onto because everything in the codomain ($\emptyset$)
has an element in domain ($\emptyset$) that maps to it. Hence, it is a nothing-to-nothing onto function.
Thus, if there is an onto function $g : \emptyset \rightarrow S$, it tells me that $S = \emptyset$.
\end{quote}

\item[]

\item[(c)]
\begin{quote}
It tells me that $S = \emptyset$. It is vacuously one-to-one because everything in the domain ($\emptyset$)
maps to a different element in the codomain ($\emptyset$). In all other cases, where $S$ is non-empty, we have
that nothing maps to some element which is vacuously not true.
Thus, if there is an onto function $g : \emptyset \rightarrow S$, it tells me that $S = \emptyset$.
\end{quote}
\end{itemize}

\item[]
\item[]

\item[79.]
\begin{quote}
Consider the set $A = \{x \in S \ | \ x \notin f(x)\}$. In other words, set $A$ is a set of all elements
which are in $S$ but not in $f(x)$. Then we know that $A \in \mathcal{P}(S)$ because $A$
is the set of all elements of $S$ except for the empty set if $S$ contains one.
Now, all we have to do is to show that $f(x) \neq A$ where $x \in S$.\\\\
To prove that $f(x) \neq A$ for all $x \in S$, consider the following two cases:\\\\
1. Case I: $x \in A$.\\
2. Case II: $x \notin A$.
\\\\
Case I Proof: If $x \in A$, then we know that $x \notin f(x)$. Thus, $f(x) \neq A$.\\
Case II Proof: If $x \notin A$, then we know that $x \in f(x)$. Thus, $f(x) \neq A$.\\\\
Hence, we got that there is no $x \in S$ such that $f(x) = A$ where $A = \{x \in S \ | \ x \notin f(x)\}$.
Finally, we found an element in $\mathcal{P}(S)$ to which no element in $S$ maps to and thus,
$f$ is not onto.
$\qed$
\end{quote}

\item[]
\item[]

\item[80.]
\begin{itemize}
\item[(a)]
\begin{quote}
Suppose, for the sake of contradiction, that $f : A \rightarrow B$ is a one-to-one function and that $\hat{f} : \hat{A} \rightarrow B : \hat{a} \mapsto f(\hat{a})$
is not one-to-one (where $\hat{A}$ is arbitrary subset of $A$). Since $f$ is one-to-one, know that for all $a_1, a_2 \in A$, if $f(a_1) = f(a_2)$,
it means that $a_1 = a_2$. Now, consider $\hat{f} : \hat{A} \rightarrow B : \hat{a} \mapsto f(\hat{a})$.
Since it is not one-to-one, we have that for some $\hat{a_1} \neq \hat{a_2}$, $f(\hat{a_1}) = f(\hat{a_2})$.
Then, since $\hat{A}$ is a subset of $A$, it means that $a_1, a_2 \in A$ and that $f(a_1) = f(a_2)$ while $a_1 \neq a_2$
and we have reached the contradiction. Hence, if $f : A \rightarrow B$ is one-to-one, then $\hat{f} : \hat{A} \rightarrow B : \hat{a} \mapsto f(\hat{a})$ is also one-to-one.
$\qed$
\end{quote}

\item[]

\item[(b)]
\begin{quote}
The corestriction of $f$ to $\hat{B}$ is $f : A \rightarrow \hat{B} : a \mapsto f(\hat{a})$.\\\\
\und{Proving that every corestriction of a one-to-one function is one-to-one.}\\
Suppose, for the sake of contradiction, that $f : A \rightarrow B$ is a one-to-one function
and that $f : A \rightarrow \hat{B} : a \mapsto f(\hat{a})$ is not (where $B \subseteq \hat{B}$).
Thus, we have that for any $a_1, a_2 \in A$, if $f(a_1) = f(a_2)$, then $a_1 = a_2$.
And we also effectively assumed that there exists $a_3, a_4 \in A$ such that $a_3 \neq a_4$ and $\hat{f(a_3)} = \hat{f(a_4)}$.
Now consider the function $f : A \rightarrow \hat{B} : a \mapsto f(\hat{a})$. We know that $f(\hat{a}) \in B$ and since we
have already proven that $f$ is one-to-one, it means that if $f(a_1) = f(a_2)$, it means that $a_1 = a_2$ and $f$ is one-to-one
as well.\\\\
If the initial function is onto, it does not mean that function $f$ is onto too. We can find $\hat{b} \in \hat{B}$
such that $\hat{b} \notin B$ and then since the domain of both of the functions is the same, the function cannot have
multiple outputs of the same input and the function $f$ is not onto (NOTE: it is not onto if $B \subset \hat{B}$, if $B = \hat{B}$, then obviously both functions are onto).
\end{quote}
\end{itemize}

\item[]
\item[]

\item[81.]
\begin{itemize}
\item[(a)]
\begin{quote}
Since the function is $sin(x)$ its codomain is the set of all values from -1 to 1. In other words, the codomain
of $f$ is a closed interval $[-1, 1]$. Then, we can let $B = 25$ and we have that for all x in $\reals$,
$-1 \leq f(x) \leq -1$ and thus for all $x \in reals$, $f(x) \leq B$.
$\qed$
\end{quote}

\item[]

\item[(b)]
\begin{quote}
Suppose, for the sake of contradiction, that there exists $B \in \reals$, such that for all $a \in A$,
we have that $f(a) \leq B$. Then it must be the case that $\mbox{ln}(x)$ has some
fixed upper bound. This is, however, false because $\lim_{x \to + \infty} \mbox{ln}(x) = +\infty$.
Hence, the function $\mbox{ln}(x)$ has no upper bound and we have reached the contradiction.
\end{quote}

\item[]

\item[(c)]
\begin{quote}
It is true so let's prove it. Suppose, for the sake of contradiction, that there exists $B \in \reals$
such that for all $x$ in the $f$'s domain, $f(x) \leq B$. On the other hand, $f$ is defined for all $\reals$
thus, with the pigeonhole principle we get that $f$ always increases. Thus, $f$ is the function that
always increases and there is no upper bound and have reached the contradiction.
$\qed$
\end{quote}

\item[]

\item[(d)]
\begin{quote}
It is true. Suppose that $f : A \rarr \reals$ and $g : \reals \rarr \reals$ are the functions such that $g$ is a bounded above function.
Since $g$ is bounded above it means that there exists $B \in \reals$ such that for all $r \in \reals$, we have $g(r) \leq B$.
$g \circ f : A \rightarrow \reals$ is a composition of functions $g$ and $f$.
We can then rewrite the composition in the following way: $g \circ f : A \rarr \reals : a \mapsto g(f(a))$ (note that we could write it only because
the image of $f$ is in $\reals$ and fortunately, function $g$ can take all elements in $\reals$ as inputs).
Let's look at the expression $g(f(a))$. We know that $f(a) \in \reals$ and $g$ is the function that takes inputs from $\reals$
and is bounded above. Hence, there exists $B \in \reals$ such that for all $f(a) \in \reals$, $g(f(a)) \leq B$.
$\qed$
\end{quote}

\item[]

\item[(e)]
\begin{quote}
It is true. Suppose that $A$ is a set and that $f : A \rarr \reals$ and $g : A \rarr \reals$ are two functions
such that for every $x \in A$, we have $f(x) \leq g(x)$ and that $g$ is bounded above. Since $g$ is bounded
above, it means that there exists $B \in \reals$ such that for all $y \in A$, $g(y) \leq B$. On the other hand,
we know that for all $x \in A$, $f(x) \leq g(x)$. Hence, we have that for all $a \in A$, $f(x) \leq g(x) \leq B$.
Finally, we get that for all $a \in A$, $f(a) \leq B$ which means that $f$ is bounded.
\qed
\end{quote}

\item[]

\item[(f)]
\begin{quote}
It is false. Consider the function $f : \pints \rarr \ints : x \mapsto x + 1$.
Then $f$ is not onto since $-1 \notin image(f)$. On the other hand, it is not bounded
above. To see why, for the sake of contradiction, suppose that there exists $B \in \reals$
such that for all $z \in \pints$, $f(z) \leq B$. In other words, we have assumed
that for all $z \in Z$, $z + 1 \leq B$ which is false because we can set $z = \floor{B} + 1$
and we get that $z + 1 = \floor{B} + 1 > B$.
$\qed$
\end{quote}

\item[]

\item[(g)]
\begin{quote}
It is false. Consider the function $f : \pints \rarr \reals : x \mapsto -x$. Then $f$
is bounded above since we can take $B = 1$ and for all $z \in \pints$, $f(z) \leq B$ (due to the fact
that all the elements in the image of $f$ are negative).
Now, consider the function $\overline{f} : \pints \rarr \reals : x \mapsto -f(x)$.
We can then rewrite $\overline{f}$ in the following manner: $\overline{f} : \pints \rarr \reals : x \mapsto x$.
And now we have that $\overline{f}$ is not bounded above. To see why, suppose, for the sake of contradiction, that
there exists $B \in \reals$ such that for all $z \in \pints$, $\overline{f}(z) \leq B$.
In other words, for all $z \in \pints$, $z \leq B$. However, this is false since we can set $z = \floor{B} + 100000$
and we get that $\overline{f}(z) = z = \floor{B} + 100000 > B$.
\end{quote}
\item[]

\item[(h)]
\begin{quote}
It is false. Consider the function $f : \pints \rarr \reals : x \mapsto x + 0.25$.
Then we know that all $f$ is not bounded above. (Proof: suppose that there exists $B \in reals$
such that for all $z \in \pints$, $f(z) \leq B$. Then we assumed that for all $z \in \pints$, $z + 0.25 \leq B$.
This is, however, false since we can set $z = \floor{B} + 500$ and $f(z) = z = \floor{B} + 500 > B$).
However, $\pints \cap \mbox{\textbf{im}}(f)$ is not infinite but rather an empty set. That is because
all the elements in the image of $f$ will be of the type $integer + float$ which will always give us a $float$.
$\qed$
\end{quote}

\item[]

\item[(i)]
\begin{quote}
It is true. If $f : A \rarr \reals$ is onto, and that there exists $B \in reals$ such that for all $a \in A$, $f(a) \leq B$.
Now, since $f$ is onto, it means that its image has all the elements of $\reals$. Then the value $B + 123123$ will also be in
the image of $f$ and we get that for some $a \in A$, $f(a) = B + 123123$. And we have reached the contradiction since $f(a) = B + 123123 > B$.
$\qed$
\end{quote}
\end{itemize}

\item[]
\item[]
\item[]

\item[82.]
\begin{itemize}
\item[(a)]
It is \und{NOT} onto because the image of $g$ will contain only integers
and $5.5 \in \reals$ while $5.5 \notin image(g)$. It is, however, weakly onto
since for every $y \in \reals$, we can take $g(x) = g(y) = \floor{y}$ and we get
$\abs{g(x) - y} = \abs{\floor{y} - y} = \abs{y - \floor{y}}  < 1$ (note that $y - \floor{y}$ is the fraction
part of $y$ of which the absolute value is always less than 1).

\item[]

\item[(b)]
If $f : \reals \rarr \reals$ is onto, it means that the image of $f$ has all of the elements
of $\reals$. Then, since $f$ is onto, for every $y \in \reals$, we can find $x \in \reals$ such that $f(x) = y + 0.25$.
And finally, $\abs{f(x) - y} = \abs{y + 0.25 - y} = 0.25 < 1$.
$\qed$

\item[]

\item[(c)]
Yes. Suppose that $f : \reals \rarr \reals$ and $g : \reals \rarr \reals$
are two weakly onto function. Now, consider $f \circ g$ which is a composition
of $f$ and $g$. We then can rewrite $f \circ g$ as $f \circ g : \reals \rarr \reals : x \mapsto f(g(x))$.
Now, since $g$ has all of $\reals$ in its image, for all $y \in \reals$, there exists
$x \in \reals$ such that $f(g(x)) = x + 0.5$ and then $\abs{f(g(x)) - x} = x + 0.5 - x = 0.5 < 1$.
$\qed$

\item[]

\item[(d)]
Yes. Suppose that $f : \reals \rarr \reals$ is weakly onto. It then means that for all $y \in \reals$
there exists $x \in \reals$ such that $\abs{f(x) - y} < 1$. Consider $g = \dfrac{1}{2} f(x)$.
To show that $g(x)$ is weakly onto, we must show that for all $y \in \reals$, there exists $x \in \reals$
such that $\abs{g(x) - y} < 1$. For every $y \in \reals$, let $g(x) = \dfrac{1}{2}f(x)$ such that $\abs{f(x) - y} < 1$.
Then, for every $y \in \reals$, we have $\abs{g(x) - y} = \abs{\dfrac{1}{2}f(x) - y} < \abs{f(x) - y} < 1$ and $g(x)$ is weakly onto.
$\qed$

\item[]

\item[(e)]
\begin{quote}
$f(x) = x + \dfrac{1}{2 + x^2}$. It is a non-trivial $jiggle$ because $\abs{f(x) - x} = \abs{x + \frac{1}{2 + x^2} - x = \frac{1}{2 + x^2}} = \dfrac{1}{2 + x^2} < \dfrac{1}{1 + x^2}$.
\end{quote}

\item[]

\item[(f)]
It is true. Suppose we have a $jiggly$ function $f(x)$. Then we know that for every $x \in \reals$,
we have $\abs{f(x) - x} < \dfrac{1}{1 + x^2}$. Notice that $\dfrac{1}{1 + x^2} < 1$ for all $x \in \reals$.
Thus, we have the $jiggle$ function, then for all $x \in \reals$, $\abs{f(x) - x} < \dfrac{1}{1 + x^2} < 1$.
Now, for a $jiggle$ to also be onto, we need to show that for every $y \in \reals$, there exists $x \in reals$
such that $\abs{f(x) - y} < 1$. Now, for every $y \in \reals$, let $x = y$ and then
we get $\abs{f(x) - y} = \abs{f(y) - y} < \dfrac{1}{1 + x^2} < 1$.
$\qed$
\end{itemize}
\newpage
{\Large Reading: A Mathematician’s Apology by G. H. Hardy}
\begin{itemize}

\item[]

\item[59.]
For hardy, a mathematical idea is a serious mathematical theorem
that can, within itself, represent a whole class of different mathematical
theorems.

\item[]

\item[60.]
Hardy states: "For me, and I suppose for most mathematicians,
there is another reality, which I will call ‘mathematical reality’;
and there is no sort of agreement about the nature of mathematical
reality among either mathematicians or philosophers." Thus, for Hardy,
there is some other reality which is called mathematics.\\\\
I certainly do not think that mathematics is another reality. In fact,
I would never study it if I knew that it was useless. Mathematics has infinitely
many application to the real world and that is why it is important to me. Solving
problems via mathematics is what it is all about.

\item[]

\item[61.]
I am interested in the fundamental problems of humanity and the universe such as ``why do we exist?", 
``how was the universe created?", ``is there some superior being like God?'', ``can we artificially create a being like human? (if so, does it mean that we are Gods?)"
etc. the list goes on and on.
And I think that mathematics might be useful. That's the primary reason of my interest in math.

\item[]

\item[62.]
Hardy did not have much applications of number theory back in his time. He thought it was virtually useless.
If he was alive, he would probably smile or do something similar to it (seeing that he was wrong).

\item[]

\item[63.]
For mathematics.
\end{itemize}


\newpage


{\Large Bookwork}
\item[1.]
\begin{itemize}
\item[(a)]
A function $f$ such that $f(1) = 1$ and $f(2) = 1$ (thus, there is only one function).

\item[]

\item[(b)]
A function $f$ such that $f(1) = 1$, $f(2) = 1$, and $f(3) = 1$ (thus, there is only one function).

\item[]

\item[(c)]
A function $f$ such that $f(a_1) = b$, $f(a_2) = b$ ... $f(a_n) = b$ (thus, there is only one function).

\item[]

\item[(d)]
A function $f_1$ such that $f_1(1) = 1$\\
A function $f_2$ such that $f_2(1) = 2$\\\\
Thus, there are two functions in total.

\item[]

\item[(e)]
A function $f_1$ such that $f_1(1) = 1$\\
A function $f_2$ such that $f_2(1) = 2$\\
A function $f_3$ such that $f_3(3) = 2$\\\\
Thus, there are three functions in total.

\item[]

\item[(f)]
A function $f_1$ such that $f_1(a) = b_1$\\
A function $f_2$ such that $f_2(a) = b_2$\\
A function $f_3$ such that $f_3(a) = b_3$\\
... ... ... ... ... ... ... ... ... ... ... ...\\
A function $f_n$ such that $f_3(a) = b_n$\\\\
Thus, there are $n$ functions in total.
\end{itemize}

\item[]
\item[]

\item[2.]
\begin{itemize}
\item[(a)]
It is bijective.\\
Explanation: If for some $x_1, x_2$, $f(x_1) = f(x_2)$, then $2x_1 = 2x_2$ and $x_1 = x_2$ and thus, it is injective.\\
If we have $r \in \reals$, then we can take $x = \dfrac{r}{2}$ and we get $f(x) = 2 \times \dfrac{r}{2} = r$ and hence, it is surjective.

\item[]

\item[(b)]
It is bijective.\\
Explanation: If for some $x_1, x_2$, $f(x_1) = f(x_2)$, then $3 - x_1 = 3 - x_2$ and $x_1 = x_2$ and thus, it is injective.\\
If we have $r \in \reals$, then we can take $x = 3 - r$ and we get $f(x) = 3 - 3 + r = r$ and hence, it is surjective.
\item[]

\item[(c)]
It is neither injective nor surjective.
Explanation: If for some $x_1, x_2$, $f(x_1) = f(x_2)$, then $3 - x_1 = 3 - x_2$ and $x_1 = x_2$ and thus, it is injective.\\
Consider the $r = 3 \in \reals$ which is in the image of $f$. Then, we have that $x^2 + 2x + 3 = 3$ and $x = -2, 0$ are two legitimate
solutions to the equations. Thus $f$ is not injective.\\
On the other hand, consider $r = 0 \in \reals$. Then $x^2 + 2x + 3 = 0$ where $\mathcal{D} = 4 - 12 = -8$ and since the discriminant is negative,
the equation has no solutions.
\item[]

\item[(d)]
It is surjective but \und{NOT} injective.\\
Explanation: $\mbox{sin}(\dfrac{\pi}{4}) = \mbox{sin}(\dfrac{3\pi}{4})$ thus it is not injective.\\
Domain $[0, \pi)$ covers everything in the interval $[0, 1]$ thus, it is surjective.

\item[]

\item[(e)]
It is neither injective nor surjective.\\
Explanation: It is not injective because $f(1) = f(-1) = e^1 = e$.\\
It is not surjective because $\dfrac{1}{e} \in \reals$ but $f(x) = \dfrac{1}{e}$
has no solutions (because $x^2 \geq 0$).
\end{itemize}

\item[]
\item[]

\item[5.]
\begin{itemize}
\item[(a)]
$g(-5) = 10 - 1 = 9$\\
$g(-4) = 8 - 1 = 7$\\
$g(-3) = 6 - 1 = 5$\\
$g(-2) = 4 - 1 = 3$\\
$g(-1) = 2 - 1 = 1$\\
$g(0) = 2 \times 0 = 0$\\
$g(1) = 2 \times 1 = 2$\\
$g(2) = 2 \times 2 = 4$\\
$g(3) = 2 \times 3 = 6$\\
$g(4) = 2 \times 4 = 8$\\
$g(5) = 2 \times 5 = 10$

\item[]

\item[(b)]
$g$ is a function which has a set of integers as a domain and the set of natural numbers as a codomain and if the input
is negative, it outputs minus twice the input minus one and in all other cases, twice the input.

\item[]

\item[(c)]
For the arbitrary $x \in \ints$, let's consider the following two cases:\\\\
Case I: $x < 0$\\
Case I: $x \geq 0$\\\\
If $x < 0$, then $g(x) = -2x - 1$ which is an integer because $x \in \ints$.\\
If $x \geq 0$, then $g(x) = 2x$ which is an integer because $x \in \ints$.
$\qed$

\item[]

\item[(d)]
Suppose that for some $x_1, x_2$, $g(x_1) = g(x_2)$. Then we have two cases to consider:\\\\
Case I: $x_1 \geq 0$ and $x_2 \geq 0$
Case II: $x_1 \geq 0$ and $x_2 \geq 0$\\\\
If $x_1, x_2 \geq 0$, then $g(x_1) = g(x_2)$ is equivalent to saying that $2x_1 = 2x_2$ and we get that $x_1 = x_2$.
If $x_1, x_2 < 0$, then $g(x_1) = g(x_2)$ is equivalent to saying that $-2x_1 -1 = -2x_2 - 1$ and we get that $x_1 = x_2$.\\\\
Thus, $g$ is injective.

\item[]

\item[(e)]
Let $y \in N$. Then if $y$ is even, let $x = \dfrac{y}{2}$ and if $y$ is odd, let $x = \dfrac{1 - y}{2}$.
In both cases $f(x) = y$. Thus $g$ is surjective.

\item[]

\item[(f)]
$g^{-1} : N \rarr \ints = \begin{cases} \dfrac{x}{2} & \mbox{if } x \mbox{ is even} \\\\ \dfrac{1 - x}{2} & \mbox{if } x \mbox{ is odd.} \end{cases}$
\end{itemize}

\item[]
\item[]

\item[6.]
\begin{itemize}
\item[(a)]
Suppose, for some $x_1, x_2 \in \reals$, $f(x_1) = f(x_2)$. Then we have that
$ax_1 + b = ax_2 + b$ where $a \neq 0$. After simplifying the expression, we get that $ax_1 = ax_2$
and finally, since $a \neq 0$, $x_1 = x_2$. Thus, $f$ is injective.

\item[]

\item[(b)]
$Ran(f) = \reals$ since any real number is in the image of the function.

\item[]

\item[(c)]
We should just solve an equation $x = f^{-1}(x)a + b$ for $f^{-1}(x)$.\\
We get that $f^{-1}x = \dfrac{x - b}{a}$.
\end{itemize}

\item[]
\item[]

\item[12.]
\begin{itemize}
\item[(b)]
This is a combinatorics question. The answer is the number of permutations that is $3! = 6$.
\end{itemize}

\item[]
\item[]

\item[19.]
\begin{itemize}
\item[(c)]
Generalized function in (a): $f : \reals - \{0\} \rarr \reals : x \mapsto \dfrac{1}{x}$. The function is still bijective.\\
Explanation: it is injective because if $f(x_1) = f(x_2)$ for some $x_1, x_2 \in \reals - \{0\}$, then we have $\dfrac{1}{x_1} = \dfrac{1}{x_2}$
and since $x_1, x_2 \neq 0$, we have $x_1 = x_2$.\\
It is also surjective because for arbitrary $y \in \reals$, we can take $x = \dfrac{1}{y}$ and we get that $f(x) = \dfrac{1}{\dfrac{1}{y}} = y$.\\
Thus, the generalized function $f$ is also bijective.\\\\\\
Generalized function in (b): $f : \reals - \{0\} \rarr \preals : x \mapsto \dfrac{1}{x}$. This function is surjective but \und{NOT} injective.\\
Explanation: it is not injective because if $f(x_1) = f(x_2)$ for some $x_1, x_2 \in \reals - \{0\}$, then we have $\dfrac{1}{x_1^2} = \dfrac{1}{x_2^2}$
and it leads to two solutions, $x_1 = x_2$ and $x_1 = -x_2$.
It is surjective because for arbitrary $y \in \preals$, we can take $x = \dfrac{1}{\sqrt{y}}$ and we get that $f(x) = \dfrac{1}{(\dfrac{1}{\sqrt{y}})^2} = y$.\\
Thus, the generalized function $f$ is neither injective nor surjective.
\end{itemize}

\item[]
\item[]

\item[20.]
I will denote floor and ceiling operations as $floor$ and $ceil$.\\\\
\und{Proof that floor function is onto but not one-to-one.}\\
The floor function will be $f : \reals \rarr \ints : x -> floor(x)$. Then we know that $floor(2.2) = floor(2.3) = 2$
thus, it is not one-to-one. On the other hand, for an arbitrary $y \in ints$, we can let $x = y + 0.1$ and get that $floor(x) = floor(y + 0.1) = y$.
Thus, the floor function is onto but not one-to-one.\\\\\\
\und{Proof that ceiling function is onto but not one-to-one.}\\
The floor function will be $f : \reals \rarr \ints : x -> ceil(x)$. Then we know that $floor(26.7) = floor(26.8) = 3$
thus, it is not one-to-one. On the other hand, for an arbitrary $y \in \ints$, we can let $x = y - 0.1$ and get that $ceil(x) = floor(y - 0.1) = y$.
Thus, the ceiling function is onto but not one-to-one.

\item[]
\item[]

\item[25.]
\begin{itemize}
\item[(a)]
Consider the function $F : \mathcal{P}(A) \rarr \mathcal{P}(B) : x \mapsto f(x)$.
Thus, we can use $f$ to define a function $F$.
\end{itemize}

\item[]
\item[]

\item[26.]
Suppose for two elements $z_1, z_2 \in \ints$, we have that $f(z_1) = f(z_2)$.
Then, it means that the sets of divisors of $z_1$ and $z_2$ are the same. This means that $z_1 = z_2$.
To see why, consider $f(z_1)$ and $f(z_2)$. Let's now consider the biggest elements in $f(z_1)$ and $f(z_2)$.
It is obvious that the biggest element in $f(z_1)$ will be $z_1$ and in the $f(z_2)$, it will be $z_2$.
Now, since the sets are equal, it must be the case that the biggest elements of $f(z_1)$ and $f(z_2)$ are also
equal and we have that $z_1 = z_2$.
$\qed$

\item[]
\item[]

\item[29.]
Notice that $m - n + m + n = 2m$ and $m - n - m - n = -2n$. Thus $m + n$
and $m - n$ are either both odd or both even. Then the range of the function $F$
will be $\{(x, y) \ | \ x, y \mbox{ mod } 2 \in \{0, 1\} \}$\\\\
\end{itemize}
\end{document}
