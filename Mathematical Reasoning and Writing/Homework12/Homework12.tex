\documentclass[12pt, a4paper]{article}
\usepackage[a4paper, margin=1in]{geometry}


\usepackage{adjustbox}
\usepackage{mathtools}
\usepackage{amsmath}
\usepackage{amssymb}
\usepackage{amsthm}

\usepackage{pgfplots}
\usepackage{listings}
\usepackage{color}
\usepackage{tikz}

\usepackage{textcomp}
\usepackage{soul}

\usepackage[hidelinks]{hyperref}
\usepgfplotslibrary{external,fillbetween}
\pgfplotsset{compat=1.14}
\usepackage[makeroom]{cancel}
\title{\bf{Homework \textnumero 11}}
\author{Author: David Oniani
\\
\ \ \ Instructor: Tommy Occhipinti}
\date{November 13, 2018}

\usepackage{listings}
\usepackage{color}

%%%%%%%%%%%%%%% S E T S %%%%%%%%%%%%%%%
\newcommand{\nats}{\mathbb{N}}
\newcommand{\ints}{\mathbb{Z}}
\newcommand{\rats}{\mathbb{Q}}
\newcommand{\reals}{\mathbb{R}}
\newcommand{\irrats}{\mathbb{I}}

\newcommand{\pnats}{\mathbb{N}^+}
\newcommand{\pints}{\mathbb{Z}^+}
\newcommand{\prats}{\mathbb{Q}^+}
\newcommand{\preals}{\mathbb{R}^+}
\newcommand{\nreals}{\mathbb{R}^-}

\newcommand{\nints}{\mathbb{Z}^-}
\newcommand{\nrats}{\mathbb{Q}^-}
%%%%%%%%%%%%%%%%%%%%%%%%%%%%%%%%%%%%%%%

% Calligraphy
\newcommand\und[1]{\underline{\smash{#1}}}

% Operators
\DeclarePairedDelimiter\abs{\lvert}{\rvert}
\DeclarePairedDelimiter\ceil{\lceil}{\rceil}
\DeclarePairedDelimiter\floor{\lfloor}{\rfloor}

% Other
\newcommand{\rarr}{\rightarrow}

\definecolor{dkgreen}{rgb}{0,0.6,0}
\definecolor{gray}{rgb}{0.5,0.5,0.5}
\definecolor{mauve}{rgb}{0.58,0,0.82}
\definecolor{backcolour}{rgb}{0.95,0.95,0.92}

\lstset{
backgroundcolor=\color{backcolour},
aboveskip=3mm,
belowskip=3mm,
showstringspaces=false,
columns=flexible,
basicstyle={\small\ttfamily},
numbers=left,
numberstyle=\normalsize\color{gray},
keywordstyle=\color{blue},
commentstyle=\color{dkgreen},
stringstyle=\color{mauve},
breaklines=true,
breakatwhitespace=true,
tabsize=4
}


\begin{document}
\maketitle
\begin{itemize}
\item[87.]
To prove that a relation is an equivalence relation, we must show that the
relation is reflexive, symmetric, and transitive.\\\\
I. Showing that $\sim$ is reflexive.\\\\
$x \sim x = x + 2x = 3x$ Thus, $\sim$ is reflexive.\\\\\\
II. Showing that $\sim$ is symmetric.\\\\
Suppose that $x \sim y$, then $x + 2y = 3k \mbox{ where } k \in \ints$. Then, if we solve the
equation for $x$, we get $x = 3k - 2y$. Now, consider $y + 2x$. Let's substitute $x$ with $3k - 2y$.
We get, $y + 6k - 4y = 6k - 3y = 3 \times (2k - y)$. Hence, if $x + 2y$ is divisible by 3, then $y + 2x$
is also divisible by 3 and the relation is symmetric.\\\\\\
III. Showing that $\sim$ is transitive.\\\\
Suppose that $x \sim y$ and $y \sim z$. Then $x + 2y = 3k \mbox{ where } k \in \ints$ and $y + 2z = 3l \mbox{ where } l \in \ints$.
Consider the relation on the variables $x$ and $z$. The relation is $x \sim z = x + 2z$. Now, from the first equation, let's substitute $x$
and from the second one, substitute $z$. We get that $x = 3k - 2y$ and $z = \dfrac{3l - y}{2}$. Finally, we get:
$x + 2z = 3k - 2y + 2 \times \dfrac{3l - y}{2} = 3k - 2y + 3l - y = 3k + 3l - 3y = 3 \times (k + l - 1)$ and $3 \times (k + l - 1)$
is clearly a multiple of 3. Hence, we got that $\sim$ is transitive.
\\\\\\
Now, we proved that the relation $\sim$ is reflexive, symmetric, and transitive and thus, the relation $\sim$
is the equivalence relation.
$\qed$

\newpage

\item[88.]
\begin{itemize}
\item[(a)]
$\Xi(S)$ is a relation on $\mathcal{P}(S)$ such that it takes the powerset of $S$
\item[]

\item[(b)]
It is not symmetric. Let $A = $
\item[]

\item[(c)]

\item[]

\item[(d)]
\end{itemize}

\newpage

{\Large Bookwork}
\item[]
\item[]

{\Large \textbf{\und{4.2}}}

\item[]

\item[2.]
$\in, \ \ \notin, \ \ \subset, \ \ =, \ \ \mathcal{P}$

\item[]
\item[]

\item[4.]
\begin{itemize}
\item[(a)]
It is not. Consider tuples $(a, b)$ and $(b, a)$. For the relation $R$
to be transitive, since $(a, b) \in R$ and $(b, a) \in R$, it must be the
case that $(a, a) \in R$. However, $(a, a) \notin R$. Thus, the relation is not transitive.
$\qed$

\item[]

\item[(b)]
\begin{itemize}
\item[i.]
It is. Suppose that $x - y = q_1$ and $y - z = q_2$ where $q_1, q_2 \in \rats$.
Let's then sum those two equations up, and we get $x - y + y - z = q_1 + q_ 2$
and finally $x - z = q_1 + q_2$. Hence, we got that $x - z$ is a sum of two rational
numbers and thus is rational itself. Therefore, the relation $R$ is transitive.
$\qed$

\item[]

\item[ii.]
It is not. Consider $x = \sqrt{2}, \ \ y = 1 \mbox{, and } z = \sqrt{2} + 1$.
Then $x - y = \sqrt{2} - 1$ thus is irrational and $y - z = 1 - (\sqrt{2} + 1) = -\sqrt{2}$
hence, is also irrational. However, $x - z = \sqrt{2} - (\sqrt{2} + 1) = \sqrt{2} - \sqrt{2} - 1 = -1$
which is rational. Therefore, the relation $R$ is not transitive.
$\qed$

\item[]

\item[iii.]
It is not. Consider $x = 1, \ \ y = 2 \mbox{, and } z = 4$. Then $\abs{x - y} = 1$
and $\abs{y - z} = 2$. However, $\abs{x - z} = 3 > 2$. Hence, the relation $R$ is not transitive.
$\qed$
\end{itemize}
\end{itemize}

\item[]
\item[]

\item[12.]
\begin{itemize}
\item[(a)]
It is reflexive. Suppose that $x \in R \cap S$. Then $x \in R$ and $x \in S$. Since $R$ and $S$
are reflexive, $(x, x) \in R$ and $(x, x) \in \ S$. Therefore, $(x,x) \in R \cap S$.
$\qed$

\item[]

\item[(b)]
It is reflexive. Suppose that $x \in R \cup S$. Then, without a loss of generality, let $x \in S$.
Now, since $S$ is reflexive, $(x, x) \in S$ and thus, $(x, x) \in R \cup S$.
$\qed$

\item[]

\item[(e)]
It is transitive. Suppose that $(x, y), (y, z) \in R \cap S$. Then $(x, y), (y, z) \in R$
and $(x, y), (y, z) \in S$. Since $R, S$ are transitive, $(x, z) \in R$ and $(x, z) \in S$.
Finally, $(x, z) \in R \cap S$ and $R \cap S$ is transitive.
$\qed$

\item[]

\item[(f)]
It is transitive. Suppose that $(x, y), (y, z) \in R \cup S$. Then, without a loss of generality, $(x, y), (y, z) \in R$.
Since $R$ is transitive, $(x, z) \in R$ and $(x, z) \in R \cup S$. Therefore, $R \cup S$ is transitive.
$\qed$
\end{itemize}

\item[]
\item[]
\item[]

{\Large \textbf{\und{4.4}}}

\item[]

\item[1.]
\begin{itemize}
\item[(a)]
Yes, it is an equivalence relation since it satisfies all the criteria: reflexive, transitive, symmetric.\\\\
1. It is reflexive, since $(a, a), \ (b, b), \ (c, c) \in R$.\\\\
2. It is transitive. $(a, a), (a,c) \in R$ and $(a, c) \in R$; $(c, a), (a, a) \in R$ and $(c, a) \in R$;
$(c, c), (c, a) \in R$ and $(c, a) \in R$; $(a, c), (c, c) \in R$ and $(a, c) \in R$.\\\\
3. It is symmetric. $(a, a) \in R \mbox{ and } (a, a) \in R$; $(b, b) \in R \mbox{ and } (b, b) \in R$;
$(c, c) \in R \mbox{ and } (c, c) \in R$; $(a, c) \in R \mbox{ and } (c, a) \in R$; $(c, a) \in R \mbox{ and } (a, c) \in R$.

\item[]

\item[(b)]
No, it is not since $(b, a), (a, c) \in R$ but $(b, c) \notin R$ (it is not transitive).
\end{itemize}

\item[]
\item[]

\item[3.]
It is not. For a relation to be the equivalence relation, it must be reflexive, transitive, symmetric.
It is not transitive since if we have three lines $x, y, z$ in the euclidean space and if $x \perp y$ and $y \perp z$,
then $y \not\perp z$ (because $y \parallel z$). As a side note, it is not reflexive either since the line cannot be perpendicular to itself.

\item[]
\item[]

\item[9]
\begin{itemize}
\item[(a)]
Equivalence relation $R$ such that $x R y$ if $x, y \leq 0$ or $x, y > 0$.
\item[]

\item[(b)]
Equivalence relation $R$ such that $x R y = 0$ if $x, y < 0$, $x, y = 0$ or $x, y > 0$.
\end{itemize}

\item[]
\item[]

\item[13.]
To show that $\equiv_2$ is an equivalence relation, we must show that it is reflexive, symmetric, and transitive.\\\\
1. It is reflexive because if $(a, b) \equiv_2 (a, b)$, then $a - a = 0$ is even and $b - b = 0$ is also even.\\
2. It is symmetric because if $(a, b) \equiv_2 (c, d)$, then $a - c = 2k$ is even and $b - d = 2l$ where $k, l \in \ints$.
Hence, $(c, d) \equiv_2 (a, b)$ because $c - a = 2 \times (-k)$ and $d - b = 2 \times (-l)$\\
3. It is transitive because if $(a, b) \equiv_2 (c, d)$ and $(c, d) \equiv_2 (e, f)$, it means that
$a - c = 2k,\ b - d = 2l, \ c - e = 2m \ \mbox{and} \ d - f = 2n \ \mbox{where} \ k,l,m,n \in \ints$.
Then, we have that $a - e = (a - c) + (c - e) = 2k + 2m = 2 \times (k + m)$. On the other hand,
$b - f = (b - d) + (d - f) = 2l + 2n = 2 \times (l + n)$. Hence, we got that if $(a, b) \equiv_2 (c, d)$ and $(c, d) \equiv_2 (e, f)$,
then $(a, b) \equiv_2 (e, f)$ and therefore, the relation is transitive.\\\\
Thus, we have now proven that the relation $\equiv_2$ is reflexive, symmetric, and transitive thus, $\equiv_2$ is an equivalence relation.\\\\
A partition of $\equiv_2$ is $\{(a, b), (c, d) \in \ints \times \ints \ | \ (a, c \mbox{ mod } 2 = 0 \mbox{ or } a, c \mbox{ mod } 2 = 1) \mbox{ and }\\ (b, d \mbox{ mod } 2 = 0 \mbox{ or } b, d \mbox{ mod } 2 = 1)\}$.

\item[]
\item[]

\item[23.]
$\{(1, 2), (2, 1), (3, 3), (4, 5), (5, 4)\}$

\item[]
\item[]

\item[30.]
To prove that the relation is the equivalence relation, we must prove that it is reflexive, symmetric, and transitive.\\\\
It is reflexive because if $f \sim f$, then $f' = f'$ which is true since given the particular value, the function and then its derivative is the same (the function cannot have more than one output for a single input).\\\\
It is symmetric because if $f \sim g$, it means that $f' = g'$ and since we are dealing with values, we know that if $x = y$, then $y = x$ and the same applied for the derivatives here. Hence, if $f' = g'$, then $g' = f'$.
Finally, we got that if $f \sim g$, then $g \sim f$ and the function is symmetric.\\\\
It is transitive because if $f' \sim g'$ and $g' \sim h'$, then $f' = g'$ and $g' = h'$ and thus $f' = g' = h'$ and $f' = h'$.
Hence, we got that if $f' \sim g'$ and $g' \sim h'$, then $f' \sim h'$ and the relation is transitive.\\\\
Thus, we have proven that the relation is reflexive, symmetric, and transitive and hence, is the equivalence relation.

\item[]
\item[]

\item[31.]
\begin{itemize}
\item[(a)]
To show that the relation is the equivalence relation, we must prove that it is reflexive, symmetric, and transitive.\\\\
It is reflexive because if $A_i \sim A_i$, then $\abs{A_i} = \abs{A_i}$ and obviously, $\abs{A_i} = \abs{A_i}$ because the size of the set is constant.
Hence, we have shown that the relation is reflexive.\\\\
It is symmetric because if $A_1 \sim A_2$, it means that $\abs{A_1} = \abs{A_2}$ and thus $\abs{A_2} = \abs{A_1}$.
Hence, it is symmetric.\\\\
It is transitive because if $A_1 \sim A_2$ and $A_2 \sim A_3$, it means that $\abs{A_1} = \abs{A_2}$ and $\abs{A_2} = \abs{A_3}$.
Therefore, $\abs{A_1} = \abs{A_2} = \abs{A_3}$ and we get that $\abs{A_1} = \abs {A_3}$. Thus, we have proven that if $A_1 \sim A_2$
and $A_2 \sim A_3$, then $A_1 \sim A_3$ and the relation is transitive.\\\\
At this point, we have shown that the relation is all three: reflexive, symmetric, and transitive and thus, is the equivalence relation.

\item[]

\item[(b)]

\end{itemize}
\end{itemize}
\end{document}
