%%%%%%%%%%%%%%%%%%%%%%%%%%%%%%%%%%%%%%%%%%%%%%%%%%%%%%%%%%%%%%%%%%%%%%%%%%%%%%%
%
% Filename: template.tex
% Author:   David Oniani
% Modified: January 24, 2021
%  _         _____   __  __
% | |    __ |_   _|__\ \/ /
% | |   / _` || |/ _ \\  /
% | |__| (_| || |  __//  \
% |_____\__,_||_|\___/_/\_\
%
%%%%%%%%%%%%%%%%%%%%%%%%%%%%%%%%%%%%%%%%%%%%%%%%%%%%%%%%%%%%%%%%%%%%%%%%%%%%%%%

%%%%%%%%%%%%%%%%%%%%%%%%%%%%%%%%%%%%%%%%%%%%%%%%%%%%%%%%%%%%%%%%%%%%%%%%%%%%%%%
% Document Definition
%%%%%%%%%%%%%%%%%%%%%%%%%%%%%%%%%%%%%%%%%%%%%%%%%%%%%%%%%%%%%%%%%%%%%%%%%%%%%%%

\documentclass[11pt]{article}

%%%%%%%%%%%%%%%%%%%%%%%%%%%%%%%%%%%%%%%%%%%%%%%%%%%%%%%%%%%%%%%%%%%%%%%%%%%%%%%
% Packages and Related Settings
%%%%%%%%%%%%%%%%%%%%%%%%%%%%%%%%%%%%%%%%%%%%%%%%%%%%%%%%%%%%%%%%%%%%%%%%%%%%%%%

% Global, document-wide settings
\usepackage[margin=1in]{geometry}
\usepackage[utf8]{inputenc}
\usepackage[english]{babel}

% Other packages
\usepackage{booktabs}
\usepackage{hyperref}
\usepackage{mathtools}
\usepackage{amsthm}
\usepackage{amssymb}
\usepackage{tikz}
\usepackage[cache=false]{minted}

%%%%%%%%%%%%%%%%%%%%%%%%%%%%%%%%%%%%%%%%%%%%%%%%%%%%%%%%%%%%%%%%%%%%%%%%%%%%%%%
% Command Definitions and Redefinitions
%%%%%%%%%%%%%%%%%%%%%%%%%%%%%%%%%%%%%%%%%%%%%%%%%%%%%%%%%%%%%%%%%%%%%%%%%%%%%%%

% Nice-looking underline
\newcommand\und[1]{\underline{\smash{#1}}}

% Line spacing is 1.5
\renewcommand{\baselinestretch}{1.5}

% Absolute value
\DeclarePairedDelimiter\abs{\lvert}{\rvert}%

% Absolute value big
\DeclarePairedDelimiter\absb{\Big\lvert}{\Big\rvert}%

% Ceiling
\DeclarePairedDelimiter{\ceil}{\lceil}{\rceil}

% Ceiling big
\DeclarePairedDelimiter{\ceilb}{\Big\lceil}{\Big\rceil}

% Floor
\DeclarePairedDelimiter\floor{\lfloor}{\rfloor}

% % Naturals, Reals, Integers, and Rationals, 
\newcommand{\nats}{\mathbb{N}}
\newcommand{\reals}{\mathbb{R}}
\newcommand{\preals}{\mathbb{R^+}}
\newcommand{\nreals}{\mathbb{R^-}}
\newcommand{\ints}{\mathbb{Z}}
\newcommand{\pints}{\mathbb{Z^+}}
\newcommand{\nints}{\mathbb{Z^-}}
\newcommand{\rats}{\mathbb{Q}}
\newcommand{\prats}{\mathbb{Q^+}}
\newcommand{\nrats}{\mathbb{Q^-}}
\newcommand{\irrats}{\mathbb{I}}
\newcommand{\pirrats}{\mathbb{I^+}}
\newcommand{\nirrats}{\mathbb{I^-}}

%%%%%%%%%%%%%%%%%%%%%%%%%%%%%%%%%%%%%%%%%%%%%%%%%%%%%%%%%%%%%%%%%%%%%%%%%%%%%%%
% Miscellaneous
%%%%%%%%%%%%%%%%%%%%%%%%%%%%%%%%%%%%%%%%%%%%%%%%%%%%%%%%%%%%%%%%%%%%%%%%%%%%%%%

% Setting stuff
\setlength{\parindent}{0pt}  % Remove indentations from paragraphs

% PDF information and nice-looking urls
\hypersetup{%
  pdfauthor={David Oniani},
  pdftitle={Real Analysis},
  pdfsubject={Mathematics, Real Analysis, Real Numbers},
  pdfkeywords={Mathematics, Real Analysis, Real Numbers},
  pdflang={English},
  colorlinks=true,
  linkcolor={black!50!blue},
  citecolor={black!50!blue},
  urlcolor={black!50!blue}
}

%%%%%%%%%%%%%%%%%%%%%%%%%%%%%%%%%%%%%%%%%%%%%%%%%%%%%%%%%%%%%%%%%%%%%%%%%%%%%%%
% Author(s), Title, and Date
%%%%%%%%%%%%%%%%%%%%%%%%%%%%%%%%%%%%%%%%%%%%%%%%%%%%%%%%%%%%%%%%%%%%%%%%%%%%%%%

% Author(s)
\author{David Oniani\\
        Luther College\\
        \href{mailto:oniada01@luther.edu}{oniada01@luther.edu}}

% Title
\title{\rule{\paperwidth - 150pt}{1pt}\textbf{\\\textit{Real Analysis
Exams}\\}\rule {\paperwidth - 150pt}{1pt}\\\textbf{Exam
\textnumero4}\\{\normalsize Instructor: Dr. Eric Westlund}}

% Date
\date{\today}

%%%%%%%%%%%%%%%%%%%%%%%%%%%%%%%%%%%%%%%%%%%%%%%%%%%%%%%%%%%%%%%%%%%%%%%%%%%%%%%
% Beginning of the Document
%%%%%%%%%%%%%%%%%%%%%%%%%%%%%%%%%%%%%%%%%%%%%%%%%%%%%%%%%%%%%%%%%%%%%%%%%%%%%%%

\begin{document}
\maketitle

%%%%%%%%%%%%%%%%%%%%%%%%%%%%%%%%%%%%%%%%%%%%%%%%%%%%%%%%%%%%%%%%%%%%%%%%%%%%%%%

\begin{itemize}
    \item[1.]
        \begin{itemize}
            \item[(a)]
                Placeholder.

            \item[(b)]
                Placeholder.

            \item[(c)]
                Placeholder.
        \end{itemize}

    \item[2.]
        Let $\epsilon > 0$ be given, $x_0$ be fixed, and let $\delta = \min
        \Big(1, \dfrac{\epsilon}{5(1 + 2\abs{x_0})}\Big)$. Then for $\abs{x -
        x_0} < \delta$ we have:
        \begin{align*}
            \abs{f(x) - f(x_0)} &= \abs{5x^2 + 3 - 5x_0^2 - 3}\\
                                &= \abs{5 (x^2 - x_0^2)}\\
                                &= 5 \abs{x - x_0} \abs{x - x_0 + 2x_0}\\
                                &< 5 \delta (\abs{x - x_0} + \abs{2x_0})\\
                                &< 5 \delta (\delta + 2\abs{x_0})\\
                                &\leq 5 \frac{\epsilon}{5(1 + 2\abs{x_0})} (\delta + 2\abs{x_0})\\
                                &\leq \frac{\epsilon}{1 + 2\abs{x_0}} (1 + 2\abs{x_0})\\
                                &= \epsilon
                                \tag{Thus, $\abs{f(x) - f(x_0)} < \epsilon$}
        \end{align*}
        Hence, $f(x) = 5x^2 + 3$ is continuous at each point $x_0 \in
        \reals$.\\
        $\qed$

    \item[3.]
        \begin{itemize}
            \item[(a)]
                Placeholder.

            \item[(b)]
                Placeholder.

            \item[(c)]
                Placeholder.
        \end{itemize}

    \item[4.]
        Placeholder.

    \item[5.]
        Placeholder.

    \item[6.]
        Placeholder.

    \item[7.]
        Since $f$ is continuous on $[a, b]$, it follows that $f$ achieves both
        the absolute maximum $M$ and the absolute minimum $m$ in $[a, b]$.
        Suppose, without a loss of generality, that these points are $c_1$ and
        $c_2$ with $c_1 < c_2$. We then have:
        \begin{align}
            m (b - a) \leq \int_a^b f \leq M (b - a)\\
            m \leq \frac{1}{b - a} \int_a^b f \leq M
        \end{align}
        Now, it follows by \textbf{Intermediate Value Theorem} that $\exists c
        \in (c_1, c_2) \subset [a, b]$ s.t. $f(c) = \frac{1}{b - a} \int_a^b
        f$.\\
        $\qed$

    \item[8.]
        Suppose, for the sake of contradiction, that $f$ has Generalized
        Riemann integrals $Q_1$ and $Q_2$ with $Q_1 \neq Q_2$ and let $\epsilon
        > 0$ be given. Then,.it follows that $\exists \delta_1(x)$ s.t.
        $\forall \delta_1(x)$-fine tagged partitions, $\abs{R(f, P) - Q_1} <
        \frac{\epsilon}{2}$. Similarly, $\exists \delta_2(x)$ s.t. $\forall
        \delta_2(x)$-fine tagged partitions, $\abs{R(f, P) - Q_2} <
        \frac{\epsilon}{2}$. Now, let $\delta(x) = \min (\delta_1(x),
        \delta_2(x))$. It follows by \textbf{Theorem 8.1.5} that there exists a
        tagged partition $(P, \{c_k\})$ s.t. it is both $\delta_1(x)$-fine and
        $\delta_2(x)$-fine. We have:
        \begin{equation*}
            \abs{Q_1 - Q_2} \leq \abs{Q_1 - R(f, P)} + \abs{R(f, P) - Q_2}
                            < \frac{\epsilon}{2} + \frac{\epsilon}{2}
                            = \epsilon
        \end{equation*}
        Hence, we got that $Q_1 = Q_2$ and we face a contradiction since we
        have assumed that $Q_1 \neq Q_2$. Finally, we conclude that if $f$ has
        a generalized Riemann integral on $[a, b]$, then the value of the
        integral $\int_a^b f$ is unique.\\
        $\qed$

    \item[9.]
        \begin{itemize}
            \item[(a)]
                Placeholder.

            \item[(b)]
                Placeholder.

            \item[(c)]
                Placeholder.

            \item[(d)]
                Placeholder.

            \item[(e)]
                Placeholder.

            \item[(f)]
                Placeholder.
        \end{itemize}
\end{itemize}

%%%%%%%%%%%%%%%%%%%%%%%%%%%%%%%%%%%%%%%%%%%%%%%%%%%%%%%%%%%%%%%%%%%%%%%%%%%%%%%
% The End of the Document
%%%%%%%%%%%%%%%%%%%%%%%%%%%%%%%%%%%%%%%%%%%%%%%%%%%%%%%%%%%%%%%%%%%%%%%%%%%%%%%

\end{document}
