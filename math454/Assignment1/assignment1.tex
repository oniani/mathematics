%%%%%%%%%%%%%%%%%%%%%%%%%%%%%%%%%%%%%%%%%%%%%%%%%%%%%%%%%%%%%%%%%%%%%%%%%%%%%%%
%
% Filename: template.tex
% Author:   David Oniani
% Modified: November 30, 2020
%  _         _____   __  __
% | |    __ |_   _|__\ \/ /
% | |   / _` || |/ _ \\  /
% | |__| (_| || |  __//  \
% |_____\__,_||_|\___/_/\_\
%
%%%%%%%%%%%%%%%%%%%%%%%%%%%%%%%%%%%%%%%%%%%%%%%%%%%%%%%%%%%%%%%%%%%%%%%%%%%%%%%

%%%%%%%%%%%%%%%%%%%%%%%%%%%%%%%%%%%%%%%%%%%%%%%%%%%%%%%%%%%%%%%%%%%%%%%%%%%%%%%
% Document Definition
%%%%%%%%%%%%%%%%%%%%%%%%%%%%%%%%%%%%%%%%%%%%%%%%%%%%%%%%%%%%%%%%%%%%%%%%%%%%%%%

\documentclass[11pt]{article}

%%%%%%%%%%%%%%%%%%%%%%%%%%%%%%%%%%%%%%%%%%%%%%%%%%%%%%%%%%%%%%%%%%%%%%%%%%%%%%%
% Packages and Related Settings
%%%%%%%%%%%%%%%%%%%%%%%%%%%%%%%%%%%%%%%%%%%%%%%%%%%%%%%%%%%%%%%%%%%%%%%%%%%%%%%

% Global, document-wide settings
\usepackage[margin=1in]{geometry}
\usepackage[utf8]{inputenc}
\usepackage[english]{babel}

% Other packages
\usepackage{booktabs}
\usepackage{hyperref}
\usepackage{mathtools}
\usepackage{amsthm}
\usepackage{amssymb}
\usepackage[cache=false]{minted}

%%%%%%%%%%%%%%%%%%%%%%%%%%%%%%%%%%%%%%%%%%%%%%%%%%%%%%%%%%%%%%%%%%%%%%%%%%%%%%%
% Command Definitions and Redefinitions
%%%%%%%%%%%%%%%%%%%%%%%%%%%%%%%%%%%%%%%%%%%%%%%%%%%%%%%%%%%%%%%%%%%%%%%%%%%%%%%

% Nice-looking underline
\newcommand\und[1]{\underline{\smash{#1}}}

% Line spacing is 1.5
\renewcommand{\baselinestretch}{1.5}

% Absolute value
\DeclarePairedDelimiter\abs{\lvert}{\rvert}%

% Floor
\DeclarePairedDelimiter\floor{\lfloor}{\rfloor}

%%%%%%%%%%%%%%%%%%%%%%%%%%%%%%%%%%%%%%%%%%%%%%%%%%%%%%%%%%%%%%%%%%%%%%%%%%%%%%%
% Miscellaneous
%%%%%%%%%%%%%%%%%%%%%%%%%%%%%%%%%%%%%%%%%%%%%%%%%%%%%%%%%%%%%%%%%%%%%%%%%%%%%%%

% Setting stuff
\setlength{\parindent}{0pt}  % Remove indentations from paragraphs

% PDF information and nice-looking urls
\hypersetup{%
  pdfauthor={David Oniani},
  pdftitle={Real Analysis},
  pdfsubject={Mathematics, Real Analysis, Real Numbers},
  pdfkeywords={Mathematics, Real Analysis, Real Numbers},
  pdflang={English},
  colorlinks=true,
  linkcolor={black!50!blue},
  citecolor={black!50!blue},
  urlcolor={black!50!blue}
}

%%%%%%%%%%%%%%%%%%%%%%%%%%%%%%%%%%%%%%%%%%%%%%%%%%%%%%%%%%%%%%%%%%%%%%%%%%%%%%%
% Author(s), Title, and Date
%%%%%%%%%%%%%%%%%%%%%%%%%%%%%%%%%%%%%%%%%%%%%%%%%%%%%%%%%%%%%%%%%%%%%%%%%%%%%%%

% Author(s)
\author{David Oniani\\
        Luther College\\
        \href{mailto:oniada01@luther.edu}{oniada01@luther.edu}}

% Title
\title{\rule{\paperwidth - 150pt}{1pt}\textbf{\\\textit{Real Analysis}\\}\rule
{\paperwidth - 150pt}{1pt}\\\textbf{Assignment \textnumero1}\\{\normalsize
Instructor: Dr. Eric Westlund.}}

% Date
\date{\today}

%%%%%%%%%%%%%%%%%%%%%%%%%%%%%%%%%%%%%%%%%%%%%%%%%%%%%%%%%%%%%%%%%%%%%%%%%%%%%%%
% Beginning of the Document
%%%%%%%%%%%%%%%%%%%%%%%%%%%%%%%%%%%%%%%%%%%%%%%%%%%%%%%%%%%%%%%%%%%%%%%%%%%%%%%

\begin{document}
\maketitle

%%%%%%%%%%%%%%%%%%%%%%%%%%%%%%%%%%%%%%%%%%%%%%%%%%%%%%%%%%%%%%%%%%%%%%%%%%%%%%%
% Homework
%
% 1.2 # 5, 8, 9
% 1.3 # 3, 8
% 1.4 # 1, 3
% 1.5 # 9
% 1.6 # 4
%%%%%%%%%%%%%%%%%%%%%%%%%%%%%%%%%%%%%%%%%%%%%%%%%%%%%%%%%%%%%%%%%%%%%%%%%%%%%%%

\begin{itemize}
    \item[5.]
    \begin{itemize}
        \item[(a)]
        Suppose $A, B \subseteq \mathbb{R}$ and $x \in (A \cap B)^c$. Then $x
        \notin A \cap B$. Hence, $x \in A^c$ or $x \in B^c$. Therefore, $x \in
        A^c \cup B^c$. Finally, we get that $(A \cap B)^c \subseteq A^c \cup
        B^c$.\\
        $\qed$

        \item[(b)]
        Suppose $A, B \subseteq \mathbb{R}$ and $x \in A^c \cup B^c$. Then $x
        \in A^c$ or $x \in B^c$, which is equivalent to $x \notin A$ or $x
        \notin B$. Hence, $x \notin A \cap B$. Therefore, $x \in (A \cap B)^c$
        and it follows that $(A \cap B)^c \supseteq A^c \cup B^c$. Finally,
        since we have already proven that $(A \cap B)^c \subseteq A^c \cup
        B^c$, we get that $(A \cap B)^c = A^c \cup B^c$.\\
        $\qed$

        \item[(c)]
        Let us first prove that $A^c \cap B^c \subseteq (A \cup B)^c$.

        Suppose $A, B \subseteq \mathbb{R}$ and $x \in A^c \cap B^c$. Then $x
        \in A^c$ and $x \in B^c$. Thus, $x \notin A$ and $x \notin B$. It
        follows that $x \notin (A \cup B)$. Hence, $x \in (A \cup B)^c$ and
        $A^c \cap B^c \subseteq (A \cup B)^c$.\\
        $\qed$

        Let us now prove that $(A \cup B)^c \subseteq A^c \cap B^c$.

        Suppose $A, B \subseteq \mathbb{R}$ and $x \in (A \cup B)^c$. Then $x
        \notin A \cup B$. Therefore, $x \notin A$ and $x \notin B$. It follows
        that $x \in A^c$ and $x \in B^c$. Thus, $x \in A^c \cap B^c$ and $(A
        \cup B)^c \subseteq A^c \cap B^c$.\\
        $\qed$

        Finally, since $A^c \cap B^c \subseteq (A \cup B)^c$ and $(A \cup B)^c
        \subseteq A^c \cap B^c$, we get that $(A \cup B)^c = A^c \cap B^c$.\\
        $\qed$
    \end{itemize}

    \item[8.]
        \begin{itemize}
            \item[(a)]
            Let us define $f : \mathbb{N} \to \mathbb{N} : x \mapsto 2x + 1$.
            We can now prove that $f$ is $1 - 1$. Suppose, for the sake of
            contradiction, that $x_1, x_2 \in \mathbb{N}$ with $x_1 \neq x_2$
            and $f(x_1) = f(x_2)$. Then $f(x_1) = f(x_2) \implies 2x_1 + 1 =
            2x_2 + 1 \implies x_1 = x_2$. Hence, we face the contradiction and
            $f$ is $1 - 1$. However, $f$ is not onto since for $f(x) = 2$, the
            solution is $0.5$ which is not in $\mathbb{N}$. Thus, $f$ is a
            function that is $1 - 1$, but not onto.\\
            $\qed$

            \item[(b)]
            Let us define $f : \mathbb{N} \rightarrow \mathbb{N} : x \mapsto
            \floor{\dfrac{x}{4}}$. Then $f$ is onto since since $\forall x \in
            \mathbb{N}$, $\exists k = 4x$ with $f(k) = x$. However, $f$ is not
            $1 - 1$ since $f(2) = f(3)$. Thus, $f$ if a function that is onto,
            but not $1 - 1$.\\
            $\qed$

            \item[(c)]
            Let us define

            \begin{equation}
                f : \mathbb{N} \rightarrow \mathbb{Z} : x \mapsto
                \begin{cases}
                    \dfrac{x - 1}{2}, & \text{if } n \text{ is odd}\\
                    -\dfrac{x}{2},    & \text{if } n \text{ is even}
                \end{cases}
            \end{equation}

            Then $f$ is both $1 - 1$ and onto.
            Let us first prove that $f$ is $1 - 1$.

            Suppose, for the sake of contradiction, that $x_1, x_2 \in
            \mathbb{N}$ with $x_1 \neq x_2$ and $f(x_1) = f(x_2)$. Then
            $f(x_1)$ and $f(x_2)$ must be of the same sign or both be zero.
            Thus, we have three cases:

            \begin{itemize}
                \item[1.]
                    $f(x_1) = \dfrac{x_1 - 1}{2}$ and $f(x_2) =
                    \dfrac{x_2 - 1}{2}$

                    $f(x_1) = f(x_2) \implies \dfrac{x_1 - 1}{2} =
                    \dfrac{x_2 - 1}{2} \implies x_1 = x_2$.\\
                    $\qed$

                \item[2.]
                    $f(x_1) = -\dfrac{x_1}{2}$ and $f(x_2) = -\dfrac{x_2}{2}$

                    $f(x_1) = f(x_2) \implies -\dfrac{x_1}{2} = -\dfrac{x_2}{2}
                    \implies x_1 = x_2$\\
                    $\qed$

                \item[3.]
                    $f(x_1) = 0$ and $f(x_2) = 0$

                    The only way for this to happen is if $f(x_1) =
                    -\dfrac{x_1}{2}$ and $f(x_2) = -\dfrac{x_2}{2}$ and we get
                    $f(x_1) = f(x_2) \implies -\dfrac{x_1}{2} = -\dfrac{x_2}{2}
                    \implies x_1 = x_2$\\
                    $\qed$
            \end{itemize}
            $\qed$

            Now let us prove that $f$ is onto. For $y \in \mathbb{Z}$, we have
            three cases:

            \begin{itemize}
                \item[1.]
                    $y$ is positive.

                    If $y > 0$, we can find $k = 2y + 1$ with $f(k) = y$.\\
                    $\qed$

                \item[2.]
                    $y$ is negative.

                    If $y < 0$, we can find $k = -2y$ with $f(k) = y$.\\
                    $\qed$

                \item[3.]
                    $y$ is zero.

                    If $y = 0$, we can find $k = 1$ with $f(k) = y = 0$.\\
                    $\qed$
            \end{itemize}
            $\qed$

            Finally, we have proven that $f$ is both $1 - 1$ and onto.\\
            $\qed$
        \end{itemize}

    \item[9.]
        \begin{itemize}
            \item[(a)]


            \item[(b)]
        \end{itemize}

    \item

\end{itemize}

%%%%%%%%%%%%%%%%%%%%%%%%%%%%%%%%%%%%%%%%%%%%%%%%%%%%%%%%%%%%%%%%%%%%%%%%%%%%%%%
% The End of the Document
%%%%%%%%%%%%%%%%%%%%%%%%%%%%%%%%%%%%%%%%%%%%%%%%%%%%%%%%%%%%%%%%%%%%%%%%%%%%%%%

\end{document}
