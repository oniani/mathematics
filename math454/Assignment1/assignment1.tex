%%%%%%%%%%%%%%%%%%%%%%%%%%%%%%%%%%%%%%%%%%%%%%%%%%%%%%%%%%%%%%%%%%%%%%%%%%%%%%%
%
% Filename: template.tex
% Author:   David Oniani
% Modified: December 01, 2020
%  _         _____   __  __
% | |    __ |_   _|__\ \/ /
% | |   / _` || |/ _ \\  /
% | |__| (_| || |  __//  \
% |_____\__,_||_|\___/_/\_\
%
%%%%%%%%%%%%%%%%%%%%%%%%%%%%%%%%%%%%%%%%%%%%%%%%%%%%%%%%%%%%%%%%%%%%%%%%%%%%%%%

%%%%%%%%%%%%%%%%%%%%%%%%%%%%%%%%%%%%%%%%%%%%%%%%%%%%%%%%%%%%%%%%%%%%%%%%%%%%%%%
% Document Definition
%%%%%%%%%%%%%%%%%%%%%%%%%%%%%%%%%%%%%%%%%%%%%%%%%%%%%%%%%%%%%%%%%%%%%%%%%%%%%%%

\documentclass[11pt]{article}

%%%%%%%%%%%%%%%%%%%%%%%%%%%%%%%%%%%%%%%%%%%%%%%%%%%%%%%%%%%%%%%%%%%%%%%%%%%%%%%
% Packages and Related Settings
%%%%%%%%%%%%%%%%%%%%%%%%%%%%%%%%%%%%%%%%%%%%%%%%%%%%%%%%%%%%%%%%%%%%%%%%%%%%%%%

% Global, document-wide settings
\usepackage[margin=1in]{geometry}
\usepackage[utf8]{inputenc}
\usepackage[english]{babel}

% Other packages
\usepackage{booktabs}
\usepackage{hyperref}
\usepackage{mathtools}
\usepackage{amsthm}
\usepackage{amssymb}
\usepackage[cache=false]{minted}

%%%%%%%%%%%%%%%%%%%%%%%%%%%%%%%%%%%%%%%%%%%%%%%%%%%%%%%%%%%%%%%%%%%%%%%%%%%%%%%
% Command Definitions and Redefinitions
%%%%%%%%%%%%%%%%%%%%%%%%%%%%%%%%%%%%%%%%%%%%%%%%%%%%%%%%%%%%%%%%%%%%%%%%%%%%%%%

% Nice-looking underline
\newcommand\und[1]{\underline{\smash{#1}}}

% Line spacing is 1.5
\renewcommand{\baselinestretch}{1.5}

% Absolute value
\DeclarePairedDelimiter\abs{\lvert}{\rvert}%

% Floor
\DeclarePairedDelimiter\floor{\lfloor}{\rfloor}

% % Naturals, Reals, Integers, and Rationals, 
\newcommand{\nats}{\mathbb{N}}
\newcommand{\reals}{\mathbb{R}}
\newcommand{\preals}{\mathbb{R^+}}
\newcommand{\nreals}{\mathbb{R^-}}
\newcommand{\ints}{\mathbb{Z}}
\newcommand{\pints}{\mathbb{Z^+}}
\newcommand{\nints}{\mathbb{Z^-}}
\newcommand{\rats}{\mathbb{Q}}
\newcommand{\prats}{\mathbb{Q^+}}
\newcommand{\nrats}{\mathbb{Q^-}}
\newcommand{\irrats}{\mathbb{I}}
\newcommand{\pirrats}{\mathbb{I^+}}
\newcommand{\nirrats}{\mathbb{I^-}}

%%%%%%%%%%%%%%%%%%%%%%%%%%%%%%%%%%%%%%%%%%%%%%%%%%%%%%%%%%%%%%%%%%%%%%%%%%%%%%%
% Miscellaneous
%%%%%%%%%%%%%%%%%%%%%%%%%%%%%%%%%%%%%%%%%%%%%%%%%%%%%%%%%%%%%%%%%%%%%%%%%%%%%%%

% Setting stuff
\setlength{\parindent}{0pt}  % Remove indentations from paragraphs

% PDF information and nice-looking urls
\hypersetup{%
  pdfauthor={David Oniani},
  pdftitle={Real Analysis},
  pdfsubject={Mathematics, Real Analysis, Real Numbers},
  pdfkeywords={Mathematics, Real Analysis, Real Numbers},
  pdflang={English},
  colorlinks=true,
  linkcolor={black!50!blue},
  citecolor={black!50!blue},
  urlcolor={black!50!blue}
}

%%%%%%%%%%%%%%%%%%%%%%%%%%%%%%%%%%%%%%%%%%%%%%%%%%%%%%%%%%%%%%%%%%%%%%%%%%%%%%%
% Author(s), Title, and Date
%%%%%%%%%%%%%%%%%%%%%%%%%%%%%%%%%%%%%%%%%%%%%%%%%%%%%%%%%%%%%%%%%%%%%%%%%%%%%%%

% Author(s)
\author{David Oniani\\
        Luther College\\
        \href{mailto:oniada01@luther.edu}{oniada01@luther.edu}}

% Title
\title{\rule{\paperwidth - 150pt}{1pt}\textbf{\\\textit{Real Analysis}\\}\rule
{\paperwidth - 150pt}{1pt}\\\textbf{Assignment \textnumero1}\\{\normalsize
Instructor: Dr. Eric Westlund.}}

% Date
\date{\today}

%%%%%%%%%%%%%%%%%%%%%%%%%%%%%%%%%%%%%%%%%%%%%%%%%%%%%%%%%%%%%%%%%%%%%%%%%%%%%%%
% Beginning of the Document
%%%%%%%%%%%%%%%%%%%%%%%%%%%%%%%%%%%%%%%%%%%%%%%%%%%%%%%%%%%%%%%%%%%%%%%%%%%%%%%

\begin{document}
\maketitle

%%%%%%%%%%%%%%%%%%%%%%%%%%%%%%%%%%%%%%%%%%%%%%%%%%%%%%%%%%%%%%%%%%%%%%%%%%%%%%%
%
% Homework
%
% 1.2 # 5, 8, 9
% 1.3 # 3, 8
% 1.4 # 1, 3
% 1.5 # 9
% 1.6 # 4
%
%%%%%%%%%%%%%%%%%%%%%%%%%%%%%%%%%%%%%%%%%%%%%%%%%%%%%%%%%%%%%%%%%%%%%%%%%%%%%%%

\begin{itemize}
    \item[1.2.5]
    \begin{itemize}
        \item[(a)]
        Suppose $A, B \subseteq \reals$ and $x \in (A \cap B)^c$. Then $x
        \notin A \cap B$. Hence, $x \in A^c$ or $x \in B^c$. Therefore, $x \in
        A^c \cup B^c$. Finally, we get that $(A \cap B)^c \subseteq A^c \cup
        B^c$.\\
        $\qed$

        \item[(b)]
        Suppose $A, B \subseteq \reals$ and $x \in A^c \cup B^c$. Then $x
        \in A^c$ or $x \in B^c$, which is equivalent to $x \notin A$ or $x
        \notin B$. Hence, $x \notin A \cap B$. Therefore, $x \in (A \cap B)^c$
        and it follows that $(A \cap B)^c \supseteq A^c \cup B^c$. Finally,
        since we have already proven that $(A \cap B)^c \subseteq A^c \cup
        B^c$, we get that $(A \cap B)^c = A^c \cup B^c$.\\
        $\qed$

        \item[(c)]
        Let us first prove that $A^c \cap B^c \subseteq (A \cup B)^c$.

        Suppose $A, B \subseteq \reals$ and $x \in A^c \cap B^c$. Then $x
        \in A^c$ and $x \in B^c$. Thus, $x \notin A$ and $x \notin B$. It
        follows that $x \notin (A \cup B)$. Hence, $x \in (A \cup B)^c$ and
        $A^c \cap B^c \subseteq (A \cup B)^c$.\\
        $\qed$

        We have to now prove the converse. Let us now prove that $(A \cup B)^c
        \subseteq A^c \cap B^c$.

        Suppose $A, B \subseteq \reals$ and $x \in (A \cup B)^c$. Then $x
        \notin A \cup B$. Therefore, $x \notin A$ and $x \notin B$. It follows
        that $x \in A^c$ and $x \in B^c$. Thus, $x \in A^c \cap B^c$ and $(A
        \cup B)^c \subseteq A^c \cap B^c$.\\
        $\qed$

        Finally, since $A^c \cap B^c \subseteq (A \cup B)^c$ and $(A \cup B)^c
        \subseteq A^c \cap B^c$, we get that $(A \cup B)^c = A^c \cap B^c$.\\
        $\qed$
    \end{itemize}

    \item[1.2.8]
        \begin{itemize}
            \item[(a)]
            Let us define $f : \mathbb{N} \to \mathbb{N} : x \mapsto 2x + 1$.
            We can now prove that $f$ is $1 - 1$. Suppose, for the sake of
            contradiction, that $x_1, x_2 \in \mathbb{N}$ with $x_1 \neq x_2$
            and $f(x_1) = f(x_2)$. Then $f(x_1) = f(x_2) \implies 2x_1 + 1 =
            2x_2 + 1 \implies x_1 = x_2$. Hence, we face the contradiction and
            $f$ is $1 - 1$. However, $f$ is not onto since for $f(x) = 2$, the
            solution is $0.5$ which is not in $\mathbb{N}$. Thus, $f$ is a
            function that is $1 - 1$, but not onto.\\
            $\qed$

            \item[(b)]
            Let us define $f : \mathbb{N} \to \mathbb{N} : x \mapsto
            \floor{\dfrac{x}{4}}$. Then $f$ is onto since since $\forall x \in
            \mathbb{N}$, $\exists k = 4x$ with $f(k) = x$. However, $f$ is not
            $1 - 1$ since $f(2) = f(3)$. Thus, $f$ if a function that is onto,
            but not $1 - 1$.\\
            $\qed$

            \item[(c)]
            Let us define

            \begin{equation*}
                f : \mathbb{N} \to \ints : x \mapsto
                \begin{cases}
                    \dfrac{x - 1}{2}, & \text{if } n \text{ is odd}\\
                    -\dfrac{x}{2},    & \text{if } n \text{ is even}
                \end{cases}
            \end{equation*}

            Then $f$ is both $1 - 1$ and onto.
            Let us first prove that $f$ is $1 - 1$.

            Suppose, for the sake of contradiction, that $x_1, x_2 \in
            \mathbb{N}$ with $x_1 \neq x_2$ and $f(x_1) = f(x_2)$. Then
            $f(x_1)$ and $f(x_2)$ must be of the same sign or both be zero.
            Thus, we have three cases:

            \begin{itemize}
                \item[1.]
                    $f(x_1) = \dfrac{x_1 - 1}{2}$ and $f(x_2) =
                    \dfrac{x_2 - 1}{2}$

                    $f(x_1) = f(x_2) \implies \dfrac{x_1 - 1}{2} =
                    \dfrac{x_2 - 1}{2} \implies x_1 = x_2$.\\
                    $\qed$

                \item[2.]
                    $f(x_1) = -\dfrac{x_1}{2}$ and $f(x_2) = -\dfrac{x_2}{2}$

                    $f(x_1) = f(x_2) \implies -\dfrac{x_1}{2} = -\dfrac{x_2}{2}
                    \implies x_1 = x_2$\\
                    $\qed$

                \item[3.]
                    $f(x_1) = 0$ and $f(x_2) = 0$

                    The only way for this to happen is if $f(x_1) =
                    -\dfrac{x_1}{2}$ and $f(x_2) = -\dfrac{x_2}{2}$ and we get
                    $f(x_1) = f(x_2) \implies -\dfrac{x_1}{2} = -\dfrac{x_2}{2}
                    \implies x_1 = x_2$\\
                    $\qed$
            \end{itemize}
            $\qed$

            Now let us prove that $f$ is onto. For $y \in \ints$, we have
            three cases:

            \begin{itemize}
                \item[1.]
                    $y$ is positive.

                    If $y > 0$, we can find $k = 2y + 1$ with $f(k) = y$.\\
                    $\qed$

                \item[2.]
                    $y$ is negative.

                    If $y < 0$, we can find $k = -2y$ with $f(k) = y$.\\
                    $\qed$

                \item[3.]
                    $y$ is zero.

                    If $y = 0$, we can find $k = 1$ with $f(k) = y = 0$.\\
                    $\qed$
            \end{itemize}
            $\qed$

            Finally, we have proven that $f$ is both $1 - 1$ and onto.\\
            $\qed$
        \end{itemize}

    \item[1.2.9]
        \begin{itemize}
            \item[(a)]
                Given, $f(x) = x^2$, $A = [0, 4]$, and $B = [-1, 1]$, we get:
                \begin{align*}
                    f^{-1}(A) &= \{x \mid f(x) \in A\} = [-2, 2]\\
                    f^{-1}(B) &= \{x \mid f(x) \in B\} = [-1, 1]
                \end{align*}

                We get $f^{-1}(A) = [-2, 2]$ as the negative values plugged
                into $x^2$ get positive, with 0 mapping to 0. Since the biggest
                value of $A$ is 4, $x$ cannot exceed 2, so we have an interval
                $[0, 2]$. And now, due to properties of $f(x) = x^2$, we add
                the negative part $[-2, 0)$. Finally, $[0, 2] \cup
                [-2, 0) = [-2, 2]$

                Similarly, $f^{-1}(B) = [-1, 1]$ as $[0, 1] \cup [-1, 0) =
                [-1, 1]$.
                
                $A \cap B = [0, 4] \cap [-1, 1] = [0, 1]$.
                \begin{align*}
                    f^{-1}(A \cap B)        &= \{x \mid f(x) \in A \cap B\} = [-1, 1]\\
                    f^{-1}(A) \cap f{-1}(B) &= [-2, 2] \cap [-1, 1] = [-1, 1]
                \end{align*}
                
                Hence, $f^{-1}(A \cap B) = f^{-1}(A) \cap f^{-1}(B)$.
                
                Additionally, we have:
                \begin{align*}
                    f^{-1}(A \cup B)        &= \{x \mid f(x) \in A \cup B\} = [-2, 2]\\
                    f^{-1}(A) \cup f{-1}(B) &= [-2, 2] \cup [-1, 1] = [-2, 2]
                \end{align*}
                
                Hence, $f^{-1}(A \cup B) = f^{-1}(A) \cup f^{-1}(B)$.

            \item[(b)]
                Let us now prove that the properties shown in $(a)$ are
                completely general.

                \begin{itemize}
                    \item[1.]
                    Suppose $g : \reals \to \reals$ is a function from
                    reals to reals and $A, B \subseteq \reals$. Prove that
                    $g^{-1}(A \cap B) = g^{-1}(A) \cap g^{-1}(B)$.\\

                    We have to show both $g^{-1}(A \cap B) \subseteq g^{-1}(A)
                    \cap g^{-1}(B)$ and $g^{-1}(A) \cap g^{-1}(B) \subseteq
                    g^{-1}(A \cap B)$.\\

                    Suppose $x \in g^{-1}(A \cap B)$. Then $g(x) \in A \cap B$.
                    It follows that $g(x) \in A$ and $g(x) \in B$. As a result,
                    $x \in g^{-1}(A)$ and $x \in g^{-1}(B)$. Hence, $x \in
                    g^{-1}(A) \cap g^{-1}(B)$ and $g^{-1}(A \cap B) \subseteq
                    g^{-1}(A) \cap g^{-1}(B)$.\\
                    $\qed$

                    We now have to prove the converse. Suppose $x \in g^{-1}(A)
                    \cap g^{-1}(B)$. Then $g(x) \in A$ and $g(x) \in B$. Hence,
                    $g(x) \in A \cap B$. Therefore, $x \in g^{-1}(A \cap B)$
                    and $g^{-1}(A) \cap g^{-1}(B) \subseteq g^{-1}(A \cap B)$.
                    \\
                    $\qed$\\

                    We have now shown that $g^{-1}(A \cap B) = g^{-1}(A) \cap
                    g^{-1}(B)$\\
                    $\qed$

                    \item[2.]
                    Suppose $g : \reals \to \reals$ is a function from
                    reals to reals and $A, B \subseteq \reals$. Prove that
                    $g^{-1}(A \cup B) = g^{-1}(A) \cup g^{-1}(B)$.\\

                    We have to show both $g^{-1}(A \cup B) \subseteq g^{-1}(A)
                    \cup g^{-1}(B)$ and $ g^{-1}(A) \cup g^{-1}(B) \subseteq
                    g^{-1}(A \cup B)$.\\

                    Suppose $x \in g^{-1}(A \cup B)$. Then $g(x) \in A \cup B$.
                    It follows that $g(x) \in A$ or $g(x) \in B$. As a result,
                    $x \in g^{-1}(A)$ or $x \in g^{-1}(B)$. Hence, $x \in
                    g^{-1}(A) \cup g^{-1}(B)$ and $g^{-1}(A \cup B) \subseteq
                    g^{-1}(A) \cup g^{-1}(B)$.\\
                    $\qed$

                    \newpage

                    We now have to prove the converse. Suppose $x \in g^{-1}(A)
                    \cup g^{-1}(B)$. Then $g(x) \in A$ or $g(x) \in B$. Hence,
                    $g(x) \in A \cup B$. Therefore, $x \in g^{-1}(A \cup B)$
                    and $g^{-1}(A) \cup g^{-1}(B) \subseteq g^{-1}(A \cup B)$.
                    \\
                    $\qed$\\

                    We have now shown that $g^{-1}(A \cup B) = g^{-1}(A) \cup
                    g^{-1}(B)$\\
                    $\qed$

                \end{itemize}

        \end{itemize}

    \item[1.3.3]
        \begin{itemize}
            \item[(a)]
                Because $B = \{b \in \reals \mid b \text{ is a lower bound for
                } A\}$, we have that $\inf{A} \in B$ as $\inf{A}$ is also a
                lower bound for $A$. Thus, $B$ cannot be empty. Furthermore,
                $\forall a \in A$ and $\forall b \in B$, $b \leq a$. Hence, per
                the Axiom of Completeness, $\sup(B)$ must exist. Now, since
                $\inf{A}$ is the greatest lowest bound for $A$, it follows that
                $\forall b \in B, b \leq \inf{A}$. Therefore, $\inf{A}$ is the
                maximum of $B$ and $\inf{A} = \max{B}$. Finally, according to
                the book, \textit{when a maximum exists, then it is also the
                supremum} and thus $\inf{A} = \sup{B}$.\\
                $\qed$

            \item[(b)]
                We have already shown in $(a)$ that if $A$ is bounded below,
                then there exists a greatest lower bound. Hence, the existence
                of the least upper bound implies the existence of the greatest
                lower bound and there is no need for the axiom to state it
                explicitly. Sometimes, implicit is better than explicit. The
                axiom is very minimalistic in design, I like it.
        \end{itemize}

    \item[1.3.8]
        \begin{itemize}
            \item[(a)]
                \begin{align*}
                    \inf{A} &= \inf{\{\dfrac{1}{n}     \mid n \in \nats\}} = 0\\
                    \sup{A} &= \sup{\{\dfrac{n}{n + 1} \mid n \in \nats\}} = 1
                \end{align*}

            \item[(b)]
                \begin{align*}
                    \inf{A} &= \inf{\{\dfrac{-1}{n} \mid n \in \nats\}} = -1\\
                    \sup{A} &= \sup{\{\dfrac{1}{n}  \mid n \in \nats\}} = 1
                \end{align*}

            \newpage

            \item[(c)]
                Notice that $\dfrac{n}{3n + 1} = \dfrac{1}{3 + \dfrac{1}{n}}$.
                We can divide by $n$ since, by convention, $\nats$ does not
                include $0$ (in pure math that is). Then we have:

                \begin{align*}
                    \inf{A} &= \dfrac{1}{4}\\
                    \sup{A} &= \dfrac{1}{3}
                \end{align*}

            \item[(d)]
                Just like in $(c)$, we, once again, divide both the numerator
                and the denominator by $m$ and get $\dfrac{1}{1 +
                \dfrac{n}{m}}$. We have:

                \begin{align*}
                    \inf{A} &= 0\\
                    \sup{A} &= 1
                \end{align*}
        \end{itemize}

    \item[1.4.1]
        \begin{itemize}
            \item[(a)]
                Let us first show that $ab \in \rats$.
                Suppose $a = \dfrac{m}{n}, b = \dfrac{p}{q} \in \rats$ s.t.
                $n, q \neq 0$ and $m, n, p, q \in \ints$. Then $ab =
                \dfrac{mp}{nq}$ and since $\ints$ is closed under
                multiplication (with $nq \neq 0$), we get that $ab \in \rats$.
                \\
                $\qed$

                Let us now show that $a + b \in \rats$.
                Suppose $a = \dfrac{m}{n}, b = \dfrac{p}{q} \in \rats$ s.t.
                $n, q \neq 0$ and $m, n, p, q \in \ints$. Then $a + b =
                \dfrac{mq + pn}{nq}$ and since $\ints$ is closed under addition
                (with $nq \neq 0$), we get that $a + b \in \rats$.
                \\
                $\qed$

            \item[(b)]
                Let us first show that $a + t \in \irrats$.
                Suppose, for the sake of contradiction, that $a = \dfrac{m}{n}
                \in \rats$ s.t. $m, n \neq 0$, $m, n \in \ints$, and $a + t =
                \dfrac{j}{k}$ with $k \neq 0$. Then $t = \dfrac{j}{k} -
                \dfrac{m}{n}$ and thus, $t \in \rats$. Hence, we face a
                contradiction and $a + t$ is indeed irrational, $a + t \in
                \irrats$.
                \\
                $\qed$

                Let us now show that $at \in \irrats$.
                Suppose, for the sake of contradiction, that $a = \dfrac{m}{n}
                \in \rats$ s.t. $m, n \neq 0$, $m, n \in \ints$, and $at =
                \dfrac{j}{k}$ with $k \neq 0$. Then $t = \dfrac{jn}{km}$
                and thus, $t \in \rats$. Hence, we face a contradiction and
                $at$ is indeed irrational, $at \in \irrats$.
                \\
                $\qed$

            \item[(c)]
                The set of irrational numbers is not closed under addition.
                For example, take $s = \sqrt{3}$ and $t = -\sqrt{3}$, then $s +
                t = 0$ is not irrational. However, in some cases, it is closed
                under addition. If $s = \sqrt{2}$ and $t = \sqrt{3}$, then
                $s + t$ is irrational ($\sqrt{2} + \sqrt{3}$ is irrational).

                The set of irrational numbers is not closed under
                multiplication either. For instance, take $s = \sqrt{3}$ and
                $t = -\sqrt{3}$ again. Then $st = -3$ which is not irrational.
                In some cases, much alike addition, it is also closed under
                multiplication. If $s = \sqrt{2}$ and $t = \sqrt{3}$, then $st$
                is irrational ($\sqrt{6}$ is irrational).
        \end{itemize}

    \item[1.4.3]
        Let us denote $\bigcap_{n = 1}^{\infty} (0, 1/n)$ as $S$. Suppose, for
        the sake of contradiction, that $x \neq \emptyset \in S$. Then, by the
        definition of $S, x \neq 0$. Hence, $x > 0$. Then, per the
        \textbf{Archimedean Property}, $\exists y \in \nats$ s.t.
        $\dfrac{1}{y} < x$. It follows that $\forall n \geq y, x \notin
        (0, 1/n)$. Thus, we face a contradiction and $S = \emptyset$.\\
        $\qed$
        
    \item[1.5.9]
        \begin{itemize}
            \item[(a)]
                Let us first show that $\sqrt{2}$ is algebraic.\\
                Notice that $\sqrt{2}$ is one of the roots of the equation $x^2
                - 2 = 0$. Hence, $\sqrt{2}$ is algebraic.\\
                $\qed$
                \\
                Let us now show that $\sqrt[3]{2}$ is algebraic.\\
                Notice that $\sqrt[3]{2}$ is the root of the equation $x^3 - 2
                = 0$. Hence, $\sqrt[3]{2}$ is algebraic.\\
                $\qed$
                \\
                Finally, let us prove that $\sqrt{3} + \sqrt{2}$ is algebraic
                \\
                Using \textbf{Vieta's formulas}, it is easy to see that
                $\sqrt{3} + \sqrt{2}$ is one of the roots to the equation $x^2
                -2\sqrt{3}x + 1 = 0$. Hence, $\sqrt{3} + \sqrt{2}$ is
                algebraic.\\
                $\qed$

            \item[(b)]
                Suppose $A_n$ with $n \in \nats$ is the set of algebraic
                numbers obtained as roots of polynomials with integer
                coefficients that have degree $n$. Then let us denote $A_{ny}$
                as the set of algebraic numbers s.t. $\forall x \in A_{ny},
                x \leq y$ with $y \in \nats$. It follows that $\forall y \in
                \nats, A_{ny}$ is countable. Furthermore, $\forall y \in \nats,
                A_{ny}$ is finite. $A_n$ can then be represented as union of
                many $A_{ny}$ sets:

                \begin{equation*}
                    A_n = \bigcup_{y = 1}^{\infty} A_{ny}
                \end{equation*}

                Now, notice that every set in the union must be countable.
                Additionally, $A_n$ is countable as well. Finally, since the
                countable union of countable sets is also countable, we
                conclude that $A_n$ is countable.\\
                $\qed$

                \textit{NOTE: The fact that the coountable union of countable
                sets is countable can be proven visually. One will need to
                arrange the sets in the form of the 2D matrix and perform
                enumeration diagonal-by-diagonal (of the small squares).}

            \item[(c)]
                Suppose $S$ is the set of all algebraic numbers. In $(b)$,
                we showed that $\forall n \in \nats, A_n$ is countable. Now,
                recall that the countable union of countable sets is countable.
                Then, since $S = \bigcap_{n = 1}^{\infty} A_n$, $S$ is
                countable too. Therefore, the set of all algebraic numbers is
                also countable.\\
                $\qed$
                \\
                The set of transcendental numbers, on the other hand, cannot be
                countable. The set of real numbers $\reals$ is a union of
                algebraic and transcendental numbers. We already know that the
                set of algebraic numbers is countable and the set of real
                numbers $\reals$ is uncountable. Since $\reals$ is uncountable
                and the set of algebraic numbers is countable, it must be
                transcendental numbers that contribute to its uncountability.
                Thus, the set of transcendental numbers is uncountable.
        \end{itemize}

    \item[1.6.4]
        Suppose, for the sake of contradiction, that $S = \{(a_1, a_2, a_3
        \dots \mid a_n = 0 \text{ or } 1\}$ is countable. Then there must
        exist an enumeration of these numbers:
        \begin{align*}
            s_1 &= (a_{11}, a_{12}, a_{13}, a_{14}, \dots)\\
            s_2 &= (a_{21}, a_{22}, a_{23}, a_{24}, \dots)\\
            s_3 &= (a_{31}, a_{32}, a_{33}, a_{34}, \dots)\\
            s_4 &= (a_{41}, a_{42}, a_{43}, a_{44}, \dots)\\
            &.....................................
        \end{align*}

        Now, let us define a sequence of 0s and 1s $t$ s.t. $\forall n \in t$,
        the following holds:
        \begin{equation*}
            t_n =
            \begin{cases}
                0 & \text{if } s_{nn} \text{ is } 1\\
                1 & \text{if } s_{nn} \text{ is } 0\\
            \end{cases}
        \end{equation*}

        with $s_{ni}$ denoting the $i^{th}$ element of $S$.

        Then, since $t$ is a sequence of 0s and 1s, it must be in $S$. Thus, $t
        \in S$. However, we have effectively constructed a sequence of numbers
        that cannot be in $S$ as $\forall s \in S, s \neq t$ (first item in $t$
        is different from $s_1$, the second one is different from $s_2$, and so
        on). Thus, we face a contradiction and $S$ cannot be enumerated. Hence,
        $S$ is uncountable.\\
        $\qed$
\end{itemize}

%%%%%%%%%%%%%%%%%%%%%%%%%%%%%%%%%%%%%%%%%%%%%%%%%%%%%%%%%%%%%%%%%%%%%%%%%%%%%%%
% The End of the Document
%%%%%%%%%%%%%%%%%%%%%%%%%%%%%%%%%%%%%%%%%%%%%%%%%%%%%%%%%%%%%%%%%%%%%%%%%%%%%%%

\end{document}
