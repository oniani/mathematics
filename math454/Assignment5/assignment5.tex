%%%%%%%%%%%%%%%%%%%%%%%%%%%%%%%%%%%%%%%%%%%%%%%%%%%%%%%%%%%%%%%%%%%%%%%%%%%%%%%
%
% Filename: assignment5.tex
% Author:   David Oniani
% Modified: December 15, 2020
%  _         _____   __  __
% | |    __ |_   _|__\ \/ /
% | |   / _` || |/ _ \\  /
% | |__| (_| || |  __//  \
% |_____\__,_||_|\___/_/\_\
%
%%%%%%%%%%%%%%%%%%%%%%%%%%%%%%%%%%%%%%%%%%%%%%%%%%%%%%%%%%%%%%%%%%%%%%%%%%%%%%%

%%%%%%%%%%%%%%%%%%%%%%%%%%%%%%%%%%%%%%%%%%%%%%%%%%%%%%%%%%%%%%%%%%%%%%%%%%%%%%%
% Document Definition
%%%%%%%%%%%%%%%%%%%%%%%%%%%%%%%%%%%%%%%%%%%%%%%%%%%%%%%%%%%%%%%%%%%%%%%%%%%%%%%

\documentclass[11pt]{article}

%%%%%%%%%%%%%%%%%%%%%%%%%%%%%%%%%%%%%%%%%%%%%%%%%%%%%%%%%%%%%%%%%%%%%%%%%%%%%%%
% Packages and Related Settings
%%%%%%%%%%%%%%%%%%%%%%%%%%%%%%%%%%%%%%%%%%%%%%%%%%%%%%%%%%%%%%%%%%%%%%%%%%%%%%%

% Global, document-wide settings
\usepackage[margin=1in]{geometry}
\usepackage[utf8]{inputenc}
\usepackage[english]{babel}

% Other packages
\usepackage{booktabs}
\usepackage{hyperref}
\usepackage{mathtools}
\usepackage{amsthm}
\usepackage{amssymb}
\usepackage[cache=false]{minted}

%%%%%%%%%%%%%%%%%%%%%%%%%%%%%%%%%%%%%%%%%%%%%%%%%%%%%%%%%%%%%%%%%%%%%%%%%%%%%%%
% Command Definitions and Redefinitions
%%%%%%%%%%%%%%%%%%%%%%%%%%%%%%%%%%%%%%%%%%%%%%%%%%%%%%%%%%%%%%%%%%%%%%%%%%%%%%%

% Nice-looking underline
\newcommand\und[1]{\underline{\smash{#1}}}

% Line spacing is 1.5
\renewcommand{\baselinestretch}{1.5}

% Absolute value
\DeclarePairedDelimiter\abs{\lvert}{\rvert}%

% Ceiling
\DeclarePairedDelimiter{\ceil}{\lceil}{\rceil}

% Floor
\DeclarePairedDelimiter\floor{\lfloor}{\rfloor}

% % Naturals, Reals, Integers, and Rationals, 
\newcommand{\nats}{\mathbb{N}}
\newcommand{\reals}{\mathbb{R}}
\newcommand{\preals}{\mathbb{R^+}}
\newcommand{\nreals}{\mathbb{R^-}}
\newcommand{\ints}{\mathbb{Z}}
\newcommand{\pints}{\mathbb{Z^+}}
\newcommand{\nints}{\mathbb{Z^-}}
\newcommand{\rats}{\mathbb{Q}}
\newcommand{\prats}{\mathbb{Q^+}}
\newcommand{\nrats}{\mathbb{Q^-}}
\newcommand{\irrats}{\mathbb{I}}
\newcommand{\pirrats}{\mathbb{I^+}}
\newcommand{\nirrats}{\mathbb{I^-}}

%%%%%%%%%%%%%%%%%%%%%%%%%%%%%%%%%%%%%%%%%%%%%%%%%%%%%%%%%%%%%%%%%%%%%%%%%%%%%%%
% Miscellaneous
%%%%%%%%%%%%%%%%%%%%%%%%%%%%%%%%%%%%%%%%%%%%%%%%%%%%%%%%%%%%%%%%%%%%%%%%%%%%%%%

% Setting stuff
\setlength{\parindent}{0pt}  % Remove indentations from paragraphs

% PDF information and nice-looking urls
\hypersetup{%
  pdfauthor={David Oniani},
  pdftitle={Real Analysis},
  pdfsubject={Mathematics, Real Analysis, Real Numbers},
  pdfkeywords={Mathematics, Real Analysis, Real Numbers},
  pdflang={English},
  colorlinks=true,
  linkcolor={black!50!blue},
  citecolor={black!50!blue},
  urlcolor={black!50!blue}
}

%%%%%%%%%%%%%%%%%%%%%%%%%%%%%%%%%%%%%%%%%%%%%%%%%%%%%%%%%%%%%%%%%%%%%%%%%%%%%%%
% Author(s), Title, and Date
%%%%%%%%%%%%%%%%%%%%%%%%%%%%%%%%%%%%%%%%%%%%%%%%%%%%%%%%%%%%%%%%%%%%%%%%%%%%%%%

% Author(s)
\author{David Oniani\\
        Luther College\\
        \href{mailto:oniada01@luther.edu}{oniada01@luther.edu}}

% Title
\title{\rule{\paperwidth - 150pt}{1pt}\textbf{\\\textit{Real Analysis}\\}\rule
{\paperwidth - 150pt}{1pt}\\\textbf{Assignment \textnumero5}\\{\normalsize
Instructor: Dr. Eric Westlund}}

% Date
\date{\today}

%%%%%%%%%%%%%%%%%%%%%%%%%%%%%%%%%%%%%%%%%%%%%%%%%%%%%%%%%%%%%%%%%%%%%%%%%%%%%%%
% Beginning of the Document
%%%%%%%%%%%%%%%%%%%%%%%%%%%%%%%%%%%%%%%%%%%%%%%%%%%%%%%%%%%%%%%%%%%%%%%%%%%%%%%

\begin{document}
\maketitle

%%%%%%%%%%%%%%%%%%%%%%%%%%%%%%%%%%%%%%%%%%%%%%%%%%%%%%%%%%%%%%%%%%%%%%%%%%%%%%%
%
% Homework
%
% 3.2 # 2, 6, 8, 14
%
%%%%%%%%%%%%%%%%%%%%%%%%%%%%%%%%%%%%%%%%%%%%%%%%%%%%%%%%%%%%%%%%%%%%%%%%%%%%%%%

\begin{itemize}
    \item[3.2.2]
        \begin{itemize}
            \item[(a)]
                The limit points of $A$ are $1$ and $-1$. The limit points of
                $B$ are all numbers in the closed interval $[0, 1]$.
                \\
                \\
                Limit points of $A$ are $1, -1$ as
                $\lim_{n \to \infty} A = 1, -1$.
                \\
                Limit points of $B$ are all numbers in the closed interval
                $[0, 1]$ since between any two rationals there are infinitely
                many irrationals, vice versa. Hence, the limit points of $B$
                form the closed interval $[0, 1]$.

            \item[(b)]
                \textbf{$A$ is neither closed nor open}. \textbf{$B$ is neither
                closed nor open}. Both are due to the fact that between any two
                rational numbers, there exists infinitely many irrational
                numbers.

            \item[(c)]
                All points in $A$ except for $1$ and $-1$ are isolated. Due to
                the fact that between any two rational numbers, there exists
                infinitely many rational numbers, $\textbf{B has no isolated
                points}$.

            \item[(d)]
                $\overline{A} = A \cup \{1, -1\} = \Big\{(-1)^n + \dfrac{2}{n}
                \mid n = 1, 2, 3, \dots \Big\} \cup \{-1\}$.\\
                $\overline{B} = A \cup [0, 1] = \{x \in \rats \mid 0 < x < 1\}
                \cup [0, 1] = [0, 1]$.
        \end{itemize}

    \newpage

    \item[3.2.6]
        \begin{itemize}
            \item[(a)]
                This is false.
                \\
                \\
                Let $S = (-\infty, \sqrt{3}) \cup (\sqrt{3}, +\infty)$. Then
                the union of two open sets $(-\infty, \sqrt{3})$ and
                $(\sqrt{3}, +\infty)$ is also open. Notice that $S$ contains
                all numbers in $\reals$ except for $\sqrt{3}$ which is
                irrational. Hence, $S$ contains all rational numbers. However,
                it does not contain $\sqrt{3}$ which means that it does not
                contain all real numbers.\\
                $\qed$

            \item[(b)]
                This is false.
                \\
                \\
                Let us define $S_n = [n, \infty)$. Then we get
                $\bigcap_n S_n = \emptyset$.

            \item[(c)]
                This is true.
                \\
                \\
                Let $S$ be a nonempty open set. Then, as $S$ is not empty,
                $\exists x$ s.t. $x \in S$. Now, since $S$ is also open,
                $\exists \epsilon > 0$ s.t. $V_\epsilon(x) \subseteq S$. Since
                $\rats$ is dense in $\reals$, $\exists r \in \rats$ s.t.
                $x - \epsilon < r < x + \epsilon$. Finally, we get
                $r \in V_\epsilon(x) \subseteq S$.\\
                $\qed$

            \item[(d)]
                This is false.
                \\
                \\
                Consider $S = \{\dfrac{1}{n} + \sqrt{3} \mid n \in \nats \}
                \cup \{\sqrt{3}\}$. Then, $S$ is a bounded infinite closed
                set, however, every number in $S$ is irrational.\\
                $\qed$

            \item[(e)]
                This is true.
                \\
                \\
                The Cantor set is the intersection of closed sets. Now, since
                the arbitrary intersection of closed sets is closed (proven
                during the class period, can be proved by taking complements
                and applying De Morgan's law), the Cantor set must be closed.
        \end{itemize}

    \newpage

    \item[3.2.8]
        \begin{itemize}
            \item[(a)]
                \textbf{$\overline{A \cup B}$ is sometimes open}. If we set $A
                = B = \reals$, then $A \cup B$ is open. On the other hand, if
                $A = B = [0, 1]$, then $\overline{A \cup B} = [0, 1]$ which
                means that $\overline{A \cup B}$ is not open.
                \\
                \\
                \textbf{$\overline{A \cup B}$ is definitely closed} since for
                any set $S$, $\overline{S}$ is definitely closed.
                \\
                \\
                \textbf{$\overline{A \cup B}$ is sometimes both open and closed
                (aka \textit{clopen})}. If $A = B = \reals$, then $\overline{A
                \cup B} = \reals$ which is both open and closed. However, if we
                set $A = B = [0, 1]$, then $\overline{A \cup B} = [0, 1]$ which
                is not open and thus, $\overline{A \cup B}$ is not both open
                and closed.
                \\
                \\
                \textbf{$\overline{A \cup B}$ can never be neither open nor
                closed} as $\overline{A \cup B}$ is always closed.

            \item[(b)]
                \textbf{$\overline{A \setminus B}$ is definitely open} since
                $\overline{A \setminus B} = \overline{A \cap B^c}$. Then, since
                $B$ is closed, its complement is open. Finally, as both $A$ and
                $B^c$ are open, $\overline{A \setminus B}$ is open too.
                \\
                \\
                \textbf{$\overline{A \setminus B}$ is sometimes closed}. If we
                let $A = \reals$ and $B = \emptyset$, then $A \setminus B =
                \reals$ is closed. However, if we let $A = (0, 5)$ and $B = [1,
                6]$, then $A \setminus B = (0, 1)$ which is not closed.
                \\
                \\
                \textbf{$\overline{A \setminus B}$ is sometimes both open and
                closed (aka \textit{clopen})}. If we let $A = \reals$ and $B =
                \emptyset$, then $A \setminus B = \reals$ is closed. Hence, in
                this case, $\overline{A \setminus B}$ is both open and closed
                (open since we showed in $(a)$ that it is always open).
                However, if we let $A = (0, 5)$ and $B = [1, 6]$, then $A
                \setminus B = (0, 1)$ is not closed.
                \\
                \\
                \textbf{$\overline{A \setminus B}$ can never be neither open
                nor closed} as $\overline{A \setminus B}$ is always open.

            \item[(c)]
                \textbf{$(A^c \cup B)^c$ is definitely open}. This is the case
                since if $A$ is open, $A^c$ is closed. Now, as $B$ is closed
                $A^c \cup B$ is also closed. Hence, $(A^c \cup B)^c$ is open.
                \\
                \\
                \textbf{$(A^c \cup B)^c$ is sometimes closed}. If we let $A = B
                = \reals$, then $(A^c \cup B)^c = \emptyset$ which is closed.
                On the other hand, if we let $A = (0, 1)$ and $B = \emptyset$,
                then $(A^c \cup B) = (0, 1)$ which is not closed.
                \\
                \\
                \textbf{$(A^c \cup B)^c$ is sometimes both open and
                closed (aka \textit{clopen})}. If we let $A = \reals$ and $B =
                \emptyset$, then $A \setminus B = \reals$ is closed. Hence, in
                this case, $\overline{A \setminus B}$ is both open and closed
                (open since we showed in $(a)$ that it is always open).
                However, if we let $A = (0, 5)$ and $B = [1, 6]$, then $A
                \setminus B = (0, 1)$ is not closed.
                \\
                \\
                \textbf{$(A^c \cup B)^c$ can never be neither open nor closed}
                as $(A^c \cup B)^c$ always open.

            \item[(d)]
                Notice that $(A \cap B) \cup (A^c \cap B) = B$.
                \\
                \\
                \textbf{$(A \cap B) \cup (A^c \cap B) = B$ is sometimes open}.
                If $B = \reals$ then it is open. However, if $B = [0, 1]$, it
                is not open.
                \\
                \\
                \textbf{$(A \cap B) \cup (A^c \cap B) = B$, by definition, is
                definitely closed}.
                \\
                \\
                \textbf{$(A \cap B) \cup (A^c \cap B) = B$ is sometimes both
                open and closed (aka \textit{clopen})}. If $B = \reals$, then
                it is both open and closed. However, if $B = [0, 1]$, then it
                is not open (but it is still open as it is always open).
                \\
                \\
                \textbf{$(A \cap B) \cup (A^c \cap B) = B$ can never be neither
                open nor closed} as $(A^c \cup B)^c$ always open.

            \item[(e)]
                Notice that since $A$ is open, $A^c$ is closed and thus,
                $\overline{A^c} = A^c$. Hence, we get $\overline{A}^c \cap A^c
                = \overline{A}^c$.
                \\
                \\
                \textbf{$\overline{A}^c \cap A^c = \overline{A}^c$ is
                definitely open}.  This is by definition. As $A$ is open,
                $\overline{A}$ is closed and its complement $\overline{A}^c$
                must be open.
                \\
                \\
                \textbf{$\overline{A}^c \cap A^c = \overline{A}^c$ is sometimes
                closed}.  If we let $A = \emptyset$, then $\overline{A}^c =
                \reals$ is closed (and open as well). However, if $A = (0, 1)$,
                then $\overline{A}^c = (-\infty, 0) \cup (1, +\infty)$ which is
                not closed.
                \\
                \\
                \textbf{$\overline{A}^c \cap A^c = \overline{A}^c$ is sometimes
                both open and closed (aka \textit{clopen})}. If we let $A =
                \emptyset$, then $\overline{A}^c = \reals$ which is both open
                and closed. However, if $A = (0, 1)$, then $\overline{A}^c =
                (-\infty, 0) \cup (1, +\infty)$ which is not closed.
                \\
                \\
                \textbf{$\overline{A}^c \cap A^c = \overline{A}^c$ can never be
                neither open nor closed} as $(A^c \cup B)^c$ always open.
        \end{itemize}

    \newpage

    \item[3.2.14]
        \begin{itemize}
            \item[(a)]
                Let us first show that $E$ is closed if and only if
                $\overline{E} = E$. We will first prove this directly and then
                prove its converse.

                Suppose that $E$ is closed. Then $E$ must contain all of its
                limit points. Let us denote the set of all limit points of $E$
                as $L$. Then it follows that $L \subseteq E$ but $\overline{E}
                = E \cup L$. Thus, $\overline{E} = E$.\\
                $\qed$

                Conversely, suppose that $\overline{E} = E$. Then it follows
                that $E$ contains all of its limit points since $\overline{E}$
                contains all of the limit points of $E$. Hence, $E$ must be
                closed.\\
                $\qed$
                
                Finally, we have shown that $E$ is closed if and only if
                $\overline{E} = E$.\\
                $\qed$
                \\
                \\
                Let us now show that $E$ is open if and only if $E^\circ = E$.
                Similarly, we will first prove this statement directly and then
                prove its convers.

                Suppose $E$ is open. Then $\forall x \in E, \exists
                V_\epsilon(x) \subseteq E$. It follows that $x \in E^\circ$ and
                thus, $E \subseteq E^\circ$. On the other hand, by definition,
                $E^\circ \subseteq E$. Hence, $E^\circ = E$\\
                $\qed$

                Conversely suppose $E^\circ = E$. Then, by definition, since
                $E^\circ$ is open, $E$ must be too.\\
                $\qed$

                Finally, we have shown that $E$ is open if and only if $E^\circ
                = E$.\\
                $\qed$

            \item[(b)]
                Let us first show that $\overline{E}^c = (E^c)^\circ$. In order
                to prove this, we first show that $\overline{E}^c \subseteq
                (E^c)^\circ$ and then show that $(E^c)^\circ \subseteq
                \overline{E}^c$.

                Let $x \in \overline{E}^c$. Then, as $\overline{E}^c$ is open,
                $\exists V_{\epsilon}(x) \subseteq \overline{E}^c$. Now, since
                $E \subseteq \overline{E}$, it follows that $\overline{E}^c
                \subseteq \overline{E^c}$.

                Now, let $x \in (E^c)^\circ$. Then $\exists V_\epsilon(x)
                \subseteq \overline{E}^c \subseteq E^c$. It follows that
                $V_\epsilon(x) \cap E = \emptyset$.

                Now, notice that we $V_\epsilon(x) \cap \overline{E} =
                \emptyset$. To prove this, suppose, for the sake of
                contradiction, that $V_\epsilon(x) \cap \overline{E} \neq
                \emptyset$. Then $\exists y \in \overline{E}$ s.t. $y \in
                V_\epsilon(x)$ (with $V_\epsilon(x)$ being open). Then there
                must exists some $\epsilon$-neighborhood of $y$ that is
                contained in $V_\epsilon(x)$. However, $\epsilon$-neighborhood
                of $y$ contains points of $E$ which contradicts $V_\epsilon(x)
                \cap E = \emptyset$ (which we have already shown). Therefore,
                $V_\epsilon(x) \cap \overline{E} = \emptyset$. Hence,
                $V_\epsilon(x) \subseteq \overline{E}^c$ and it follows that
                $(E^c)^\circ \subseteq \overline{E}^c$.

                Finally, we have shown both $\overline{E}^c \subseteq
                (E^c)^\circ$ and then show that $(E^c)^\circ \subseteq
                \overline{E}^c$. Hence, $\overline{E}^c = (E^c)^\circ$.\\
                $\qed$
                \\
                \\
                Let us now show that $\overline{E}^c = (E^c)^\circ$. Recall
                that we have already shown $\overline{E}^c = (E^c)^\circ$.  If
                we simply substitute $E$ with $E^c$ in this equality, we get
                $\overline{E^c}^c = ((E^c)^c)^\circ = E^\circ$. Taking the
                complement of both sides gives us $\overline{E^c} =
                \overline{E^\circ}^c$.\\
                $\qed$
        \end{itemize}
\end{itemize}

%%%%%%%%%%%%%%%%%%%%%%%%%%%%%%%%%%%%%%%%%%%%%%%%%%%%%%%%%%%%%%%%%%%%%%%%%%%%%%%
% The End of the Document
%%%%%%%%%%%%%%%%%%%%%%%%%%%%%%%%%%%%%%%%%%%%%%%%%%%%%%%%%%%%%%%%%%%%%%%%%%%%%%%

\end{document}
