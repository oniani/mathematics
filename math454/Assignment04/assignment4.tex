%%%%%%%%%%%%%%%%%%%%%%%%%%%%%%%%%%%%%%%%%%%%%%%%%%%%%%%%%%%%%%%%%%%%%%%%%%%%%%%
%
% Filename: assignment4.tex
% Author:   David Oniani
% Modified: December 14, 2020
%  _         _____   __  __
% | |    __ |_   _|__\ \/ /
% | |   / _` || |/ _ \\  /
% | |__| (_| || |  __//  \
% |_____\__,_||_|\___/_/\_\
%
%%%%%%%%%%%%%%%%%%%%%%%%%%%%%%%%%%%%%%%%%%%%%%%%%%%%%%%%%%%%%%%%%%%%%%%%%%%%%%%

%%%%%%%%%%%%%%%%%%%%%%%%%%%%%%%%%%%%%%%%%%%%%%%%%%%%%%%%%%%%%%%%%%%%%%%%%%%%%%%
% Document Definition
%%%%%%%%%%%%%%%%%%%%%%%%%%%%%%%%%%%%%%%%%%%%%%%%%%%%%%%%%%%%%%%%%%%%%%%%%%%%%%%

\documentclass[11pt]{article}

%%%%%%%%%%%%%%%%%%%%%%%%%%%%%%%%%%%%%%%%%%%%%%%%%%%%%%%%%%%%%%%%%%%%%%%%%%%%%%%
% Packages and Related Settings
%%%%%%%%%%%%%%%%%%%%%%%%%%%%%%%%%%%%%%%%%%%%%%%%%%%%%%%%%%%%%%%%%%%%%%%%%%%%%%%

% Global, document-wide settings
\usepackage[margin=1in]{geometry}
\usepackage[utf8]{inputenc}
\usepackage[english]{babel}

% Other packages
\usepackage{booktabs}
\usepackage{hyperref}
\usepackage{mathtools}
\usepackage{amsthm}
\usepackage{amssymb}
\usepackage[cache=false]{minted}

%%%%%%%%%%%%%%%%%%%%%%%%%%%%%%%%%%%%%%%%%%%%%%%%%%%%%%%%%%%%%%%%%%%%%%%%%%%%%%%
% Command Definitions and Redefinitions
%%%%%%%%%%%%%%%%%%%%%%%%%%%%%%%%%%%%%%%%%%%%%%%%%%%%%%%%%%%%%%%%%%%%%%%%%%%%%%%

% Nice-looking underline
\newcommand\und[1]{\underline{\smash{#1}}}

% Line spacing is 1.5
\renewcommand{\baselinestretch}{1.5}

% Absolute value
\DeclarePairedDelimiter\abs{\lvert}{\rvert}%

% Ceiling
\DeclarePairedDelimiter{\ceil}{\lceil}{\rceil}

% Floor
\DeclarePairedDelimiter\floor{\lfloor}{\rfloor}

% % Naturals, Reals, Integers, and Rationals, 
\newcommand{\nats}{\mathbb{N}}
\newcommand{\reals}{\mathbb{R}}
\newcommand{\preals}{\mathbb{R^+}}
\newcommand{\nreals}{\mathbb{R^-}}
\newcommand{\ints}{\mathbb{Z}}
\newcommand{\pints}{\mathbb{Z^+}}
\newcommand{\nints}{\mathbb{Z^-}}
\newcommand{\rats}{\mathbb{Q}}
\newcommand{\prats}{\mathbb{Q^+}}
\newcommand{\nrats}{\mathbb{Q^-}}
\newcommand{\irrats}{\mathbb{I}}
\newcommand{\pirrats}{\mathbb{I^+}}
\newcommand{\nirrats}{\mathbb{I^-}}

%%%%%%%%%%%%%%%%%%%%%%%%%%%%%%%%%%%%%%%%%%%%%%%%%%%%%%%%%%%%%%%%%%%%%%%%%%%%%%%
% Miscellaneous
%%%%%%%%%%%%%%%%%%%%%%%%%%%%%%%%%%%%%%%%%%%%%%%%%%%%%%%%%%%%%%%%%%%%%%%%%%%%%%%

% Setting stuff
\setlength{\parindent}{0pt}  % Remove indentations from paragraphs

% PDF information and nice-looking urls
\hypersetup{%
  pdfauthor={David Oniani},
  pdftitle={Real Analysis},
  pdfsubject={Mathematics, Real Analysis, Real Numbers},
  pdfkeywords={Mathematics, Real Analysis, Real Numbers},
  pdflang={English},
  colorlinks=true,
  linkcolor={black!50!blue},
  citecolor={black!50!blue},
  urlcolor={black!50!blue}
}

%%%%%%%%%%%%%%%%%%%%%%%%%%%%%%%%%%%%%%%%%%%%%%%%%%%%%%%%%%%%%%%%%%%%%%%%%%%%%%%
% Author(s), Title, and Date
%%%%%%%%%%%%%%%%%%%%%%%%%%%%%%%%%%%%%%%%%%%%%%%%%%%%%%%%%%%%%%%%%%%%%%%%%%%%%%%

% Author(s)
\author{David Oniani\\
        Luther College\\
        \href{mailto:oniada01@luther.edu}{oniada01@luther.edu}}

% Title
\title{\rule{\paperwidth - 150pt}{1pt}\textbf{\\\textit{Real Analysis}\\}\rule
{\paperwidth - 150pt}{1pt}\\\textbf{Assignment \textnumero4}\\{\normalsize
Instructor: Dr. Eric Westlund}}

% Date
\date{\today}

%%%%%%%%%%%%%%%%%%%%%%%%%%%%%%%%%%%%%%%%%%%%%%%%%%%%%%%%%%%%%%%%%%%%%%%%%%%%%%%
% Beginning of the Document
%%%%%%%%%%%%%%%%%%%%%%%%%%%%%%%%%%%%%%%%%%%%%%%%%%%%%%%%%%%%%%%%%%%%%%%%%%%%%%%

\begin{document}
\maketitle

%%%%%%%%%%%%%%%%%%%%%%%%%%%%%%%%%%%%%%%%%%%%%%%%%%%%%%%%%%%%%%%%%%%%%%%%%%%%%%%
%
% Homework
%
% 2.6 # 2, 3
% 2.7 # 5, 8, 9
% 2.8 # 1
%
%%%%%%%%%%%%%%%%%%%%%%%%%%%%%%%%%%%%%%%%%%%%%%%%%%%%%%%%%%%%%%%%%%%%%%%%%%%%%%%

\begin{itemize}
    \item[2.6.2]
        \begin{itemize}
            \item[(a)]
                Such sequence exists. $a_n = \dfrac{(-1)^n}{n}$ is a Cauchy
                sequence that is not monotone since it alternates, but
                converges to $0$.

            \item[(b)]
                Per \textbf{Lemma 2.6.3}, such sequence cannot exist.

            \item[(c)]
                Such sequences cannot exist. A divergent monotone sequence
                implies that the sequence is unbounded. Unbounded and monotone
                sequence, on the other hand, cannot contain a convergent
                subsequence. Hence, by \textbf{Cauchy Criterion (Theorem
                2.6.4)}, it cannot contain a Cauchy subsequence.

            \item[(d)]
                Such sequence exists. Let us define
                \begin{equation*}
                    a_n =
                    \begin{cases}
                        n \text{ if } n \in \nats \text{ is odd}\\
                        0 \text{ if } n \in \nats \text{ is even}
                    \end{cases}
                \end{equation*}
                Then it is easy to see that sequence is unbounded since
                $\forall k \in \nats, a_{2k + 1} = 2k + 1 > k$. On the other
                hand the subsequence formed by the even-termed elements is
                comprised of only zeros and hence, converges to $0$. Therefore,
                the subsequence is Cauchy. Hence, we found an unbounded
                sequence containing a subsequence that is Cauchy.
        \end{itemize}

    \item[2.6.3]
        \begin{enumerate}
            \item[(a)]
                Since $x_n$ and $y_n$ are both Cauchy sequences, $\forall
                \epsilon > 0, \exists N_1, N_2 \in \nats$ s.t. $\abs{x_{m_1} -
                x_{n_1}} < \dfrac{\epsilon}{2}$ and $\abs{y_{m_2} - y_{n_2}} <
                \dfrac{\epsilon}{2}$ with $m_1, n_1 \geq N_1$ and $m_2, n_2
                \geq N_2$. Then let $N = \max{\{N_1, N_2\}}$. It follows that
                $\forall m, n \geq N, \abs{x_m - x_n} < \dfrac{\epsilon}{2}$
                and $\abs{y_m - y_n} < \dfrac{\epsilon}{2}$. Finally, using the
                triangle inequality, we get:
                \begin{align*}
                    \abs{(x_m + y_m) - (x_n + y_n)} &=
                    \abs{(x_m - x_n) - (y_m - y_n)}\\
                    &\leq \abs{x_m - x_n} + \abs{y_m - y_n}\\
                    &< \dfrac{\epsilon}{2} + \dfrac{\epsilon}{2} = \epsilon
                \end{align*}
                Thus, we conclude that $(x_n + y_n)$ is a Cauchy sequence.\\
                $\qed$

            \item[(b)]
                Since $x_n$ and $y_n$ are both Cauchy sequence, they are also
                bounded and hence, $\exists X, Y  >0$ s.t. $\forall n \in
                \nats, \abs{x_n} < X$ and $\abs{y_n} < Y$. Additionally,
                $\forall \epsilon > 0, \exists N_1, N_2 \in \nats$ s.t.
                $\abs{x_{m_1} - x_{n_1}} < \dfrac{\epsilon}{2}$ and
                $\abs{y_{m_2} - y_{n_2}} < \dfrac{\epsilon}{2}$ with $m_1, n_1
                \geq N_1$ and $m_2, n_2 \geq N_2$. Then let $N = \max{\{N_1,
                N_2\}}$. We get:
                \begin{align*}
                    \abs{x_my_m - x_ny_n} &=
                    \abs{x_m(y_m - y_n) + y_n(x_m - x_n)}\\
                    &\leq \abs{x_m}\abs{y_m - y_n} +
                    \abs{y_n}\abs{x_m - x_n}\\
                    &< \dfrac{X\epsilon}{2} + \dfrac{Y\epsilon}{2} =
                    \dfrac{\epsilon}{2}(X + Y).
                \end{align*}
                Thus, we found $\epsilon^{\prime} = \dfrac{\epsilon}{2}(X + Y)
                > 0$ with $N$ s.t. $m, n \geq N$ and $\abs{x_my_m - x_ny_n} <
                \epsilon^\prime$.  Hence, we conclude that $(x_ny_n)$ is a
                Cauchy sequence.\\
                $\qed$
        \end{enumerate}

    \item[2.7.5]
        We have to prove that the series $\sum_{n = 1}^\infty 1/n^p$ converges
        if and only if $p > 1$. By \textbf{Cauchy Condensation Test}, the
        series converges if and only if $\sum_{n = 1}^\infty
        \dfrac{2^n}{2^{np}} = \sum_{n = 1}^\infty 2^{n(1 - p)}$ converges. Now,
        we need to get $2^{n(1 - p)}$ less than $1$ as otherwise, the sequence
        will diverge (all terms will be defferent and $\geq 1$). $2^{n(1 - p)}
        < 1$ if and only if $n(1 - p) < 0$. Since $n \in \nats$, we can safely
        divide both sides of the inequality by $n$. We get $1 - p < 0$ and
        thus, $p > 1$.  Hence, we have proven that the series $\sum_{n =
        1}^\infty 1/n^p$ converges if and only if $p > 1$.
        \\\\
        \textit{NOTE: We did not take the classic ``prove it directly and prove
        its converse'' approach since every statement used in the proof was if
        and only if statement. \textbf{Cauchy Condensation Test} is if and only
        if and $2^{n(1 - p)} < 1$ when $n(1 - p) < 0$ is if and only if}.

    \item[2.7.8]
        \begin{itemize}
            \item[(a)]
                True. By \textbf{Theorem 2.7.3}, $\sum{a_n}$ converges
                absolutely. It follows that $\lim{a_n} = 0$. Thus, $\abs{a_n}$
                is bounded and $\exists B > 0$ s.t. $\forall n \in \nats,
                \abs{a_n} \leq B$. Now, by \textbf{Algebraic Limit Theorem for
                Series (Theorem 2.7.1)}, $\sum{B\abs{a_n}}$ and $B\abs{a_n}
                \geq a^2_n$. Finally, per \textbf{Comparison Test (Theorem
                2.7.4)}, $\sum{a^2_n}$ converges.\\
                $\qed$

            \item[(b)]
                False. Let $a_n = b_n = \dfrac{(-1)^n}{\sqrt{n}}$. Now,
                $\lim{b_n} = 0$ and thus, $(b_n)$ converges. Additionally,
                $\lim{\dfrac{1}{\sqrt{n}}} = 0$ and $\dfrac{1}{\sqrt{n}}$ is
                decreasing. Hence, by \textbf{Alternating Series Test (Theorem
                2.7.7)}, we get that $\sum{a_n}$ converges. Thus, both
                $\sum{a_n}$ and $\lim{(b_n)}$ converge. However, $a_nb_n =
                \dfrac{1}{n}$ which is harmonic series and it does not
                converge.

            \item[(c)]
                True. Suppose, for the sake of contradiction, that $\sum{a_n}$
                converges conditionally and $\sum{n^2a_n}$ converges. Then,
                $\lim{n^2a_n} = 0$ and $\exists N$ s.t. $\forall n \geq N,
                \abs{n^2a_n} < 1$. We then get $\abs{a_n} < \dfrac{1}{n^2}$.
                Now, per \textbf{Comparison Test (Theorem 2.7.4)}, $\sum{a_n}$
                converges absolutely and we face the contradiction since
                $\sum{a_n}$ converges conditionally. Thus, $\sum{a_n}$
                converges conditionally, then $\sum{n^2a_n}$ diverges.\\
                $\qed$
        \end{itemize}

    \item[2.7.9]
        \begin{itemize}
            \item[(a)]
                Suppose $r < r^\prime < 1$. Since $\lim_{n \to
                \infty}\abs{\dfrac{a_{n + 1}}{a_n}} = r < 1$, let $\epsilon =
                r^\prime - r > 0$. Then $\exists N \in \nats$ s.t. $\forall n
                \geq \nats$, the following is true:
                \begin{align*}
                    &\abs{\abs{\dfrac{a_{n + 1}}{a_n}} - r} < \epsilon\\
                    &\abs{\dfrac{a_{n + 1}}{a_n}} < r + \epsilon\\
                    &\abs{\dfrac{a_{n + 1}}{a_n}} < r + r^\prime - r\\
                    &\abs{\dfrac{a_{n + 1}}{a_n}} < r^\prime\\
                    &\abs{a_{n + 1}} \leq \abs{{a_n}}r^\prime
                \end{align*}
                $\qed$

            \item[(b)]
                Since $\abs{r^\prime} < 1$, $\sum(r^\prime)^n$ is a convergent
                geometric series. Then, by \textbf{Algebraic Limit Theorem for
                Series (Theorem 2.7.1)}, $\abs{a_N}\sum{(r^\prime)^n}$.\\
                $\qed$

            \item[(c)]
                Notice that $\sum{\abs{a_n}} = \sum_{n = 1}^N \abs{a_n}  +
                \sum_{n = N + 1}^\infty\abs{a_n}$. Now, it is easy to see that
                $N\sum_{n = N + 1}^\infty a_n$ converges by \textbf{Comparison
                Test (Theorem 2.7.4)} since $N\sum_{n = N + 1}^\infty \leq
                \abs{a_N}\sum_{n = N + 1}^\infty {r^\prime}^{n - N}$ (the fact
                that $\abs{a_N}\sum_{n = N + 1}^\infty {r^\prime}^{n - N}$
                converges was proved in part (b) of this exercise). Hence,
                $\sum{a_n}$ converges as well. Finally, as $\sum{\abs{a_n}}$
                converges absolutely, by \textbf{Absolute Convergence Test
                (Theorem 2.7.6)}, $\sum{a_n}$ conveges absolutely as well.\\
                $\qed$
        \end{itemize}

    \item[2.8.1]
        From \textbf{Section 2.1}, we know

        \begin{equation*}
            (a_{ij}) =
            \begin{cases}
                \dfrac{1}{2^{j - i}} \text{ if } j > i\\
                -1 \text{ if } j = i\\
                0 \text{ if } j < i
            \end{cases}
        \end{equation*}

        Now, if we set $j = 1$, it is easy to see that $a_{i1} = (-1,
        \dfrac{1}{2}, \dfrac{1}{4}, \dots)$ and excluding the first term
        ($-1$), $a_{i1}$ is a sequence whose sum converges. Recall that
        the formula for the sum is $\dfrac{b_nq - b_1}{q - 1}$ where $b_1$ is
        the first term, $b_n$ is the last term, and $q$ is the
        quotient/ratio (current element over previous element). Thus, we get:
        $$\sum_{i = 1}^k a_{i1} =  -1 + \dfrac{1}{2} + \dfrac{1}{4} + \dots +
        \dfrac{1}{2^{n - 1}} = -1 + \dfrac{\frac{1}{2^{n - 1}} \times
        \frac{1}{2} - \frac{1}{2}}{\frac{1}{2} - 1} = -1 + \dfrac{\frac{1}{2^{n
        - 1}} \times \frac{1}{2} - \frac{1}{2}}{-\frac{1}{2}} = -1 + 1 -
        \dfrac{1}{2^{n - 1}} = -\dfrac{1}{2^{n - 1}}$$
        In general, $\forall j < k$, we have:
        $$\sum_{i = 1}^k a_{ij} =  -1 + \dfrac{1}{2} + \dfrac{1}{4} + \dots +
        \dfrac{1}{2^{n - j}} = -1 + \dfrac{\frac{1}{2^{n - j}} \times
        \frac{1}{2} - \frac{1}{2}}{\frac{1}{2} - 1} = -1 + \dfrac{\frac{1}{2^{n
        - j}} \times \frac{1}{2} - \frac{1}{2}}{-\frac{1}{2}} = -1 + 1 -
        \dfrac{1}{2^{n - j}} = -\dfrac{1}{2^{n - j}}$$
        Finally, we get:
        $$s_{nn} = \sum_{j = 1}^{n}\sum_{i = 1}^{n} a_{ij} = -1 + \sum_{j =
        1}^{n - 1} -\dfrac{1}{2^{n - j}} = -1 - (1 - \dfrac{1}{2^{n - 1}}) = -2
        + \dfrac{1}{2^{n - 1}}$$
        Now, since $\lim_{n \to \infty} \dfrac{1}{2^{n - 1}} = 0$, we get that
        $\sum_{i, j}^\infty a_{ij} = \lim_{n \to \infty} s_{nn} = \lim{n \to
        \infty} -2 + \dfrac{1}{2^{n - 1}} = -2 + \lim_{n \to \infty}
        \dfrac{1}{2^{n - 1}} = -2 + 0 = -2$. Hence, we get $s_{nn} = -2$.

        The iterated sums would give us $\sum_{i = 1}^\infty\sum_{j = 1}^\infty
        a_{ij} = -1 + \sum_{i = 2}^\infty -\dfrac{1}{2^{i - 1}}$.  Now, recall
        that the sum can be computed by the infinite geometric series formula
        $\dfrac{b_1}{q - 1}$. Finally, we get $\sum_{i = 1}^\infty\sum_{j =
        1}^\infty a_{ij} = -1 + \dfrac{-\frac{1}{2}}{-\frac{1}{2}} = -1 + 1 =
        0$. Similarly, $\sum_{j = 1}^\infty\sum_{i = 1}^\infty a_{ij} = \sum_{j
        = 1}^\infty (-1 + \sum_{i = 1}^\infty\dfrac{1}{2^i}) = \sum_{j =
        1}^\infty 0 = 0$.
\end{itemize}

%%%%%%%%%%%%%%%%%%%%%%%%%%%%%%%%%%%%%%%%%%%%%%%%%%%%%%%%%%%%%%%%%%%%%%%%%%%%%%%
% The End of the Document
%%%%%%%%%%%%%%%%%%%%%%%%%%%%%%%%%%%%%%%%%%%%%%%%%%%%%%%%%%%%%%%%%%%%%%%%%%%%%%%

\end{document}
