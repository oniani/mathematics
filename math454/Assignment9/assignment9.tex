%%%%%%%%%%%%%%%%%%%%%%%%%%%%%%%%%%%%%%%%%%%%%%%%%%%%%%%%%%%%%%%%%%%%%%%%%%%%%%%
%
% Filename: assignment9.tex
% Author:   David Oniani
% Modified: January 03, 2020
%  _         _____   __  __
% | |    __ |_   _|__\ \/ /
% | |   / _` || |/ _ \\  /
% | |__| (_| || |  __//  \
% |_____\__,_||_|\___/_/\_\
%
%%%%%%%%%%%%%%%%%%%%%%%%%%%%%%%%%%%%%%%%%%%%%%%%%%%%%%%%%%%%%%%%%%%%%%%%%%%%%%%

%%%%%%%%%%%%%%%%%%%%%%%%%%%%%%%%%%%%%%%%%%%%%%%%%%%%%%%%%%%%%%%%%%%%%%%%%%%%%%%
% Document Definition
%%%%%%%%%%%%%%%%%%%%%%%%%%%%%%%%%%%%%%%%%%%%%%%%%%%%%%%%%%%%%%%%%%%%%%%%%%%%%%%

\documentclass[11pt]{article}

%%%%%%%%%%%%%%%%%%%%%%%%%%%%%%%%%%%%%%%%%%%%%%%%%%%%%%%%%%%%%%%%%%%%%%%%%%%%%%%
% Packages and Related Settings
%%%%%%%%%%%%%%%%%%%%%%%%%%%%%%%%%%%%%%%%%%%%%%%%%%%%%%%%%%%%%%%%%%%%%%%%%%%%%%%

% Global, document-wide settings
\usepackage[margin=1in]{geometry}
\usepackage[utf8]{inputenc}
\usepackage[english]{babel}

% Other packages
\usepackage{booktabs}
\usepackage{hyperref}
\usepackage{mathtools}
\usepackage{amsthm}
\usepackage{amssymb}
\usepackage[cache=false]{minted}

%%%%%%%%%%%%%%%%%%%%%%%%%%%%%%%%%%%%%%%%%%%%%%%%%%%%%%%%%%%%%%%%%%%%%%%%%%%%%%%
% Command Definitions and Redefinitions
%%%%%%%%%%%%%%%%%%%%%%%%%%%%%%%%%%%%%%%%%%%%%%%%%%%%%%%%%%%%%%%%%%%%%%%%%%%%%%%

% Nice-looking underline
\newcommand\und[1]{\underline{\smash{#1}}}

% Line spacing is 1.5
\renewcommand{\baselinestretch}{1.5}

% Absolute value
\DeclarePairedDelimiter\abs{\lvert}{\rvert}%

% Absolute value big
\DeclarePairedDelimiter\absb{\Big\lvert}{\Big\rvert}%

% Ceiling
\DeclarePairedDelimiter{\ceil}{\lceil}{\rceil}

% Floor
\DeclarePairedDelimiter\floor{\lfloor}{\rfloor}

% % Naturals, Reals, Integers, and Rationals, 
\newcommand{\nats}{\mathbb{N}}
\newcommand{\reals}{\mathbb{R}}
\newcommand{\preals}{\mathbb{R^+}}
\newcommand{\nreals}{\mathbb{R^-}}
\newcommand{\ints}{\mathbb{Z}}
\newcommand{\pints}{\mathbb{Z^+}}
\newcommand{\nints}{\mathbb{Z^-}}
\newcommand{\rats}{\mathbb{Q}}
\newcommand{\prats}{\mathbb{Q^+}}
\newcommand{\nrats}{\mathbb{Q^-}}
\newcommand{\irrats}{\mathbb{I}}
\newcommand{\pirrats}{\mathbb{I^+}}
\newcommand{\nirrats}{\mathbb{I^-}}

%%%%%%%%%%%%%%%%%%%%%%%%%%%%%%%%%%%%%%%%%%%%%%%%%%%%%%%%%%%%%%%%%%%%%%%%%%%%%%%
% Miscellaneous
%%%%%%%%%%%%%%%%%%%%%%%%%%%%%%%%%%%%%%%%%%%%%%%%%%%%%%%%%%%%%%%%%%%%%%%%%%%%%%%

% Setting stuff
\setlength{\parindent}{0pt}  % Remove indentations from paragraphs

% PDF information and nice-looking urls
\hypersetup{%
  pdfauthor={David Oniani},
  pdftitle={Real Analysis},
  pdfsubject={Mathematics, Real Analysis, Real Numbers},
  pdfkeywords={Mathematics, Real Analysis, Real Numbers},
  pdflang={English},
  colorlinks=true,
  linkcolor={black!50!blue},
  citecolor={black!50!blue},
  urlcolor={black!50!blue}
}

%%%%%%%%%%%%%%%%%%%%%%%%%%%%%%%%%%%%%%%%%%%%%%%%%%%%%%%%%%%%%%%%%%%%%%%%%%%%%%%
% Author(s), Title, and Date
%%%%%%%%%%%%%%%%%%%%%%%%%%%%%%%%%%%%%%%%%%%%%%%%%%%%%%%%%%%%%%%%%%%%%%%%%%%%%%%

% Author(s)
\author{David Oniani\\
        Luther College\\
        \href{mailto:oniada01@luther.edu}{oniada01@luther.edu}}

% Title
\title{\rule{\paperwidth - 150pt}{1pt}\textbf{\\\textit{Real Analysis}\\}\rule
{\paperwidth - 150pt}{1pt}\\\textbf{Assignment \textnumero9}\\{\normalsize
Instructor: Dr. Eric Westlund}}

% Date
\date{\today}

%%%%%%%%%%%%%%%%%%%%%%%%%%%%%%%%%%%%%%%%%%%%%%%%%%%%%%%%%%%%%%%%%%%%%%%%%%%%%%%
% Beginning of the Document
%%%%%%%%%%%%%%%%%%%%%%%%%%%%%%%%%%%%%%%%%%%%%%%%%%%%%%%%%%%%%%%%%%%%%%%%%%%%%%%

\begin{document}
\maketitle

%%%%%%%%%%%%%%%%%%%%%%%%%%%%%%%%%%%%%%%%%%%%%%%%%%%%%%%%%%%%%%%%%%%%%%%%%%%%%%%
%
% Homework
%
% 5.2 # 3, 7
% 5.3 # 1a, 3, 7, 11a
%
%%%%%%%%%%%%%%%%%%%%%%%%%%%%%%%%%%%%%%%%%%%%%%%%%%%%%%%%%%%%%%%%%%%%%%%%%%%%%%%

\begin{itemize}
    \item[5.2.3]
        \begin{itemize}
            \item[(a)]
                \begin{equation*}
                    \boxed{
                        h^\prime(x) = \lim_{x \to c}{\frac{h(x) - h(c)}{x - c}}
                                    = \lim_{x \to c}{\frac{\frac{1}{x} - \frac{1}{c}}{x - c}}
                                    = \lim_{x \to c}{-\frac{1}{cx}}
                                    = -\frac{1}{x^2}
                        }
                        \qed
                \end{equation*}

            \item[]
            \item[]
            \item[]
            \item[]
            \item[]

            \item[(b)]
                Assuming $g(c) \neq 0$, we have:
                \begin{equation*}
                    \boxed{
                        \Big(\frac{f}{g}\Big)^\prime(c) = f^\prime(c)\frac{1}{g(c)} + \Big(- \frac{1}{(g(c))^2}g^\prime(c)f(c)\Big)
                                                        = \frac{f^\prime(c)g(c) - g^\prime(c)f(c)}{(g(c))^2}
                        }
                        \qed
                \end{equation*}

            \item[(c)]
                Assuming $g(c) \neq 0$, we have:
                \begin{align*}
                    \Big(\frac{f}{g}\Big)^\prime(c) &= \lim_{x \to c}{\frac{\Big(\frac{f}{g}\Big)(x) - \Big(\frac{f}{g}\Big)(c)}{x - c}}\\
                                                    &= \lim_{x \to c}{\frac{\frac{f(x)}{g(x)} - \frac{f(c)}{g(x)} + \frac{f(c)}{g(x)} - \frac{f(c)}{g(c)}}{x - c}}\\
                                                    &= \lim_{x \to c}{\frac{\frac{f(x)}{g(x)} - \frac{f(c)}{g(c)}}{x - c}}\\
                                                    &= \lim_{x \to c}{\frac{f(x)g(c) - f(c)g(x)}{g(x)g(c)(x - c)}}\\
                                                    &= \lim_{x \to c}{\frac{g(c)\Big(f(x) - f(c)\Big) - f(c)\Big(g(x) - g(c)\Big)}{g(x)g(c)(x - c)}}\\
                                                    &= \lim_{x \to c}{\frac{g(c)}{g(x)g(c)}} \times \lim_{x \to c}{\frac{f(x) - f(c)}{x - c}} - \lim_{x \to c}{\frac{f(c)}{g(x)g(c)}} \times \lim_{x \to c}{\frac{g(x) - g(c)}{x - c}}\\
                                                    &= \frac{g(c)}{\Big(g(c)\Big)^2} \times f^\prime(c) - \frac{f(c)}{\Big(g(c\Big)^2} \times g^\prime(c)\\
                                                    &= \boxed{\frac{g(c)f^\prime(c) - f(c)g^\prime(c)}{\Big(g(c)\Big)^2}}
                    \qed
                \end{align*}
        \end{itemize}

    \item[5.2.7]
        \begin{itemize}
            \item[(a)]
                Let $a = \frac{5}{4}$. For $x = 0$ we have:
                \begin{equation*}
                    \lim{x \to 0}\frac{x^\frac{5}{4}\sin{\frac{1}{x}} - 0}{x - 0}
                        = \lim_{x \to 0}\sqrt[4]{x}\sin{\frac{1}{x}}
                \end{equation*}
                Notice that $\sqrt[4]{x} \leq \sqrt[4]{x}\sin{\frac{1}{x}} \leq
                \sqrt[4]{x}$ and $\lim_{x \to 0}\sqrt[4]{x} = 0$. Then it
                follows by the \textbf{Squeeze Theorem} that $\lim_{x \to
                0}\sqrt[4]{x}\frac{1}{x} = 0$ and hence, $g_{\frac{5}{4}}(x)$
                is differentiable at $0$.
                \\
                \\
                Now, for $x \neq 0$ we get:
                \begin{equation*}
                    g_{\frac{5}{4}}^\prime(x) = \Big(x^\frac{5}{4}\sin{\frac{1}{x}}\Big)^\prime
                                              = \frac{5}{4}\sqrt[4]{x}\sin{\frac{1}{x}} - \frac{1}{\sqrt[4]{x^3}}\cos{\frac{1}{x}}
                \end{equation*}
                Set $x_n = \frac{1}{2n\pi}$ and we have
                $g_{\frac{5}{4}}^\prime(x) =
                -\frac{1}{\sqrt[4]{\Big(\frac{1}{2n\pi}\Big)^3}} =
                -\sqrt[4]{(2n\pi)^3}$ which is unbounded on $[0, 1]$.
                \\
                \\
                Hence, for $a = \frac{5}{4}$, function $g_a$ is differentiable
                on $\reals$ with $g_a^\prime$ unbounded on $[0, 1]$.
                \\
                \\
                Finally, we got that $g_{\frac{5}{4}}$ is an example of a
                function that is differentiable on $\reals$ with
                $g_{\frac{5}{4}}^\prime$ being unbounded on $[0, 1]$.

            \item[(b)]
                Let $a = \frac{5}{2}$. For $x = 0$ we have:
                \begin{equation*}
                    \lim{x \to 0}\frac{x^\frac{5}{2}\sin{\frac{1}{x}} - 0}{x - 0}
                        = \lim_{x \to 0}\sqrt{x^3}\sin{\frac{1}{x}}
                \end{equation*}
                Then, once again, per the \textbf{Squeeze Theorem}, the limit
                is $0$.
                \\
                \\
                Now, for $x \neq 0$ we get:
                \begin{equation*}
                    g_{\frac{5}{2}}^\prime(x) = \Big(x^\frac{5}{2}\sin{\frac{1}{x}}\Big)^\prime
                                              = \frac{5}{2}\sqrt{x^3}\sin{\frac{1}{x}} - \sqrt{x}\cos{\frac{1}{x}}
                \end{equation*}
                Functions $\sin$ and $\cos$ are both bounded and it follows
                that $\lim_{x \to 0}g_{\frac{5}{2}}^\prime(x) = 0 =
                g_\frac{5}{2}^\prime(0)$. Thus, we have that
                $g_{\frac{5}{2}}^\prime$ is continuous. Similar to part $(a)$,
                let $x_n = \frac{1}{2n\pi}$. Then we get
                $g_{\frac{5}{2}}^{\prime\prime} = 3\sqrt{2n\pi}$ which is
                unbounded. Hence, $g^\prime$ is not differentiable at $0$.
                \\
                \\
                Finally, we got that $g_{\frac{5}{2}}$ is an example of a
                function that is differentiable on $\reals$ with
                $g_{\frac{5}{2}}^\prime$ being continuous but not
                differentiable at $0$.

            \item[(c)]
                Let $a = 4$. For $x = 0$ we have:
                \begin{align*}
                    g_4^\prime(x)         &= 4x^3\sin{\frac{1}{x}} - x^2\cos{\frac{1}{x}}\\
                    g_4^{\prime\prime}(x) &= 12x^2\sin{\frac{1}{x}} - 6x\cos{\frac{1}{x}} + \sin{\frac{1}{x}}
                \end{align*}
                Then, notice that
                \begin{equation*}
                    g_4^{\prime\prime} = \lim_{x \to 0}{\frac{4x^3\sin{\frac{1}{x}} - x^2\cos{\frac{1}{x}} - 0}{x - 0}}
                                       = \lim_{x \to 0}{4x^2\sin{\frac{1}{x}} - x\cos{\frac{1}{x}}}
                                       = 0
                \end{equation*}
                On the other hand, $\lim_{x \to 0}{12x^2\sin{\frac{1}{x}}} -
                6x\cos{\frac{1}{x}} + \sin{\frac{1}{x}}$ does not exist, as the
                the third term fluctuates between $1$ and $-1$ (the first two
                do go to $0$, but the third one does not).
                \\
                \\
                Finally, we got that $g_4$ is an example of a function that is
                differentiable on $\reals$ with $g_4^\prime$ being
                differentiable on $\reals$, but $g_4^{\prime\prime}$ not
                continuous at $0$.
        \end{itemize}

    \newpage

    \item[5.3.1]
        \begin{itemize}
            \item[(a)]
                Suppose that $f$ is differentiable on a closed interval $[a,
                b]$ and that $f^\prime$ is continuous on a closed interval $[a,
                b]$. It follows that $\abs{f^\prime}$ is also continuous on
                $[a, b]$. Now, per \textbf{Theorem 4.4.2 (Extreme Value
                Theorem)}, $\exists x_0 \in [a, b]$ s.t. $\forall x \in [a, b],
                \abs{f^\prime(x)} \leq f^\prime(x_0)$. Then, if some $m, n \in
                [a, b]$ with $m \neq n$, by the \textbf{Mean Value Theorem},
                $\exists x \in [a, b]$ s.t. $\absb{\dfrac{f(m) - f(n)}{m - n}}
                = \abs{f^\prime(x)} \leq f^\prime(x_0)$. Hence, we got that $f$
                is Lipschitz on $[a, b]$ with $M = \abs{f^\prime(x_0)}$.\\
                $\qed$

        \end{itemize}

    \item[5.3.3]
        \begin{itemize}
            \item[(a)]
                As $h$ is differentiable on $[0, 3]$, it follows that $h$ is
                also continuous on $[0, 3]$. Hence, the function $g(x) = h(x) -
                x$ is also continuous on $[0, 3]$. Now, notice that $g(0) =
                h(0) = 1$ and $g(3) = h(3) - 3 = -1$. Then, per \textbf{Theorem
                4.5.1 (Intermediate Value Theorem)}, there exists $d \in [0,
                3]$ s.t. $g(d) = 0$ which means that $h(d) = d$.\\
                $\qed$

            \item[(b)]
                Once again, since $h$ is differentiable on $[0, 3]$, it follows
                that $h$ is also continuous on $[0, 3]$. Now, by the
                \textbf{Mean Value Theorem}, $\exists c \in (0, 3)$ s.t.
                $h^\prime(c) = \dfrac{h(3) - h(0)}{3 - 0} = \frac{1}{3}$.\\
                $\qed$

            \item[(c)]
                As $h(1) = h(3)$, per \textbf{Theorem 5.3.1 (Rolle’s Theorem)},
                $\exists b \in (1, 3)$ s.t. $h^\prime(b) = 0$. Now, since $0 <
                \frac{1}{4} < \frac{1}{3}$, by \textbf{Theorem 5.2.7 (Darboux’s
                Theorem)}, $\exists x \in A = [b, c]$ (could be $[c, b]$ if $b
                > c$, but this does not change the logic) s.t. $h^\prime(x) =
                \frac{1}{4}$ and since $A \subset [0, 3]$, we get $x \in [0,
                3]$.\\
                $\qed$
        \end{itemize}

    \item[5.3.7]
        Suppose, for the sake of contradiction, that $f$ is differentiable on
        an interval with $f^\prime(x) \neq 1$ and has two fixed points $x$ and
        $y$. Then, we have $f(x) = x$ and $f(y) = y$. Now, by the \textbf{Mean
        Value Theorem}, $\exists c \in (x, y)$ s.t. $\dfrac{f(y) - f(x)}{y - x}
        = f^\prime(c)$. Substituting $f(x)$ with $x$ and $f(y)$ with $y$ gives
        us $f^\prime(c) = \dfrac{y - x}{y - x} = 1$. Now, by assumption, we
        know that $\forall x, f^\prime(x) \neq 1$, however, if we set $x = c$,
        we get $f^\prime(c) = 1$ and we face a contradiction. Finally, we got
        that if $f$ is differentiable on an interval with $f^\prime(x) \neq 1$,
        $f$ can only have at most one fixed point.\\
        $\qed$

    \newpage

    \item[5.3.11]
        \begin{itemize}
            \item[(a)]
                Let $f$ and $g$ be continuous on an interval containing $a$,
                and assume $f$ and $g$ are differentiable on this interval with
                the possible exception of the point $a$. Let us consider the
                following two cases:
                \begin{itemize}
                    \item[(i)]
                        $x < 0$
                        \\
                        \\
                        If $x < 0$, for $c \in (x, 0)$, it follows by the
                        \textbf{Theorem 5.3.5 (Generalized Mean Value Theorem)}
                        that
                        \begin{equation*}
                            \frac{f^\prime(c)}{g^\prime(c)} = \frac{f(x) - f(0)}{g(x) - g(0)}
                                                            = \frac{f(x)}{g(x)}
                        \end{equation*}
                        Now, since $x \to 0^-$, it follows that $c \to 0^-$ and
                        thus, we have
                        \begin{equation*}
                            \lim_{c \to 0^-}{\frac{f^\prime(c)}{g^\prime(c)}} = \lim_{x \to 0^-}{\frac{f(x)}{g(x)}}
                        \end{equation*}

                    \item[(ii)]
                        $x > 0$
                        \\
                        \\
                        Similarly, if $x > 0$, for $c \in (0, x)$, it follows
                        by the \textbf{Theorem 5.3.5 (Generalized Mean Value
                        Theorem)} that
                        \begin{equation*}
                            \frac{f^\prime(c)}{g^\prime(c)} = \frac{f(x) - f(0)}{g(x) - g(0)}
                                                            = \frac{f(x)}{g(x)}
                        \end{equation*}
                        Now, since $x \to 0^+$, it follows that $c \to 0^+$ and
                        thus, we have
                        \begin{equation*}
                            \lim_{c \to 0^+}{\frac{f^\prime(c)}{g^\prime(c)}} = \lim_{x \to 0^+}{\frac{f(x)}{g(x)}}
                        \end{equation*}
                \end{itemize}
                Finally, we got that $\lim_{x \to
                0}{\frac{f^\prime(x)}{g^\prime(x)}} = \lim_{x \to
                0}\frac{f(x)}{g(x)}$.\\
                $\qed$
        \end{itemize}
\end{itemize}

%%%%%%%%%%%%%%%%%%%%%%%%%%%%%%%%%%%%%%%%%%%%%%%%%%%%%%%%%%%%%%%%%%%%%%%%%%%%%%%
% The End of the Document
%%%%%%%%%%%%%%%%%%%%%%%%%%%%%%%%%%%%%%%%%%%%%%%%%%%%%%%%%%%%%%%%%%%%%%%%%%%%%%%

\end{document}
