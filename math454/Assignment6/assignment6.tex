%%%%%%%%%%%%%%%%%%%%%%%%%%%%%%%%%%%%%%%%%%%%%%%%%%%%%%%%%%%%%%%%%%%%%%%%%%%%%%%
%
% Filename: assignment6.tex
% Author:   David Oniani
% Modified: December 18, 2020
%  _         _____   __  __
% | |    __ |_   _|__\ \/ /
% | |   / _` || |/ _ \\  /
% | |__| (_| || |  __//  \
% |_____\__,_||_|\___/_/\_\
%
%%%%%%%%%%%%%%%%%%%%%%%%%%%%%%%%%%%%%%%%%%%%%%%%%%%%%%%%%%%%%%%%%%%%%%%%%%%%%%%

%%%%%%%%%%%%%%%%%%%%%%%%%%%%%%%%%%%%%%%%%%%%%%%%%%%%%%%%%%%%%%%%%%%%%%%%%%%%%%%
% Document Definition
%%%%%%%%%%%%%%%%%%%%%%%%%%%%%%%%%%%%%%%%%%%%%%%%%%%%%%%%%%%%%%%%%%%%%%%%%%%%%%%

\documentclass[11pt]{article}

%%%%%%%%%%%%%%%%%%%%%%%%%%%%%%%%%%%%%%%%%%%%%%%%%%%%%%%%%%%%%%%%%%%%%%%%%%%%%%%
% Packages and Related Settings
%%%%%%%%%%%%%%%%%%%%%%%%%%%%%%%%%%%%%%%%%%%%%%%%%%%%%%%%%%%%%%%%%%%%%%%%%%%%%%%

% Global, document-wide settings
\usepackage[margin=1in]{geometry}
\usepackage[utf8]{inputenc}
\usepackage[english]{babel}

% Other packages
\usepackage{booktabs}
\usepackage{hyperref}
\usepackage{mathtools}
\usepackage{amsthm}
\usepackage{amssymb}
\usepackage[cache=false]{minted}

%%%%%%%%%%%%%%%%%%%%%%%%%%%%%%%%%%%%%%%%%%%%%%%%%%%%%%%%%%%%%%%%%%%%%%%%%%%%%%%
% Command Definitions and Redefinitions
%%%%%%%%%%%%%%%%%%%%%%%%%%%%%%%%%%%%%%%%%%%%%%%%%%%%%%%%%%%%%%%%%%%%%%%%%%%%%%%

% Nice-looking underline
\newcommand\und[1]{\underline{\smash{#1}}}

% Line spacing is 1.5
\renewcommand{\baselinestretch}{1.5}

% Absolute value
\DeclarePairedDelimiter\abs{\lvert}{\rvert}%

% Ceiling
\DeclarePairedDelimiter{\ceil}{\lceil}{\rceil}

% Floor
\DeclarePairedDelimiter\floor{\lfloor}{\rfloor}

% % Naturals, Reals, Integers, and Rationals, 
\newcommand{\nats}{\mathbb{N}}
\newcommand{\reals}{\mathbb{R}}
\newcommand{\preals}{\mathbb{R^+}}
\newcommand{\nreals}{\mathbb{R^-}}
\newcommand{\ints}{\mathbb{Z}}
\newcommand{\pints}{\mathbb{Z^+}}
\newcommand{\nints}{\mathbb{Z^-}}
\newcommand{\rats}{\mathbb{Q}}
\newcommand{\prats}{\mathbb{Q^+}}
\newcommand{\nrats}{\mathbb{Q^-}}
\newcommand{\irrats}{\mathbb{I}}
\newcommand{\pirrats}{\mathbb{I^+}}
\newcommand{\nirrats}{\mathbb{I^-}}

%%%%%%%%%%%%%%%%%%%%%%%%%%%%%%%%%%%%%%%%%%%%%%%%%%%%%%%%%%%%%%%%%%%%%%%%%%%%%%%
% Miscellaneous
%%%%%%%%%%%%%%%%%%%%%%%%%%%%%%%%%%%%%%%%%%%%%%%%%%%%%%%%%%%%%%%%%%%%%%%%%%%%%%%

% Setting stuff
\setlength{\parindent}{0pt}  % Remove indentations from paragraphs

% PDF information and nice-looking urls
\hypersetup{%
  pdfauthor={David Oniani},
  pdftitle={Real Analysis},
  pdfsubject={Mathematics, Real Analysis, Real Numbers},
  pdfkeywords={Mathematics, Real Analysis, Real Numbers},
  pdflang={English},
  colorlinks=true,
  linkcolor={black!50!blue},
  citecolor={black!50!blue},
  urlcolor={black!50!blue}
}

%%%%%%%%%%%%%%%%%%%%%%%%%%%%%%%%%%%%%%%%%%%%%%%%%%%%%%%%%%%%%%%%%%%%%%%%%%%%%%%
% Author(s), Title, and Date
%%%%%%%%%%%%%%%%%%%%%%%%%%%%%%%%%%%%%%%%%%%%%%%%%%%%%%%%%%%%%%%%%%%%%%%%%%%%%%%

% Author(s)
\author{David Oniani\\
        Luther College\\
        \href{mailto:oniada01@luther.edu}{oniada01@luther.edu}}

% Title
\title{\rule{\paperwidth - 150pt}{1pt}\textbf{\\\textit{Real Analysis}\\}\rule
{\paperwidth - 150pt}{1pt}\\\textbf{Assignment \textnumero6}\\{\normalsize
Instructor: Dr. Eric Westlund}}

% Date
\date{\today}

%%%%%%%%%%%%%%%%%%%%%%%%%%%%%%%%%%%%%%%%%%%%%%%%%%%%%%%%%%%%%%%%%%%%%%%%%%%%%%%
% Beginning of the Document
%%%%%%%%%%%%%%%%%%%%%%%%%%%%%%%%%%%%%%%%%%%%%%%%%%%%%%%%%%%%%%%%%%%%%%%%%%%%%%%

\begin{document}
\maketitle

%%%%%%%%%%%%%%%%%%%%%%%%%%%%%%%%%%%%%%%%%%%%%%%%%%%%%%%%%%%%%%%%%%%%%%%%%%%%%%%
%
% Homework
%
% 3.3 # 1, 2, 11
% 3.4 # 5, 7, 9ab
%
%
%%%%%%%%%%%%%%%%%%%%%%%%%%%%%%%%%%%%%%%%%%%%%%%%%%%%%%%%%%%%%%%%%%%%%%%%%%%%%%%

\begin{itemize}
    \item[3.3.1]
        Per \textbf{Heine-Borel Theorem}, $K$ is closed and bounded. Then since
        $K$ is bounded, $\sup{K}$ and $\inf{K}$ must both exist.
        \\
        \\
        Let us first prove that $\sup{K} \in K$. Suppose, for the sake of
        contradiction, that $\sup{K} \notin K$. Then it follows that $\sup{K}$
        is the limit point of $K$. However, since $K$ is closed, $\sup{K} \in
        K$ and we face a contradiction. Thus, $\sup{K} \in K$\\
        $\qed$
        \\
        Let us now prove that $\inf{K} \in K$. Similarly, suppose, for the sake
        of contradiction, that $\inf{K} \notin K$. Then it follows that
        $\inf{K}$ is the limit point of $K$. However, since $K$ is closed,
        $\inf{K} \in K$ and we face a contradiction. Thus, $\inf{K} \in K$\\
        $\qed$
        \\
        Finally, we have shown that if $K$ is compact and nonempty, then
        $\sup{K}$ and $\inf{K}$ both exist and are elements of $K$.\\
        $\qed$

    \newpage

    \item[3.3.2]
        \begin{itemize}
            \item[(a)]
                \textbf{$\nats$ is not compact}.
                \\
                This is the case since a sequence $a_n = (n)$ has no
                subsequence that is convergent and \textbf{by Theorem 3.3.1},
                $\nats$ is not compact.

            \item[(b)]
                \textbf{$\rats \cap [0, 1]$ is not compact.}
                \\
                Recall that $\rats$ is dense in $\reals$. Then there must exist
                a sequence $(a_n) \subseteq \rats$ s.t. $(a_n) \to
                \dfrac{1}{\sqrt{3}}$. Now, as every $\epsilon$-neighborhood
                contains rational numbers, it is a limit point. However,
                $\dfrac{1}{\sqrt{3}} \notin \rats$ as $\dfrac{1}{\sqrt{3}}$ is
                irrational. Thus, \textbf{by Theorem 3.3.1} $\rats \cap [0, 1]$
                cannot be compact.

            \item[(c)]
                \textbf{The Cantor set is compact.}
                \\
                The Cantor set is an intersection of closed sets, and hence, it
                is closed. Additionally, the Cantor set is a subset of $[0, 1]$
                and thus, it is bounded. Now, since the Cantor set is both
                closed and bounded, it follows $\textbf{by Theorem 3.3.1}$ that
                the Cantor set is compact.

            \item[(d)]
                \textbf{The set $\{1 + \dfrac{1}{2^2} + \dfrac{1}{3^2} + \dots
                + \dfrac{1}{n^2} \mid n \in \nats\}$ is not compact.}
                \\
                Notice that $\lim_{n \to \infty} \{1 + \dfrac{1}{2^2} +
                \dfrac{1}{3^2} + \dots + \dfrac{1}{n^2} \mid n \in \nats\} =
                \lim_{n \to \infty} \sum_{n = 1}^{\infty} \dfrac{1}{n^2}$
                converges to $2$. Hence, the sum is a limit point of the set.
                However, the sum is not in the set and thus, \textbf{by Theorem
                3.3.1}, it follows that the set $\{1 + \dfrac{1}{2^2} +
                \dfrac{1}{3^2} + \dots + \dfrac{1}{n^2} \mid n \in \nats\}$ is
                not compact.

            \item[(e)]
                \textbf{The set $\{1, \dfrac{1}{2}, \dfrac{2}{3}, \dfrac{3}{4},
                \dfrac{4}{5}, \dots \mid n \in \nats\}$ is compact.}
                \\
                Let us denote this set by $S$. Then notice that $S_1 = 1$ and
                the rest of the terms can be calculated using the formula $S_n
                = \dfrac{n - 1}{n}$ with $n > 1$. Also notice that $\lim_{n \to
                \infty} \dfrac{n}{n - 1} = 1$. Hence, the set $S$ contains its
                own only limit point. Additionally, it is easy to see that
                every element of the set lies between 0 and 1 (inclusive).
                Thus, \textbf{by Theorem 3.3.1}, we get that The set $\{1,
                \dfrac{1}{2}, \dfrac{2}{3}, \dfrac{3}{4}, \dfrac{4}{5}, \dots
                \mid n \in \nats\}$ is compact.
        \end{itemize}

    \newpage

    \item[3.3.11]
        From $3.3.2$ we know that there are three sets which are not compact:
        \begin{itemize}
            \item[1.]
                $\nats$
            \item[2.]
                $\rats \cap [0, 1]$
            \item[3.]
                $\{1 + \dfrac{1}{2^2} + \dfrac{1}{3^2} + \dots + \dfrac{1}{n^2}
                \mid n \in \nats\}$
        \end{itemize}

        For $(1)\ \nats$, consider an open cover $C_1 = \{(n - 0.5, n + 0.5)
        \mid n \in \nats\}$. Then $C_1$ has no finite subcover.
        \\
        \\
        For $(2)\ \rats \cap [0, 1]$, consider an open cover $C_2 = \{(-3,
        \dfrac{1}{\sqrt{3}} - \dfrac{1}{n}), (\dfrac{1}{\sqrt{3}} +
        \dfrac{1}{n}, 3) \mid n \in \nats\}$. Then $C_2$ has no finite
        subcover.
        \\
        \\
        For $(3)\ \{1 + \dfrac{1}{2^2} + \dfrac{1}{3^2} + \dots +
        \dfrac{1}{n^2} \mid n \in \nats\}$, consider an open cover $C_3 = \{
        (0, \sum_{m = 1}^n \dfrac{1}{m^2}) \mid n \in \nats\}$. Then $C_3$ has
        no finite subcover.

    \item[3.4.5]
        Suppose, for the sake of contradiction and without a loss of
        generality, that $A$ and $B$ are nonempty subsets of $\reals$ s.t. $A
        \cap B \neq \emptyset$ and there exist disjoint open sets $U$ and $V$
        s.t. $A \subseteq U$ and $B \subseteq V$. Let $x \in A \cap
        \overline{B}$. Then it follows that $x \in U \cap \overline{V}$.  Now,
        since $U \cap V = \emptyset$, we get that $x \in U$ and $x \in
        \overline{V}$. Now, notice that any $\epsilon$-neighborhood
        $V_\epsilon(x)$ contains an element of $V$. Thus, $V_\epsilon(x)$ is
        not contained in $U$. We get $A \cap B = \emptyset$ and we face a
        contradiction since we assumed that $A \cap B \neq \emptyset$. Hence,
        $A$ and $B$ are separated.\\
        $\qed$

    \item[3.4.7]
        \begin{itemize}
            \item[(a)]
                Recall that the set of irrational numbers $\irrats$ is dense in
                $\reals$. This means that $\forall r_1, r_2 \in \reals$ with
                $r_1 < r_2, \exists i \in \irrats$ s.t. $r_1 < i < r_2$. Let $U
                = \rats \cap (-\infty, i)$ and let $V = \rats \cap (i,
                +\infty)$. Then it is easy to see that $Q = U \cap V$ where $U$
                and $V$ are separated. Now, it is clear that $\overline{U}
                \subset (-\infty, i]$ and thus, $\overline{U} \cap V =
                \emptyset$.  Similarly, $\overline{V} \subset [i, -\infty)$ and
                therefore, $U \cap \overline{V} = \emptyset$. Finally, we get
                that $\rats$ is totally disconnected since given any two
                distinct points $r_1, r_2$ there exist separated sets $U$ and
                $V$ with $r_1 \in U, r_2 \in V$, and $U \cap V = \rats$.\\
                $\qed$

            \item[(b)]
                \textbf{Yes, the set of irrational numbers is totally
                disconnected}.
                \\
                Let us now show this fact.
                \\
                \\
                Notice that $\irrats = \reals - \rats$. Recall that $\rats$ is
                dense in $\reals$. It follows that $\forall i_1, i_2 \in
                \irrats$ with $i_1 < i_2, \exists q \in \rats$ s.t. $i_1 < q <
                i_2$. Now, let $U = \irrats \cap (-\infty, q)$ and let $V =
                \irrats \cap (q, +\infty)$. Then $U \cup V = \irrats$. Let $X =
                (-\infty, q)$ and let $Y = (q, +\infty)$. Then $U \subseteq X$
                and $Y \subseteq V$. Notice that $X$ and $Y$ are totally
                disconnected sets. Then, according to what we showed in
                \textbf{Exercise 3.4.5}, $U$ and $V$ must be separated. Hence,
                $\irrats$ is totally disconnected.\\
                $\qed$
        \end{itemize}

    \item[3.4.9]
        \begin{itemize}
            \item[(a)]
                Notice that for the length of $O$, the following stands:
                \begin{equation*}
                    \abs{O} \leq \sum_{n = 1}^\infty 2(\dfrac{1}{2^n})
                \end{equation*}
                It follows that $O$ does not fully cover $\reals$ and thus, $F
                = O^c$ is nonempty. Now, $O$ contains all rational numbers and
                thus $F = O^c \subseteq \irrats$. Furthermore, $O$ is an open
                set as it is constructed by an arbirary countable union of open
                neighborhoods. Hence, $F = O^c$ must be closed. Hence, we have
                shown that $F$ is a closed, nonempty set consisting only of
                irrational numbers.\\
                $\qed$

            \item[(b)]
                \textbf{No, $F$ does not contain any nonempty open intevals and
                yes, $F$ is totally disconnected.}
                \\
                Let us now prove these facts.
                \\
                \\
                Let $r_1, r_2 \in \reals$ with $r_1 < r_2$. Recall that $\rats$
                is dense in $\reals$. Then $\exists q \in \rats$ s.t. $r_1 < q
                < r_2$.  Now, since $q \in O$, $(r_1, r_2) \cap O = \emptyset$
                and thus $(r_1, r_2) \not\subseteq F$. Hence, $F$ does not
                contain any nonempty open intevals.\\
                $\qed$
                \\
                \\
                $F$ is totally disconnected since it is a subset of irrational
                numbers $\irrats$ (shown in \textbf{Exercise 3.4.9 (a)}). We
                have shown in \textbf{Exercise 3.4.7 (b)} that the set of
                irrational numbers $\irrats$ is totally disconnected. Hence,
                $F$ is also totally disconnected.\\
                $\qed$
        \end{itemize}
\end{itemize}

%%%%%%%%%%%%%%%%%%%%%%%%%%%%%%%%%%%%%%%%%%%%%%%%%%%%%%%%%%%%%%%%%%%%%%%%%%%%%%%
% The End of the Document
%%%%%%%%%%%%%%%%%%%%%%%%%%%%%%%%%%%%%%%%%%%%%%%%%%%%%%%%%%%%%%%%%%%%%%%%%%%%%%%

\end{document}
