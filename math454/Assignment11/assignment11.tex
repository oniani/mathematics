%%%%%%%%%%%%%%%%%%%%%%%%%%%%%%%%%%%%%%%%%%%%%%%%%%%%%%%%%%%%%%%%%%%%%%%%%%%%%%%
%
% Filename: assignment11.tex
% Author:   David Oniani
% Modified: January 12, 2020
%  _         _____   __  __
% | |    __ |_   _|__\ \/ /
% | |   / _` || |/ _ \\  /
% | |__| (_| || |  __//  \
% |_____\__,_||_|\___/_/\_\
%
%%%%%%%%%%%%%%%%%%%%%%%%%%%%%%%%%%%%%%%%%%%%%%%%%%%%%%%%%%%%%%%%%%%%%%%%%%%%%%%

%%%%%%%%%%%%%%%%%%%%%%%%%%%%%%%%%%%%%%%%%%%%%%%%%%%%%%%%%%%%%%%%%%%%%%%%%%%%%%%
% Document Definition
%%%%%%%%%%%%%%%%%%%%%%%%%%%%%%%%%%%%%%%%%%%%%%%%%%%%%%%%%%%%%%%%%%%%%%%%%%%%%%%

\documentclass[11pt]{article}

%%%%%%%%%%%%%%%%%%%%%%%%%%%%%%%%%%%%%%%%%%%%%%%%%%%%%%%%%%%%%%%%%%%%%%%%%%%%%%%
% Packages and Related Settings
%%%%%%%%%%%%%%%%%%%%%%%%%%%%%%%%%%%%%%%%%%%%%%%%%%%%%%%%%%%%%%%%%%%%%%%%%%%%%%%

% Global, document-wide settings
\usepackage[margin=1in]{geometry}
\usepackage[utf8]{inputenc}
\usepackage[english]{babel}

% Other packages
\usepackage{booktabs}
\usepackage{hyperref}
\usepackage{mathtools}
\usepackage{amsthm}
\usepackage{amssymb}
\usepackage[cache=false]{minted}

%%%%%%%%%%%%%%%%%%%%%%%%%%%%%%%%%%%%%%%%%%%%%%%%%%%%%%%%%%%%%%%%%%%%%%%%%%%%%%%
% Command Definitions and Redefinitions
%%%%%%%%%%%%%%%%%%%%%%%%%%%%%%%%%%%%%%%%%%%%%%%%%%%%%%%%%%%%%%%%%%%%%%%%%%%%%%%

% Nice-looking underline
\newcommand\und[1]{\underline{\smash{#1}}}

% Line spacing is 1.5
\renewcommand{\baselinestretch}{1.5}

% Absolute value
\DeclarePairedDelimiter\abs{\lvert}{\rvert}%

% Absolute value big
\DeclarePairedDelimiter\absb{\Big\lvert}{\Big\rvert}%

% Ceiling
\DeclarePairedDelimiter{\ceil}{\lceil}{\rceil}

% Floor
\DeclarePairedDelimiter\floor{\lfloor}{\rfloor}

% % Naturals, Reals, Integers, and Rationals, 
\newcommand{\nats}{\mathbb{N}}
\newcommand{\reals}{\mathbb{R}}
\newcommand{\preals}{\mathbb{R^+}}
\newcommand{\nreals}{\mathbb{R^-}}
\newcommand{\ints}{\mathbb{Z}}
\newcommand{\pints}{\mathbb{Z^+}}
\newcommand{\nints}{\mathbb{Z^-}}
\newcommand{\rats}{\mathbb{Q}}
\newcommand{\prats}{\mathbb{Q^+}}
\newcommand{\nrats}{\mathbb{Q^-}}
\newcommand{\irrats}{\mathbb{I}}
\newcommand{\pirrats}{\mathbb{I^+}}
\newcommand{\nirrats}{\mathbb{I^-}}

%%%%%%%%%%%%%%%%%%%%%%%%%%%%%%%%%%%%%%%%%%%%%%%%%%%%%%%%%%%%%%%%%%%%%%%%%%%%%%%
% Miscellaneous
%%%%%%%%%%%%%%%%%%%%%%%%%%%%%%%%%%%%%%%%%%%%%%%%%%%%%%%%%%%%%%%%%%%%%%%%%%%%%%%

% Setting stuff
\setlength{\parindent}{0pt}  % Remove indentations from paragraphs

% PDF information and nice-looking urls
\hypersetup{%
  pdfauthor={David Oniani},
  pdftitle={Real Analysis},
  pdfsubject={Mathematics, Real Analysis, Real Numbers},
  pdfkeywords={Mathematics, Real Analysis, Real Numbers},
  pdflang={English},
  colorlinks=true,
  linkcolor={black!50!blue},
  citecolor={black!50!blue},
  urlcolor={black!50!blue}
}

%%%%%%%%%%%%%%%%%%%%%%%%%%%%%%%%%%%%%%%%%%%%%%%%%%%%%%%%%%%%%%%%%%%%%%%%%%%%%%%
% Author(s), Title, and Date
%%%%%%%%%%%%%%%%%%%%%%%%%%%%%%%%%%%%%%%%%%%%%%%%%%%%%%%%%%%%%%%%%%%%%%%%%%%%%%%

% Author(s)
\author{David Oniani\\
        Luther College\\
        \href{mailto:oniada01@luther.edu}{oniada01@luther.edu}}

% Title
\title{\rule{\paperwidth - 150pt}{1pt}\textbf{\\\textit{Real Analysis}\\}\rule
{\paperwidth - 150pt}{1pt}\\\textbf{Assignment \textnumero11}\\{\normalsize
Instructor: Dr. Eric Westlund}}

% Date
\date{\today}

%%%%%%%%%%%%%%%%%%%%%%%%%%%%%%%%%%%%%%%%%%%%%%%%%%%%%%%%%%%%%%%%%%%%%%%%%%%%%%%
% Beginning of the Document
%%%%%%%%%%%%%%%%%%%%%%%%%%%%%%%%%%%%%%%%%%%%%%%%%%%%%%%%%%%%%%%%%%%%%%%%%%%%%%%

\begin{document}
\maketitle

%%%%%%%%%%%%%%%%%%%%%%%%%%%%%%%%%%%%%%%%%%%%%%%%%%%%%%%%%%%%%%%%%%%%%%%%%%%%%%%
%
% Homework
%
% 6.5 # 1, 2, 7, 8a
% 6.6 # 5
%
%%%%%%%%%%%%%%%%%%%%%%%%%%%%%%%%%%%%%%%%%%%%%%%%%%%%%%%%%%%%%%%%%%%%%%%%%%%%%%%

\begin{itemize}
    \item[6.5.1]
        \begin{itemize}
            \item[(a)]
                Notice that $g(x) = \frac{(-1)^{n - 1}x^n}{n} \leq x^n$. Then
                $x^n$ is a geometric series $\sum_{n = 1}^\infty x^n$ which
                converges absolutely on $(-1, 1)$. Therefore, $g(x)$ is defined
                on set $(-1, 1)$. Now, since every term in $g(x)$ is
                continuous, by uniform convergence $g(x)$ is also continuous.
                Furthermore, at point $x = 1$, we have:
                \begin{align*}
                    g(1) &= 1 - \frac{1}{2} + \frac{1}{3} - \frac{1}{4} + \dots\\
                         &= \frac{1}{1 \times 2} + \frac{1}{3 \times 4} + \frac{1}{5 \times 6}\dots\\
                         &\leq \frac{1}{1^2} + \frac{1}{3^2} + \frac{1}{5^2} + \dots\\
                         &\leq \frac{1}{1^2} + \frac{1}{2^2} + \frac{1}{3^2} + \dots
                \end{align*}
                Now, notice that $\frac{1}{1^2} + \frac{1}{2^2} + \frac{1}{3^2}
                + \dots$ is a $p$-series with $p = 2$. Thus, $g(x)$ is also
                defined on $(-1, 1]$. If $x = -1$, we have $g(-1) = -(1 +
                \frac{1}{2} + \frac{1}{3} + \dots)$ which is a harmonic series
                that diverges. Hence, $g(x)$ diverges on $[-1, 1]$. Finally,
                if $\abs{x} > 1, g(x)$ diverges since $\lim_{n \to
                \infty}{\frac{(-x)^{n - 1}}{n}} \neq 0$.

            \item[(b)]
                Notice that $g^\prime(x) = 1 + (-x) + (-x)^2 + (-x)^3 + \dots$
                which is a geometric series that converges if $\abs{(-x)} < 1$.
                Hence, $g^\prime(x)$ is defined on $(-1, 1)$. The sum of the
                series is $\frac{1}{1 + x}$.
        \end{itemize}

    \item[6.5.2]
        \begin{itemize}
            \item[(a)]
                Let $a_n = 0\ \forall n \geq 0$.

            \item[(b)]
                This is impossible since all power series converge at $0$.

            \item[(c)]
                Let $a_0 = 0$ and let $a_n = \frac{1}{n^2}\ \forall n \geq 1$.
                Then $\sum a_nx^n$ converges since $\frac{1}{n^2}$ converges
                absolutely if $x = \pm 1$ (we showed that $\frac{1}{n^2}$
                converges multiple times over the course of the class).
                However, if $\abs{x} > 1$, we have $\lim_{n \to \infty}
                \frac{x^n}{n^2} = \infty$ and the power series diverges.

            \item[(d)]
                This is impossible since $\sum{\abs{a_n(-1)^n}} =
                \sum{\abs{a_n(1)^n}}$.

            \item[(e)]
                Let $a_n$ be the following:
                \begin{equation*}
                    a_n =
                    \begin{cases}
                        \frac{(-1)^{\frac{n}{2}}}{n} \text{ if $n$ is even}\\
                        0 \text{ otherwise}
                    \end{cases}
                \end{equation*}
                Then we have $\sum_{n = 1}^\infty = \frac{(-1)^nx^{2n}}{2n}$
                which converges conditionally at both $x \pm 1$.
        \end{itemize}

    \item[6.5.7]
        \begin{itemize}
            \item[(a)]
                Let $\sum a_nx^n$ be a power series s.t. $a_n \neq 0$ and let
                $L = \lim_{n \to \infty}\abs{{\frac{a_{n + 1}}{a_n}}}$. Then,
                if $\abs{x} < \frac{1}{L}$ and $L \neq 0$, we have:
                \begin{equation*}
                    \lim_{n \to \infty}{\absb{\frac{a_{n + 1}x^{n + 1}}{a_nx^n}}}
                        = \lim_{n \to \infty}{\absb{\frac{a_{n + 1}}{a_n}} \times \abs{x}}
                        < L \times \frac{1}{L} = 1
                \end{equation*}
                Finally, by \textbf{Exercise 2.7.9 (Ratio Test)} $\sum a_nx^n$
                converges if $x \in (-\frac{1}{L}, \frac{1}{L})$.

            \item[(b)]
                Let $L = 0$. Then we have:
                \begin{equation*}
                    \lim_{n \to \infty}{\absb{\frac{a_{n + 1}x^{n + 1}}{a_nx^n}}}
                        = L \times \abs{x} = 0 < 1
                \end{equation*}
                It follows that $\sum a_nx^n$ converges $\forall x \in \reals$.

            \item[(c)]
                Let $L^\prime = \lim_{n \to \infty} s_n$ where $s_n =
                \sup\Big\{\absb{\frac{a_{k + 1}}{a_k}} : k \geq n \Big\}$.
                Now, $\forall \epsilon > 0, L^\prime + \epsilon > 1, \exists N
                \in \nats$ s.t. $\abs{\frac{a_n}{a_{n + 1}}} < L^\prime +
                \epsilon \ \forall n \geq N$. For $x \in (-\frac{1}{L},
                \frac{1}{L})$, let $0 < \delta < \frac{1}{L^\prime} - \abs{x}$.
                We have:
                \begin{equation*}
                    \absb{\frac{a_{n + 1}x^{n + 1}}{a_nx^n}}
                        < (L^\prime + \epsilon)(\frac{1}{L^\prime} - \delta)
                        = 1 + \epsilon \times \Big(\frac{1}{L^\prime}
                            - \delta\Big) - \delta L^\prime
                \end{equation*}
                Now, choose $\epsilon$ s.t. the following holds:
                \begin{equation*}
                    \epsilon < \frac{\delta L^\prime}{\frac{1}{L^\prime} - \delta}
                        \text{ with } L^\prime + \epsilon < 1
                \end{equation*}
                Then $\abs{\frac{a_{n + 1}x^{n + 1}}{a_nx_n}} < 1 \ \forall n
                \geq N$ and thus, $\sum_{1}^{\infty} a_nx^n$ converges.
        \end{itemize}

    \item[6.5.8]
        \begin{itemize}
            \item[(a)]
                Suppose that $\sum_{n = 0}^\infty a_nx^n = \sum_{n = 0}^\infty
                b_nx^n$ converges in an interval $(-R, R)$. Let $x = 0$, then
                $a_0 = b_0$. It follows by \textbf{Theorem 6.5.6} that the
                differentiated series also converges and hence for $x = 0$, the
                equality of the differentiated series gives us $a_1 + \sum_{n =
                2}^\infty na_n0^{n - 1} = b_1 + \sum_{n = 2}^\infty nb_n0^{n -
                1}$. Thus, we have $a_1 = b_1$. Then, once again, employing
                \textbf{Theorem 6.5.6}, we get that $a_2 = b_2$. If we continue
                in this fashion, we get $a_n = b_n \ \forall n = 0, 1, 2,
                \dots$.\\
                $\qed$
        \end{itemize}

    \item[6.6.5]
        \begin{itemize}
            \item[(a)]
                Let $f(x) = e^x$. Then we have $f^n(x) = e^x$ and the Taylor
                coefficients are of the form $a_n = \frac{e^0}{n!} =
                \frac{1}{n!}$. Thus, $f(x) = 1 + x + \frac{x^2}{2!} +
                \frac{x^3}{3!} + \dots$. Now, applying the \textbf{ratio test},
                we get:
                \begin{equation*}
                    L = \lim_{n \to \infty}{\absb{\frac{a_{n + 1}}{a_n}}}
                        = \lim_{n \to \infty}{\absb{\frac{n!}{(n + 1)!}}}
                        = 0
                \end{equation*}
                Hence, the Taylor series converges absolutely in all of
                $\reals$. Furthermore, if we let $x = \pm 1$, it follows by the
                \textbf{Theorem 6.5.4 (Abel’s Theorem)} that the Taylor series
                converges on $[-R, R]$.

            \item[(b)]
                Using the fact that the series converges unifotmly on $[-R, R$]
                (shown in $(a)$), $f^\prime(x)$ can be given by differentiating
                every term in the sequence. We have:
                \begin{equation*}
                    f^\prime(x) =
                        \sum_{n = 1}^\infty \frac{x^{n - 1}}{(n - 1)!}
                            = \sum_{n = 1}^\infty \frac{x^n}{n!}
                            = f(x)
                \end{equation*}

            \item[(c)]
                Notice that $e^{-c}$ is given by $f(x) = 1 + (-x) +
                \frac{(-x)^2}{2!} + \frac{(-x)^3}{3!} + \dots = 1 - x +
                \frac{x^2}{2!} - \frac{x^3}{3!} + \dots$.
                Then we have:
                \begin{align*}
                    e^x \times e^{-x} &=
                        (1 + x + \frac{x^2}{2!} + \frac{x^3}{3!} + \dots)
                            \times
                        (1 - x + \frac{x^2}{2!} - \frac{x^3}{3!} + \dots)\\
                        &= 1 + x(1 - 1) + x^2(\frac{1}{2!} + \frac{1}{2!} -1)
                            + x^3(\frac{1}{3!} - \frac{1}{3!} + \frac{1}{2!}
                                               - \frac{1}{2!})\\
                        &= 1 + 0 + 0 + 0 + \dots\\
                        &= 1
                \end{align*}
                Hence, $e^x \times e^{-x} = 1$.
        \end{itemize}
\end{itemize}

%%%%%%%%%%%%%%%%%%%%%%%%%%%%%%%%%%%%%%%%%%%%%%%%%%%%%%%%%%%%%%%%%%%%%%%%%%%%%%%
% The End of the Document
%%%%%%%%%%%%%%%%%%%%%%%%%%%%%%%%%%%%%%%%%%%%%%%%%%%%%%%%%%%%%%%%%%%%%%%%%%%%%%%

\end{document}
