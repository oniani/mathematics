%%%%%%%%%%%%%%%%%%%%%%%%%%%%%%%%%%%%%%%%%%%%%%%%%%%%%%%%%%%%%%%%%%%%%%%%%%%%%%%
%
% Filename: template.tex
% Author:   David Oniani
% Modified: December 10, 2020
%  _         _____   __  __
% | |    __ |_   _|__\ \/ /
% | |   / _` || |/ _ \\  /
% | |__| (_| || |  __//  \
% |_____\__,_||_|\___/_/\_\
%
%%%%%%%%%%%%%%%%%%%%%%%%%%%%%%%%%%%%%%%%%%%%%%%%%%%%%%%%%%%%%%%%%%%%%%%%%%%%%%%

%%%%%%%%%%%%%%%%%%%%%%%%%%%%%%%%%%%%%%%%%%%%%%%%%%%%%%%%%%%%%%%%%%%%%%%%%%%%%%%
% Document Definition
%%%%%%%%%%%%%%%%%%%%%%%%%%%%%%%%%%%%%%%%%%%%%%%%%%%%%%%%%%%%%%%%%%%%%%%%%%%%%%%

\documentclass[11pt]{article}

%%%%%%%%%%%%%%%%%%%%%%%%%%%%%%%%%%%%%%%%%%%%%%%%%%%%%%%%%%%%%%%%%%%%%%%%%%%%%%%
% Packages and Related Settings
%%%%%%%%%%%%%%%%%%%%%%%%%%%%%%%%%%%%%%%%%%%%%%%%%%%%%%%%%%%%%%%%%%%%%%%%%%%%%%%

% Global, document-wide settings
\usepackage[margin=1in]{geometry}
\usepackage[utf8]{inputenc}
\usepackage[english]{babel}

% Other packages
\usepackage{booktabs}
\usepackage{hyperref}
\usepackage{mathtools}
\usepackage{amsthm}
\usepackage{amssymb}
\usepackage[cache=false]{minted}

%%%%%%%%%%%%%%%%%%%%%%%%%%%%%%%%%%%%%%%%%%%%%%%%%%%%%%%%%%%%%%%%%%%%%%%%%%%%%%%
% Command Definitions and Redefinitions
%%%%%%%%%%%%%%%%%%%%%%%%%%%%%%%%%%%%%%%%%%%%%%%%%%%%%%%%%%%%%%%%%%%%%%%%%%%%%%%

% Nice-looking underline
\newcommand\und[1]{\underline{\smash{#1}}}

% Line spacing is 1.5
\renewcommand{\baselinestretch}{1.5}

% Absolute value
\DeclarePairedDelimiter\abs{\lvert}{\rvert}%

% Absolute value big
\DeclarePairedDelimiter\absb{\Big\lvert}{\Big\rvert}%

% Ceiling
\DeclarePairedDelimiter{\ceil}{\lceil}{\rceil}

% Ceiling big
\DeclarePairedDelimiter{\ceilb}{\Big\lceil}{\Big\rceil}

% Floor
\DeclarePairedDelimiter\floor{\lfloor}{\rfloor}

% % Naturals, Reals, Integers, and Rationals, 
\newcommand{\nats}{\mathbb{N}}
\newcommand{\reals}{\mathbb{R}}
\newcommand{\preals}{\mathbb{R^+}}
\newcommand{\nreals}{\mathbb{R^-}}
\newcommand{\ints}{\mathbb{Z}}
\newcommand{\pints}{\mathbb{Z^+}}
\newcommand{\nints}{\mathbb{Z^-}}
\newcommand{\rats}{\mathbb{Q}}
\newcommand{\prats}{\mathbb{Q^+}}
\newcommand{\nrats}{\mathbb{Q^-}}
\newcommand{\irrats}{\mathbb{I}}
\newcommand{\pirrats}{\mathbb{I^+}}
\newcommand{\nirrats}{\mathbb{I^-}}

%%%%%%%%%%%%%%%%%%%%%%%%%%%%%%%%%%%%%%%%%%%%%%%%%%%%%%%%%%%%%%%%%%%%%%%%%%%%%%%
% Miscellaneous
%%%%%%%%%%%%%%%%%%%%%%%%%%%%%%%%%%%%%%%%%%%%%%%%%%%%%%%%%%%%%%%%%%%%%%%%%%%%%%%

% Setting stuff
\setlength{\parindent}{0pt}  % Remove indentations from paragraphs

% PDF information and nice-looking urls
\hypersetup{%
  pdfauthor={David Oniani},
  pdftitle={Real Analysis},
  pdfsubject={Mathematics, Real Analysis, Real Numbers},
  pdfkeywords={Mathematics, Real Analysis, Real Numbers},
  pdflang={English},
  colorlinks=true,
  linkcolor={black!50!blue},
  citecolor={black!50!blue},
  urlcolor={black!50!blue}
}

%%%%%%%%%%%%%%%%%%%%%%%%%%%%%%%%%%%%%%%%%%%%%%%%%%%%%%%%%%%%%%%%%%%%%%%%%%%%%%%
% Author(s), Title, and Date
%%%%%%%%%%%%%%%%%%%%%%%%%%%%%%%%%%%%%%%%%%%%%%%%%%%%%%%%%%%%%%%%%%%%%%%%%%%%%%%

% Author(s)
\author{David Oniani\\
        Luther College\\
        \href{mailto:oniada01@luther.edu}{oniada01@luther.edu}}

% Title
\title{\rule{\paperwidth - 150pt}{1pt}\textbf{\\\textit{Real Analysis
Exams}\\}\rule {\paperwidth - 150pt}{1pt}\\\textbf{Exam
\textnumero1}\\{\normalsize Instructor: Dr. Eric Westlund}}

% Date
\date{\today}

%%%%%%%%%%%%%%%%%%%%%%%%%%%%%%%%%%%%%%%%%%%%%%%%%%%%%%%%%%%%%%%%%%%%%%%%%%%%%%%
% Beginning of the Document
%%%%%%%%%%%%%%%%%%%%%%%%%%%%%%%%%%%%%%%%%%%%%%%%%%%%%%%%%%%%%%%%%%%%%%%%%%%%%%%

\begin{document}
\maketitle

%%%%%%%%%%%%%%%%%%%%%%%%%%%%%%%%%%%%%%%%%%%%%%%%%%%%%%%%%%%%%%%%%%%%%%%%%%%%%%%

\begin{itemize}
    \item[1.]
        Let $a_n = \dfrac{6n + 1}{3n + 2}$.

        Now, let $N \in \nats = \ceilb{\dfrac{1 + \epsilon}{\epsilon}}
        \Big(\ceilb{\dfrac{a}{b}}$ is the \textbf{ceiling} of
        $\dfrac{a}{b}\Big)$.

        Then $\forall \epsilon > 0$ and $\forall n \geq N$ and we have:
        \begin{align*}
            \abs{a_n - 2} &= \absb{\dfrac{6n + 1}{3n + 2} - 2}\\
                          &= \absb{\dfrac{-3}{3n + 2}}\\
                          &\leq \dfrac{3}{3N + 2}\\
                          &\leq \dfrac{3}{3\dfrac{1 + \epsilon}{\epsilon} + 2}
                               = \dfrac{3\epsilon}{3 + 3\epsilon + 2\epsilon}
                               = \dfrac{3\epsilon}{5\epsilon + 3}\\
                          &< \dfrac{3}{5}\epsilon\\
                          &< \epsilon
        \end{align*}

        We have now shown that $\forall \epsilon > 0, n \geq N, \abs{a_n - 2}
        < \epsilon$ and thus, $\displaystyle\lim_{n \to \infty} a_n = 2$.\\
        $\qed$

    \item[2.]
        \begin{itemize}
            \item[(a)]
                This is true. We can prove this visually.
                In order to prove that $A \times B$ is countable, it suffices to
                show that there exists an enumeration of this set.\\
                First, recall that $A \times B = \{(a, b) : a \in A, b \in B\}$.\\
                Now, let $P = A \times B, A = \{a_1, a_2, a_3, \dots\}$, and
                $B = \{b_1, b_2, b_3, \dots\}$.\\
                We have:
                \begin{align*}
                    P_1 &= (a_1, b_1), (a_1, b_2), (a_1, b_3), (a_1, b_4), \dots\\
                    P_2 &= (a_2, b_1), (a_2, b_2), (a_2, b_3), (a_2, b_4), \dots\\
                    P_3 &= (a_3, b_1), (a_3, b_2), (a_3, b_3), (a_3, b_4), \dots\\
                    P_4 &= (a_4, b_1), (a_4, b_2), (a_4, b_3), (a_4, b_4), \dots\\
                        &........................................................
                \end{align*}
                We enumerate the sequence diagonal-by-diagonal (of small
                squares) as follows:
                $$(a_1, b_1), (a_2, b_1), (a_1, b_2), (a_3, b_1), (a_2, b_2),
                (a_1, b_3), \dots$$
                This way, all of the elements in $P$ will eventually be listed
                and hence, we have successfully enumerated $P$. Thus, $P$ is
                countable, which means that $A \times B$ is countable.\\
                $\qed$

            \item[(b)]
                This is true. Let the finite product of countable sets be
                denoted as
                \begin{align*}
                    FP &= A_1 \times A_2 \times A_3 \times \dots \times A_{n - 1} \times A_n\\
                       &= \{(a_1, a_2, a_3, \dots, a_{n - 1}, a_n) \mid a_1 \in A_1, a_2 \in A_2, a_3 \in A_3, \dots, a_{n - 1} \in A_{n - 1}, a_n \in A_n\}
                \end{align*}
                Let us use induction to show that $FP$ is countable.

                \textbf{Base case}: for $n = 1$, we get $FP = A_1$. $A_1$ is
                countable by definition and thus, $FP$ is clearly countable.

                \textbf{Inductive step}: suppose that $A_1 \times A_2 \times
                A_3 \times \dots \times A_{n - 1}$ is countable and prove that
                $A_1 \times A_2 \times A_3 \times \dots \times A_{n - 1} \times
                A_n$ is countable as well (with $A_1, A_2, A_3, \dots, A_{n -
                1}, A_n$ all being countable.
                \\
                Let $A^\prime = A_1 \times A_2 \times A_3 \times \dots \times
                A_{n - 1}$. Then we have to show that $A^\prime \times A_n$ is
                countable where $A^\prime$ and $A_n$ are both countable. Now,
                recall that we have already shown in $(a)$ that the product of
                two countable sets is countable and thus, $A^\prime \times A_n$
                is countable. Hence, we have shown that the finite product of
                countable sets is countable.\\
                $\qed$

            \item[(c)]
                This is false. Suppose, for the sake of contradiction, that a
                countable product of countable sets is countable and let this
                product be $P = A_1 \times A_2 \times A_3 \times \dots$. Then
                there must exist an enumeration of $P$:
                \begin{align*}
                    P_1 = p_{11}, p_{12}, p_{13}, p_{14}, \dots\\
                    P_2 = p_{21}, p_{22}, p_{23}, p_{24}, \dots\\
                    P_3 = p_{31}, p_{32}, p_{33}, p_{34}, \dots\\
                    P_4 = p_{41}, p_{42}, p_{43}, p_{44}, \dots\\
                    .......................................
                \end{align*}
                where $p_{ij}$ is the $j^{th}$ element of $P_i$.
                \\
                \\
                Then, if $P$ is countable, this enumeration should contain all
                elements of $P$.
                \\
                Now, let us define a sequence $p^\prime$ s.t. the following
                holds:
                \begin{align*}
                    p^\prime &= p^\prime_1, p^\prime_2, p^\prime_3, p^\prime_4, \dots\\
                             &\text{where}\\
                             & p^\prime_1 \neq p_{11}\\
                             & p^\prime_2 \neq p_{22}\\
                             & p^\prime_3 \neq p_{33}\\
                             & .............................\\
                             & p^\prime_{n - 1} \neq p_{n - 1}\\
                             & p^\prime_n \neq p_{nn}\\
                             & .............................
                \end{align*}
                
                Then $p^\prime \neq P_1$ since $p^\prime_1 \neq p_{11}$,
                $p^\prime \neq P_2$ since $p^\prime_2 \neq p_{22}$, $p^\prime
                \neq P_3$ since $p^\prime_3 \neq p_{33}$, etc.  Hence, we have
                effectively constructed a sequence that is not in $P$ and we
                face a contradiction since $P$ has enumerated all sequences.
                Thus, a countable product of countable sets is not countable.\\
                $\qed$
        \end{itemize}

        \newpage
        \item[3.]
            Suppose $\sum_{a_n}$ converges conditionally. Then notice that the
            following stands (thanks for the hint!):
            \begin{align*}
                p_n &= \dfrac{a_n + \abs{a_n}}{2}\\
                q_n &= \dfrac{a_n - \abs{a_n}}{2}
            \end{align*}

            Now, let us split our proof in two parts: first prove that $p_n$
            diverges and then show that $q_n$ diverges as well.

            Let us first prove that $p_n$ diverges. Now, suppose for the sake
            of contradiction, that $p_n$ converges. Then $p_n = \dfrac{a_n +
            \abs{a_n}}{2}$ and it follows that $\abs{a_n} = 2p_n - a_n$. We
            have:
            \begin{align*}
                \sum{\abs{a_n}} = \sum{2p_n} - \sum{a_n}
            \end{align*}
            Since $a_n$ and $p_n$ converges, we know that the
            addition/subtraction/multiplication by scalar of convergent
            sequence will yield a convergent sequence (proved in the homework).
            Finally, we get that $\abs{a_n}$ converges absolutely and we face a
            contradiction since it converges conditionally.\\
            $\qed$

            Similarly, we can prove that $q_n$ diverges as well. Suppose for
            the sake of contradiction, that $q_n$ converges. Then $q_n =
            \dfrac{a_n - \abs{a_n}}{2}$ and it follows that $\abs{a_n} = a_n -
            2q_n$. We get:
            \begin{align*}
                \sum{\abs{a_n}} = \sum{a_n} - \sum{2q_n}
            \end{align*}
            Since $a_n$ and $q_n$ converges, we know that the
            addition/subtraction/multiplication by scalar of convergent
            sequence will yield a convergent sequence (proved in the homework).
            Finally, we get that $\abs{a_n}$ converges absolutely and we face a
            contradiction since it converges conditionally.\\
            $\qed$

            Hence, we have proven that if $a_n$ converges conditionally, then
            both $p_n$ and $q_n$ must diverge.\\
            $\qed$

        \newpage
        \item[4.]
            Let $(a_n) = \Big(\sqrt{5}, \sqrt{5\sqrt{5}}, \sqrt{5\sqrt{5\sqrt{5}}}, \dots\Big)$.

            Now, it is easy to notice the pattern:
            \begin{align*}
                a_1 &= \sqrt{5}                         &&= a_1         &= 5^{\frac{1}{2}}   = 5^{1 - \frac{1}{2^{1}}}\\
                a_2 &= \sqrt{5\sqrt{5}}                 &&= \sqrt{5a_1} &= 5^{\frac{3}{4}}   = 5^{1 - \frac{1}{2^{2}}}\\
                a_3 &= \sqrt{5\sqrt{5\sqrt{5}}}         &&= \sqrt{5a_2} &= 5^{\frac{7}{8}}   = 5^{1 - \frac{1}{2^{3}}}\\
                a_4 &= \sqrt{5\sqrt{5\sqrt{5\sqrt{5}}}} &&= \sqrt{5a_3} &= 5^{\frac{15}{16}} = 5^{1 - \frac{1}{2^{4}}}\\
                    & .....................             && ..................... & .....................
            \end{align*}
            Then the recursive definition of the sequence can be written as follows:
            \begin{align*}
                a_{n + 1} = \sqrt{5a_{n}}
            \end{align*}
            Additionally, the formula for the $n^{th}$ element of the sequence
            is $a_n = 5^{1 - \frac{1}{2^{n}}}$. Then we have:
            \begin{align*}
                \lim_{x \to \infty}a_n = \lim_{x \to \infty}5^{1 - \frac{1}{2^{n}}} = 5^{1 - 0} = 5^1 = 5
            \end{align*}
            We now need to prove that $\lim_{n \to \infty}a_n = 5$ (the
            sequences converges to $5$).

            Let $l = \dfrac{\epsilon}{2} + 5$ and $k = \dfrac{1}{1 - \log_5^{l}}$.
            We then define $N \in \nats = \ceilb{\log_2^{k}}$
            $\Big(\ceilb{\dfrac{a}{b}}$ is the \textbf{ceiling} of $\dfrac{a}{b}\Big)$.

            Then $\forall \epsilon > 0$ and $\forall n \geq N$ and we have:
            \begin{align*}
                \abs{a_n - 5} &= \absb{5^{1 - \frac{1}{2^{n}}} - 5}\\
                              &\leq \absb{5^{1 - \frac{1}{2^{N}}} - 5}
                                   = \absb{5^{1 - \frac{1}{k}} - 5}\\
                              &= \absb{5^{1 - (1 - \log_5^l)} - 5}
                                   = \absb{5^{\log_5^l} - 5}
                                   = \absb{5^{\log_5^l} - 5}\\
                              &= \absb{l - 5}
                                   = \absb{\frac{\epsilon}{2} + 5 - 5}
                                   = \absb{\frac{\epsilon}{2}}\\
                              &\leq \frac{\epsilon}{2}\\
                              &< \epsilon
            \end{align*}

            We have now shown that $\forall \epsilon > 0, n \geq N, \abs{a_n - 5}
            < \epsilon$ and thus, $\displaystyle\lim_{n \to \infty} a_n = 5$.\\
            $\qed$

            Finally, we have that the recursive definition of the sequence is
            $a_{n + 1} = \sqrt{5a_{n}}$, the limit of the sequence is $5$, and
            we have also proved this fact.

        \newpage
        \item[5.]
            Suppose, for the sake of contradiction, that every convergent
            subsequence of the sequence $(x_n)$ converges to the same value
            $L$, but the sequence does not converge to $L$. Since $x(n)$ does
            not converge to $L$, it follows that $\exists n \geq N$ s.t.
            $\forall \epsilon > 0, \abs{x_n - L} \geq 0$. Thus, we have found a
            subsequence $x_{n_t}$ s.t. $\forall n \geq N, \abs{x_{nt} - L} \geq
            \epsilon$. Let us now find a subsequence that is not in the
            $\epsilon$-neighborhood of $L$. We proceed by constructing a
            subsequence in the following manner:
            \begin{equation*}
                \begin{cases}
                    \text{If } N = 1, n_1 = 1                  \text{ with } a_{n_1} \text{ not in the $\epsilon$-neighborhood of } L\\
                    \text{If } N = 2, n_2 = \max{(n_1 + 1, 2}) \text{ with } a_{n_2} \text{ not in the $\epsilon$-neighborhood of } L\\
                    \text{If } N = 3, n_3 = \max{(n_2 + 1, 3}) \text{ with } a_{n_3} \text{ not in the $\epsilon$-neighborhood of } L\\
                    \text{If } N = 4, n_3 = \max{(n_3 + 1, 4}) \text{ with } a_{n_4} \text{ not in the $\epsilon$-neighborhood of } L\\
                    ..........................................
                \end{cases}
            \end{equation*}
            This way, we end up with a subsequence $x_{nk}$ s.t. $\forall k \in
            \nats, x_{n_k}$ is not in the $\epsilon$-neighborhood of $L$.
            Furthermore, since $(x_n)$ is bounded, it follows that $(x_{nk})$
            is also bounded. Now, per \textbf{Bolzano-Weierstrass Theorem}, we
            get that $(x_{nk})$ must contain some convergent subsequence. Thus,
            we have found a convergent subsequence $(x_{nkj})$ that converges
            to $L$ (by definition) and we face a contradiction since the
            subsequence was constructed in a way that none of its terms are in
            the $\epsilon$-neighborhood of $L$. Hence, if $(x_n)$ is a bounded
            sequence of real numbers such that every convergent subsequence of
            $(x_n)$ converges to the same value $L$, $(x_n)$ also converges to
            $L$.\\
            $\qed$

        \newpage
        \item[6.]
            \begin{itemize}
                \item[(a)]
                    Let $f : A \to B : x \mapsto (x, 0.25)$. Then this function
                    is one-to-one. However, $f$ is not onto.

                    Let us first show that it is one-to-one. Suppose, for the
                    sake of contradiction, that $f$ is not one-to-one. Then
                    $\exists x_1 \neq x_2$ s.t. $f(x_1) = f(x_2)$. We have:
                    \begin{align*}
                        f(x_1)      &= f(x_2)\\
                        (x_1, 0.25) &= (x_2, 0.25)\\
                        x_1 = x_2
                    \end{align*}
                    Hence, we got that $x_1 = x_2$ and we face a contradiction
                    since we assumed $x_1 \neq x_2$. Thus, $f : A \to B : x
                    \mapsto (x, 0.25)$ is one-to-one.\\
                    $\qed$

                    On the other hand, $f$ is not onto since there is no value
                    of $x$ such that $f(x) = (x, 0.5)$.

                    Finally, we found a function $f : A \to B : x \mapsto (x,
                    0.25)$ that is one-to-one, but not onto.

                \item[(b)]
                    Suppose $(a, b)$ is an input to the function and we want to
                    make the output unique as well. Let us consider the decimal
                    expansions of $a$ and $b$. We get $a = 0.a_1a_2a_3\dots$
                    and $b = 0.b_1b_2b_3\dots$ with $a_1, b_1, a_2, b_2, \dots
                    \in \nats$. We can then construct number $c =
                    0.a_1b_1a_2b_2a_3b_3\dots$. Now, since the decimal
                    expansions of $a$ and $b$ are unique, the number built by
                    alternating the digits in the decimal expansions is also
                    unique. Thus, $g : B \to A : (a, b) \mapsto c = g : B \to A
                    : (0.a_1a_2a_3\dots, b_1b_2b_3\dots) \mapsto
                    0.a_1b_1a_2b_2a_3b_3\dots$ is one-to-one.
                    \\
                    \\
                    However, this function is not onto. Suppose, for the sake
                    of contradiction that $g$ is onto. Consider the following
                    output in the codomain $A$: $g(x) = 0.9b_19b_29b_3\dots$.
                    The only way for this to happen is if the function has the
                    following form: $g(0.999\dots, 0.b_1b_2b_3\dots)$. Now,
                    recall that $0.999\dots = \dfrac{9}{9} = 1$, but $1 \notin
                    A$ and hence, we face a contradiction. Thus, $g$ is not
                    onto.\\
                    $\qed$

                    Finally, we have found a function $g : B \to A :
                    (0.a_1a_2a_3\dots, b_1b_2b_3\dots) \mapsto
                    0.a_1b_1a_2b_2a_3b_3\dots$ that is one-to-one, but not
                    onto.

                \item[(c)]
                    Let $u(x) : A \to \reals : x \mapsto x + 1$ and let $v(x) :
                    \reals \to A : x \mapsto \dfrac{1}{2^x + 1}$. If we now
                    show that both $u : A \to \reals$ and $v : \reals \to A$
                    are one-to-one, we have effectively shown that $A \sim
                    \reals$.

                    Let us first show that $u(x) : A \to \reals : x \mapsto y$
                    is one-to-one. Suppose, for the sake of contradiction, that
                    $u$ is not one-to-one. The $\exists x_1 \neq x_2$ s.t.
                    $u(x_1) = u(x_2)$. We have:
                    \begin{align*}
                        u(x_1)  &= u(x_2)\\
                        x_1 + 1 &= x_2 + 1\\
                        x_1 = x_2
                    \end{align*}
                    Hence, we got that $x_1 = x_2$ and we face a contradiction
                    since we assumed $x_1 \neq x_2$. Thus, $u : A \to \reals :
                    x \mapsto x + 1$ is one-to-one.\\
                    $\qed$

                    Let us now prove that $v(x) : \reals \to A : x \mapsto
                    \dfrac{1}{2^x + 1}$ is one-to-one. Suppose, for the sake of
                    contradiction, that $v$ is not one-to-one. The $\exists x_1
                    \neq x_2$ s.t. $v(x_1) = u(v_2)$. We have:
                    \begin{align*}
                        v(x_1)                 &= v(x_2)\\
                        \dfrac{1}{2^{x_1} + 1} &= \dfrac{1}{2^{x_2} + 1}\\
                        2^{x_1} + 1            &= 2^{x_2} + 1\\
                        2^{x_1}                &= 2^{x_2}\\
                        x_1                    &= x_2
                    \end{align*}
                    Hence, we got that $x_1 = x_2$ and we face a contradiction
                    since we assumed $x_1 \neq x_2$. Thus, $u : \reals \to A :
                    x \mapsto \dfrac{1}{2^x + 1}$ is one-to-one.\\
                    $\qed$

                    Thus, we found functions $u : A \to \reals$ and $v : \reals
                    \to A$ such that both $u$ and $v$ are one-to-one. Finally,
                    by \textbf{Schroeder-Bernstein Theorem}, we get that $A
                    \sim \reals$.\\
                    $\qed$

                \item[(d)]
                    Using the results obtained in $(a)$ and $(b)$, we get that
                    $A \sim (0, 1) \times (0, 1) \sim A \times A$. Thus, $A
                    \sim A \times A$. In $(c)$, we have proved that $A \sim
                    \reals$.  It follows that $\reals \sim \reals \times \reals
                    \sim \reals^2$. Hence, $\reals \sim \reals^2$.\\
                    $\qed$
            \end{itemize}
\end{itemize}

%%%%%%%%%%%%%%%%%%%%%%%%%%%%%%%%%%%%%%%%%%%%%%%%%%%%%%%%%%%%%%%%%%%%%%%%%%%%%%%
% The End of the Document
%%%%%%%%%%%%%%%%%%%%%%%%%%%%%%%%%%%%%%%%%%%%%%%%%%%%%%%%%%%%%%%%%%%%%%%%%%%%%%%

\end{document}
