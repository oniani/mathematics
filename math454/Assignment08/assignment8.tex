%%%%%%%%%%%%%%%%%%%%%%%%%%%%%%%%%%%%%%%%%%%%%%%%%%%%%%%%%%%%%%%%%%%%%%%%%%%%%%%
%
% Filename: assignment8.tex
% Author:   David Oniani
% Modified: January 01, 2020
%  _         _____   __  __
% | |    __ |_   _|__\ \/ /
% | |   / _` || |/ _ \\  /
% | |__| (_| || |  __//  \
% |_____\__,_||_|\___/_/\_\
%
%%%%%%%%%%%%%%%%%%%%%%%%%%%%%%%%%%%%%%%%%%%%%%%%%%%%%%%%%%%%%%%%%%%%%%%%%%%%%%%

%%%%%%%%%%%%%%%%%%%%%%%%%%%%%%%%%%%%%%%%%%%%%%%%%%%%%%%%%%%%%%%%%%%%%%%%%%%%%%%
% Document Definition
%%%%%%%%%%%%%%%%%%%%%%%%%%%%%%%%%%%%%%%%%%%%%%%%%%%%%%%%%%%%%%%%%%%%%%%%%%%%%%%

\documentclass[11pt]{article}

%%%%%%%%%%%%%%%%%%%%%%%%%%%%%%%%%%%%%%%%%%%%%%%%%%%%%%%%%%%%%%%%%%%%%%%%%%%%%%%
% Packages and Related Settings
%%%%%%%%%%%%%%%%%%%%%%%%%%%%%%%%%%%%%%%%%%%%%%%%%%%%%%%%%%%%%%%%%%%%%%%%%%%%%%%

% Global, document-wide settings
\usepackage[margin=1in]{geometry}
\usepackage[utf8]{inputenc}
\usepackage[english]{babel}

% Other packages
\usepackage{booktabs}
\usepackage{hyperref}
\usepackage{mathtools}
\usepackage{amsthm}
\usepackage{amssymb}
\usepackage[cache=false]{minted}

%%%%%%%%%%%%%%%%%%%%%%%%%%%%%%%%%%%%%%%%%%%%%%%%%%%%%%%%%%%%%%%%%%%%%%%%%%%%%%%
% Command Definitions and Redefinitions
%%%%%%%%%%%%%%%%%%%%%%%%%%%%%%%%%%%%%%%%%%%%%%%%%%%%%%%%%%%%%%%%%%%%%%%%%%%%%%%

% Nice-looking underline
\newcommand\und[1]{\underline{\smash{#1}}}

% Line spacing is 1.5
\renewcommand{\baselinestretch}{1.5}

% Absolute value
\DeclarePairedDelimiter\abs{\lvert}{\rvert}%

% Absolute value big
\DeclarePairedDelimiter\absb{\Big\lvert}{\Big\rvert}%

% Ceiling
\DeclarePairedDelimiter{\ceil}{\lceil}{\rceil}

% Floor
\DeclarePairedDelimiter\floor{\lfloor}{\rfloor}

% % Naturals, Reals, Integers, and Rationals, 
\newcommand{\nats}{\mathbb{N}}
\newcommand{\reals}{\mathbb{R}}
\newcommand{\preals}{\mathbb{R^+}}
\newcommand{\nreals}{\mathbb{R^-}}
\newcommand{\ints}{\mathbb{Z}}
\newcommand{\pints}{\mathbb{Z^+}}
\newcommand{\nints}{\mathbb{Z^-}}
\newcommand{\rats}{\mathbb{Q}}
\newcommand{\prats}{\mathbb{Q^+}}
\newcommand{\nrats}{\mathbb{Q^-}}
\newcommand{\irrats}{\mathbb{I}}
\newcommand{\pirrats}{\mathbb{I^+}}
\newcommand{\nirrats}{\mathbb{I^-}}

%%%%%%%%%%%%%%%%%%%%%%%%%%%%%%%%%%%%%%%%%%%%%%%%%%%%%%%%%%%%%%%%%%%%%%%%%%%%%%%
% Miscellaneous
%%%%%%%%%%%%%%%%%%%%%%%%%%%%%%%%%%%%%%%%%%%%%%%%%%%%%%%%%%%%%%%%%%%%%%%%%%%%%%%

% Setting stuff
\setlength{\parindent}{0pt}  % Remove indentations from paragraphs

% PDF information and nice-looking urls
\hypersetup{%
  pdfauthor={David Oniani},
  pdftitle={Real Analysis},
  pdfsubject={Mathematics, Real Analysis, Real Numbers},
  pdfkeywords={Mathematics, Real Analysis, Real Numbers},
  pdflang={English},
  colorlinks=true,
  linkcolor={black!50!blue},
  citecolor={black!50!blue},
  urlcolor={black!50!blue}
}

%%%%%%%%%%%%%%%%%%%%%%%%%%%%%%%%%%%%%%%%%%%%%%%%%%%%%%%%%%%%%%%%%%%%%%%%%%%%%%%
% Author(s), Title, and Date
%%%%%%%%%%%%%%%%%%%%%%%%%%%%%%%%%%%%%%%%%%%%%%%%%%%%%%%%%%%%%%%%%%%%%%%%%%%%%%%

% Author(s)
\author{David Oniani\\
        Luther College\\
        \href{mailto:oniada01@luther.edu}{oniada01@luther.edu}}

% Title
\title{\rule{\paperwidth - 150pt}{1pt}\textbf{\\\textit{Real Analysis}\\}\rule
{\paperwidth - 150pt}{1pt}\\\textbf{Assignment \textnumero8}\\{\normalsize
Instructor: Dr. Eric Westlund}}

% Date
\date{\today}

%%%%%%%%%%%%%%%%%%%%%%%%%%%%%%%%%%%%%%%%%%%%%%%%%%%%%%%%%%%%%%%%%%%%%%%%%%%%%%%
% Beginning of the Document
%%%%%%%%%%%%%%%%%%%%%%%%%%%%%%%%%%%%%%%%%%%%%%%%%%%%%%%%%%%%%%%%%%%%%%%%%%%%%%%

\begin{document}
\maketitle

%%%%%%%%%%%%%%%%%%%%%%%%%%%%%%%%%%%%%%%%%%%%%%%%%%%%%%%%%%%%%%%%%%%%%%%%%%%%%%%
%
% Homework
%
% 4.4 # 1, 6, 11
% 4.5 # 7
%
%%%%%%%%%%%%%%%%%%%%%%%%%%%%%%%%%%%%%%%%%%%%%%%%%%%%%%%%%%%%%%%%%%%%%%%%%%%%%%%

\begin{itemize}
    \item[4.4.1]
        \begin{itemize}
            \item[(a)]
                \textbf{Per Theorem 4.3.4}, we know that products of continuous
                functions are continuous. Hence, it suffices to show that the
                function $g(x) = x$ is continuous on all of $\reals$. Now, let
                $\epsilon < 0$ be given and choose $\delta = \epsilon$.  Then,
                we have that $\forall x, y \in \reals, \abs{x - y} < \delta$.
                It follows that $\abs{g(x) - g(y)} = \abs{x - y} < \epsilon$.\\
                $\qed$

            \item[(b)]
                Let $x_n = n$ and $y_n = n - \dfrac{1}{n}$. Then $\abs{x_n -
                y_n} \to 0$. However, we have:
                \begin{equation*}
                    \abs{f(x_n) - f(y_n)} = \absb{n^3 - \Big(n - \dfrac{1}{n}
                    \Big)^3} = \absb{3n + \dfrac{1}{n^3} - \dfrac{3}{n}} \geq n
                    \to \infty
                \end{equation*}
                Hence, \textbf{by Theorem 4.4.5}, the function $f$ is not
                uniformly continuous on $\reals$.\\
                $\qed$

            \item[(c)]
                Let $A \subseteq \reals$ be any bounded subset. Then
                $\overline{A}$, the closure of $A$ in $\reals$, is compact.
                \textbf{By Theorem 4.4.7 (Uniform Continuity on Compact Sets)},
                $f$ is uniformly continuous on $\overline{A}$, and hence on any
                subset of $\overline{A}$. Thus, $f$ is uniformly continuous on
                $A$. Finally, since $A$ was chosen arbitrarily, we get that $f$
                is uniformly continuous on any bounded subset of $\reals$.\\
                $\qed$
        \end{itemize}

    \item[4.4.6]
        \begin{itemize}
            \item[(a)]
                Such request is possible.\\
                Let $f(x) = \dfrac{1}{x}$ for $x \in (0, 1)$. Then $f$ is
                continuous on $(0, 1)$. Now, let $x_n = \dfrac{1}{n}$.  Then
                $x_n$ is Cauchy. However, $f(x_n) = n$ and thus, $(f(x_n))$ is
                not Cauchy.

            \item[(b)]
                Such request is impossible.\\ Suppose, for the sake of
                contradiction, that $f$ is a uniformly continuous function on
                $(0, 1)$, $x_n$ is the Cauchy sequence, and $f(x_n)$ is not a
                Cauchy sequence. Then $\forall \epsilon > 0, \exists \delta >
                0$ s.t.  given $\abs{x - y} < \delta$, $\abs{f(x) - f(y)} <
                \epsilon$.  Now, since $(x_n)$ is Cauchy, $\exists N > 0$ s.t.
                $\abs{x_n - x_m} < \delta$ with $n, m > N$. Then, it follows
                that $\abs{f(x_n) - f(x_m)} < \epsilon$ which implies that
                $f(x_n)$ is a Cauchy sequence and we face a contradiction as we
                have assumed that $f(x)$ is not a Cauchy sequence.\\
                $\qed$

            \item[(c)]
                Such request is impossible.\\
                Suppose, for the sake of contradiction, that $f : [0, \infty)
                \to \reals$ is a continuous function, $(x_n)$ is a Cauchy
                sequence, and $f(x_n)$ is not a Cauchy sequence. If $(x_n) \to
                x$, then $[0, \infty)$ is closed and containts $x$. Finally,
                since $f$ is continuous on $x$, by definition, we get $(x_n)
                \to x$ and it follows that $f(x_n) \to f(x)$. Thus, such
                request is impossible.\\
                $\qed$
        \end{itemize}

    \item[4.4.11]
        We will first prove the statement directly and then prove its converse.
        \\
        Suppose that $g$ is continuous function on $O \subseteq \reals$. Let $x
        \in g^{-1}(O)$ which implies that $g(x) \in O$. Since $O$ is open,
        $\exists \epsilon > 0$ s.t. $V_\epsilon(g(x)) \subseteq O$.
        Furthermore, as $g$ is continuous, it follows that $\exists \delta > 0$
        s.t. if $y \in V_\delta(\epsilon)$, we have $g(y) \in V_\epsilon(g(x))
        \subseteq O$. We get that $\forall y \in V_\delta(x), g(y) \in O$ and
        thus, $y \in g^{-1}(O)$. Now, since $y$ is chosen arbitrarily,
        $V_\delta(x) \subseteq g^{-1}(O)$. Furthermore, since the choice of $x$
        is also arbitrary, every element of $g^{-1}(O)$ has some
        $\delta$-neighborhood contained in $g^{-1}(O)$. Hence, $g^{-1}(O)$ is
        open.\\
        $\qed$
        \\
        Conversely, suppose, $g^{-1}(O)$ is open whenever $O \subseteq \reals$
        with $O$ being open. Let $O = V_\epsilon(g(c))$. Then it follows that
        $g^{-1}(O)$ is open. Now, since $c \in g^{-1}(O)$ and $c$ is an
        interior point, $\exists \delta > 0$ s.t $V_\delta(c) \subseteq
        g^{-1}(O)$. Then, by definition, if $x \in V_\delta(c)$, it follows
        that $g(x) \in O = V_\epsilon(g(c))$ and we finally got that $g$ is a
        continuous function.\\
        $\qed$
        \\
        Finally, we have proven both the direct statement and its converse and
        hence, $g$ is continuous if and only if $g^{-1}(O)$ is open whenever $O
        \subseteq \reals$ is an open set.\\
        $\qed$

    \item[4.5.7]
        Notice that $\forall 0 \leq x \leq 1$, we have $0 \leq f(x) \leq 1$.
        Now, consider the following function:
        \begin{equation*}
            g(x) : [0, 1] \to \reals : x \mapsto f(x) - x
        \end{equation*}
        Then $g$ is continuous on $[0, 1]$. Furthermore, we have $g(0) = f(0)
        \geq 0$. However, $g(1) = f(1) - 1 \leq 0$. Hence, \textbf{by the
        Intermediate Value Theorem (Theorem 4.5.1)}, $\exists x_0 \in [0, 1]$
        s.t.  $g(x_0) = 0$.  Hence, we have that $f(x_0) = x_0$.\\
        $\qed$
\end{itemize}

%%%%%%%%%%%%%%%%%%%%%%%%%%%%%%%%%%%%%%%%%%%%%%%%%%%%%%%%%%%%%%%%%%%%%%%%%%%%%%%
% The End of the Document
%%%%%%%%%%%%%%%%%%%%%%%%%%%%%%%%%%%%%%%%%%%%%%%%%%%%%%%%%%%%%%%%%%%%%%%%%%%%%%%

\end{document}
