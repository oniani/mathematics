%%%%%%%%%%%%%%%%%%%%%%%%%%%%%%%%%%%%%%%%%%%%%%%%%%%%%%%%%%%%%%%%%%%%%%%%%%%%%%%
%
% Filename: assignment3.tex
% Author:   David Oniani
% Modified: December 14, 2020
%  _         _____   __  __
% | |    __ |_   _|__\ \/ /
% | |   / _` || |/ _ \\  /
% | |__| (_| || |  __//  \
% |_____\__,_||_|\___/_/\_\
%
%%%%%%%%%%%%%%%%%%%%%%%%%%%%%%%%%%%%%%%%%%%%%%%%%%%%%%%%%%%%%%%%%%%%%%%%%%%%%%%

%%%%%%%%%%%%%%%%%%%%%%%%%%%%%%%%%%%%%%%%%%%%%%%%%%%%%%%%%%%%%%%%%%%%%%%%%%%%%%%
% Document Definition
%%%%%%%%%%%%%%%%%%%%%%%%%%%%%%%%%%%%%%%%%%%%%%%%%%%%%%%%%%%%%%%%%%%%%%%%%%%%%%%

\documentclass[11pt]{article}

%%%%%%%%%%%%%%%%%%%%%%%%%%%%%%%%%%%%%%%%%%%%%%%%%%%%%%%%%%%%%%%%%%%%%%%%%%%%%%%
% Packages and Related Settings
%%%%%%%%%%%%%%%%%%%%%%%%%%%%%%%%%%%%%%%%%%%%%%%%%%%%%%%%%%%%%%%%%%%%%%%%%%%%%%%

% Global, document-wide settings
\usepackage[margin=1in]{geometry}
\usepackage[utf8]{inputenc}
\usepackage[english]{babel}

% Other packages
\usepackage{booktabs}
\usepackage{hyperref}
\usepackage{mathtools}
\usepackage{amsthm}
\usepackage{amssymb}
\usepackage[cache=false]{minted}

%%%%%%%%%%%%%%%%%%%%%%%%%%%%%%%%%%%%%%%%%%%%%%%%%%%%%%%%%%%%%%%%%%%%%%%%%%%%%%%
% Command Definitions and Redefinitions
%%%%%%%%%%%%%%%%%%%%%%%%%%%%%%%%%%%%%%%%%%%%%%%%%%%%%%%%%%%%%%%%%%%%%%%%%%%%%%%

% Nice-looking underline
\newcommand\und[1]{\underline{\smash{#1}}}

% Line spacing is 1.5
\renewcommand{\baselinestretch}{1.5}

% Absolute value
\DeclarePairedDelimiter\abs{\lvert}{\rvert}%

% Ceiling
\DeclarePairedDelimiter{\ceil}{\lceil}{\rceil}

% Floor
\DeclarePairedDelimiter\floor{\lfloor}{\rfloor}

% % Naturals, Reals, Integers, and Rationals, 
\newcommand{\nats}{\mathbb{N}}
\newcommand{\reals}{\mathbb{R}}
\newcommand{\preals}{\mathbb{R^+}}
\newcommand{\nreals}{\mathbb{R^-}}
\newcommand{\ints}{\mathbb{Z}}
\newcommand{\pints}{\mathbb{Z^+}}
\newcommand{\nints}{\mathbb{Z^-}}
\newcommand{\rats}{\mathbb{Q}}
\newcommand{\prats}{\mathbb{Q^+}}
\newcommand{\nrats}{\mathbb{Q^-}}
\newcommand{\irrats}{\mathbb{I}}
\newcommand{\pirrats}{\mathbb{I^+}}
\newcommand{\nirrats}{\mathbb{I^-}}

%%%%%%%%%%%%%%%%%%%%%%%%%%%%%%%%%%%%%%%%%%%%%%%%%%%%%%%%%%%%%%%%%%%%%%%%%%%%%%%
% Miscellaneous
%%%%%%%%%%%%%%%%%%%%%%%%%%%%%%%%%%%%%%%%%%%%%%%%%%%%%%%%%%%%%%%%%%%%%%%%%%%%%%%

% Setting stuff
\setlength{\parindent}{0pt}  % Remove indentations from paragraphs

% PDF information and nice-looking urls
\hypersetup{%
  pdfauthor={David Oniani},
  pdftitle={Real Analysis},
  pdfsubject={Mathematics, Real Analysis, Real Numbers},
  pdfkeywords={Mathematics, Real Analysis, Real Numbers},
  pdflang={English},
  colorlinks=true,
  linkcolor={black!50!blue},
  citecolor={black!50!blue},
  urlcolor={black!50!blue}
}

%%%%%%%%%%%%%%%%%%%%%%%%%%%%%%%%%%%%%%%%%%%%%%%%%%%%%%%%%%%%%%%%%%%%%%%%%%%%%%%
% Author(s), Title, and Date
%%%%%%%%%%%%%%%%%%%%%%%%%%%%%%%%%%%%%%%%%%%%%%%%%%%%%%%%%%%%%%%%%%%%%%%%%%%%%%%

% Author(s)
\author{David Oniani\\
        Luther College\\
        \href{mailto:oniada01@luther.edu}{oniada01@luther.edu}}

% Title
\title{\rule{\paperwidth - 150pt}{1pt}\textbf{\\\textit{Real Analysis}\\}\rule
{\paperwidth - 150pt}{1pt}\\\textbf{Assignment \textnumero3}\\{\normalsize
Instructor: Dr. Eric Westlund}}

% Date
\date{\today}

%%%%%%%%%%%%%%%%%%%%%%%%%%%%%%%%%%%%%%%%%%%%%%%%%%%%%%%%%%%%%%%%%%%%%%%%%%%%%%%
% Beginning of the Document
%%%%%%%%%%%%%%%%%%%%%%%%%%%%%%%%%%%%%%%%%%%%%%%%%%%%%%%%%%%%%%%%%%%%%%%%%%%%%%%

\begin{document}
\maketitle

%%%%%%%%%%%%%%%%%%%%%%%%%%%%%%%%%%%%%%%%%%%%%%%%%%%%%%%%%%%%%%%%%%%%%%%%%%%%%%%
%
% Homework
%
% 2.4 # 1, 7
% 2.5 # 1, 9
%
%%%%%%%%%%%%%%%%%%%%%%%%%%%%%%%%%%%%%%%%%%%%%%%%%%%%%%%%%%%%%%%%%%%%%%%%%%%%%%%

\begin{itemize}
    \item[2.4.1]
        \begin{itemize}
            \item[(a)]
                Let us first show that $(x_n)$ is monotonically decreasing. We
                can use induction for this proof.
                \\\\
                \textit{Base case}: $x_1 = 3$ and $x_2 = \dfrac{1}{4 - x_1} =
                \dfrac{1}{1} = 1$.  It follows that $x_2 - x_1 = 1 - 3 = -2 <
                0$. Hence, the base case is satisfied.
                \\\\
                \textit{Inductive step}: suppose $x_n - x_{n + 1} > 0$.
                We now have to show that $x_{n + 1} - x_{n + 2} > 0$.

                \begin{align*}
                    x_{n + 1} - x_{n + 2} & = \dfrac{1}{4 - x_n} -
                        \dfrac{1}{4 - x_{n + 1}}\\
                                          & = \dfrac{x_n -x_{n + 1}}{(4 -
                                              x_n)(4 - x_{n + 1})} > 0
                \end{align*}

                Thus, by assuming that $x_n - x_{n + 1} > 0$, we got that $x_{n
                + 1} - x_{n + 2} > 0$ as well. Hence, $\forall n \in \nats, x_n
                > x_{n + 1}$. Additionally, $(x_n)$ is a bounded sequence since
                $\forall n \in \nats, 0 < x_n < 5$. Finally, by
                \textbf{Monotone Convergence Theorem}, we get that $(x_n)$
                converges.\\
                $\qed$

            \item[(b)]
                As $\lim{x_n}$ exists, let $\lim{x_n} = X$. Then $\forall
                \epsilon > 0, \exists N \in \nats$ s.t. if $n \geq N, \abs{x_n
                - X} < \epsilon$. Now, since $n + 1 > n \geq N$, we get that
                $\abs{x_{n + 1} - X} < \epsilon$ and hence, $\lim{x_{n + 1}} =
                X$.\\
                $\qed$

            \item[(c)]
                If we take the limits of both sides, we know from $(b)$ that
                $\lim{x_n} = \lim{x_{n + 1}}$. Then, by \textbf{Algebraic Limit
                Theorem}, we get the following:
                \begin{align}
                    &\lim{x_{n + 1}} = \dfrac{1}{4 - \lim{x_n}} = \lim{x_n}\\
                    &\lim{x_n} = \dfrac{1}{4 - \lim{x_{n + 1}}}\\
                    &\lim{x^2_n} - 4\lim{x_n} + 1 = 0
                \end{align}

                From $(3)$, we have that $\lim{x_n} = 2 \pm \sqrt{3}$. Finally,
                since $x_1 < 2 + \sqrt{3}$ and the sequence is monotonically
                decreasing, it follows that the $\lim{x_n} = 2 - \sqrt{3}$.
        \end{itemize}

    \item[2.4.7]
        \begin{itemize}
            \item[(a)]
                Since $(a_n)$ is bounded, $\exists B_1, B_2$ s.t. $\forall n
                \in N, B_1 < a_n < B_2$. It follows that\\ $\{a_m \mid m \geq n
                + 1\} \subseteq \{a_m \mid m \geq n\}$. We get:

                $$A \leq \sup{\{ a_m \mid m \geq n + 1\}} = y_{n + 1} \leq
                \sup{\{a_m \mid m \geq n\}} = y_n \leq B_2$$

                Finally, since $(y_n)$ is both bounded and decreasing, by
                \textbf{Monotone Convergence Theorem}, we conclude that $(y_n)$
                converges.

            \item[(b)]
                Let $x_n = \inf{\{a_m \mid m \geq n\}}$. Similar to part $(a)$,
                $x_n$ converges by \textbf{Monotone Convergence Theorem}. To
                see this, all one needs to do is to reverse all the
                inequalities in part $(a)$. In this case, it is both bounded
                and increasing (in lieu of decreasing). We can then define
                limit inferior as $\lim{\inf{{a_n}}} = \lim{x_n}$ and it will
                always exist.
                
            \item[(c)]
                As $(a_n)$ is bounded, we get:

                $$x_n = \inf{\{a_m \mid m \geq n\}} \leq \sup{\{a_m \mid m \geq
                n\}} = y_n$$

                Then, by \textbf{Order Limit Theorem}, it follows that $\forall
                n \in \nats$:

                $$\lim{\inf{a_n}} = \lim_{n \to \infty}x_n \leq \lim_{n \to
                \infty} y_n = \lim{\sup{a_n}}$$\\
                $\qed$

                One such example can be a sequence $t_n = (-1)^n$. In this
                case, we would have $\lim{\inf{a_n}} = -1$ and $\lim{\sup{a_n}}
                = 1$. Thus, we get that $\lim{\inf{a_n}} \leq \lim{\sup{a_n}}$
                since $-1 < 1$ and the inequality is strict.

            \item[(d)]
                Let us first show that if $\lim{\inf{a_n}} = \lim{\sup{a_n}}$,
                then $\lim a_n$ exists. Suppose that $\lim{\inf{a_n}} =
                \lim{\sup{a_n}}$. Then, $\forall n \in \nats$, we get: $$x_n =
                \inf{\{a_m \mid m \geq n\}} \leq a_n \leq \sup{\{a_m \mid m
                \geq n\}} = y_n$$ Then, we get that $\lim_{n \to \infty}x_n =
                \lim{\inf{a_n}} = \lim{\sup{a_n}} = \lim_{n \to \infty}y_n$ and
                it follows by the \textbf{Squeeze Theorem} (proven in the
                previous homework) that $a_n$ converges to $\lim{\inf{a_n}} =
                \lim{\sup{a_n}} = \lim{a_n}$.\\
                $\qed$
                \\\\
                Conversely, suppose that $\lim{\inf{a_n}}$ exists. Then $\lim
                a_n = A$ and $\exists N \in \nats$ s.t. $\forall n \geq N, A -
                \epsilon < a_n < A + \epsilon, \forall \epsilon > 0$. Then $A -
                \epsilon \leq \lim{\inf{a_n}}$ and $\lim{\sup{a_n}} \leq A +
                \epsilon$. Finally, we get that: $$A - \epsilon \leq
                \lim{\inf{a_n}} \leq \lim{\sup{a_n}} \leq A + \epsilon$$

                and since $\epsilon$ is chosen arbitrarily, we get that $\lim{
                    \inf{a_n}} = \lim{\sup{a_n}} = \lim{a_n}$.\\
                $\qed$
                \\\\
                Thus, we have now shown that $\lim{\inf{a_n}} =
                \lim{\sup{a_n}}$ if and only if $\lim a_n$ exists. And in this
                case, all three share the same value.\\
                $\qed$
        \end{itemize}

    \item[2.5.1]
        \begin{itemize}
            \item[(a)]
                Per Bolzano-Weierstrass theorem, such sequence cannot exist.

            \item[(b)]
                Such sequence exists. Let $(a_n) = \Big(\dfrac{1}{2}, -1,
                \dfrac{1}{4}, \dfrac{1}{2}, \dfrac{1}{8}, -\dfrac{1}{3},
                \dfrac{1}{16}, \dfrac{1}{4},\dots\Big)$. Then notice that the
                odd terms make up a sequence $(b_n) = \Big(\dfrac{1}{2},
                \dfrac{1}{4}, \dfrac{1}{8}, \dfrac{1}{16}, \dots\Big)$ which
                converges to $\dfrac{\frac{1}{2}}{1 - \frac{1}{2}} = 1$. On the
                other hand, the even terms make up a sequence $(c_n) = \Big(-1,
                \dfrac{1}{2}, -\dfrac{1}{3}, \dfrac{1}{4}, \dots\Big)$, which
                represents a sequence $\dfrac{(-1)^{n}}{n}$ and converges $0$.
                Thus, we found a sequence that does not contain 0, but its
                subsequences converge to 0 and 1.

            \item[(c)]
                Such sequence exists. To construct such sequence it suffices to
                make each number appear infinitely many times. Let $(s_n) =
                \Big(1, 1, \dfrac{1}{2}, 1, \dfrac{1}{2}, \dfrac{1}{3}, 1,
                \dfrac{1}{2}, \dfrac{1}{3}, \dfrac{1}{4}, 1, \dfrac{1}{2},
                \dfrac{1}{3}, \dfrac{1}{4}, \dfrac{1}{5}\dots\Big)$. Hence, at
                a step $k \in \nats$, we add a new number $\dfrac{1}{k}$ to the
                sequence. This ensures that the sequence contains subsequences
                converging to every point in the infinite set $\{1,
                \dfrac{1}{2}, \dfrac{1}{3}, \dfrac{1}{4}, \dfrac{1}{5},
                \dots\}$ and thus, such sequence exists.

            \item[(d)]
                There is no such sequence. Let $S = \{1, \dfrac{1}{2},
                \dfrac{1}{3}, \dfrac{1}{4}, \dfrac{1}{5}, \dots\}$. Then, since
                $\lim_{n \to \infty} S = 0$, there must exist at least one
                sequence that converges to 0. On the other hand, 0 is not in
                the set $S$. Hence, such sequence cannot exist.
        \end{itemize}

    \item[2.5.9]
        As sequence $(a_n)$ is bounded, $\exists B > 0$ s.t. $\forall n \in
        \nats, -B < a_n < B$. It follows that $\forall x$ s.t. $x < -B, x \in
        S$. Hence, $S \neq \emptyset$. Additionally, since $x < a_n < B$ for
        infinitely many $n$, $B$ is the upper bound for $S$. Then we get, per
        the \textbf{Axiom of Completeness}, that $s = \sup{S}$ exists. Now,
        suppose, for the sake of contradiction, that the sequence does not
        converge to $s = \sup{S}$ and $a_{n_m} \geq \sup{S}$ ($a_{n_m}$
        represents the $m^{th}$ element of the sequence $a_n$). Then $\exists
        \epsilon > 0$ s.t. $\forall m \geq M$, we get $a_{n_m} - s \geq
        \epsilon$. Since there are infinitely many of such $m$, let $M =
        \max{\{m_{t}\}} + 1$ where $\{m_{t}\}$ is any finite subset of $m$s.
        However, now we have effectively found $e = \sup{S} + \epsilon$ which
        is an element the set $S$ ($e \in S$) and we face a contradiction since
        $\forall x \in S, x \leq \sup{S}$. Thus, the sequence $(a_n)$ converges
        to $\sup{S}$.\\
        $\qed$
\end{itemize}

%%%%%%%%%%%%%%%%%%%%%%%%%%%%%%%%%%%%%%%%%%%%%%%%%%%%%%%%%%%%%%%%%%%%%%%%%%%%%%%
% The End of the Document
%%%%%%%%%%%%%%%%%%%%%%%%%%%%%%%%%%%%%%%%%%%%%%%%%%%%%%%%%%%%%%%%%%%%%%%%%%%%%%%

\end{document}
