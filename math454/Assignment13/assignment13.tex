%%%%%%%%%%%%%%%%%%%%%%%%%%%%%%%%%%%%%%%%%%%%%%%%%%%%%%%%%%%%%%%%%%%%%%%%%%%%%%%
%
% Filename: assignment13.tex
% Author:   David Oniani
% Modified: January 19, 2020
%  _         _____   __  __
% | |    __ |_   _|__\ \/ /
% | |   / _` || |/ _ \\  /
% | |__| (_| || |  __//  \
% |_____\__,_||_|\___/_/\_\
%
%%%%%%%%%%%%%%%%%%%%%%%%%%%%%%%%%%%%%%%%%%%%%%%%%%%%%%%%%%%%%%%%%%%%%%%%%%%%%%%

%%%%%%%%%%%%%%%%%%%%%%%%%%%%%%%%%%%%%%%%%%%%%%%%%%%%%%%%%%%%%%%%%%%%%%%%%%%%%%%
% Document Definition
%%%%%%%%%%%%%%%%%%%%%%%%%%%%%%%%%%%%%%%%%%%%%%%%%%%%%%%%%%%%%%%%%%%%%%%%%%%%%%%

\documentclass[11pt]{article}

%%%%%%%%%%%%%%%%%%%%%%%%%%%%%%%%%%%%%%%%%%%%%%%%%%%%%%%%%%%%%%%%%%%%%%%%%%%%%%%
% Packages and Related Settings
%%%%%%%%%%%%%%%%%%%%%%%%%%%%%%%%%%%%%%%%%%%%%%%%%%%%%%%%%%%%%%%%%%%%%%%%%%%%%%%

% Global, document-wide settings
\usepackage[margin=1in]{geometry}
\usepackage[utf8]{inputenc}
\usepackage[english]{babel}

% Other packages
\usepackage{booktabs}
\usepackage{hyperref}
\usepackage{mathtools}
\usepackage{amsthm}
\usepackage{amssymb}
\usepackage[cache=false]{minted}

%%%%%%%%%%%%%%%%%%%%%%%%%%%%%%%%%%%%%%%%%%%%%%%%%%%%%%%%%%%%%%%%%%%%%%%%%%%%%%%
% Command Definitions and Redefinitions
%%%%%%%%%%%%%%%%%%%%%%%%%%%%%%%%%%%%%%%%%%%%%%%%%%%%%%%%%%%%%%%%%%%%%%%%%%%%%%%

% Nice-looking underline
\newcommand\und[1]{\underline{\smash{#1}}}

% Line spacing is 1.5
\renewcommand{\baselinestretch}{1.5}

% Absolute value
\DeclarePairedDelimiter\abs{\lvert}{\rvert}%

% Absolute value big
\DeclarePairedDelimiter\absb{\Big\lvert}{\Big\rvert}%

% Ceiling
\DeclarePairedDelimiter{\ceil}{\lceil}{\rceil}

% Floor
\DeclarePairedDelimiter\floor{\lfloor}{\rfloor}

% % Naturals, Reals, Integers, and Rationals, 
\newcommand{\nats}{\mathbb{N}}
\newcommand{\reals}{\mathbb{R}}
\newcommand{\preals}{\mathbb{R^+}}
\newcommand{\nreals}{\mathbb{R^-}}
\newcommand{\ints}{\mathbb{Z}}
\newcommand{\pints}{\mathbb{Z^+}}
\newcommand{\nints}{\mathbb{Z^-}}
\newcommand{\rats}{\mathbb{Q}}
\newcommand{\prats}{\mathbb{Q^+}}
\newcommand{\nrats}{\mathbb{Q^-}}
\newcommand{\irrats}{\mathbb{I}}
\newcommand{\pirrats}{\mathbb{I^+}}
\newcommand{\nirrats}{\mathbb{I^-}}

%%%%%%%%%%%%%%%%%%%%%%%%%%%%%%%%%%%%%%%%%%%%%%%%%%%%%%%%%%%%%%%%%%%%%%%%%%%%%%%
% Miscellaneous
%%%%%%%%%%%%%%%%%%%%%%%%%%%%%%%%%%%%%%%%%%%%%%%%%%%%%%%%%%%%%%%%%%%%%%%%%%%%%%%

% Setting stuff
\setlength{\parindent}{0pt}  % Remove indentations from paragraphs

% PDF information and nice-looking urls
\hypersetup{%
  pdfauthor={David Oniani},
  pdftitle={Real Analysis},
  pdfsubject={Mathematics, Real Analysis, Real Numbers},
  pdfkeywords={Mathematics, Real Analysis, Real Numbers},
  pdflang={English},
  colorlinks=true,
  linkcolor={black!50!blue},
  citecolor={black!50!blue},
  urlcolor={black!50!blue}
}

%%%%%%%%%%%%%%%%%%%%%%%%%%%%%%%%%%%%%%%%%%%%%%%%%%%%%%%%%%%%%%%%%%%%%%%%%%%%%%%
% Author(s), Title, and Date
%%%%%%%%%%%%%%%%%%%%%%%%%%%%%%%%%%%%%%%%%%%%%%%%%%%%%%%%%%%%%%%%%%%%%%%%%%%%%%%

% Author(s)
\author{David Oniani\\
        Luther College\\
        \href{mailto:oniada01@luther.edu}{oniada01@luther.edu}}

% Title
\title{\rule{\paperwidth - 150pt}{1pt}\textbf{\\\textit{Real Analysis}\\}\rule
{\paperwidth - 150pt}{1pt}\\\textbf{Assignment \textnumero13}\\{\normalsize
Instructor: Dr. Eric Westlund}}

% Date
\date{\today}

%%%%%%%%%%%%%%%%%%%%%%%%%%%%%%%%%%%%%%%%%%%%%%%%%%%%%%%%%%%%%%%%%%%%%%%%%%%%%%%
% Beginning of the Document
%%%%%%%%%%%%%%%%%%%%%%%%%%%%%%%%%%%%%%%%%%%%%%%%%%%%%%%%%%%%%%%%%%%%%%%%%%%%%%%

\begin{document}
\maketitle

%%%%%%%%%%%%%%%%%%%%%%%%%%%%%%%%%%%%%%%%%%%%%%%%%%%%%%%%%%%%%%%%%%%%%%%%%%%%%%%
%
% Homework
%
% 7.5 # 8ab
% 7.6 # 1, 3
%
%%%%%%%%%%%%%%%%%%%%%%%%%%%%%%%%%%%%%%%%%%%%%%%%%%%%%%%%%%%%%%%%%%%%%%%%%%%%%%%

\begin{itemize}
    \item[7.5.8]
        \begin{itemize}
            \item[(a)]
                $L_1 = \int_1^1 \frac{1}{x} = 0$. $L(x)$ is differentiable
                since $\frac{1}{t}$ is continuous and it follows by
                \textbf{Theorem 7.5.1 (Fundamental Theorem of Calculus) (part
                (ii))} that $L(x)$ is differentiable with $L(x)^\prime =
                \frac{1}{x}$.

            \item[(b)]
                Keeping $y$ constant, we have:
                \begin{equation*}
                    \frac{d}{dx} L(xy) = yL^\prime(xy)
                                       = y \times \frac{1}{xy}
                                       = \frac{1}{x}
                \end{equation*}
                Now, integrating with respect to $x$ get us the following:
                \begin{equation*}
                    L(xy) = \int_1^{x} \frac{1}{t} dt + c(y)
                    \tag{$c(y)$ is a function of only $y$}
                \end{equation*}
                If we now differentiate with respect to $y$, we get:
                \begin{align}
                    \frac{1}{y} &= c^\prime(y)\\
                    c(y)        &= \int_1^y \frac{1}{t} dt
                \end{align}
                Finally, we have:
                \begin{equation*}
                    L(xy) = \int_0^x \frac{1}{t} dt + \int_0^y \frac{1}{t} dt
                          = L(x) + L(y)
                \end{equation*}
        \end{itemize}

    \item[7.6.1]
        \begin{itemize}
            \item[(a)]
                It follows by the \textbf{density property} that every
                subinterval of any partition has an irrational number in it.
                Hence, the infinum on this interval is $0$. Thus, for any
                partition $P$, we have $L(t, P) = 0$.\\
                $\qed$

            \item[(b)]
                The set of points $\geq \epsilon/2$ are:
                \begin{align*}
                    x &= 0\\
                    x &= \frac{1}{1}\\
                    x &= \frac{1}{2}\\
                    x &= \frac{1}{3}\\
                    &\cdots\\
                    x &= \frac{1}{\floor{\frac{2}{\epsilon}}}
                \end{align*}
                Thus, the size of $D_{\frac{\epsilon}{2}}$ is
                $\floor{\frac{2}{\epsilon} + 1}$.

            \item[(c)]
                Pick the following partition:
                \begin{equation*}
                    \Big\{0, \frac{1}{\floor{2/\epsilon}}\Big\} \cup
                        \Big\{V_{\frac{\epsilon^2}{9}}(x)\Big\}
                \end{equation*}
                Then we have:
                \begin{align*}
                  U(t, P_\epsilon) &= \frac{\epsilon}{2} \cdot 1 +
                        \Big[{\Big({\floor{\frac{2}{\epsilon}} + 1}\Big)
                        \cdot \frac{\epsilon^2}{9}}\Big]\\
                    &= \frac{\epsilon}{2} + \frac{\epsilon^2}{3}\\
                    &\leq \frac{\epsilon}{2} + \frac{\epsilon}{3}
                        \tag{for $\epsilon < 1$}\\
                    &< \epsilon
                \end{align*}
                Since $\sup U(t, P) = 1$, showing this $\forall \epsilon \geq
                1$ is trivial. Hence, any partition will work for $\epsilon
                \geq 1$. Finally, we have constructed a partition $P_\epsilon$
                of $[0, 1]$ s.t. $U(t, P_\epsilon) < \epsilon$.
        \end{itemize}

    \item[7.6.3]
        Let $S = \{s_1, s_2, s_3, \dots\}$ be an arbitary countable set. Then
        $\forall \epsilon > 0$ pick the sequence of intervals $I_n = \Big[s_n -
        \frac{\epsilon}{2^{n + 1}}, s_n + \frac{\epsilon}{2^{n + 1}}\Big]$.
        Notice that $\abs{I_n} = \frac{\epsilon}{2^n}$ and $\{s_1, s_2, s_3,
        \dots \} \subseteq \bigcup_{n = 1}^\infty I_n$. Now, define $I =
        \bigcup_{n = 1}^\infty I_n$ and we have $\abs{I} = \frac{\epsilon}{2} +
        \frac{\epsilon}{2^2} + \frac{\epsilon}{2^3} + \dots =
        \frac{\frac{\epsilon}{2}}{1 - \frac{1}{2}} =
        \frac{\frac{\epsilon}{2}}{\frac{1}{2}} = \epsilon$. Hence, we got that
        $\forall \epsilon > 0, \abs{S} < \epsilon$ and thus, $S$ has the
        measure of $0$.\\
        $\qed$
\end{itemize}

%%%%%%%%%%%%%%%%%%%%%%%%%%%%%%%%%%%%%%%%%%%%%%%%%%%%%%%%%%%%%%%%%%%%%%%%%%%%%%%
% The End of the Document
%%%%%%%%%%%%%%%%%%%%%%%%%%%%%%%%%%%%%%%%%%%%%%%%%%%%%%%%%%%%%%%%%%%%%%%%%%%%%%%

\end{document}
