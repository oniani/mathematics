%%%%%%%%%%%%%%%%%%%%%%%%%%%%%%%%%%%%%%%%%%%%%%%%%%%%%%%%%%%%%%%%%%%%%%%%%%%%%%%
%
% Filename: assignment10.tex
% Author:   David Oniani
% Modified: January 09, 2020
%  _         _____   __  __
% | |    __ |_   _|__\ \/ /
% | |   / _` || |/ _ \\  /
% | |__| (_| || |  __//  \
% |_____\__,_||_|\___/_/\_\
%
%%%%%%%%%%%%%%%%%%%%%%%%%%%%%%%%%%%%%%%%%%%%%%%%%%%%%%%%%%%%%%%%%%%%%%%%%%%%%%%

%%%%%%%%%%%%%%%%%%%%%%%%%%%%%%%%%%%%%%%%%%%%%%%%%%%%%%%%%%%%%%%%%%%%%%%%%%%%%%%
% Document Definition
%%%%%%%%%%%%%%%%%%%%%%%%%%%%%%%%%%%%%%%%%%%%%%%%%%%%%%%%%%%%%%%%%%%%%%%%%%%%%%%

\documentclass[11pt]{article}

%%%%%%%%%%%%%%%%%%%%%%%%%%%%%%%%%%%%%%%%%%%%%%%%%%%%%%%%%%%%%%%%%%%%%%%%%%%%%%%
% Packages and Related Settings
%%%%%%%%%%%%%%%%%%%%%%%%%%%%%%%%%%%%%%%%%%%%%%%%%%%%%%%%%%%%%%%%%%%%%%%%%%%%%%%

% Global, document-wide settings
\usepackage[margin=1in]{geometry}
\usepackage[utf8]{inputenc}
\usepackage[english]{babel}

% Other packages
\usepackage{booktabs}
\usepackage{hyperref}
\usepackage{mathtools}
\usepackage{amsthm}
\usepackage{amssymb}
\usepackage[cache=false]{minted}

%%%%%%%%%%%%%%%%%%%%%%%%%%%%%%%%%%%%%%%%%%%%%%%%%%%%%%%%%%%%%%%%%%%%%%%%%%%%%%%
% Command Definitions and Redefinitions
%%%%%%%%%%%%%%%%%%%%%%%%%%%%%%%%%%%%%%%%%%%%%%%%%%%%%%%%%%%%%%%%%%%%%%%%%%%%%%%

% Nice-looking underline
\newcommand\und[1]{\underline{\smash{#1}}}

% Line spacing is 1.5
\renewcommand{\baselinestretch}{1.5}

% Absolute value
\DeclarePairedDelimiter\abs{\lvert}{\rvert}%

% Absolute value big
\DeclarePairedDelimiter\absb{\Big\lvert}{\Big\rvert}%

% Ceiling
\DeclarePairedDelimiter{\ceil}{\lceil}{\rceil}

% Floor
\DeclarePairedDelimiter\floor{\lfloor}{\rfloor}

% % Naturals, Reals, Integers, and Rationals, 
\newcommand{\nats}{\mathbb{N}}
\newcommand{\reals}{\mathbb{R}}
\newcommand{\preals}{\mathbb{R^+}}
\newcommand{\nreals}{\mathbb{R^-}}
\newcommand{\ints}{\mathbb{Z}}
\newcommand{\pints}{\mathbb{Z^+}}
\newcommand{\nints}{\mathbb{Z^-}}
\newcommand{\rats}{\mathbb{Q}}
\newcommand{\prats}{\mathbb{Q^+}}
\newcommand{\nrats}{\mathbb{Q^-}}
\newcommand{\irrats}{\mathbb{I}}
\newcommand{\pirrats}{\mathbb{I^+}}
\newcommand{\nirrats}{\mathbb{I^-}}

%%%%%%%%%%%%%%%%%%%%%%%%%%%%%%%%%%%%%%%%%%%%%%%%%%%%%%%%%%%%%%%%%%%%%%%%%%%%%%%
% Miscellaneous
%%%%%%%%%%%%%%%%%%%%%%%%%%%%%%%%%%%%%%%%%%%%%%%%%%%%%%%%%%%%%%%%%%%%%%%%%%%%%%%

% Setting stuff
\setlength{\parindent}{0pt}  % Remove indentations from paragraphs

% PDF information and nice-looking urls
\hypersetup{%
  pdfauthor={David Oniani},
  pdftitle={Real Analysis},
  pdfsubject={Mathematics, Real Analysis, Real Numbers},
  pdfkeywords={Mathematics, Real Analysis, Real Numbers},
  pdflang={English},
  colorlinks=true,
  linkcolor={black!50!blue},
  citecolor={black!50!blue},
  urlcolor={black!50!blue}
}

%%%%%%%%%%%%%%%%%%%%%%%%%%%%%%%%%%%%%%%%%%%%%%%%%%%%%%%%%%%%%%%%%%%%%%%%%%%%%%%
% Author(s), Title, and Date
%%%%%%%%%%%%%%%%%%%%%%%%%%%%%%%%%%%%%%%%%%%%%%%%%%%%%%%%%%%%%%%%%%%%%%%%%%%%%%%

% Author(s)
\author{David Oniani\\
        Luther College\\
        \href{mailto:oniada01@luther.edu}{oniada01@luther.edu}}

% Title
\title{\rule{\paperwidth - 150pt}{1pt}\textbf{\\\textit{Real Analysis}\\}\rule
{\paperwidth - 150pt}{1pt}\\\textbf{Assignment \textnumero10}\\{\normalsize
Instructor: Dr. Eric Westlund}}

% Date
\date{\today}

%%%%%%%%%%%%%%%%%%%%%%%%%%%%%%%%%%%%%%%%%%%%%%%%%%%%%%%%%%%%%%%%%%%%%%%%%%%%%%%
% Beginning of the Document
%%%%%%%%%%%%%%%%%%%%%%%%%%%%%%%%%%%%%%%%%%%%%%%%%%%%%%%%%%%%%%%%%%%%%%%%%%%%%%%

\begin{document}
\maketitle

%%%%%%%%%%%%%%%%%%%%%%%%%%%%%%%%%%%%%%%%%%%%%%%%%%%%%%%%%%%%%%%%%%%%%%%%%%%%%%%
%
% Homework
%
% 6.2 # 1, 3, 5
% 6.3 # 1, 3
% 6.4 # 3a, 5a
%
%%%%%%%%%%%%%%%%%%%%%%%%%%%%%%%%%%%%%%%%%%%%%%%%%%%%%%%%%%%%%%%%%%%%%%%%%%%%%%%

\begin{itemize}
    \item[6.2.1]
        \begin{itemize}
            \item[(a)]
                \begin{equation*}
                    \boxed{f(x) = \lim_{n \to \infty}{f_n(x)}
                                = \lim_{n \to \infty}{\frac{nx}{1 + nx^2}}
                                = \frac{1}{x}}
                \end{equation*}

            \item[(b)]
                Notice that
                \begin{equation*}
                    f(x) = \frac{1}{x} - \frac{1}{\frac{1}{nx} + x}
                         = \frac{1}{nx^3 + x}
                \end{equation*}
                Then it is easy to see that the convergence is \textbf{not
                uniform} since $\forall \epsilon > 0$ and $\forall n \in
                \nats$, we can always have $x = \frac{1}{2n}$ s.t.
                $\frac{1}{nx^3 + x} = \frac{8n^2}{4n + 1} > \epsilon$ which
                shows that $\abs{f(x) - f_n(x)} > \epsilon$.

            \item[(c)]
                Similar to $(b)$, it is \textbf{not uniform} on $(0, 1)$ since
                $\forall \epsilon > 0$ and $\forall n \in \nats$, we can always
                have $x = \frac{1}{2n}$ s.t.  $\frac{1}{nx^3 + x} =
                \frac{8n^2}{4n + 1} > \epsilon$ which shows that $\abs{f(x) -
                f_n(x)} > \epsilon$.

            \item[(d)]
                We have:
                \begin{equation*}
                    \absb{\frac{x}{nx^2 + x} - \frac{1}{x}}
                        = \absb{\frac{1}{nx^3 + x}}
                        < \frac{1}{n}
                \end{equation*}
                Hence, $\forall \epsilon > 0$ and $\exists N$ s.t. $\forall n
                \geq N, \frac{1}{n} < \epsilon$ and $\abs{f(x) - f_n(x) <
                \epsilon}$. Finally, we got that the convergence \textbf{is
                uniform} on $(1, \infty)$.
        \end{itemize}

    \newpage

    \item[6.2.3]
        \begin{itemize}
            \item[(a)]
                We have:
                \begin{align*}
                    \lim_{n \to \infty}{g_n(x)}
                        &= \lim_{n \to \infty}{\frac{x}{1 + x^n}}
                        = x \text{ if $x < 1$}\\
                    \lim_{n \to \infty}{g_n(x)}
                        &= \lim_{n \to \infty}{\frac{x}{1 + x^n}}
                        = \frac{1}{2} \text{ if $x = 1$}\\
                    \lim_{n \to \infty}{g_n(x)}
                        &= \lim_{n \to \infty}{\frac{x}{1 + x^n}}
                        = 0 \text{ if $x > 1$}\\
                    &and\\
                    \lim_{n \to \infty}{h_n(x)}
                        &= 0 \text{ if $x = 0$}\\
                    \lim_{n \to \infty}{h_n(x)}
                        &= 1 \text{ if $x > 0$}
                \end{align*}
                Hence, $(g_n)$ converges pointwise to
                \begin{equation*}
                    g(x) = 
                    \begin{cases}
                        x \text{ if $x < 1$}\\
                        \frac{1}{2} \text{ if $x = 1$}\\
                        0 \text{ if $x > 1$}\\
                    \end{cases}
                \end{equation*}
                And $(h_n)$ converges pointwise to
                \begin{equation*}
                    g(x) = 
                    \begin{cases}
                        0 \text{ if $x - 1$}\\
                        1 \text{ if $x > 1$}
                    \end{cases}
                \end{equation*}

            \item[(b)]
                It follows by \textbf{Theorem 6.2.6 (Continuous Limit
                Theorem)}, that both $(g_n)$ (pick $x = 1$) and $(h_n)$ (pick
                $x = 0$) do not converge uniformly on $[0, \infty)$.

            \item[(c)]
                For $(g_n)$ consider a half-open interval $[0, 1)$. Then we
                have:
                \begin{equation*}
                    \abs{g_n(x) - g(x)} = \frac{x^n}{1 + x^n} < x^n
                \end{equation*}
                Now, as $x < 1$, it follows that $\forall \epsilon > 0, \exists
                N$ s.t. $\forall n \geq N, x^n < \epsilon$ and therefore the
                convergence is uniform on $[0, 1)$.
                \\
                \\
                For $(h_n)$ consider a half-open interval $[1, \infty)$. Pick
                $N = 2$, then $\forall \epsilon > 0$ and $\forall n \geq N$ we
                have:
                \begin{equation*}
                    \abs{h_n(x) - h(x)} = 1 - 1 = 0 < \epsilon
                \end{equation*}
                Hence, the convergence is uniform on $[1, \infty)$.
        \end{itemize}
    
    \newpage

    \item[6.2.5]
        We first prove the the theorem directly and then its converse. Suppose
        $(f_n)$ converges uniformly on $A$ to some function $f$. Let $\epsilon
        > 0$. Then, by definition, $\exists N \in \nats$ s.t. $\forall x \in A,
        n \geq N$ implies $\abs{f_n(x) - f(x)} < \frac{\epsilon}{2}$. Now, if
        $m, n \geq N$, then, by the triangle inequality, we have:
        \begin{align*}
            \abs{f_n(x) - f_m(x)}
                &= \abs{f_n(x) - f(x) + f(x) - f_m(x)}\\
                &\leq \abs{f_n(x) - f(x)} + \abs{f(x) - f_m(x)}\\
                &< \frac{\epsilon}{2} + \frac{\epsilon}{2}\\
                &< \epsilon
        \end{align*}
        Hence, $\abs{f_n(x) - f_m(x) < \epsilon}$.\\
        $\qed$
        \\
        \\
        Conversely, suppose that $\forall \epsilon > 0, \exists N \in \nats$
        s.t. $\forall x \in A$ and $\forall m, n \geq N, \abs{f_n(x) - f_m(x)}
        < \epsilon$. Now, notice that $(f_n(x))$ is a Cauchy sequence and per
        \textbf{Theorem 2.6.4 (Cauchy Criterion)}, it converges. Now, since
        this is true $\forall x \in A$, we can define $f(x) = \lim_{n \to
        \infty}(f_n(x))$ and we now have to show that $f_n$ converges to $f$
        uniformly. Let $\epsilon > 0$ be given. Then we know that $\exists N
        \in \nats$ s.t. $x \in A$ and $\forall m, n \geq N, \abs{ f_n(x) -
        f_m(x)} < \epsilon$. Then it follows by the \textbf{Algebraic Limit
        Theorem} that $\lim_{m \to \infty}{f_n(x) - f_m(x)} = f_n(x) - f_m(x)$.
        Finally, per the \textbf{Order Limit Theorem}, we get that for $x \in
        A$ and $\forall n \geq N, \abs{f_n(x) - f(x)} < \epsilon$. Hence, we
        got that $f_n$ uniformly converges to $f$.\\
        $\qed$
        \\
        \\
        Finally, we have shown that a sequence of functions $(f_n)$ defined on
        a set $A \subseteq \reals$ converges uniformly on $A$ if and only if
        for every $\epsilon > 0$ there exists an $N \in \nats$ such that
        $\abs{f_n(x) - f_m(x)} < \epsilon$ whenever $m, n \geq N$ and $x \in
        A$.\\
        $\qed$

    \item[6.3.1]
        \begin{itemize}
            \item[(a)]

            \item[(b)]
        \end{itemize}

    \item[6.3.3]
        \begin{itemize}
            \item[(a)]

            \item[(b)]
        \end{itemize}

    \item[6.4.3]
        \begin{itemize}
            \item[(a)]
                Placeholder
        \end{itemize}

    \item[6.4.5]
        \begin{itemize}
            \item[(a)]
                Placeholder
        \end{itemize}
\end{itemize}

%%%%%%%%%%%%%%%%%%%%%%%%%%%%%%%%%%%%%%%%%%%%%%%%%%%%%%%%%%%%%%%%%%%%%%%%%%%%%%%
% The End of the Document
%%%%%%%%%%%%%%%%%%%%%%%%%%%%%%%%%%%%%%%%%%%%%%%%%%%%%%%%%%%%%%%%%%%%%%%%%%%%%%%

\end{document}
