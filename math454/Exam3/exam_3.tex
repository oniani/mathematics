%%%%%%%%%%%%%%%%%%%%%%%%%%%%%%%%%%%%%%%%%%%%%%%%%%%%%%%%%%%%%%%%%%%%%%%%%%%%%%%
%
% Filename: template.tex
% Author:   David Oniani
% Modified: January 13, 2020
%  _         _____   __  __
% | |    __ |_   _|__\ \/ /
% | |   / _` || |/ _ \\  /
% | |__| (_| || |  __//  \
% |_____\__,_||_|\___/_/\_\
%
%%%%%%%%%%%%%%%%%%%%%%%%%%%%%%%%%%%%%%%%%%%%%%%%%%%%%%%%%%%%%%%%%%%%%%%%%%%%%%%

%%%%%%%%%%%%%%%%%%%%%%%%%%%%%%%%%%%%%%%%%%%%%%%%%%%%%%%%%%%%%%%%%%%%%%%%%%%%%%%
% Document Definition
%%%%%%%%%%%%%%%%%%%%%%%%%%%%%%%%%%%%%%%%%%%%%%%%%%%%%%%%%%%%%%%%%%%%%%%%%%%%%%%

\documentclass[11pt]{article}

%%%%%%%%%%%%%%%%%%%%%%%%%%%%%%%%%%%%%%%%%%%%%%%%%%%%%%%%%%%%%%%%%%%%%%%%%%%%%%%
% Packages and Related Settings
%%%%%%%%%%%%%%%%%%%%%%%%%%%%%%%%%%%%%%%%%%%%%%%%%%%%%%%%%%%%%%%%%%%%%%%%%%%%%%%

% Global, document-wide settings
\usepackage[margin=1in]{geometry}
\usepackage[utf8]{inputenc}
\usepackage[english]{babel}

% Other packages
\usepackage{booktabs}
\usepackage{hyperref}
\usepackage{mathtools}
\usepackage{amsthm}
\usepackage{amssymb}
\usepackage{tikz}
\usepackage[cache=false]{minted}

%%%%%%%%%%%%%%%%%%%%%%%%%%%%%%%%%%%%%%%%%%%%%%%%%%%%%%%%%%%%%%%%%%%%%%%%%%%%%%%
% Command Definitions and Redefinitions
%%%%%%%%%%%%%%%%%%%%%%%%%%%%%%%%%%%%%%%%%%%%%%%%%%%%%%%%%%%%%%%%%%%%%%%%%%%%%%%

% Nice-looking underline
\newcommand\und[1]{\underline{\smash{#1}}}

% Line spacing is 1.5
\renewcommand{\baselinestretch}{1.5}

% Absolute value
\DeclarePairedDelimiter\abs{\lvert}{\rvert}%

% Absolute value big
\DeclarePairedDelimiter\absb{\Big\lvert}{\Big\rvert}%

% Ceiling
\DeclarePairedDelimiter{\ceil}{\lceil}{\rceil}

% Ceiling big
\DeclarePairedDelimiter{\ceilb}{\Big\lceil}{\Big\rceil}

% Floor
\DeclarePairedDelimiter\floor{\lfloor}{\rfloor}

% % Naturals, Reals, Integers, and Rationals, 
\newcommand{\nats}{\mathbb{N}}
\newcommand{\reals}{\mathbb{R}}
\newcommand{\preals}{\mathbb{R^+}}
\newcommand{\nreals}{\mathbb{R^-}}
\newcommand{\ints}{\mathbb{Z}}
\newcommand{\pints}{\mathbb{Z^+}}
\newcommand{\nints}{\mathbb{Z^-}}
\newcommand{\rats}{\mathbb{Q}}
\newcommand{\prats}{\mathbb{Q^+}}
\newcommand{\nrats}{\mathbb{Q^-}}
\newcommand{\irrats}{\mathbb{I}}
\newcommand{\pirrats}{\mathbb{I^+}}
\newcommand{\nirrats}{\mathbb{I^-}}

%%%%%%%%%%%%%%%%%%%%%%%%%%%%%%%%%%%%%%%%%%%%%%%%%%%%%%%%%%%%%%%%%%%%%%%%%%%%%%%
% Miscellaneous
%%%%%%%%%%%%%%%%%%%%%%%%%%%%%%%%%%%%%%%%%%%%%%%%%%%%%%%%%%%%%%%%%%%%%%%%%%%%%%%

% Setting stuff
\setlength{\parindent}{0pt}  % Remove indentations from paragraphs

% PDF information and nice-looking urls
\hypersetup{%
  pdfauthor={David Oniani},
  pdftitle={Real Analysis},
  pdfsubject={Mathematics, Real Analysis, Real Numbers},
  pdfkeywords={Mathematics, Real Analysis, Real Numbers},
  pdflang={English},
  colorlinks=true,
  linkcolor={black!50!blue},
  citecolor={black!50!blue},
  urlcolor={black!50!blue}
}

%%%%%%%%%%%%%%%%%%%%%%%%%%%%%%%%%%%%%%%%%%%%%%%%%%%%%%%%%%%%%%%%%%%%%%%%%%%%%%%
% Author(s), Title, and Date
%%%%%%%%%%%%%%%%%%%%%%%%%%%%%%%%%%%%%%%%%%%%%%%%%%%%%%%%%%%%%%%%%%%%%%%%%%%%%%%

% Author(s)
\author{David Oniani\\
        Luther College\\
        \href{mailto:oniada01@luther.edu}{oniada01@luther.edu}}

% Title
\title{\rule{\paperwidth - 150pt}{1pt}\textbf{\\\textit{Real Analysis
Exams}\\}\rule {\paperwidth - 150pt}{1pt}\\\textbf{Exam
\textnumero3}\\{\normalsize Instructor: Dr. Eric Westlund}}

% Date
\date{\today}

%%%%%%%%%%%%%%%%%%%%%%%%%%%%%%%%%%%%%%%%%%%%%%%%%%%%%%%%%%%%%%%%%%%%%%%%%%%%%%%
% Beginning of the Document
%%%%%%%%%%%%%%%%%%%%%%%%%%%%%%%%%%%%%%%%%%%%%%%%%%%%%%%%%%%%%%%%%%%%%%%%%%%%%%%

\begin{document}
\maketitle

%%%%%%%%%%%%%%%%%%%%%%%%%%%%%%%%%%%%%%%%%%%%%%%%%%%%%%%%%%%%%%%%%%%%%%%%%%%%%%%

\begin{itemize}
    \item[1.]
        \begin{itemize}
            \item[(a)]
                Placeholder.

            \item[(b)]
                Placeholder.

            \item[(c)]
                Placeholder.

            \item[(d)]
                Placeholder.
        \end{itemize}

    \item[2.]
        \begin{itemize}
            \item[(a)]
                Placeholder.

            \item[(b)]
                Placeholder.
        \end{itemize}

    \item[3.]
        \begin{itemize}
            \item[(a)]
                Placeholder.

            \item[(b)]
                Placeholder.
        \end{itemize}

    \item[4.]
        \begin{itemize}
            \item[(a)]
                Placeholder.

            \item[(b)]
                Placeholder.

            \item[(c)]
                Placeholder.

            \item[(d)]
                Placeholder.

            \item[(e)]
                Placeholder.
        \end{itemize}

    \item[5.]
        \begin{itemize}
            \item[(a)]
                Notice that the following holds:
                \begin{equation*}
                    \absb{\frac{\cos{(3^nx)}}{2^n}} \leq \frac{1}{2^n}
                \end{equation*}
                Now, recall that $\frac{1}{2^n}$ converges (showed many times
                over the course of the class). Then, it follows by
                \textbf{Corollary 6.4.5 (Weierstrass M-Test)} that $g(x) =
                \sum_{n = 1}^\infty \frac{\cos{(3^nx)}}{2^n}$ converges
                uniformly on $\reals$. And since the uniform convergence
                implies continuity, it follows that $g(x) = \sum_{n = 1}^\infty
                \frac{\cos{(3^nx)}}{2^n}$ is continuous on $\reals$.

            \item[(b)]
                Notice that we have:
                \begin{equation*}
                    g^\prime(x) =
                        \sum_{n = 1}^\infty
                            -\Big(\frac{3}{2}\Big)^n\sin{(3^nx)}
                \end{equation*}
                Unfortunately, in this case we cannot apply \textbf{Corollary
                6.4.5 (Weierstrass M-Test)} as $\Big(\frac{3}{2}\Big)^n$ is not
                bounded. However, recall that this is the Weierstrass function
                of the form $\sum_{n = 0}^\infty a^n\cos{(b^nx)}$ which is a
                nowhere-differentiable function. Hence, $g^\prime(x)$ is not
                differentiable on $\reals$.
        \end{itemize}

    \item[6.]
        For $x \not\in \mathbb{Q}$, we can show $f_n(x)$ is continuous, since
        for $x < r_n$, we can choose a small enough $\delta$ such that $f_n(y)
        = 0$ for $y \in V_\delta(x)$. Similar logic can be applied for when $x
        > r_n$.
        Now, notice that
        \begin{equation*}
            f_n(x) \leq \frac{1}{2^n}
        \end{equation*}
        Then it follows by \textbf{Corollary 6.4.5 (Weierstrass M-Test)} that
        $f(x)$ converges uniformly.

        Now, since $f_n$ are all continuous, and $f$ converges uniformly, we
        have that $f$ is continuous.

        Furthermore, since every $f_n(x)$ is increasing, $f$ is monotonely
        increasing. Thus, for $x < y$, we get:
        \begin{align*}
          \forall n\ f_n(x) &\leq f_n(y)\\
          \sum_{n = 1}^k f_n(x) &\leq \sum_{n = 1}^k f_n(y)\\
          \lim_k \sum_{n = 1}^k f_n(x) &\leq \lim_k \sum_{n = 1}^k f_n(y)\\
          f(x) &\leq f(y)
        \end{align*}
        Hence, we got that $f$ is increasing on $\reals$.\\
        $\qed$

    \item[7.]
        \begin{itemize}
            \item[(a)]
                Placeholder.

            \item[(b)]
                Placeholder.

            \item[(c)]
                Placeholder.
        \end{itemize}
\end{itemize}

%%%%%%%%%%%%%%%%%%%%%%%%%%%%%%%%%%%%%%%%%%%%%%%%%%%%%%%%%%%%%%%%%%%%%%%%%%%%%%%
% The End of the Document
%%%%%%%%%%%%%%%%%%%%%%%%%%%%%%%%%%%%%%%%%%%%%%%%%%%%%%%%%%%%%%%%%%%%%%%%%%%%%%%

\end{document}
