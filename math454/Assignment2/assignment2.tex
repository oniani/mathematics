%%%%%%%%%%%%%%%%%%%%%%%%%%%%%%%%%%%%%%%%%%%%%%%%%%%%%%%%%%%%%%%%%%%%%%%%%%%%%%%
%
% Filename: template.tex
% Author:   David Oniani
% Modified: December 03, 2020
%  _         _____   __  __
% | |    __ |_   _|__\ \/ /
% | |   / _` || |/ _ \\  /
% | |__| (_| || |  __//  \
% |_____\__,_||_|\___/_/\_\
%
%%%%%%%%%%%%%%%%%%%%%%%%%%%%%%%%%%%%%%%%%%%%%%%%%%%%%%%%%%%%%%%%%%%%%%%%%%%%%%%

%%%%%%%%%%%%%%%%%%%%%%%%%%%%%%%%%%%%%%%%%%%%%%%%%%%%%%%%%%%%%%%%%%%%%%%%%%%%%%%
% Document Definition
%%%%%%%%%%%%%%%%%%%%%%%%%%%%%%%%%%%%%%%%%%%%%%%%%%%%%%%%%%%%%%%%%%%%%%%%%%%%%%%

\documentclass[11pt]{article}

%%%%%%%%%%%%%%%%%%%%%%%%%%%%%%%%%%%%%%%%%%%%%%%%%%%%%%%%%%%%%%%%%%%%%%%%%%%%%%%
% Packages and Related Settings
%%%%%%%%%%%%%%%%%%%%%%%%%%%%%%%%%%%%%%%%%%%%%%%%%%%%%%%%%%%%%%%%%%%%%%%%%%%%%%%

% Global, document-wide settings
\usepackage[margin=1in]{geometry}
\usepackage[utf8]{inputenc}
\usepackage[english]{babel}

% Other packages
\usepackage{booktabs}
\usepackage{hyperref}
\usepackage{mathtools}
\usepackage{amsthm}
\usepackage{amssymb}
\usepackage[cache=false]{minted}

%%%%%%%%%%%%%%%%%%%%%%%%%%%%%%%%%%%%%%%%%%%%%%%%%%%%%%%%%%%%%%%%%%%%%%%%%%%%%%%
% Command Definitions and Redefinitions
%%%%%%%%%%%%%%%%%%%%%%%%%%%%%%%%%%%%%%%%%%%%%%%%%%%%%%%%%%%%%%%%%%%%%%%%%%%%%%%

% Nice-looking underline
\newcommand\und[1]{\underline{\smash{#1}}}

% Line spacing is 1.5
\renewcommand{\baselinestretch}{1.5}

% Absolute value
\DeclarePairedDelimiter\abs{\lvert}{\rvert}%

% Ceiling
\DeclarePairedDelimiter{\ceil}{\lceil}{\rceil}

% Floor
\DeclarePairedDelimiter\floor{\lfloor}{\rfloor}

% % Naturals, Reals, Integers, and Rationals, 
\newcommand{\nats}{\mathbb{N}}
\newcommand{\reals}{\mathbb{R}}
\newcommand{\preals}{\mathbb{R^+}}
\newcommand{\nreals}{\mathbb{R^-}}
\newcommand{\ints}{\mathbb{Z}}
\newcommand{\pints}{\mathbb{Z^+}}
\newcommand{\nints}{\mathbb{Z^-}}
\newcommand{\rats}{\mathbb{Q}}
\newcommand{\prats}{\mathbb{Q^+}}
\newcommand{\nrats}{\mathbb{Q^-}}
\newcommand{\irrats}{\mathbb{I}}
\newcommand{\pirrats}{\mathbb{I^+}}
\newcommand{\nirrats}{\mathbb{I^-}}

%%%%%%%%%%%%%%%%%%%%%%%%%%%%%%%%%%%%%%%%%%%%%%%%%%%%%%%%%%%%%%%%%%%%%%%%%%%%%%%
% Miscellaneous
%%%%%%%%%%%%%%%%%%%%%%%%%%%%%%%%%%%%%%%%%%%%%%%%%%%%%%%%%%%%%%%%%%%%%%%%%%%%%%%

% Setting stuff
\setlength{\parindent}{0pt}  % Remove indentations from paragraphs

% PDF information and nice-looking urls
\hypersetup{%
  pdfauthor={David Oniani},
  pdftitle={Real Analysis},
  pdfsubject={Mathematics, Real Analysis, Real Numbers},
  pdfkeywords={Mathematics, Real Analysis, Real Numbers},
  pdflang={English},
  colorlinks=true,
  linkcolor={black!50!blue},
  citecolor={black!50!blue},
  urlcolor={black!50!blue}
}

%%%%%%%%%%%%%%%%%%%%%%%%%%%%%%%%%%%%%%%%%%%%%%%%%%%%%%%%%%%%%%%%%%%%%%%%%%%%%%%
% Author(s), Title, and Date
%%%%%%%%%%%%%%%%%%%%%%%%%%%%%%%%%%%%%%%%%%%%%%%%%%%%%%%%%%%%%%%%%%%%%%%%%%%%%%%

% Author(s)
\author{David Oniani\\
        Luther College\\
        \href{mailto:oniada01@luther.edu}{oniada01@luther.edu}}

% Title
\title{\rule{\paperwidth - 150pt}{1pt}\textbf{\\\textit{Real Analysis}\\}\rule
{\paperwidth - 150pt}{1pt}\\\textbf{Assignment \textnumero2}\\{\normalsize
Instructor: Dr. Eric Westlund}}

% Date
\date{\today}

%%%%%%%%%%%%%%%%%%%%%%%%%%%%%%%%%%%%%%%%%%%%%%%%%%%%%%%%%%%%%%%%%%%%%%%%%%%%%%%
% Beginning of the Document
%%%%%%%%%%%%%%%%%%%%%%%%%%%%%%%%%%%%%%%%%%%%%%%%%%%%%%%%%%%%%%%%%%%%%%%%%%%%%%%

\begin{document}
\maketitle

%%%%%%%%%%%%%%%%%%%%%%%%%%%%%%%%%%%%%%%%%%%%%%%%%%%%%%%%%%%%%%%%%%%%%%%%%%%%%%%
%
% Homework
%
% 2.2 # 1, 2, 7
% 2.3 # 3, 7, 13a
%
%%%%%%%%%%%%%%%%%%%%%%%%%%%%%%%%%%%%%%%%%%%%%%%%%%%%%%%%%%%%%%%%%%%%%%%%%%%%%%%

\begin{itemize}
    \item[2.2.1]
        For instance, the sequence $(1, 0, 1, 0, 1, 0, \dots)$ is vercongent.
        The sequence verconges at $2$ since for $\epsilon = 4, \forall N \in
        \nats, n \geq N$ implies $\abs{x_n - 2} \leq 1 < \epsilon$. This
        vercongent sequence is also a divergent sequence. Thus, divergent
        vercongent sequences exist. Notice that the sequence verconges at $3$
        as well since $\forall x \in (1, 0, 1, 0, 1, 0, \dots), \abs{x_n - 3}
        \leq 1 < \epsilon (\epsilon = 4)$. Therefore, the sequence also
        verconges to two different values. This definition describes
        \textbf{bounded sequences.}

    \item[2.2.2]
        \begin{itemize}
            \item[(a)]
                Let $N = \ceil{\dfrac{1}{4\epsilon}}$. Then, $\forall n \geq N$
                and $\forall \epsilon > 0$, $\abs{\dfrac{2n + 1}{5n + 4} -
                \dfrac{2}{5}} < \dfrac{1}{5n} < \dfrac{1}{5N} < \epsilon$.\\
                \textit{NOTE: $\dfrac{1}{5N} \leq \dfrac{4\epsilon}{5} < 
                \epsilon$}.\\\\
                Hence, $\lim\dfrac{2n + 1}{5n + 4} = \dfrac{2}{5}$.\\
                $\qed$


            \item[(b)]
                Let $N = \ceil{\dfrac{3}{\epsilon}}$. Then, $\forall n \geq N$
                and $\forall \epsilon > 0$, $\abs{\dfrac{2n^2}{n^3 + 3} -
                0} = \dfrac{2n^2}{n^3 + 3} < \dfrac{2}{n} < \dfrac{2}{N} <
                \epsilon$.\\
                \textit{NOTE: $\dfrac{2}{N} \leq \dfrac{2\epsilon}{3} <
                \epsilon$}.\\\\
                Hence, $\lim\dfrac{2n^2}{n^3 + 3} = 0$.\\
                $\qed$

            \item[(c)]
                Let $N = \ceil{\dfrac{2}{\epsilon^3}}$. Then, $\forall n \geq N$
                and $\forall \epsilon > 0$, $\abs{\dfrac{\sin{n^2}}{\sqrt[3]{n}}
                - 0} = \dfrac{\sin{n^2}}{\sqrt[3]{n}} \leq \dfrac{1}{\sqrt[3]{n}}
                < \dfrac{1}{\sqrt[3]{N}} < \epsilon$.\\
                \textit{NOTE: $\dfrac{1}{\sqrt[3]{N}} \leq
                \dfrac{\epsilon}{\sqrt[3]{2}} < \epsilon$}.\\\\
                Hence, $\lim\dfrac{\sin{n^2}}{\sqrt[3]{n}} = 0$.\\
                $\qed$
        \end{itemize}

    \item[2.2.7]
        \begin{itemize}
            \item[(a)]
                It is frequently in $\{1\}$ since it alternates, but it is not
                eventually in $\{1\}$. Let us now prove this statement. We
                first prove that it is \textit{frequently} in $\{1\}$ and then
                prove that it cannot be \textit{eventually} in $\{1\}$.
                \\\\
                $\forall N \in \nats, \exists n = 2N$ with $(-1)^{2N} = 1 \in
                \{1\}$. Hence, the sequence is frequently in $\{1\}$\\
                $\qed$
                \\
                On the other hand, $\forall N \in \nats, \exists n = 2N + 1$
                with $(-1)^{2N + 1} = -1 \notin \{1\}$. Hence, the sequence is
                not eventually in $\{1\}$\\
                $\qed$
                \\
                Therefore, the sequence $(-1)^n$ is frequently, but not
                eventually in $\{1\}$.

            \item[(b)]
                Eventually is certainly stronger. Eventually imples frequently.
                This is true since if $\exists N \in \nats$ s.t $\forall n \geq
                N, a_n \in A$ also implies that $\forall N \in \nats, \exists n
                \geq N$ s.t. $a_n \in A$. Put it simply, $\forall$ statement is
                stronger than $\exists$ statement since it generalizes and
                applies to all numbers from a certain point while satisfying
                exists condition only requires finding a single case.

            \item[(c)]
                We need to use \textit{eventually}. Here is an alternate
                rephrasing of Definition 2.2.3B:

                ``A sequence $(a_n)$ converges to a real number a if $\forall
                \epsilon > 0$, the sequence is eventually in the
                $\epsilon$-neighborhood of $a$.''

            \item[(d)]
                It is frequently in $(1.9, 2.1)$. This is the case since
                $\forall N \in \nats, \exists n \geq N$ s.t. $a_n \in (1.9,
                2.1)$ (as the number of $2$s is infinite).\\
                On the other hand, it is not necessarily eventually in $(1.9,
                2.1)$. A counterexample would be a sequence $(2, 0, 2, 0,
                \dots)$. The sequence is frequently in $(1.9, 2.1)$ (as will be
                all sequences with infinite number of $2$s, but is not
                eventually in $(1.9, 2.1)$ as $2s$ and $0s$ alternate and
                $\forall N \in \nats, \exists n = 2N$ with $0
                \neq 2$.
        \end{itemize}

    \item[2.3.3]
        $\forall \epsilon > 0, \exists N_1, N_2 \in \nats$ s.t. $\forall n_1
        \geq N_1$ and $\forall n_2 \geq N_2, \abs{x_{n_1} - l} < \epsilon$ and
        $\abs{z_{n_2} - l} < \epsilon$. Then, let us define $N = \max{(N_1,
        N_2)}$. It follows that $\forall n \geq N, \abs{x_{n} - l} < \epsilon$
        and $\abs{z_{n} - l} < \epsilon$. We get $-\epsilon < x_n - l <
        \epsilon$ and $-\epsilon < z_n - l < \epsilon$. After adding $l$ to
        all three sides of the inequalities, we then get $l -\epsilon < x_n <
        l + \epsilon$ and $l -\epsilon < z_n  < l + \epsilon$. Now, since we
        know that $x_n \leq y_n \leq z_n$, we have $l - \epsilon < x_n \leq
        y_n$ and $y_n \leq z_n < l + \epsilon$. Therefore, $l - \epsilon < y_n
        < l + \epsilon$ and it follows that $\abs{y_n - l} < \epsilon$. Hence,
        $\lim y_n = l$.\\
        $\qed$

    \item[2.3.7]
        \begin{itemize}
            \item[(a)]
                Let $(x_n) = (-1, -1, -1 \dots)$ and $(y_n) = (1, 1, 1,
                \dots)$. Then both $x_n$ and $y_n$ diverge, but their sum
                $(x_n + y_n) = (0, 0, 0, \dots)$ converges to $0$. Hence,
                such sequences do exist.

            \item[(b)]
                Such sequences cannot exist. Suppose, for the sake of
                contradiction, that $(x_n), (x_n + y_n)$ are convergent and
                $(y_n)$ is divergent. Then, since $y_n = (x_n + y_n) - (x_n)$,
                it follows by \textbf{Algebraic Limit Theorem} that $y_n$ is
                also convergent and we face a contradiction. Hence, such
                sequences do not exist.

            \item[(c)]
                Let $b_n = \dfrac{1}{n}$. Then $b_n$ converges to $0$ with $b_n
                \neq 0 \forall n \in \nats$. However, $1 / b_n = 1 / (1 / n)
                = n$ which is a divergent sequence (sequence $(n)$ diverges).
                In other words, since $(n)$ is not bounded, it is divergent by
                \textbf{Theorem 2.3.2}. Hence, such sequence does exist.

            \item[(d)]
                Such sequences cannot exist. Suppose, for the sake of
                contradiction, that $(a_n)$ and $(b_n)$ are unbounded and
                convergent sequences respectively with $(a_n - b_n)$ being a
                bounded sequence. Since $b_n$ is convergent (by \textbf{Theorem
                2.3.2, it is also has to be bounded}), let its bound be $B$ and
                let $(a_n - b_n)$ be bounded by $D$. Then it follows that
                $\forall n \in \nats, \abs{a_n} \leq \abs{a_n - b_n} \leq
                \abs{a_n - b_n} + \abs{b_n} \leq D + B$. Thus, we got that
                $(a_n)$ is bounded too and we face the contradiction. Hence,
                such sequences do not exist.

            \item[(e)]
                Let $(a_n) = (0, 0, 0, \dots)$ and let $(b_n) = (1, -1, 1,
                \dots)$. Then $(a_n)$ and $(a_nb_n)$ both converge to $0$, but
                $(b_n)$ is divergent. Hence, such sequences do exist.
        \end{itemize}

    \item[2.3.13]
        \begin{itemize}
            \item[(a)]
                Notice that $a_{mn} = \dfrac{m}{m + n} =
                \dfrac{\frac{m}{m}}{\frac{m}{m} + \frac{n}{m}} =
                \dfrac{1}{1 + \frac{n}{m}}$.\\

                Then we have $\lim_{m \to \infty}a_{mn} = 1$ and
                $\lim_{n \to \infty}\Big(\lim_{m \to \infty}a_{mn}\Big) =
                \lim_{n \to \infty}1 = 1$.

                Similarly, $\lim_{n \to \infty}a_{mn} = 0$ and
                $\lim_{m \to \infty}\Big(\lim_{n \to \infty}a_{mn}\Big) =
                \lim_{m \to \infty}0 = 0$.

                Finally, we got that
                $\lim_{n \to \infty}\Big(\lim_{m \to \infty}a_{mn}\Big) = 1$
                and
                $\lim_{m \to \infty}\Big(\lim_{n \to \infty}a_{mn}\Big) = 0$
        \end{itemize}
\end{itemize}

%%%%%%%%%%%%%%%%%%%%%%%%%%%%%%%%%%%%%%%%%%%%%%%%%%%%%%%%%%%%%%%%%%%%%%%%%%%%%%%
% The End of the Document
%%%%%%%%%%%%%%%%%%%%%%%%%%%%%%%%%%%%%%%%%%%%%%%%%%%%%%%%%%%%%%%%%%%%%%%%%%%%%%%

\end{document}
