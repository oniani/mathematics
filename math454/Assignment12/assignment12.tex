%%%%%%%%%%%%%%%%%%%%%%%%%%%%%%%%%%%%%%%%%%%%%%%%%%%%%%%%%%%%%%%%%%%%%%%%%%%%%%%
%
% Filename: assignment12.tex
% Author:   David Oniani
% Modified: January 19, 2020
%  _         _____   __  __
% | |    __ |_   _|__\ \/ /
% | |   / _` || |/ _ \\  /
% | |__| (_| || |  __//  \
% |_____\__,_||_|\___/_/\_\
%
%%%%%%%%%%%%%%%%%%%%%%%%%%%%%%%%%%%%%%%%%%%%%%%%%%%%%%%%%%%%%%%%%%%%%%%%%%%%%%%

%%%%%%%%%%%%%%%%%%%%%%%%%%%%%%%%%%%%%%%%%%%%%%%%%%%%%%%%%%%%%%%%%%%%%%%%%%%%%%%
% Document Definition
%%%%%%%%%%%%%%%%%%%%%%%%%%%%%%%%%%%%%%%%%%%%%%%%%%%%%%%%%%%%%%%%%%%%%%%%%%%%%%%

\documentclass[11pt]{article}

%%%%%%%%%%%%%%%%%%%%%%%%%%%%%%%%%%%%%%%%%%%%%%%%%%%%%%%%%%%%%%%%%%%%%%%%%%%%%%%
% Packages and Related Settings
%%%%%%%%%%%%%%%%%%%%%%%%%%%%%%%%%%%%%%%%%%%%%%%%%%%%%%%%%%%%%%%%%%%%%%%%%%%%%%%

% Global, document-wide settings
\usepackage[margin=1in]{geometry}
\usepackage[utf8]{inputenc}
\usepackage[english]{babel}

% Other packages
\usepackage{booktabs}
\usepackage{hyperref}
\usepackage{mathtools}
\usepackage{amsthm}
\usepackage{amssymb}
\usepackage[cache=false]{minted}

%%%%%%%%%%%%%%%%%%%%%%%%%%%%%%%%%%%%%%%%%%%%%%%%%%%%%%%%%%%%%%%%%%%%%%%%%%%%%%%
% Command Definitions and Redefinitions
%%%%%%%%%%%%%%%%%%%%%%%%%%%%%%%%%%%%%%%%%%%%%%%%%%%%%%%%%%%%%%%%%%%%%%%%%%%%%%%

% Nice-looking underline
\newcommand\und[1]{\underline{\smash{#1}}}

% Line spacing is 1.5
\renewcommand{\baselinestretch}{1.5}

% Absolute value
\DeclarePairedDelimiter\abs{\lvert}{\rvert}%

% Absolute value big
\DeclarePairedDelimiter\absb{\Big\lvert}{\Big\rvert}%

% Ceiling
\DeclarePairedDelimiter{\ceil}{\lceil}{\rceil}

% Floor
\DeclarePairedDelimiter\floor{\lfloor}{\rfloor}

% % Naturals, Reals, Integers, and Rationals, 
\newcommand{\nats}{\mathbb{N}}
\newcommand{\reals}{\mathbb{R}}
\newcommand{\preals}{\mathbb{R^+}}
\newcommand{\nreals}{\mathbb{R^-}}
\newcommand{\ints}{\mathbb{Z}}
\newcommand{\pints}{\mathbb{Z^+}}
\newcommand{\nints}{\mathbb{Z^-}}
\newcommand{\rats}{\mathbb{Q}}
\newcommand{\prats}{\mathbb{Q^+}}
\newcommand{\nrats}{\mathbb{Q^-}}
\newcommand{\irrats}{\mathbb{I}}
\newcommand{\pirrats}{\mathbb{I^+}}
\newcommand{\nirrats}{\mathbb{I^-}}

%%%%%%%%%%%%%%%%%%%%%%%%%%%%%%%%%%%%%%%%%%%%%%%%%%%%%%%%%%%%%%%%%%%%%%%%%%%%%%%
% Miscellaneous
%%%%%%%%%%%%%%%%%%%%%%%%%%%%%%%%%%%%%%%%%%%%%%%%%%%%%%%%%%%%%%%%%%%%%%%%%%%%%%%

% Setting stuff
\setlength{\parindent}{0pt}  % Remove indentations from paragraphs

% PDF information and nice-looking urls
\hypersetup{%
  pdfauthor={David Oniani},
  pdftitle={Real Analysis},
  pdfsubject={Mathematics, Real Analysis, Real Numbers},
  pdfkeywords={Mathematics, Real Analysis, Real Numbers},
  pdflang={English},
  colorlinks=true,
  linkcolor={black!50!blue},
  citecolor={black!50!blue},
  urlcolor={black!50!blue}
}

%%%%%%%%%%%%%%%%%%%%%%%%%%%%%%%%%%%%%%%%%%%%%%%%%%%%%%%%%%%%%%%%%%%%%%%%%%%%%%%
% Author(s), Title, and Date
%%%%%%%%%%%%%%%%%%%%%%%%%%%%%%%%%%%%%%%%%%%%%%%%%%%%%%%%%%%%%%%%%%%%%%%%%%%%%%%

% Author(s)
\author{David Oniani\\
        Luther College\\
        \href{mailto:oniada01@luther.edu}{oniada01@luther.edu}}

% Title
\title{\rule{\paperwidth - 150pt}{1pt}\textbf{\\\textit{Real Analysis}\\}\rule
{\paperwidth - 150pt}{1pt}\\\textbf{Assignment \textnumero12}\\{\normalsize
Instructor: Dr. Eric Westlund}}

% Date
\date{\today}

%%%%%%%%%%%%%%%%%%%%%%%%%%%%%%%%%%%%%%%%%%%%%%%%%%%%%%%%%%%%%%%%%%%%%%%%%%%%%%%
% Beginning of the Document
%%%%%%%%%%%%%%%%%%%%%%%%%%%%%%%%%%%%%%%%%%%%%%%%%%%%%%%%%%%%%%%%%%%%%%%%%%%%%%%

\begin{document}
\maketitle

%%%%%%%%%%%%%%%%%%%%%%%%%%%%%%%%%%%%%%%%%%%%%%%%%%%%%%%%%%%%%%%%%%%%%%%%%%%%%%%
%
% Homework
%
% 7.2 # 1, 2
% 7.3 # 1, 3
% 7.4 # 3
%
%%%%%%%%%%%%%%%%%%%%%%%%%%%%%%%%%%%%%%%%%%%%%%%%%%%%%%%%%%%%%%%%%%%%%%%%%%%%%%%

\begin{itemize}
    \item[7.2.1]
        Let $f$ be a bounded function on $[a, b]$ and let $P$ be an arbitraty
        partition of $[a, b]$. Now, let $A$ be a collection of all possible
        partitions of $[a, b]$. Then $U(f) = \inf \{U(f, Q) \mid Q \in A \}$
        and it follows by \textbf{Lemma 7.2.4} that $L(f, A) \leq \inf \{U(f,
        Q) \mid Q \in A\} = U(f)$. Thus, $U(f) \geq L(f, P)$. Now, we have
        $\forall A^\prime \in A, L(f, A^\prime) \leq U(f)$ which means that
        $U(f)$ is an upper bound for $\{L(f, A^\prime) \mid A^\prime \in A\}$.
        Finally, we get $L(f) = \{L(f, A^\prime) \mid A^\prime \in A\} \leq
        U(f)$. Hence, we have proven \textbf{Lemma 7.2.6}.\\
        $\qed$

    \item[7.2.2]
        \begin{itemize}
            \item[(a)]
                Notice that the function $f$ is decreasing on interval $[1,
                4]$. Then we have $m_k = \inf \{f(x) \mid x \in [x_{k - 1},
                x_k]\} = f(x)$ and $M_k = \sup \{f(x) \mid x \ in [x_{k - 1},
                x_k]\} = f(x_{k - 1})$. We have:
                \begin{equation*}
                    L(f, P) = f\Big(\frac{3}{2}\Big) \times
                                  \Big(\frac{3}{2} - 1\Big) + f(2)
                                  \times (2 - \frac{3}{2}) + f(4)(4 - 2)
                            = \frac{13}{12}
                \end{equation*}
                With the similar line of reasoning, we get:
                \begin{equation*}
                    U(f, P) = f(1) \times (\frac{3}{2} - 1) +
                                  f\Big(\frac{3}{2}\Big) \times
                                      (2 - \frac{3}{2}) + f(2)(4 - 2)
                            = \frac{11}{6}
                \end{equation*}
                Finally, we have $L(f, P) = \frac{13}{12}, U(f, P) =
                \frac{11}{6}$, and $U(f, P) - L(f, P) = \frac{3}{4}$.

            \item[(b)]
                Notice that interval $[2, 3]$ contributes $f(2)(4 - 2) - f(4)(4
                - 2) = \frac{1}{2}$ to $U(f, P) - L(f, P)$. If the point $3$ is
                added to the partition, the subpartition contribution will be
                $f(2) + f(3) - f(3) - f(4) = f(2) - f(4) = \frac{1}{4}$. Hence,
                when we add add the point $3$ to the partition, $U(f, P) - L(f,
                P)$ decreases by $\frac{1}{2} - \frac{1}{4} = \frac{1}{4}$ and
                we get $U(f, P) - L(f, P) = \frac{3}{4} - \frac{1}{4} =
                \frac{1}{2}$.
                
            \item[(c)]
                Let $P^\prime = \Big\{1, \frac{5}{4}, \frac{3}{2}, \frac{7}{4},
                \dots\Big\}$. Then we have $U(f, P^\prime) - L(f, P^\prime) =
                \frac{3}{8}$ which is less than $\frac{2}{5}$.
        \end{itemize}

    \item[7.3.1]
        \begin{itemize}
            \item[(a)]
                Let $P$ be a partition of $[0, 1]$ s.t. it is comprised of
                points $0 = x_0 < x_1 < \dots < x_n = 1$. Then, $\forall [x_{k
                - 1}, x_{k}], m_k(h) = \inf \{h(t) : t \in [x_{k - 1}, x_k\} =
                1$. It follows that $L(h, P) = \sum_{k = 1}^n m_k(h)\Delta x_k
                = \sum_{k = 1}^n 1 \times \Delta x_k = \sum_{k = 1}^n \Delta
                x_k = 1$.

            \item[(b)]
                Let $P = \{0, \frac{19}{20}, 1\}$. Then we get:
                \begin{equation*}
                    U(h, P) = \sup_{i \in [0, \frac{19}{20}]} h(i) \times
                    \frac{19}{20} + \sup_{i \in [\frac{19}{20}, 1]} h(i) \times
                    \frac{1}{20} = \frac{19}{20} + \frac{2}{20} = 1 +
                    \frac{1}{20} < 1 + \frac{1}{10}
                \end{equation*}

            \item[(c)]
                Let $\epsilon > 0$ be given. Choose $k \in \reals$ s.t. $0 < 1
                - k\epsilon < \epsilon$ and define $P_{\epsilon} = \{0,
                k\epsilon, 1\}$. We then have:
                \begin{equation*}
                    U(h, P_{\epsilon}) = \sup_{i \in [0, k\epsilon]} h(i)
                    \times k\epsilon + \sup_{i \in [k\epsilon, 1]} h(i) \times
                    (1 - k\epsilon) = 2 - k\epsilon = 1 + (1 - k\epsilon) < 1 +
                    \epsilon
                \end{equation*}
        \end{itemize}

    \item[7.3.3]
        Notice that $L(f, P) = 0$ for any partition $P$. Now, take a partition
        $P_n$ comprised of points of the form $x_i = \frac{i}{n^2}$ (with $x_0
        = 0$). Then the length of the interval is $\Delta x_i = \frac{1}{n^2}$
        and we have:
        \begin{equation*}
            U(f, P_n) = \frac{1}{n^2} \times (1 + 1 + 1 + \dots + 1) +
                            \frac{1}{n^2} \times \sup \{f(j) \mid f
                                \leq \frac{1}{n}\}
                      = \frac{1}{n} + \frac{1}{n^2}
        \end{equation*}
        Now, let $\epsilon > 0$ be given. Then it is easy to see that $\exists
        N$ s.t. $\frac{1}{N} + \frac{1}{N^2} < \epsilon$. Hence, $\forall n
        \geq N, U(f, P_n) < \epsilon$ and we have $U(f, P) = 0$. Finally, it
        follows that $\int_0^1 = L(f, P) = U(f, P) = 0$. Therefore, $f$ is
        integrable on $[0, 1]$ and $\int_0^1 f = 0$.

    \item[7.4.3]
        \begin{itemize}
            \item[(a)]
                Not true.
                \\
                \\
                Counterxample: Let us define function $f$ as follows:
                \begin{equation*}
                    f(x) =
                    \begin{cases}
                        1 \text{ if } x \in \rats\\
                        -1 \text{ if } x \notin \rats
                    \end{cases}
                \end{equation*}
                Then $\abs{f}$ is integrable on $[0, 1]$, but $f$ is not
                integrable on $[0, 1]$.

            \item[(b)]
                Not true.
                \\
                \\
                Counterxample: Let us define function $g$ as follows:
                \begin{equation*}
                    f(x) =
                    \begin{cases}
                        \frac{1}{n} \text{ if } x = \frac{1}{n}, n \in \nats\\
                        0 \text{ otherwise}
                    \end{cases}
                \end{equation*}
                Then $\forall x_n = \frac{1}{n}, g(x_n) > 0 \ (n \in \nats)$,
                however, $\int_0^1 = g(x) = 0$.

            \item[(c)]
                This is true.
                \\
                \\
                Proof:\\
                If $g$ is continuous on $[a, b]$ and if $g(y_0) > 0$, then
                $\exists \delta > 0$ s.t. $\forall (y_0 - \delta, y_0 +
                \delta), g(x) > 0$. Pick $m = \inf \{g(y) \mid y_0 - \delta
                \leq y \leq y_0 + \delta\}$. Then we have:
                \begin{equation*}
                    \int_a^b g \geq \int_{y_0 - \delta}^{y_0 + \delta} g
                               \geq 2m\delta > 0
                \end{equation*}
                Hence, if $g$ is continuous on $[a, b]$ and $g(x) \geq 0$ with
                $g(y_0) > 0$ for at least one point $y_0 \in [a, b]$, then
                $\int_a^b g > 0$.\\
                $\qed$
        \end{itemize}
\end{itemize}

%%%%%%%%%%%%%%%%%%%%%%%%%%%%%%%%%%%%%%%%%%%%%%%%%%%%%%%%%%%%%%%%%%%%%%%%%%%%%%%
% The End of the Document
%%%%%%%%%%%%%%%%%%%%%%%%%%%%%%%%%%%%%%%%%%%%%%%%%%%%%%%%%%%%%%%%%%%%%%%%%%%%%%%

\end{document}
