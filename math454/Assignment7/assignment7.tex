%%%%%%%%%%%%%%%%%%%%%%%%%%%%%%%%%%%%%%%%%%%%%%%%%%%%%%%%%%%%%%%%%%%%%%%%%%%%%%%
%
% Filename: assignment7.tex
% Author:   David Oniani
% Modified: December 18, 2020
%  _         _____   __  __
% | |    __ |_   _|__\ \/ /
% | |   / _` || |/ _ \\  /
% | |__| (_| || |  __//  \
% |_____\__,_||_|\___/_/\_\
%
%%%%%%%%%%%%%%%%%%%%%%%%%%%%%%%%%%%%%%%%%%%%%%%%%%%%%%%%%%%%%%%%%%%%%%%%%%%%%%%

%%%%%%%%%%%%%%%%%%%%%%%%%%%%%%%%%%%%%%%%%%%%%%%%%%%%%%%%%%%%%%%%%%%%%%%%%%%%%%%
% Document Definition
%%%%%%%%%%%%%%%%%%%%%%%%%%%%%%%%%%%%%%%%%%%%%%%%%%%%%%%%%%%%%%%%%%%%%%%%%%%%%%%

\documentclass[11pt]{article}

%%%%%%%%%%%%%%%%%%%%%%%%%%%%%%%%%%%%%%%%%%%%%%%%%%%%%%%%%%%%%%%%%%%%%%%%%%%%%%%
% Packages and Related Settings
%%%%%%%%%%%%%%%%%%%%%%%%%%%%%%%%%%%%%%%%%%%%%%%%%%%%%%%%%%%%%%%%%%%%%%%%%%%%%%%

% Global, document-wide settings
\usepackage[margin=1in]{geometry}
\usepackage[utf8]{inputenc}
\usepackage[english]{babel}

% Other packages
\usepackage{booktabs}
\usepackage{hyperref}
\usepackage{mathtools}
\usepackage{amsthm}
\usepackage{amssymb}
\usepackage[cache=false]{minted}

%%%%%%%%%%%%%%%%%%%%%%%%%%%%%%%%%%%%%%%%%%%%%%%%%%%%%%%%%%%%%%%%%%%%%%%%%%%%%%%
% Command Definitions and Redefinitions
%%%%%%%%%%%%%%%%%%%%%%%%%%%%%%%%%%%%%%%%%%%%%%%%%%%%%%%%%%%%%%%%%%%%%%%%%%%%%%%

% Nice-looking underline
\newcommand\und[1]{\underline{\smash{#1}}}

% Line spacing is 1.5
\renewcommand{\baselinestretch}{1.5}

% Absolute value
\DeclarePairedDelimiter\abs{\lvert}{\rvert}%

% Absolute value big
\DeclarePairedDelimiter\absb{\Big\lvert}{\Big\rvert}%

% Ceiling
\DeclarePairedDelimiter{\ceil}{\lceil}{\rceil}

% Floor
\DeclarePairedDelimiter\floor{\lfloor}{\rfloor}

% % Naturals, Reals, Integers, and Rationals, 
\newcommand{\nats}{\mathbb{N}}
\newcommand{\reals}{\mathbb{R}}
\newcommand{\preals}{\mathbb{R^+}}
\newcommand{\nreals}{\mathbb{R^-}}
\newcommand{\ints}{\mathbb{Z}}
\newcommand{\pints}{\mathbb{Z^+}}
\newcommand{\nints}{\mathbb{Z^-}}
\newcommand{\rats}{\mathbb{Q}}
\newcommand{\prats}{\mathbb{Q^+}}
\newcommand{\nrats}{\mathbb{Q^-}}
\newcommand{\irrats}{\mathbb{I}}
\newcommand{\pirrats}{\mathbb{I^+}}
\newcommand{\nirrats}{\mathbb{I^-}}

%%%%%%%%%%%%%%%%%%%%%%%%%%%%%%%%%%%%%%%%%%%%%%%%%%%%%%%%%%%%%%%%%%%%%%%%%%%%%%%
% Miscellaneous
%%%%%%%%%%%%%%%%%%%%%%%%%%%%%%%%%%%%%%%%%%%%%%%%%%%%%%%%%%%%%%%%%%%%%%%%%%%%%%%

% Setting stuff
\setlength{\parindent}{0pt}  % Remove indentations from paragraphs

% PDF information and nice-looking urls
\hypersetup{%
  pdfauthor={David Oniani},
  pdftitle={Real Analysis},
  pdfsubject={Mathematics, Real Analysis, Real Numbers},
  pdfkeywords={Mathematics, Real Analysis, Real Numbers},
  pdflang={English},
  colorlinks=true,
  linkcolor={black!50!blue},
  citecolor={black!50!blue},
  urlcolor={black!50!blue}
}

%%%%%%%%%%%%%%%%%%%%%%%%%%%%%%%%%%%%%%%%%%%%%%%%%%%%%%%%%%%%%%%%%%%%%%%%%%%%%%%
% Author(s), Title, and Date
%%%%%%%%%%%%%%%%%%%%%%%%%%%%%%%%%%%%%%%%%%%%%%%%%%%%%%%%%%%%%%%%%%%%%%%%%%%%%%%

% Author(s)
\author{David Oniani\\
        Luther College\\
        \href{mailto:oniada01@luther.edu}{oniada01@luther.edu}}

% Title
\title{\rule{\paperwidth - 150pt}{1pt}\textbf{\\\textit{Real Analysis}\\}\rule
{\paperwidth - 150pt}{1pt}\\\textbf{Assignment \textnumero7}\\{\normalsize
Instructor: Dr. Eric Westlund}}

% Date
\date{\today}

%%%%%%%%%%%%%%%%%%%%%%%%%%%%%%%%%%%%%%%%%%%%%%%%%%%%%%%%%%%%%%%%%%%%%%%%%%%%%%%
% Beginning of the Document
%%%%%%%%%%%%%%%%%%%%%%%%%%%%%%%%%%%%%%%%%%%%%%%%%%%%%%%%%%%%%%%%%%%%%%%%%%%%%%%

\begin{document}
\maketitle

%%%%%%%%%%%%%%%%%%%%%%%%%%%%%%%%%%%%%%%%%%%%%%%%%%%%%%%%%%%%%%%%%%%%%%%%%%%%%%%
%
% Homework
%
% 4.2 # 5, 8ab, 9, 11
% 4.3 # 3, 7, 8
%
%%%%%%%%%%%%%%%%%%%%%%%%%%%%%%%%%%%%%%%%%%%%%%%%%%%%%%%%%%%%%%%%%%%%%%%%%%%%%%%

\begin{itemize}
    \item[]

    \item[]

    \item[4.2.5]
        \begin{itemize}
            \item[(a)]
                Prove that $\lim_{x \to 2} (3x + 4) = 10$.
                \\
                Let $\epsilon > 0$ be given and let $\delta =
                \dfrac{\epsilon}{3}$. Then $0 < \abs{x - 2} < \delta$ and we
                have that $\abs{x - 2} < \dfrac{\epsilon}{3}$ and it follows
                that $\abs{3x - 6} < \epsilon$. Now, notice that $3x - 6 = 3x +
                4 - 10$ and thus, $\abs{(3x + 4) - 10} < \epsilon$. Hence,
                $\lim_{x \to 2} (3x + 4) = 10$.\\
                $\qed$

            \item[(b)]
                Prove that $\lim_{x \to 0} x^3 = 0$.
                \\
                Let $\epsilon > 0$ be given and let $\delta =
                \sqrt[3]{\epsilon}$. Then $0 < \abs{x - 0} < \delta$ and we
                have that $\abs{x} < \sqrt[3]{\epsilon}$. Now, notice that
                $\abs{x}^3 = \abs{x^3}$.  and thus, $\abs{x^3 - 0} < \epsilon$.
                Hence, $\lim_{x \to 0} x^3 = 0$.\\
                $\qed$

            \item[(c)]
                Prove that $\lim_{x \to 2} (x^2 + x - 1) = 5$.
                \\
                Let $\epsilon > 0$ be given and let $\abs{x - 2} < \delta$.
                Then $\abs{x - 2} < \delta$, $\abs{x + 3} < 5 + \delta$. We
                have $0 < \abs{x^2 + x - 6} < \delta$. Notice that $x^2 + x - 6
                = (x - 2)(x + 3)$. Then $\abs{(x - 2)(x + 3)} < \delta(\delta +
                5)$.  Notice that the equation $\delta(\delta + 5) = \epsilon$
                has a discriminant $\mathbb{D} = 25 + 4\epsilon > 0$ and the
                equation always has at least one solution since $\epsilon > 0$.
                Thus, $\exists \delta$ s.t. $\delta(\delta + 5) < \epsilon$ and
                it follows that $\abs{(x - 2)(x + 3)} < \epsilon$. Finally, we
                have that $\abs{(x - 2)(x + 3)} = \abs{(x^2 + x - 1) - 5} <
                \epsilon$.  Hence, $\lim_{x \to 2} (x^2 + x - 1) = 5$.\\
                $\qed$

            \item[(d)]
                Let $\epsilon > 0$ be given and let $\abs{x - 3} < \delta$.
                Then notice that $\absb{\dfrac{1}{x} - \dfrac{1}{3}} =
                \absb{\dfrac{x - 3}{3x}} < \dfrac{\delta}{3(\delta + 3)}$. Now,
                the equation $\dfrac{\delta}{3(\delta + 3)} = \epsilon$ always
                has at least one solution as the discriminant $\mathbb{D} =
                9\epsilon^2 + 36 \epsilon > 0$ and thus, $\forall \epsilon > 0,
                \exists \delta$ s.t. $\dfrac{\delta}{3(\delta + 3)} <
                \epsilon$. Finally, we get that $\absb{\dfrac{1}{x} -
                \dfrac{1}{3}} < \epsilon$.  Hence, $\lim_{x \to 3} \dfrac{1}{x}
                = \dfrac{1}{3}$.\\
                $\qed$
                
        \end{itemize}

    \item[4.2.8]
        \begin{itemize}
            \item[(a)]
                $\lim_{x \to 2} \dfrac{\abs{x - 2}}{x - 2}$ does not exist.
                \\
                To show this, let $x_n = \dfrac{1}{n} + n$ and let $y_n =
                -\dfrac{1}{n} + 2$ with $n \in \nats$..

                Then we have $\dfrac{\abs{x_n - 2}}{x_n - 2} =
                \dfrac{\frac{1}{n}}{\frac{1}{n}} = 1$ and $\dfrac{\abs{y_n -
                2}}{y_n - 2} = \dfrac{\frac{1}{n}}{-\frac{1}{n}} = -1$.
                Now, \textbf{by Corollary 4.2.5}, the limit does not exist.\\
                $\qed$

            \item[(b)]
                $\lim_{x \to \frac{7}{4}} \dfrac{\abs{x - 2}}{x - 2} = -1$.
                \\
                Notice that $\lim_{x \to \frac{7}{4}} \abs{x - 2} = 0.25$ and
                $\lim_{x \to \frac{7}{4}} (x - 2) = -0.25$. Then, \textbf{per
                Algebraic Limit Theorem}, we get that $\lim_{x \to \frac{7}{4}}
                \dfrac{\abs{x - 2}}{x - 2} = -1$.\\
                $\qed$
        \end{itemize}

    \newpage

    \item[4.2.9]
        \begin{itemize}
            \item[(a)]
                Let $M > 0$ be given and let $\delta = \dfrac{1}{\sqrt{M}}$.
                Then, if $\abs{x} < \delta$, we have $\dfrac{1}{x^2} >
                \dfrac{1}{\delta^2}$ and it follows that $\dfrac{1}{x^2} > M$.
                Hence, we showed that $\lim_{x \to 0} \dfrac{1}{x^2} =
                \infty$.\\
                $\qed$

            \item[(b)]
                The definition would read as follows:

                ``We say $\lim_{x \to \infty} f(x) = L$ if $\forall \epsilon >
                0, \exists M > 0$ s.t. if $x > M$ we have $\abs{f(x) - L} <
                \epsilon$.''

                Let us now prove that $\lim_{x \to \infty} \dfrac{1}{x} = 0$.
                Let $\epsilon > 0$ be given and let $M = \dfrac{1}{\epsilon}$.
                Then if $x > M$, we have $x > \dfrac{1}{\epsilon}$. We have
                $\dfrac{1}{x} = \abs{\dfrac{1}{x} - 0} < \epsilon$. Hence,
                $\lim_{x \to \infty} \dfrac{1}{x} = 0$.\\
                $\qed$

            \item[(c)]
                The definition would read as follows:

                ``$\lim_{x \to \infty} f(x) = \infty$ if $\forall M > 0,
                \exists N > 0$ s.t. $\forall x > N, f(x) > M$.''

                For instance, $\lim_{x \to \infty} x = \infty$ is one example.
                In this case, given $M > 0$, we can pick $N = M$.
        \end{itemize}

    \item[4.2.11]
        Let us first $\forall c \in \reals$ define $S_\delta(c) = \{x \in
        \reals \mid 0 < \abs{x - c} < \delta\}$
        \\
        Let $\epsilon > 0$ be given. Then, since $\lim_{x \to c} f(x) = L$, by
        definition, $\exists \delta_1 > 0$ s.t. $\forall x \in S_{\delta_1}(c),
        \abs{f(x) - L} < \epsilon$. Similarly, we can also find $\delta_2$ s.t.
        $\forall x \in S_{\delta_2}(c), \abs{h(x) - L} < \epsilon$. Then let
        $\delta = \min{(\delta_1, \delta_2)}$. We get that $\forall x \in
        S_{\delta}(c)$, we have $L - \epsilon < f(x) \leq g(x) \leq h(x) < L +
        \epsilon$. And finally, it follows that $\abs{g(x) - L} < \epsilon$.
        Hence, $\lim_{x \to c}g(x) = L$.\\
        $\qed$

    \item[4.3.3]
        \begin{itemize} 
            \item[(a)]
                As $g$ is continuous at $f(c)$, it follows that $\forall
                \epsilon > 0, \exists \delta^\prime > 0$ s.t. if $\abs{f(x) -
                f(c)}$, we have $\abs{g(f(c)) - g(f(x))} < \epsilon$. Now, as
                $f$ is continuous at $c$, $\forall \delta^\prime > 0$, we get
                $\exists \delta > 0$ s.t. if $\abs{x - c} < \delta$, we get
                $\abs{f(x) - f(c)} < \delta^\prime$. Hence, $g \circ f$ is
                continuous at $c$.\\
                $\qed$

            \item[(b)]
                Let $(x_n) \to c$. Then, as $f$ is continuous at $c$, we have
                that $(f(x_n)) \to f(c)$. We get that $(f(x_n))$ is a
                convergent sequence with $(f(x_n)) \to f(c)$. Since $g$ is
                continuous at $f(c)$, we get $(g(f(x_n))) \to g(f(c))$ and
                thus, $\lim_{x \to c} g(f(x)) = g(f(c))$.\\
                $\qed$
        \end{itemize} 

    \item[4.3.7]
        \begin{itemize}
            \item[(a)]
                First consider an arbitrary $r \in \rats$. Now, recall that
                $\irrats$ is dense in $\reals$. Then, due to the density
                property, there exists a sequence $(x_n) \subseteq \irrats$
                s.t. $(x_n) \to r$. Then it follows that $g(x_n) = 0$ for all
                $n \in N$ with $g(r) = 1$. Since $\lim g(x_n) = 0 \neq g(r)$,
                \textbf{by Corollary 4.3.3 (Criterion for Discontinuity)}, we
                conclude that $g(x)$ is not continuous at $r \in \rats$.

                Let us now consider an arbitrary $i \in \irrats$. Recall that
                $\rats$ is dense in $\reals$. Then, due to the density
                property, we can find a sequence $(y_n) \subseteq \rats$ s.t.
                $(y_n) \to i$. This time $g(y_n) = 1$ for all $n \in N$
                with $g(i) = 0$. Now, since $\lim g(y_n) = 1 \neq g(i)$, we can
                conclude that $g$ is not continuous at $i$.

                Now, since $g(x)$ is not continuous at any $r \in \rats$ as
                well as at any $i \in \irrats$, we conclude that Dirichlet's
                function is nowhere continuous on $\reals$.\\
                $\qed$

            \item[(b)]
                Consider an arbitrary rational number $r \in \rats$. Then,
                notice that $t(r) \neq 0$. Now, since $\irrats$ is dense in
                $\rats$, there exists a sequence $(x_n) \subseteq I$ s.t.
                $(x_n) \to r$.  It follows that $n \in N, t(x_n) = 0$ with
                $t(r) \neq 0$.  Thus $\lim t(x_n) \neq t(r)$ and $t(x)$ is not
                continuous at $r$.  Hence, Thomae's function fails to be
                continuous at every rational point.\\
                $\qed$

            \item[(c)]
                Let $c \in \irrats$ be an arbitrary irrational number.  Given
                $\epsilon > 0$, set $T = \{x \in \reals \mid t(x) \geq
                \epsilon\}$. If $x \in T$, then $x$ is a rational number of the
                form $x = \dfrac{m}{n}$ with $m \in \ints$ and $n \in \nats$
                where $n \leq \dfrac{1}{\epsilon}$. The restriction on the size
                of $n$ implies that the intersection of $T$ with the interval
                $[c – 1, c + 1]$ is finite. In a finite set, all points are
                isolated, so we can pick a neighborhood $V_\delta(c)$ around
                $c$ such that $x \in V_\delta(c)$ implies $x \notin T$. But if
                $x \notin T$, then $t(x) < \epsilon$, i.e., $t(x) \in
                V_\delta(0)$ = $V_\delta(t(c))$. Finally, \textbf{by Theorem
                4.3.2 (iii)}, we conclude that $t(x)$ is continuous at $c$.\\
                $\qed$
        \end{itemize}

    \item[4.3.8]
        \begin{itemize}
            \item[(a)]
                It is true that $g(1) \geq 0$.

                Suppose, for the sake of contradiction, that $g$ is continuous
                on $\reals$ and $g(1) < 0$. Then $\exists \epsilon =
                \abs{g(1)}$ s.t. $\abs{g(x) - g(1) \geq \epsilon}$ for every
                choice $\delta = \abs{x - c}$. And we face a contradiction
                since we assumed that $g$ is continuous at $x = 1 \in \reals$.
                Hence, $g(1) \geq 0$ as well.\\
                $\qed$

            \item[(b)]
                It is true that $g(x) = 0$ for all $x \in \reals$.

                Let $x \in \reals$ be given. Recall that $\rats$ is dense in
                $\reals$. Then, there exists a sequence of rational numbers
                $(s_n)$ s.t. $(s_n) \to x$. Now, as $g(x)$ is continuous at
                $x$, \textbf{per Theorem 4.3.2}, if follows that $g(x) =
                \lim_{n \to \infty} g(s_n) = 0$ and thus $g(x) = 0$ for all $x
                \in \reals$.\\
                $\qed$

            \item[(c)]
                It is true that $g(x)$ is strictly positive for uncountably
                many points.

                Let $c = g(x_0) > 0$. Now, since $g(x)$ continuous at $x_0$,
                $\exists \delta > 0$ s.t. $\abs{f(x) - f(x_0)} < \dfrac{c}{2}$
                for all $\abs{x - x_0} < \delta$. Hence, $\forall \abs{x -
                x_0}$ we get $-\dfrac{c}{2} < f(x) - c < \dfrac{c}{2}$ and
                thus, $\dfrac{c}{2} < f(x) < \dfrac{3c}{2}$. Hence, we get
                $\forall \abs{x - x_0} < \delta, f(x) > 0$ and it follows that
                $g(x)$ is strictly positive for uncountably many points.\\
                $\qed$
        \end{itemize}
\end{itemize}

%%%%%%%%%%%%%%%%%%%%%%%%%%%%%%%%%%%%%%%%%%%%%%%%%%%%%%%%%%%%%%%%%%%%%%%%%%%%%%%
% The End of the Document
%%%%%%%%%%%%%%%%%%%%%%%%%%%%%%%%%%%%%%%%%%%%%%%%%%%%%%%%%%%%%%%%%%%%%%%%%%%%%%%

\end{document}
