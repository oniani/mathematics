\documentclass[11pt, a4paper]{article}
\usepackage[a4paper, margin=1in]{geometry}


\usepackage{adjustbox}
\usepackage{mathtools}
\usepackage{amsmath}
\usepackage{amssymb}
\usepackage{amsthm}

\usepackage{pgfplots}
\usepackage{pst-func, pstricks-add}
\usepackage{listings}
\usepackage{color}
\usepackage{tikz}

\usepackage{textcomp}
\usepackage{soul}

\usepackage[hidelinks]{hyperref}
\pgfplotsset{width=7.5cm,compat=1.12}
\usepgfplotslibrary{fillbetween}
\usepackage[makeroom]{cancel}
\title{\bf{Homework \textnumero 3}}
\author{Author: David Oniani
\\
\ \ \ Instructor: Dr. Eric Westlund}
\date{February 12, 2019}

\usepackage{listings}
\usepackage{color}

%%%%%%%%%%%%%%% S E T S %%%%%%%%%%%%%%%
\newcommand{\nats}{\mathbb{N}}
\newcommand{\ints}{\mathbb{Z}}
\newcommand{\rats}{\mathbb{Q}}
\newcommand{\reals}{\mathbb{R}}
\newcommand{\irrats}{\mathbb{I}}

\newcommand{\pnats}{\mathbb{N}^+}
\newcommand{\pints}{\mathbb{Z}^+}
\newcommand{\prats}{\mathbb{Q}^+}
\newcommand{\preals}{\mathbb{R}^+}
\newcommand{\nreals}{\mathbb{R}^-}

\newcommand{\nints}{\mathbb{Z}^-}
\newcommand{\nrats}{\mathbb{Q}^-}
%%%%%%%%%%%%%%%%%%%%%%%%%%%%%%%%%%%%%%%

% Calligraphy
\newcommand\und[1]{\underline{\smash{#1}}}

% Operators
\DeclarePairedDelimiter\abs{\lvert}{\rvert}
\DeclarePairedDelimiter\ceil{\lceil}{\rceil}
\DeclarePairedDelimiter\floor{\lfloor}{\rfloor}

% Other
\newcommand{\rarr}{\rightarrow}

\definecolor{dkgreen}{rgb}{0,0.6,0}
\definecolor{gray}{rgb}{0.5,0.5,0.5}
\definecolor{mauve}{rgb}{0.58,0,0.82}
\definecolor{backcolour}{rgb}{0.95,0.95,0.92}

\lstset{
backgroundcolor=\color{backcolour},
aboveskip=3mm,
belowskip=3mm,
showstringspaces=false,
columns=flexible,
basicstyle={\small\ttfamily},
numbers=left,
numberstyle=\normalsize\color{gray},
keywordstyle=\color{blue},
commentstyle=\color{dkgreen},
stringstyle=\color{mauve},
breaklines=true,
breakatwhitespace=true,
tabsize=4
}


\begin{document}
\maketitle

\begin{itemize}
\item[3.4]
\begin{itemize}
\item[(a)]
$C$ is the mean and $B$ is the median (since the distribution is right-skewed).

\item[]

\item[(b)]
$B$ is both the mean and the median (since the distribution is symmetric).

\item[]

\item[(c)]
$A$ is the mean and $B$ is the median (since the distribution is left-skewed).
\end{itemize}

\item[]
\item[]

\item[3.6]
\begin{itemize}
\item[(a)]
The area for $99.7\%$ corresponds to the three standard deviations and therefore, the range for lengths
that cover almost all $(99.7\%)$ of this distribution is from $35.8 - 3 \times 2.1$
to $35.8 + 3 \times 2.1$. That is the range from $29.5$ to $42.1$.

\item[]

\item[(b)]
Notice that $33.7 = 35.8 - 2.1$. Therefore, the datapoint is located one deviation to the left from the center.
Hence, we got that $\dfrac{32}{2}\% = 16\%$ of women over $20$ have the arm length less than $33.7$cm.
\end{itemize}

\item[]
\item[]

\item[3.7]
\begin{itemize}
\item[(a)]
According to the $68 - 95 - 99.7$ rule, it will be between
$852 - 2 \times 82$ and $852 + 2 \times 82$. That is,
between $688$ and $1016$.

\item[]

\item[(b)]
According to the $68 - 95 - 99.7$ rule, it will be
$852 - 2 \times 82 = 688$ (this is since $95\%$ leaves
us with $2.5\%$ on both sides and we need the left one).
\end{itemize}

\item[]
\item[]

\item[3.8]
$z_{\text{Idonna}} = \dfrac{x - \mu}{\sigma} = \dfrac{670 - 514}{118} = 1.32$\\\\
$z_{\text{Jonathan}} = \dfrac{x - \mu}{\sigma} = \dfrac{26 - 20.9}{5.3} = 0.96$\\\\
Since $z_{\text{Idonna}} > z_{\text{Jonathan}} (1.32 > 0.96)$, it appears that Idonna did better.

\newpage

\item[3.10]
\begin{itemize}
\pgfmathdeclarefunction{gauss}{2}{%
    \pgfmathparse{1/(#2*sqrt(2*pi))*exp(-((x-#1)^2)/(2*#2^2))}%
}

\item[(a)]
From the Table A, we get that the value for for $z = -0.76$ is $0.2236$.
This means that $22.36\%$ area will be covered to the left of the
point $-0.76$.
\item[]
\item[]
\item[]
\begin{tikzpicture}
\begin{axis}[
  no markers, domain=-5:5, samples=1000,
  axis lines*=left,
  every axis y label/.style={at=(current axis.above origin),anchor=south},
  every axis x label/.style={at=(current axis.right of origin),anchor=west},
  height=5cm, width=12cm,
  xtick={-0.76}, ytick=\empty,
  enlargelimits=false, clip=false, axis on top,
  grid = major
  ]
  \addplot [fill=blue!20, draw=none, domain=-5:-0.76] {gauss(0,1)} \closedcycle;
  \addplot [very thick,blue!50!black] {gauss(0,1)};
\end{axis}
\end{tikzpicture}

\item[]

\item[(b)]
From the Table A, we get that the value for for $z = -0.76$ is $0.2236$.
This means that $22.36\%$ area will be covered to the left of the
point $-0.76$ leaving $100\% - 22.36\% = 77.64\%$ to the right.
\item[]
\item[]
\item[]
\begin{tikzpicture}
\begin{axis}[
  no markers, domain=-5:5, samples=1000,
  axis lines*=left,
  every axis y label/.style={at=(current axis.above origin),anchor=south},
  every axis x label/.style={at=(current axis.right of origin),anchor=west},
  height=5cm, width=12cm,
  xtick={-0.76}, ytick=\empty,
  enlargelimits=false, clip=false, axis on top,
  grid = major
  ]
  \addplot [fill=blue!20, draw=none, domain=-0.76:5] {gauss(0,1)} \closedcycle;
  \addplot [very thick,blue!50!black] {gauss(0,1)};
\end{axis}
\end{tikzpicture}

\item[]

\item[(c)]
From the Table A, we get that the value for for $z = 1.45$ is $0.9265$.
This means that $92.65\%$ area will be covered to the left of the
point $1.45$.
\item[]
\item[]
\item[]
\begin{tikzpicture}
    \begin{axis}[
      no markers, domain=-5:5, samples=1000,
      axis lines*=left,
      every axis y label/.style={at=(current axis.above origin),anchor=south},
      every axis x label/.style={at=(current axis.right of origin),anchor=west},
      height=5cm, width=12cm,
      xtick={1.45}, ytick=\empty,
      enlargelimits=false, clip=false, axis on top,
      grid = major
      ]
      \addplot [fill=blue!20, draw=none, domain=-5:1.45] {gauss(0,1)} \closedcycle;
      \addplot [very thick,blue!50!black] {gauss(0,1)};
    \end{axis}
\end{tikzpicture}

\newpage

\item[(d)]
From the Table A, we get that the value for for $z = -0.76$ is $0.2236$
and for $z = 1.45$ is $0.9265$. Therefore, we shade the region in-between
$-0.76$ and $1.45$ and get the total area in percentages
equal to $(0.9265 - 0.2236) \times 100\% = 70.30\%$.
\item[]
\item[]
\item[]
\begin{tikzpicture}
    \begin{axis}[
      no markers, domain=-5:5, samples=1000,
      axis lines*=left,
      every axis y label/.style={at=(current axis.above origin),anchor=south},
      every axis x label/.style={at=(current axis.right of origin),anchor=west},
      height=5cm, width=12cm,
      xtick={-0.76, 1.45}, ytick=\empty,
      enlargelimits=false, clip=false, axis on top,
      grid = major
      ]
      \addplot [fill=blue!20, draw=none, domain=-0.76:1.45] {gauss(0,1)} \closedcycle;
      \addplot [very thick,blue!50!black] {gauss(0,1)};
    \end{axis}
\end{tikzpicture}
\end{itemize}

\item[]
\item[]

\item[3.11]
\begin{itemize}
\item[(a)]
$z = \dfrac{x - \mu}{\sigma} = \dfrac{697 - 852}{82} = -1.89$.
From the Table A, we have that the standard normal cumulative
proportion is $0.0294$ which in percentages, is $0.0294 \times 100 \% = 2.94 \%$.
Therefore, we have that in $2.94 \%$ of all years, India will have 697 mm or less monsoon rain.
\item[]

\item[(b)]
$z$ for $683$ is $\dfrac{683 - 852}{82} = -2.06$.\\\\
$z$ for $1022$ is $\dfrac{1022 - 852}{82} = 2.07$\\\\
We now look at the Table $A$ and find that for $z = -2.06$,
the standard normal cumulative proportion value is $0.0197$
and for $z = 2.07$, $0.9808$. Finally, we get that
$(0.9808 - 0.0197) \times 100 \% = 96.11 \%$ of all years,
the rainfall is ``normall,'' or between $683$mm and $1022$mm.
\end{itemize}

\item[]
\item[]

\item[3.14]
\begin{itemize}
\item[(a)]
Since the distribution is normal, its mean and median are equal.
Therefore, $M = \overline{x} = 25.3$. To calculate the IQR,
we need $Q_1$ and $Q_3$. To calculate $Q_1$, we look at
the Table A to find the value closest to $0.25$. It
appears to be $0.2514$ which corresponds to $z = -0.67$.
For $Q_3$, we find the value closest to $0.75$ which
turns out to be $z = 0.67$ (it makes sense since
these two points must match if we fold the graph
across the $y - \text{axis}$). Finally, we have
that $Q_1 = -0.67\sigma = -0.67 \times 6.5 = -4.355$,
$Q_2 = 0.67\sigma = 0.67 \times 6.5 = 4.355$, and $\text{IQR} = 0.67\sigma - (-0.67\sigma) = 1.34\sigma = 1.34 \times 6.5 = 8.71$

\item[]

\item[(b)]
We will first need to find a values close to $0.10$. It appears
to be $0.1003$ for $z = -1.28$. Since the normal distribution
is \und{perfectly} symmetric, value close to $0.90$ will be
$z = 1.28$ (one can verify this by looking up the value in the Table A). Therefore, we have that the interval is $(-1.28, 1.28)$.
\end{itemize}

\item[]
\item[]

\item[3.28]
\begin{itemize}
\pgfmathdeclarefunction{gauss}{2}{%
    \pgfmathparse{1/(#2*sqrt(2*pi))*exp(-((x-#1)^2)/(2*#2^2))}%
}

\item[(a)]
From the Table A, we get that the value for for $z = -2.15$ is $0.0158$.
This means that $1.58\%$ area will be covered to the left of the
point $-2.15$.
\item[]
\item[]
\item[]
\begin{tikzpicture}
\begin{axis}[
  no markers, domain=-5:5, samples=1000,
  axis lines*=left,
  every axis y label/.style={at=(current axis.above origin),anchor=south},
  every axis x label/.style={at=(current axis.right of origin),anchor=west},
  height=5cm, width=12cm,
  xtick={-2.15}, ytick=\empty,
  enlargelimits=false, clip=false, axis on top,
  grid = major
  ]
  \addplot [fill=blue!20, draw=none, domain=-5:-2.15] {gauss(0,1)} \closedcycle;
  \addplot [very thick,blue!50!black] {gauss(0,1)};
\end{axis}
\end{tikzpicture}

\item[]

\item[(b)]
From the Table A, we get that the value for for $z = -2.15$ is $0.0158$.
This means that $1.58\%$ area will be covered to the left of the
point $-2.15$ leaving $100\% - 1.58\% = 98.42\%$ to the right.
\item[]
\item[]
\item[]
\begin{tikzpicture}
\begin{axis}[
  no markers, domain=-5:5, samples=1000,
  axis lines*=left,
  every axis y label/.style={at=(current axis.above origin),anchor=south},
  every axis x label/.style={at=(current axis.right of origin),anchor=west},
  height=5cm, width=12cm,
  xtick={-2.15}, ytick=\empty,
  enlargelimits=false, clip=false, axis on top,
  grid = major
  ]
  \addplot [fill=blue!20, draw=none, domain=-2.15:5] {gauss(0,1)} \closedcycle;
  \addplot [very thick,blue!50!black] {gauss(0,1)};
\end{axis}
\end{tikzpicture}

\item[]

\item[(c)]
From the Table A, we get that the value for for $z = 1.57$ is $0.9418$.
This means that $94.18\%$ area will be covered to the left of the
point $1.45$ leaving $100 - 94.18 = 5.82\%$ to the left.
\item[]
\item[]
\item[]
\begin{tikzpicture}
    \begin{axis}[
      no markers, domain=-5:5, samples=1000,
      axis lines*=left,
      every axis y label/.style={at=(current axis.above origin),anchor=south},
      every axis x label/.style={at=(current axis.right of origin),anchor=west},
      height=5cm, width=12cm,
      xtick={1.57}, ytick=\empty,
      enlargelimits=false, clip=false, axis on top,
      grid = major
      ]
      \addplot [fill=blue!20, draw=none, domain=1.57:5] {gauss(0,1)} \closedcycle;
      \addplot [very thick,blue!50!black] {gauss(0,1)};
    \end{axis}
\end{tikzpicture}

\newpage

\item[(d)]
From the Table A, we get that the value for for $z = -2.15$ is $0.0158$
and for $z = 1.57$ is $0.9418$. Therefore, we shade the region in-between
$-2.15$ and $1.57$ and get the total area in percentages
equal to $(0.9418 - 0.0158) \times 100\% = 92.6\%$.
\item[]
\item[]
\item[]
\begin{tikzpicture}
    \begin{axis}[
      no markers, domain=-5:5, samples=1000,
      axis lines*=left,
      every axis y label/.style={at=(current axis.above origin),anchor=south},
      every axis x label/.style={at=(current axis.right of origin),anchor=west},
      height=5cm, width=12cm,
      xtick={-2.15, 1.57}, ytick=\empty,
      enlargelimits=false, clip=false, axis on top,
      grid = major
      ]
      \addplot [fill=blue!20, draw=none, domain=-2.15:1.57] {gauss(0,1)} \closedcycle;
      \addplot [very thick,blue!50!black] {gauss(0,1)};
    \end{axis}
\end{tikzpicture}
\end{itemize}

\item[]
\item[]

\item[3.30]
\begin{itemize}
\item[(a)]
First, let's find the $z$-value. We have, $z = \dfrac{0.6 - 0.8}{0.078} \approx -2.56$. From the Table A,\\
\vspace{0.05cm}
\\
we get that this value corresponding to $z = -2.56$ is $0.0052$. Then since this value shows us proportion to the left of the normal distribution, we got that $0.0052$ is the proportion of flies that have thorax length less than 0.6mm.

\item[]

\item[(b)]
First, let's find the $z$-value. We have, $z = \dfrac{0.9 - 0.8}{0.078} \approx 1.28$. From the Table A, we\\
\vspace{0.05cm}
\\
get that this value corresponding to $z = 1.28$ is $0.8997$. Then since this value shows us proportion to the left of the normal distribution,
we got that $1 - 0.8997 = 0.1003$ is the proportion of flies that have thorax length greater than 0.9mm.

\item[]

\item[(c)]
The $z$-value for $0.6$ is $2.56$ and for $0.9$ is $1.28$.
From the Table A, we get that the proportion for $z = 0.6$ is $0.0052$
and for $z = 0.9$ is $0.8997$. The proportion in-between is
$0.8997 - 0.0052 = 0.8945$.
\end{itemize}

\end{itemize}

\end{document}
