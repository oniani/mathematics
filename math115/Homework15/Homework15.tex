\documentclass[11pt, a4paper]{article}
\usepackage[a4paper, margin=1in]{geometry}

\usepackage{adjustbox}
\usepackage{mathtools}
\usepackage{amsmath}
\usepackage{amssymb}
\usepackage{amsthm}

\usepackage{pgfplots}
\usepackage{listings}
\usepackage{color}
\usepackage{tikz}

\usepackage{textcomp}
\usepackage{soul}

\usepackage[hidelinks]{hyperref}
\pgfplotsset{width=7.5cm,compat=1.12}
\usepgfplotslibrary{fillbetween}
\pgfplotsset{compat=1.8}
\usepgfplotslibrary{statistics}
\usepackage[makeroom]{cancel}
\title{\bf{Homework \textnumero 15}}
\author{Author: David Oniani
\\
\ \ \ Instructor: Dr. Eric Westlund}
\date{April 8, 2019}

\usepackage{listings}
\usepackage{color}

%%%%%%%%%%%%%%% S E T S %%%%%%%%%%%%%%%
\newcommand{\nats}{\mathbb{N}}
\newcommand{\ints}{\mathbb{Z}}
\newcommand{\rats}{\mathbb{Q}}
\newcommand{\reals}{\mathbb{R}}
\newcommand{\irrats}{\mathbb{I}}

\newcommand{\pnats}{\mathbb{N}^+}
\newcommand{\pints}{\mathbb{Z}^+}
\newcommand{\prats}{\mathbb{Q}^+}
\newcommand{\preals}{\mathbb{R}^+}
\newcommand{\nreals}{\mathbb{R}^-}

\newcommand{\nints}{\mathbb{Z}^-}
\newcommand{\nrats}{\mathbb{Q}^-}
%%%%%%%%%%%%%%%%%%%%%%%%%%%%%%%%%%%%%%%

% Calligraphy
\newcommand\und[1]{\underline{\smash{#1}}}

% Operators
\DeclarePairedDelimiter\abs{\lvert}{\rvert}
\DeclarePairedDelimiter\ceil{\lceil}{\rceil}
\DeclarePairedDelimiter\floor{\lfloor}{\rfloor}

% Other
\newcommand{\rarr}{\rightarrow}

\definecolor{dkgreen}{rgb}{0,0.6,0}
\definecolor{gray}{rgb}{0.5,0.5,0.5}
\definecolor{mauve}{rgb}{0.58,0,0.82}
\definecolor{backcolour}{rgb}{0.95,0.95,0.92}

\lstset{
backgroundcolor=\color{backcolour},
aboveskip=3mm,
belowskip=3mm,
showstringspaces=false,
columns=flexible,
basicstyle={\small\ttfamily},
numbers=left,
numberstyle=\normalsize\color{gray},
keywordstyle=\color{blue},
commentstyle=\color{dkgreen},
stringstyle=\color{mauve},
breaklines=true,
breakatwhitespace=true,
tabsize=4
}


\begin{document}
\maketitle
\begin{itemize}
\item[21.1]
This is a matched-pairs design (2).

\item[]
\item[]

\item[21.2]
This is a two independent samples design (3).

\item[]
\item[]

\item[21.3]
This is a single sample design (1).

\item[]
\item[]

\item[21.4]
This is a two independent samples design (3).

\item[]
\item[]

\item[21.5]
We start with the \und{\textbf{PLAN}} part as the \und{\textbf{STATE}}
part is the description of the problem itself.\\\\
\und{\textbf{PLAN}}\\
We perform a two-sample $t$ test for the hypotheses $\text{H}_0: \mu_1 = \mu_2$ and $\text{H}_{\text{a}}: \mu_1 > \mu_2$.
\\\\\\
\und{\textbf{SOLVE}}\\
We get that $\overline{x_1} \approx 132.71$ and $\overline{x_2} \approx 59.67$.
We also get that $s_1 \approx 98.44$ and $s_2 \approx 63.04$.
Now, we calculate the test statistic $t$.
We get that
$$t = \dfrac{\overline{x_1} - \overline{x_2}}{\sqrt{\dfrac{s_1^2}{n_1} + \dfrac{s_2^2}{n_2}}} = \dfrac{132.71 - 59.67}{\sqrt{\dfrac{98.44^2}{21} + \dfrac{63.04^2}{21}}} \approx 2.86$$
Now, $\text{df} = n - 1 = 21 - 1 = 20$ and from Table C, we get that $0.0025 < P < 0.005$.
Now, since $P < 0.05$, we reject the null hypothesis $\text{H}_0$.
\\\\\\
\und{\textbf{CONCLUDE}}\\
There is a sufficient evidence to claim that playing with Nintendo $\text{Wii}^{\text{TM}}$ helps improve the skills
of student doctors. More particularly, it means that the mean improvement time shows the significant increase.

\item[]
\item[]

\item[21.7]
From the previous exercise we know that $\overline{x_1} \approx 132.71$ and $\overline{x_2} \approx 59.67$.
We also know that $s_1 \approx 98.44$ and $s_2 \approx 63.04$.
Then the 99\% confidence interval will be from $(132.71 - 59.67) - 2.845 \times 25.509$.
to $(132.71 - 59.67) + 2.845 \times 25.509$ which is approximately same
as the confidence interval from 0.467 to 145.613. Therefore, the 99\% confidence
interval is from 0.467 to 145.613.

\item[]
\item[]

\item[21.13]
NOTE: $\text{SE}$ here means the Standard Error.\\\\
$\text{SE}_1 = \dfrac{s_1}{\sqrt{n_1}} = \dfrac{0.68}{\sqrt{37}} \approx 0.11$.\\
$\text{SE}_2 = \dfrac{s_2}{\sqrt{n_2}} = \dfrac{0.83}{\sqrt{21}} \approx 0.18$.\\\\
Now, let calculate the test statistic. We have:
$$t = \dfrac{\overline{x_1} - \overline{x_2}}{\sqrt{\dfrac{s_1^2}{n_1} + \dfrac{s_2^2}{n_2}}} = \dfrac{0.29 - (-0.19)}{\sqrt{\dfrac{0.68^2}{37} + \dfrac{0.83^2}{21}}} \approx 2.255$$
For the degrees of freedom, we can use the smaller value and get that $\text{df} = n_2 - 1 = 21 - 1 = 20$ (this is called the conservative inference procedure).
Obviously, this won't give us the exact number of degrees of freedom, so, unfortunately, we cannot really verify this value. Though, it seems like
35.3 is the right value (I verified it using the online software). Now, from Table C, we get that $0.01 < P < 0.02$.
As a side note, since $P < 0.05$ we can reject the null hypothesis $\text{H}_0$.
\item[]
\item[]

\item[21.15]
From the software output, we see the significant difference in mean perceived formidability for men alone
versus with friends. Since bigger scores mean bigger perceived formidability, foes appear
more formidable when alone in comparison with when with friends.

\item[]
\item[]

\item[21.29]
\begin{itemize}
\item[(a)]
NOTE: $\text{SEM}$ here means the Standard Error of The Mean.\\\\
$s_1 = \text{SEM}_1 \times \sqrt{n_1} = 3 \times \sqrt{20} \approx 13.416$\\
$s_2 = \text{SEM}_2 \times \sqrt{n_2} = 2 \times \sqrt{51} \approx 14.283$.

\item[]

\item[(b)]
$\text{df} = \text{min}(n_1 - 1, n_2 - 1) = \text{min}(20 - 1, 51 - 1) = \text{min}(19, 50) = 19$.

\newpage

\item[(c)]
We are testing $\text{H}_0: \mu_1 = \mu_2$ against $\text{H}_{\text{a}}: \mu_1 \neq \mu_2$.
We have:
$$t = \dfrac{26 - 32}{\sqrt{\dfrac{13.416^2}{20} + \dfrac{14.283^2}{51}}} \approx -1.664$$
Now, from Table C, we get that $0.10 < P < 0.20$. Since $P > 0.05$, we fail to reject the
the null hypothesis $\text{H}_0$ and there is no sufficient evidence to support the claim.
\end{itemize}

\item[]
\item[]

\item[21.43]
\begin{itemize}
\item[(a)]
NOTE: In both stemplots, we round the values to the nearest thousands.\\\\
Below is the stemplot for the data of 27 women.
\item[]
\begin{tabular}{r | *{120}{c}}
    0 & 6 & 7 & 8 & 8 & 9\\
    1 & 0 & 0 & 0 & 1 & 2 & 3 & 3 & 4 & 5 & 6 & 7 & 8 & 9 & 9\\
    2 & 0 & 3 & 5 & 5 & 5 & 6 & 7\\
    3 & \\
    4 & 0\\
\end{tabular}
\item[]
\item[]
KEY: $0 \mid 6$ is 6000 and $2 \mid 5$ is 25000.
\item[]
\item[]

Below is the stemplot for the data of 20 men.
\item[]
\begin{tabular}{r | *{120}{c}}
    0 & 4 & 4 & 5 & 6 & 7 & 8 & 9\\
    1 & 0 & 0 & 1 & 1 & 1 & 3 & 3 & 3 & 6 & 8\\
    2 & 2 & 8\\
    3 & 8
\end{tabular}
\item[]
\item[]
KEY: $0 \mid 6$ is 6000 and $2 \mid 5$ is 25000.
\item[]
\item[]
Notice that both of the stemplots show the right-skewed distributions
for populations. However, the sample sizes are large enough to overcome
the problem of skewness. Therefore, we can use the $t$-procedures.

\item[]

\item[(b)]
NOTE: Subscript 1 is for the population of women and subscript 2 is for the population of men.\\\\
We are testing $\text{H}_0: \mu_1 = \mu_2$ against $\text{H}_{\text{a}}: \mu_1 > \mu_2$.
We have that $\overline{x_1} \approx 16496.1$ and $\overline{x_2} \approx 12866.7$.
Then we have
$$t = \dfrac{16496.1 - 12866.7}{\sqrt{\dfrac{7914.35^2}{27} + \dfrac{8342.47^2}{20}}} \approx 1.51$$
Now, using the conservative inference procedure, we get that
$\text{df} = \text{min}(n_1 - 1, n_2 - 1) = \text{min}(27 - 1, 20 - 1) = \text{min}(26, 19) = 19$.
Using Table C, we get that $0.05 < P < 0.1$ and $P > 0.05$ which means that
we failed to reject the null hypothesis $\text{H}_0$. At last, we got that there is some evidence
that, on average, women say more words than men, however, this evidence is weak and we cannot reject
the null hypothesis $\text{H}_0$.
\end{itemize}

\end{itemize}
\end{document}
