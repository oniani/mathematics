\documentclass[12pt, a4paper]{article}                      % use "amsart" instead of "article" for AMSLaTeX format
\usepackage[a4paper,margin=1in]{geometry}                   % Adjust margins
\usepackage{amsmath,amssymb,listings,color,textcomp}        % Math packages: amsmath, amssymb, listings, color
\usepackage[makeroom]{cancel}

\title{\bf{Homework \textnumero 6}}
\author{Author: David Oniani
\\
\ \ \ Instructor: Tommy Occhipinti}
\date{October 05, 2018}

\usepackage{listings}
\usepackage{color}

\newcommand{\natn}{\mathbb{N}}
\newcommand{\intz}{\mathbb{Z}}
\newcommand{\intzp}{\mathbb{Z^+}}
\newcommand{\intzn}{\mathbb{Z^-}}

\definecolor{dkgreen}{rgb}{0,0.6,0}
\definecolor{gray}{rgb}{0.5,0.5,0.5}
\definecolor{mauve}{rgb}{0.58,0,0.82}
\definecolor{backcolour}{rgb}{0.95,0.95,0.92}

\lstset{
backgroundcolor=\color{backcolour},
aboveskip=3mm,
belowskip=3mm,
showstringspaces=false,
columns=flexible,
basicstyle={\small\ttfamily},
numbers=left,
numberstyle=\normalsize\color{gray},
keywordstyle=\color{blue},
commentstyle=\color{dkgreen},
stringstyle=\color{mauve},
breaklines=true,
breakatwhitespace=true,
tabsize=4
}


\begin{document}
\maketitle


\begin{itemize}
\item[36.]
Prove that if $S$ and $T$ are shifty sets (in the sense of a previous exercise) then $S \cup T$
is also a shifty set.
\begin{quote}
Let's first remember what it means to be shifty.
A subset $S$ of $\intz$ is called shifty if for every $x \in S$, $x - 1 \in S$ or $x + 1 \in S$.\\
Let's consider the following two cases:
\begin{itemize}
\item[1.]
Let $x \in S$ and prove that $x - 1$ or $x + 1$ is in $S \cup T$
\item[2.]
Let $y \in T$ and prove that $y - 1$ or $y + 1$ is in $S \cup T$
\\\\
Let's first consider the case when $x \in S$ and prove that $x - 1$ or $x + 1$ is in $S \cup T$.
If $x \in S$, since $S$ is shifty, it means that either $x - 1 \in S$ or $x + 1 \in S$. Union $S \cup T$
will have all the members of $S$ thus, it means that either $x - 1 \in S \cup T$ or $x + 1 \in S \cup T$.
\\\\
Now let's show that if $y \in T$, $y - 1$ or $y + 1$ is in $S \cup T$.
If $y \in T$, since $T$ is shifty, it means that either $y - 1 \in T$ or $y + 1 \in T$. Union $S \cup T$
will have all the members of $T$ thus, it means that either $y - 1 \in S \cup T$ or $y + 1 \in S \cup T$.
\\\\
Thus, we considered all the cases and proved that if $S$ and $T$ are shifty sets (in the sense of a previous exercise) then $S \cup T$
is also a shifty set.
\end{itemize}
\begin{flushright}
\textit{Q.E.D.}
\end{flushright}
\end{quote}

\item[37.]
Prove that $n \in \intz$ then $1 + (-1)^n(2n-1)$ is a multiple of 4.
\begin{quote}
Since $n \in \intz$, it is either even or odd. Let's consider two cases:
\begin{itemize}
\item[1.]
$n$ is even
\item[2.]
$n$ is odd
\end{itemize}

If $n$ is even, then $n = 2k$ where $k \in \intz$. We get:
$$1 + (-1)^n(2n-1) = 1 + (-1)^{2k}(2 \times 2k-1) = 1 + 4k - 1 = 4k \mbox{, where } k \in \intz$$
Hence, we got that if $n$ is even, $1 + (-1)^n(2n-1) = 4k$ where $k \in \intz$. Thus, if $n$ is even,
is a multiple of 4.

If $n$ is odd, then $n = 2k$ where $k \in \intz$. We have:
$$1 + (-1)^n(2n-1) = 1 + (-1)^{2k+1}(2 \times (2k + 1)-1) = 1  - (4k + 2 - 1) = -4k \mbox{ where } k \in \intz$$
Thus, we got that if $n$ is odd, $1 + (-1)^n(2n-1) = -4k$ where $k \in \intz$. Hence, if $n$ is odd,
$1 + (-1)^n(2n-1) = -4k$ is a multiple of 4.

Finally, since integers could either be odd or even, we considered all the cases ($n$ is even and $n$ is odd) and in
both of the cases, $1 + (-1)^n(2n-1)$ is a multiple of 4.
\begin{flushright}
\textit{Q.E.D.}
\end{flushright}
\end{quote}

\item[38]
Prove that if $n \in \intzp$ is odd then $n^2 - 1$ is divisible by 8.
\begin{quote}
Suppose $n$ is odd. Then $n = 2k + 1$ where $k \in \intzp \cup \{0\}$.
Then we have:
$$n^2 - 1 = (2k + 1)^2 - 1 = 4k^2 + 4k + 1 - 1 = 4k^2 + 4k = 4k \times (k + 1)$$
Thus, if $n$ is odd, $n^2 - 1 = 4k \times (k + 1)$ where $k \in \intzp \cup \{0\}$.
Since $k$ is in the set of positive integers or equls 0, we can consider the following
three cases:
\begin{itemize}
\item[1.]
$k$ is 0
\item[2.]
$k$ is even
\item[3.]
$k$ is odd\\
\end{itemize}
If $k = 0$, $n^2 - 1 = 4k \times (k + 1) = 4 \times 0 \times (0 + 1) = 0$ which is divisible by 8.
\\\\
If $k$ is even, $k = 2l$ where $l \in \intzp$ (not including zero since we already considered that case above).
Then $n^2 - 1 = 4k \times (k + 1) = 8 \times (l \times (2l + 1))$ which is divisible by 8.
\\\\
If $k$ is even, $k = 2l + 1$ where $l \in \intzp$.
Then $n^2 - 1 = 4k \times (k + 1) = 4 \times (2l + 1) \times (2l + 2) = 8 \times ((2l + 1) \times (l + 1))$ which
is divisible by 8.
\\\\
Thus, we considered all the cases and we proved that if $n \in \intzp$ is odd then $n^2 - 1$ is divisible by 8.
\begin{flushright}
\textit{Q.E.D.}
\end{flushright}
\end{quote}

\item[39.]
Prove that every integer can be written as the sum of exactly 3 distinct integers. (For
example, $5 = 4 + 2 + (-1)$).
\begin{quote}
Suppose $k$ is an integer. Then, we should find three distinct integers such that they sum to $k$.
Now, let's consider the following integers:\\
$x = k + 1$, $y = -k - 1$, and $z = k$. It is clear that $x \neq z$ and $y \neq z$.\\
If $x = z$, we have $k + 1 = k \iff 1 = 0$ which is nonsensical.\\
If $y = z$, we have $-k - 1 = k \iff k = \dfrac{-1}{2}$ which is also
not possible because $k$ is an integer.\\
Thus $x \neq z$ and $y \neq z$, but we do not know if $x \neq y$.
Hence, we have to consider cases.
\\\\
Now, consider two cases:
\begin{itemize}
\item[1.]
$x \neq y$
\item[2.]
$x = y$
\end{itemize}
If $x \neq y$, we already know that $x \neq z$ and $y \neq z$ and thus we found three distinct integers which sum up to $k$, namely $x = k + 1$, $y = -k - 1$, and $z = k$.
\\\\
If $x = y$, we get $k + 1 = -k - 1$ and $k = -1$. However, even if $k = -1$, we can find three distinct integers, namely $x = -2$, $y = 2$, and $z = -1$.\\\\
Thus, we considered all the possible cases and we always find three integers which sum up to $k$.
\begin{flushright}
\textit{Q.E.D.}
\end{flushright}
\end{quote}
\item[40.]
Prove that if $x, y, \mbox{ and } z$ are integers then at least one of $x + y$, $x + z$, and $y + z$ is even.
\begin{quote}
Suppose, for the sake of contradiction, that $x + y$, $x + z$, and $y + z$ are all odd.\\\\
Now, without a loss of generality, consider the following three cases:
\begin{itemize}
\item[1.]
$x,y,z$ are all even
\item[2.]
$x,y,z$ are all odd
\item[3.]
$x,y$ are even and $z$ is odd
\item[4.]
$x,y$ are odd and $z$ is even
\end{itemize}
If $x,y,z$ are all even, $x = 2j$, $y = 2k$, $z = 2l$ where $j,k,l \in \intz$. Thus, we get:
$x + y = 2j + 2k = 2 \times (j + k)$ which is even.
\\\\
If $x,y,z$ are all odd, $x = 2j + 1$, $y = 2k + 1$, $z = 2l + 1$ where $j,k,l \in \intz$. Thus, we get:
$x + y = 2j + 1 + 2k + 1 = 2j + 2k + 2 = 2 \times (j + k + 1)$ which is even.
\\\\
If $x,y$ are even and $z$ is odd, $x = 2j$, $y = 2k$, $z = 2l + 1$ where $j,k,l \in \intz$. Thus, we get:
$x + y = 2j + 2k = 2 \times (j + k)$ which is even.
\\\\
If $x,y$ are odd and $z$ is even, $x = 2j + 1$, $y = 2k + 1$, $z = 2l$ where $j,k,l \in \intz$. Thus, we get:
$x + y = 2j + 1 + 2k + 1 = 2j + 2k + 2 = 2 \times (j + k) + 1$ which is even.
\\
Thus, we've considered all the possible cases and proved that if $x, y, \mbox{ and } z$ are integers then at
least one of $x + y$, $x + z$, and $y + z$ is even.
\begin{flushright}
\textit{Q.E.D.}
\end{flushright}
\end{quote}

\item[41.]
Prove or Disprove: There exist prime number $p$ and $q$ such that $p - q = 97$.
\begin{quote}
There are no prime numbers $p$ and $q$ such that $p - q = 97$. Let's prove it!
\\
First, notice that since $p$ and $q$ are prime, $p,q > 0$. Also, from $p - q = 97$,
we get $p = 97 + q$ and since $p,q > 0$, $p > q$. Thus, we have $0 < q  < p$.
Now, since $p - q$ is odd, it is the case that one of $p,q$ is odd and the other one is even
(because difference of even as well as odd numbers is even). However, we know that there is only
one even prime number, namely 2. Let's consider following two cases:
\begin{itemize}
\item[1.]
$p = 2$
\item[2.]
$q = 2$
\end{itemize}
Since $0 < q < p$, we know that $p$ cannot be 2 (if $p = 2$,
there is no prime number below 2, so $q$ cannot be prime).
\\\\
If $q = 2$, $p - 2 = 97$ and $p = 95 = 5 \times 19$ which is also not prime.
\\\\
Thus, we considered all the cases and there are no prime numbers, $p$ and $q$,
which sum up to 95.
\begin{flushright}
\textit{Q.E.D.}
\end{flushright}
\end{quote}
\item[42.]
For $n \in \intzp$, we define the $n^{th}$ triangular number to be $T_n := 1 + 2 + ... + n$.
Thus we have $T_1 = 1, T_2 = 3, T_3 = 6, T_4 = 10, T_5 = 15$, and so on. We will prove later
that $T_n = \dfrac{n(n+1)}{2}$, and you should use that formula for this problem. Prove that
for $n \in \intzp$, $T_n$ is odd if and only if $n$ is 1 or 2 more than a multiple of $4$.
\begin{quote}
Since it is the "if and only if" proof, let's split the proof in two cases:
\begin{itemize}
\item[1.]
Prove that if $n$ is 1 or 2 more than a multiple of 4, $T_n$ is odd.
\item[2.]
Prove that if $n$ is not 1 and is not 2 more than a multiple of 4, $T_n$ is not odd (thus is even).
\end{itemize}
\textbf{Let's first prove that if $n$ is 1 or 2 more than a multiple of 4, $T_n$ is odd.}\\\\
If $n$ is 1 more than a multiple of 4, $n = 4k + 1$ where $k \in \intzp$. Then we have:
$$T_n = \dfrac{n(n+1)}{2} = \dfrac{(4k + 1)((4k + 1) + 1)}{2} = \dfrac{(4k + 1)((4k + 2)}{2} = (2k + 1) \times (2k + 1)$$.
Thus, we have that if $n$ is 1 more than a multiple of 4, $T_n = (2k + 1) \times (2k + 1)$ which is a multiplication of two
odd numbers and thus is odd.
\\\\
If $n$ is 2 more than a multiple of 4, $n = 4k + 2$ where $k \in \intzp$. Then we have:
$$T_n = \dfrac{n(n+1)}{2} = \dfrac{(4k + 2)((4k + 2) + 1)}{2} = \dfrac{(4k + 2)((4k + 3)}{2} = (2k + 1) \times (4k + 1)$$
Hence, we got that $T_n = (2k + 1) \times (4k + 1) = (2k + 1) \times (2(2k) + 1)$ which the multiplication of two odd
numbers and thus, if $n$ is 2 more than a multiple of 4, $T_n$ is odd.
\\\\
\textbf{Now, let's prove that if $n$ is not 1 or 2 more than a multiple of 4, $T_n$ is even.}\\\\
If $n$ is not $1$ and is not two more than a multiple of four, we have the following five cases to consider:
\begin{itemize}
\item[1.]
$n = 4k$ where $k \in \intzp$
\item[2.]
$n = 4k + 3$ where $k \in \intzp$
\end{itemize}
If $n = 4k$ where $k \in \intzp$, $T_n = \dfrac{4k(4k+1)}{2} = 2 \times (k(4k + 1))$ which is even.
\\\\
If $n = 4k + 3$ where $k \in \intzp$, $T_n = \dfrac{(4k+3)(4k+4)}{2} = 2 \times ((4k + 3)(2k + 2))$ which is even.
\\\\
NOTE: SINCE WE CONSIDERED INTEGERS STARTING AT 4 (SINCE WE ASSUMED THAT $n = 4k,4k+1,4k+2,4k+3$ where $k \in \intzp$),
NOW WE NEED TO CHECK $T_n$ for 1,2, and 3.
\\\\
If $n = 1$, $T_n = \dfrac{1(1+1)}{2} = 1$ which is odd and indeed, 1 = 0 + 1 and 0 is a multiple of 4 (thus, 3 has $4l + 1$ form where $l \in \intz$)..
\\
If $n = 2$, $T_n = \dfrac{2(2+1)}{2} = 3$ which is odd and indeed, 2 = 0 + 2 and 0 is a multiple of 4 (thus, 3 has $4l + 2$ form where $l \in \intz$)..
\\
If $n = 3$, $T_n = \dfrac{3(3+1)}{2} = 6$ which is even and indeed, 3 = 0 + 3 and 0 is a multiple of 4 (thus, 3 has $4l + 3$ form where $l \in \intz$).
\\\\
Thus, we considered all the cases and proved that for $n \in \intzp$, $T_n$ is odd if and only if $n$ is 1 or 2 more
than a multiple of $4$.
\begin{flushright}
\textit{Q.E.D.}
\end{flushright}
\end{quote}
\newpage
{\large Bookwork}
\item[1.]
TODO\\
TODO\\
TODO\\
TODO\\
TODO\\
TODO\\
\\
Solve the equality $|x + 1| < |x^2 - 1|$. Interpret results geometrically.
\begin{quote}
$x + 1$ 
\begin{flushright}
\textit{Q.E.D.}
\end{flushright}
\end{quote}
\end{itemize}
\end{document}
