\documentclass[12pt, a4paper]{article}                      % use "amsart" instead of "article" for AMSLaTeX format
\usepackage[a4paper,margin=1in]{geometry}                   % Adjust margins
\usepackage{amsmath,amssymb,mathtools,listings,color,textcomp,adjustbox,tikz,pgfplots}        % Math packages: amsmath, amssymb, listings, color
\usepgfplotslibrary{external,fillbetween}
\pgfplotsset{compat=1.14}
\usepackage[makeroom]{cancel}
\title{\bf{Homework \textnumero 6}}
\author{Author: David Oniani
\\
\ \ \ Instructor: Tommy Occhipinti}
\date{October 12, 2018}

\usepackage{listings}
\usepackage{color}

\newcommand{\natn}{\mathbb{N}}
\newcommand{\intz}{\mathbb{Z}}
\newcommand{\intzp}{\mathbb{Z^+}}
\newcommand{\intzn}{\mathbb{Z^-}}

\definecolor{dkgreen}{rgb}{0,0.6,0}
\definecolor{gray}{rgb}{0.5,0.5,0.5}
\definecolor{mauve}{rgb}{0.58,0,0.82}
\definecolor{backcolour}{rgb}{0.95,0.95,0.92}

\lstset{
backgroundcolor=\color{backcolour},
aboveskip=3mm,
belowskip=3mm,
showstringspaces=false,
columns=flexible,
basicstyle={\small\ttfamily},
numbers=left,
numberstyle=\normalsize\color{gray},
keywordstyle=\color{blue},
commentstyle=\color{dkgreen},
stringstyle=\color{mauve},
breaklines=true,
breakatwhitespace=true,
tabsize=4
}


\begin{document}
\maketitle


\begin{itemize}
\item[43.]
Prove that for all $n \in \intzp$, $11^n - 6$ is divisible by 5.
\begin{quote}
Let's prove this by induction. For this we have to have a base case
and an inductive hypothesis.\\\\
Base case: if $n = 1$, $11^n - 6 = 11 - 6 = 5$ and thus, since 5 is divisible by 5, $11^n - 6$ is divisible by 5. Hence, the base case check is done.\\\\
Inductive case: suppose for all $k \in \intzp$, $11^k - 6$ is divisible by 5 and prove that $11^{k + 1} - 6$ is divisible by 5.
Notice that
$$11^{k + 1} - 6 = 11 \times 11^k - 6 = 10 \times 11^k + (11^k + 6)$$
Now, once again, notice that $10 \times 11^k$ is divisible by 5 since 10 is a multiple of 5. According to our assumption,
$11^k - 6$ is also divisible by 5.
Finally, we get that $11^{k + 1} - 6 = 10\times 11^k + (11^k + 6)$ which means that $11^{k + 1} - 6$ is the sum of two numbers which
are both divisible by 5.
\begin{flushright}
\textit{Q.E.D.}
\end{flushright}
\end{quote}

\item[]

\item[44.]
Prove that for $n \in \intzp$, we have $1^2 + 2^2 + 3^2 + ... + n^2 = \dfrac{n(n + 1)(2n + 1)}{6}$.
\begin{quote}
Let's prove this by taking an inductive approach.\\\\
Base case: if $n = 1$, $1^2 = \dfrac{1(1 + 1)(2 + 1)}{6} = \dfrac{1 \times 2 \times 3}{6} = 1$ and thus, the base case check is done.\\\\
Inductive hypothesis: suppose for $k \in \intzp$, $1^2 + 2^2 + 3^2 + ... + k^2 = \dfrac{k(k + 1)(2k + 1)}{6}$ and prove that $1^2 + 2^2 + 3^2 + ... + k^2 + (k + 1)^2 = \dfrac{(k + 1)(k + 2)(2k + 3)}{6}$.
According to our initial assumption, we have:\\
\begin{align*}
1^2 + 2^2 + 3^2 + ... + k^2 + (k + 1)^2 = (1^2 + 2^2 + 3^2 + ... + k^2) + (k + 1)^2 =\\
1^2 + 2^2 + 3^2 + ... + k^2 + (k + 1)^2 = \dfrac{k(k + 1)(2k + 1)}{6} + (k + 1)^2 =\\
\dfrac{k(k + 1)(2k + 1)}{6} + \dfrac{6(k + 1)^2}{6} =\\
\dfrac{k(k + 1)(2k + 1) + 6(k + 1)^2}{6} =\\
\dfrac{(k + 1)(k(2k + 1) + 6(k + 1))}{6} =\\
\dfrac{(k + 1)(2k^2 + k + 6k + 6)}{6} =\\
\dfrac{(k + 1)(2k^2 + 7k + 6)}{6} =\\
\dfrac{(k + 1)((k + 2)(2k + 3))}{6}
\end{align*}
Thus, we proved that $1^2 + 2^2 + 3^2 + ... + k^2 + (k + 1)^2 = \dfrac{(k + 1)(k + 2)(2k + 3)}{6}$ which finishes
our proof.
\begin{flushright}
\textit{Q.E.D.}
\end{flushright}
\end{quote}

\item[]

\item[45.]
Prove that for all $n \in \intzp$, $2^{n + 2} + 3^{2n + 1}$ is divisible by 7.
\begin{quote}
To prove it using induction, we'll need a base case and an inductive hypothesis.\\\\
Base case: let $n = 1$, then $2^{1 + 2} + 3^{2 \times 1 + 1} = 2^3 + 3^3 = 8 + 27 = 35 = 7 \times 5$ thus is divisible
by 7. And we've checked the base case.\\\\
Inductive hypothesis: now, we have to prove the inductive hypothesis. Here is the hypothesis:
suppose that for all $k \in \intzp$, $2^{k + 2} + 3^{2k + 1}$ is divisible by 7 and show that
$2^{k + 3} + 3^{2k + 3}$ is also divisible by 7.\\\\
Notice that
\begin{align*}
2^{k + 3} + 3^{2k + 3} = \underbrace{2^{k + 2} + 2^{k + 2}}_{2^{k + 3}} + \underbrace{3^{2k + 1} + 3^{2k + 1} + 7 \times 3^{2k + 1}}_{3^{2k + 3}} =\\
2 \times (2^{k + 2} + 3 ^ {2k + 1}) + 7 \times 3^{2k + 1}
\end{align*}
And now it's clear that since $2^{k + 2} + 3 ^ {2k + 1}$ and $7 \times 3^{2k + 1}$ are the multiples of 7,
$2 \times (2^{k + 2} + 3 ^ {2k + 1}) + 7 \times 3^{2k + 1}$ is also a multiple of 7 and thus is divisible by 7.
Hence, we proved the inductive hypothesis and it concludes our proof.
\begin{flushright}
\textit{Q.E.D.}
\end{flushright}
\end{quote}

\item[]

\item[46.]
Let $F_n$ denote the $n^{th}$ Fibonacci number. Prove that $\displaystyle\sum_{k=1}^{n} F^2_k = F_nF_{n + 1}$.
\begin{quote}
Let's prove this using induction. For this we have to have a base case and inductive hypothesis.\\\\
Let's first look at the Fibonacci sequence. It is 1, 1, 2, 3, 5, 8, 13, 21 ... where each element is the
sum of two previous elements.\\\\
Base case: let $k = 1$, then $\displaystyle\sum_{k=1}^{1} F^2_k = 1 \times 1 = 1$ and indeed the second
element of the sequence is 1.\\\\
Inductive hypothesis: let $F_m$ denote the $m^{th}$ Fibonacci number and also suppose that $\displaystyle\sum_{j=1}^{m} F^2_j = F_mF_{m + 1}$.\\
Then, we have to prove that $\displaystyle\sum_{j=1}^{m + 1} F^2_j = F_{m + 1}F_{m + 2}$.\\\\
Notice that $\displaystyle\sum_{j=1}^{m + 1} F^2_j = F_1^2 + F_2^2 + F_3^2 + ... + F_{m + 1}^2$ and using our assumption, we get:
$$F_1^2 + F_2^2 + F_3^2 + ... + F_{m + 1}^2 = F_mF_{m + 1} + F_{m + 1}^2 = F_{m + 1}(F_m + F_{m + 1})$$.
Now, let's remember the way Fibonacci numbers are created. To get the next Fibonacci number, one must sum up the previous two
thus, $F_m + F_{m + 1} = F_{m + 2}$ and we finally get that $\displaystyle\sum_{j=1}^{m + 1} F^2_j = F_1^2 + F_2^2 + F_3^2 + ... + F_{m + 1}^2 = F_mF_{m + 1} + F_{m + 1}^2 = F_{m + 1}(F_m + F_{m + 1}) = F_{m + 1}F_{m + 2}$.
Hence, we proved that $\displaystyle\sum_{j=1}^{m + 1} F^2_j = F_{m + 1}F_{m + 2}$ and this concludes our inductive reasoning as well as the proof.\\
Finally, if $F_n$ denotes the $n^{th}$ Fibonacci number, $\displaystyle\sum_{k=1}^{n} F^2_k = F_nF_{n + 1}$.
\begin{flushright}
\textit{Q.E.D.}
\end{flushright}
\end{quote}

\item[]

\item[47.]
Prove that for all $n \in \intzp_{\geq 12}$ we have $n! \geq 5^n$.
\begin{quote}
To prove the statement above, we can use induction. We need to have a base case as well as an inductive hypothesis.\\\\
Base case: let $n = 12$, then we have $12! > 5^{12}$ which is true since $12! = 479001600$ and $5^{12} = 244140625$ (thus $12! > 5^{12}$).\\\\
Inductive hypothesis: suppose that for every $k \in \intzp_{\geq 12}$ we have $k! \geq 5^k$ and prove that $(k + 1)! > 5^{k + 1}$.\\\\
Since we assumed that $k! \geq 5^k$, by multiplying both sides on $k + 1$, we have $(k + 1)! \geq 5^k (k + 1)$.
Now, since $k \geq 12$, we know that $k + 1 \geq 13 > 5$ and thus, since $(k + 1)! \geq 5^k (k + 1)$, we also know that
$(k + 1)! \geq 5^{k + 1}$.
\begin{flushright}
\textit{Q.E.D.}
\end{flushright}
\end{quote}

\item[]

\item[48.]
Suppose that $x \neq 0$ is a real number and $x + \dfrac{1}{x} \in \intz$.
Prove that $x^n + 1/x^n \in \intz$ for all $n \in \intzp$. (Hint: $(x + \dfrac{1}{x})(x^n + \dfrac{1}{x^n})$ is probably nicer than you think!).
\begin{quote}
Let's use induction to prove this. Hence, we need the base case and an inductive hypothesis.\\\\
Base case: if $x = 1$, $x + \dfrac{1}{x} = 1 + \dfrac{1}{1} = 2$ thus, we have that if $x = 1$
$x + \dfrac{1}{x}$ is also in $\intz$. Now, we have to check $x^n + \dfrac{1}{x^n}$.
$x^n + \dfrac{1}{x^n} = 1^n + \dfrac{1}{1^n} = 1 + \dfrac{1}{1} = 2$ and the base case is checked.\\\\
Now, we have to come up with an inductive hypothesis.\\
Inductive hypothesis is that if $y \neq 0$ and $y + \dfrac{1}{y} \in \intz$, we have $y^m + \dfrac{1}{y^m} \in \intz$ and we have to prove that for all $m \in \intzp$,
$y^{m + 1} + \dfrac{1}{y^{m + 1}} \in \intz$.\\\\
Notice that
$$(y + 1/y)(y^m + 1/y^m) =  y^{m + 1} + \dfrac{1}{y^{m + 1}} + y^{m - 1} + \dfrac{1}{y^{m - 1}}$$
and
$$y^{m + 1} + \dfrac{1}{y^{m + 1}} = \underbrace{(y + \dfrac{1}{y})(y^m + \dfrac{1}{y^m})}_{\mbox{we know it's an integer}} - \underbrace{(y^{m - 1} + \dfrac{1}{y^{m - 1}})}_{\mbox{it's an integer too}}$$
By the assumption we made in the inductive hypothersis, we know that $(y + \dfrac{1}{y})(y^m + \dfrac{1}{y^m})$ is an integer.
According to our assumption, $y^{m - 1} + \dfrac{1}{y^{m - 1}}$ is also an integer and if we subtract one integer from the other, we always end up with an integer.
Thus, $y^{m + 1} + \dfrac{1}{y^{m + 1}} = (y + \dfrac{1}{y})(y^m + \dfrac{1}{y^m}) - (y^{m - 1} + \dfrac{1}{y^{m - 1}})$ which means that $y^{m + 1} + \dfrac{1}{y^{m + 1}}$
is an integer and this concludes our proof.
\begin{flushright}
\textit{Q.E.D.}
\end{flushright}
\end{quote}

{\Large Bookwork}
\item[]
\item[5.]
Show: $n^2 \leq 2^n$ for each integer $n \geq 4$.
\begin{quote}
Let's use induction.\\\\
Base case: let $n = 4$, then $2^4 \leq 2^4$ which is equivalent to $16 \leq 16$ which is true
thus, the base case check is done.\\\\
Inductive hypothesis: suppose that for $k \geq 4$, $k^2 \leq 2^k$ and prove that $(k + 1)^2 \leq 2^{k + 1}$.\\\\
Notice that since we assumed that $k^2 \leq 2^k$ for $k \geq 4$, we have effectively assumed that $2k^2 \leq 2^{k + 1}$.
Thus, if we now prove that for $k \geq 4$, $2k^2 \geq (k + 1)^2$, the proof is done. So, let's just solve the inequality $k \geq 4$, $2k^2 \geq (k + 1)^2$
and see if it is true for $k \geq 4$. We have:
\begin{align}
2k^2 \geq (k + 1)^2\\
2k^2 \geq k^2 + 2k + 1\\
k^2 \geq 2k + 1\\
k^2 - 2k - 1 \geq 0\\
(k - (1 + \sqrt{2}))(k - (1 - \sqrt{2})) \geq 0
\end{align}
And finally, we get that $k \in (-\infty, 1 - \sqrt{2}) \cup (1 + \sqrt{2}, +\infty)$.
Now, notice that $4 > 1 + \sqrt{2}$ and thus, for all $k \geq 4$, this inequality holds
and this concludes our proof.
\end{quote}

\item[]
\item[12.]
Let $b_n = 1/3 + 1/15 + ... + 1/(4n^2 - 1)$.
\begin{itemize}
\item[(a)]
Compute $b_1, b_2, b_3, b_4, \mbox{ and } b_5$.
\begin{quote}

$b_1 = \dfrac{1}{4 \times 1^2 - 1} = 1/3$\\
$b_2 = b_1 + \dfrac{1}{4 \times 2^2 - 1} = 1/3 + 1/15 = 2/3$\\
$b_3 = b_1 + b_2 + \dfrac{1}{4 \times 3^2 - 1} = 1/3 + 1/15 + 1/35 = 73/105$\\
$b_4 = b_1 + b_2 + b_3 + b_4 = 73/105 + \dfrac{1}{4 \times 4^2 - 1} = 73/105 + 1/63 = 56/2205$
$b_5 = b_1 + b_2 + b_3 + b_4 + \dfrac{1}{4 \times 5^2 - 1} = 56/2205 + 1/99 = 861/24255$.
\end{quote}
\end{itemize}
\begin{flushright}
\textit{Q.E.D.}
\end{flushright}
\end{itemize}
\end{document}
