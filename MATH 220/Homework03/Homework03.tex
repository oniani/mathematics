\documentclass[12pt, a4paper]{article}                  % use "amsart" instead of "article" for AMSLaTeX format
\usepackage[a4paper,margin=1in]{geometry}               % Adjust margins
\usepackage{amsmath,amssymb,textcomp,listings,graphicx,adjustbox}    % Math packages: amsmath, amssymb, listings, color
\usepackage[makeroom]{cancel}

\title{\bf{Homework \textnumero 3}}
\author{Author: David Oniani
\\
\ \ \ Instructor: Tommy Occhipinti}
\date{September 7, 2018}

\usepackage{listings}
\usepackage{color}

\definecolor{dkgreen}{rgb}{0,0.6,0}
\definecolor{gray}{rgb}{0.5,0.5,0.5}
\definecolor{mauve}{rgb}{0.58,0,0.82}
\definecolor{backcolour}{rgb}{0.95,0.95,0.92}

\lstset{
backgroundcolor=\color{backcolour},
aboveskip=3mm,
belowskip=3mm,
showstringspaces=false,
columns=flexible,
basicstyle={\small\ttfamily},
numbers=left,
numberstyle=\normalsize\color{gray},
keywordstyle=\color{blue},
commentstyle=\color{dkgreen},
stringstyle=\color{mauve},
breaklines=true,
breakatwhitespace=true,
tabsize=4
}


\begin{document}
\maketitle

{\large Section 2.2}
\\
\begin{enumerate}
\item[7.]
List three elements of each of the following sets.

\begin{enumerate}
\item[(a)]
$\mathbb{Q} \cap (2,3)$\\\\
Since $(2,3)$ is an open range from $2$ to $3$ (not including $2$ and $3$),
all of it is in the $\mathbb{Q}$ and the answer of the intersection is $(2,3)$.\\

Here are three elements that are the result set: $2.1,2.2,2.3$
\\
\item[(b)]
$\{2^n - 1 \ \vert \ n \in \mathbb{Z^+} \}$\\

If $n=1$, then \ $2^n - 1 = 1$\\
If $n=2$, then \ $2^n - 1 = 3$\\
If $n=3$, then \ $2^n - 1 = 7$\\\\
Here are three elements that are the result set: $1,3,7$
\\
\item[(c)]
$\{n \in \mathbb{Z^+} \ \vert \ n^2 + 1 \ \textnormal{is prime} \}$\\

If $n=1$, then \ $n^2 + 1 = 2$ and 2 is a prime\\
If $n=2$, then \ $n^2 + 1 = 5$ and 3 is a prime\\
If $n=4$, then \ $n^2 + 1 = 17$ and 17 is a prime\\\\
Here are three elements that are the result set: $2,3,17$
\end{enumerate}

\

\item[8.]
Let $A = [4,7]$ and $B = (6,8)$, both subsets of the universal set
$\mathbb{R}$. Write each of the following sets as naturally as possible:
\begin{enumerate}
\item[(a)]
$A \cup B$\\

$A \cup B = [4,8)$. Thus, the set $A \cup B$ is the set of all real numbers
from 4 to 8, including 4 but not including 8.
\\
\item[(b)]
$A \cap B$\\

$A \cap B = (6,7]$. Thus, the set $A \cap B = (6,7]$ is the set of all real numbers
from 6 to 7, not including 6 but including 7.
\\
\item[(c)]
$A^C$\\

$A^C = (-\infty, 4) \ \cup \ (7, +\infty)$. Thus, the set $A^C$ is the set of all real numbers
but those in the range $[4,7]$, not including $4$ and $7$ (also called the \textit{complement} of $A$).
\\
\item[(d)]
$B^C$\\

$B^C = (-\infty, 6] \ \cup \ [8, +\infty)$. Thus, the set $B^C$ is the set of all real numbers
but those in the range $(6,8)$, including $4$ and $7$ (also called the \textit{complement} of $B$).
\\

\item[(e)]
$A - B$\\

$A - B = [4,6]$. Thus, the set $B^C$ is the set of all real numbers
in the range $[4,6]$, $4$ and $6$ inclusive.
\end{enumerate}

\

\item[9.]
A \textbf{bi-partition} of a set $S$ is a set $\{A,B\}$ of two subsets
$A$ and $B$ of $S$ such that $A \cup B = S$ and $A \cap B = \emptyset$.
\begin{enumerate}
\item[(a)]
List all bi-partitions of the set $\{1,2,3\}$.
\\

The list of all bi-partitions of the set $\{1,2,3\}$ is:
\begin{center}
$\{1\}$ and $\{2,3\}$\\
$\{2\}$ and $\{1,3\}$\\
$\{3\}$ and $\{1,2\}$\\
\end{center}
The list above shows that there are three bi-partitions for the given set.
\\
\item[(b)]
Explain why, every subset $A$ of a set $S$ is an element of exactly one bi-partition of
$S$. (Hint: First explain why every such $A$ is an element of at least one bi-partition
of $S$, then explain why it cannot be be an element of more than one bi-partition
of $S$.)
\\

$A$ is an element of at least one bi-partition since the list of bi-partitions of $B$
are really the sets of paired subsets of $B$. If we list all the bi-partition subset pairs,
we will get all the subsets of $B$. Then if we have all the subsets of $B$, $A$ is just one
of them and it has to be one of the bi-partitions.
\\
\item[(c)]
It is a fact that we will prove later that if $S = \{1, 2, 3, 4, 5, 6, 7, 8, 9, 10 \}$, then $S$
has exactly 1024 subsets. How many bi-partitions does S have?\\

Actually, let's prove it now and then count the number of bi-partitions.\\
So, for each element of the set, we have two options, either put in the subset
or leave it off. We have $10$ elements in the set $S$. $2^{10} = 1024$.
\begin{flushright}
\textit{Q.E.D}
\end{flushright}
Now, let's go ahead and count the number of bi-partitions for the set.\\
\begin{center}
$\{1\}$ and $\{2,3,4,5,6,7,8,9,10\}$\\
$\{2\}$ and $\{1,3,4,5,6,7,8,9,10\}$\\
$\{3\}$ and $\{1,2,4,5,6,7,8,9,10\}$\\
$\{4\}$ and $\{1,2,3,5,6,7,8,9,10\}$\\
$\{5\}$ and $\{1,2,3,4,6,7,8,9,10\}$\\
$\{6\}$ and $\{1,2,3,4,5,7,8,9,10\}$\\
$\{7\}$ and $\{1,2,3,4,5,6,8,9,10\}$\\
$\{8\}$ and $\{1,2,3,4,5,6,7,9,10\}$\\
$\{9\}$ and $\{1,2,3,4,5,6,7,8,10\}$\\
$\{10\}$ and $\{1,2,3,4,5,6,7,8,9\}$\\
\end{center}
Thus, we see that the number of the bi-partitions for the set $S$ is 10.
\end{enumerate}
\item[2.]
Write down two sets each having infinitely many elements.
Do parts (a)--(c) for these sets.\\

Let's look at two infinite sets $A = \{x \in \mathbb{Z^+} \ \vert \ 2x \}$
and $B = \{x \in \mathbb{Z^+} \ \vert \ 2x - 1 \}$.

\begin{enumerate}
\item[(a)]
What is their union?\\
\\
$A \cup B = \mathbb{Z}^+$
\\
\item[(a)]
What is their intersection?\\
\\
$A \cap B = \emptyset$
\\
\item[(a)]
What is their set difference? (both of them)\\

$A - B = A$\\
$B - A = B$
\end{enumerate}

\item[6.]
List the elements in each of the following sets:
\begin{enumerate}
\item[(h)]
$\{x \in \mathbb{R} \ \vert \ x^3 - 3x = 0\}$\\

If $x^3 - 3x = 0$, then the solutions to the equation
are $0,\sqrt{3}$ and $-\sqrt{3}$.\\
Thus the elements of the set are $0,\sqrt{3},-\sqrt{3}$.
\\
\item[(h)]
$\{x \in \mathbb{R} \ \vert \ x^2 + 4x + 5 = 0\}$\\

Equation $x^2 + 4x + 5 = 0$ has no solutions.\\
Thus the set is $\emptyset$.
\\
\end{enumerate}

\item[7.]
\begin{enumerate}
\item[(c).]
Show that $\{x \in \mathbb{R} \ \vert \ (x-1/2)(x-1/3)<0\} \subseteq \{x \in \mathbb{R} \ \vert \ 0 < x < 1\}$.
Check your answer.\\

The solution set for the equation $(x-1/2)(x-1/3)$ is $\{1/2,1/3\}$. Thus, $\{x \in \mathbb{R} \ \vert \ (x-1/2)(x-1/3)<0\} = \{1/2,1/3\}$.
$\{x \in \mathbb{R} \ \vert \ 0 < x < 1\}$ the set of all elements of the open range $(0,1)$.
$1/2 \in (0,1)$ and $1/3 \in (0,1)$.
\begin{flushright}
\textit{Q.E.D}
\end{flushright}
\end{enumerate}

\item[14.]
A set $A$ is called \textit{full} if any element of $A$ is also a subset of $A$.
In other words, $A$ is full if $x \in A$ implies $x \subseteq A$.
\begin{enumerate}
\item[(a)]
Show that $\{\emptyset\}$ is full.\\

Set $\{\emptyset\}$ has only one element, namely $\emptyset$ (empty set), and $\emptyset \subseteq \{\emptyset\}$.
Hence, $\{\emptyset\}$ is full.
\begin{flushright}
\textit{Q.E.D}
\end{flushright}
\item[(b)]
Find a full set having exactly two elements.\\

$\{\emptyset,\{\emptyset\}\}$.
\\
\item[(c)]
Find a full set having exactly three elements.\\

$\{\emptyset,\{\emptyset\},\{\{\emptyset\}\}\}$.
\end{enumerate}

\

\item[19.]
Russell's Paradox. Let $S = \{A \ \vert \ A \ \textnormal{is a set and} \ A \notin A\}$.
Suppose that $S$ itself is a set.
\begin{enumerate}
\item[(a)]
Show that if $S$ is a member of itself, then $S$ cannot be a member of itself.\\

If $S \in S$, then $S \notin S$ as $S$ is a subset of $S$ and for every subset $A$
of $S$, if $A \in S$, then $A \notin S$.
\begin{flushright}
\textit{Q.E.D}
\end{flushright}
\item[(b)]
Show that if $S$ is not a member of itself, then $S$ must be a member of itself.\\

Since $S$ is a set, it has to have members.
Thus, if $S$ is not a member of itself, then $S$ must be a member of itself.
\begin{flushright}
\textit{Q.E.D}
\end{flushright}
\end{enumerate}

{\large Section 2.3}
\\

\item[10.]
An integer $n$ is called doubly even if there exist even integers $x$ and $y$ such that $n = xy$.
\begin{enumerate}
\item[(a)]
Is 12 doubly even? Prove your answer.\\

Yes, it is doubly even since $12 = 2 \times 6$ where $2$ and $6$ are even.
\begin{flushright}
\textit{Q.E.D.}
\end{flushright}
\item[(b)]
Is 98 doubly even? Prove your answer.? Prove your answer.\\

No, it is not doubly even since $98 = 2 \times 47$ where $2$ and $47$ is odd.
Thus, since 47 cannot be further divided by 2 (and obtain integer), it is not double even.
\begin{flushright}
\textit{Q.E.D.}
\end{flushright}
\item[(c)]
Write the negation of the statement “n is doubly even” without using the word
“not.”\\

$n$ is not divisible by 4.\\

Explanation:\\
Notice that if $n$ is doubly even, by the definition, it is the multiplication
of two even integers that is $n = xy$. Let $x = 2l$ and $y = 2k$ $(l, \ k \in \mathbb{Z}^+)$, then we have
$n = 2l * 2k = 4lk$. Thus, if $n$ is doubly even, then it is divisible by 4.
\\
\item[(d)]
For what positive integers $n$ is $n!$ doubly even? Prove your answer.\\

For $n \geq 4$. Let's prove it!

For a number $n$ to be doubly-even, it should be divisible by $4$
(if $n = xy$ where $x$ and $y$ are even, then let $x = 2i$ and $y = 2j$
where $i,j \in \mathbb{Z}$ and we get $n = 4ij$ which proves that $n$ is
indeed divisible by $4$). From this fact, we can deduce that $n!$ must
be divisible by $4$ which happens for $n \geq 4$. Thus, if $n \geq 4$, $n!$
is doubly even.
\begin{flushright}
\textit{Q.E.D.}
\end{flushright}
\end{enumerate}

\item[11.]
A subset $S$ of $\mathbb{Z}$ is called \textit{shifty} if for every $x \in S, \ x - 1 \in S, \ \textnormal{or} \ x + 1 \in S$.
\begin{enumerate}
\item[(a)]
Give an example of a shifty set with 5 elements.\\

$S = \{1,2,3,4,5\}$
\\
\item[(b)]
Give an example of a shifty set that contains $10$ and $-10$ but does not contain $0$.\\

$S = \{-10,-9,-8,-7,-6,-5,-4,-3,-2,-1,1,2,3,4,5,6,7,8,9,10\}$
\\
\item[(c)]
Is $\{n \in \mathbb{Z}^+ \ \vert \ n \ \textnormal{is not a multiple of 5 or 11} \}$ shifty? Why or why not?\\

For the set not to be shifty, we should have an element $n$ for which either $n-1$
must be either divisible by $5$ or $11$ and $n + 1$ should either be divisible by $5$
or $11$. Then, all we have to do is solve the systems of the following Diophantine equations:

$$
\begin{cases}
n - 1 = 5k, \ where \ k \in \mathbb{Z}^+\\
n + 1 = 11l, \ where \ l \in \mathbb{Z}^+
\end{cases}
$$
\begin{center}
or
\end{center}
$$
\begin{cases}
n - 1 = 11k, \ where \ k \in \mathbb{Z}^+\\
n + 1 = 5l, \ where \ l \in \mathbb{Z}^+
\end{cases}
$$
\end{enumerate}
Let's only consider the first system. From the first equation, we get:
$$n = 5k + 1$$
Then, if we substitute $n$ in the second equation, we get the following Diophantine equation:
$$11l - 5k = 2$$
It's easy to see that $l = 2$ and $k = 4$. Then we have $n = 5 \times 4 + 1 = 21$.
And we finally, $21$ is an element of the set for which $21 - 1 = 20$ which is divisible
by $5$ and $21 + 1 = 22$ is divisible by $11$.
Thus,  $\{n \in \mathbb{Z}^+ \ \vert \ n \ \textnormal{is not a multiple of 5 or 11} \}$ is not shifty.
\\
\item[(d)]
$\{n \in \mathbb{Z}^+ \ \vert \ n \ \textnormal{is a multiple of 5 or $n + 1$ is a multiple of 5} \}$.
Why or why not?\\

This set will be shifty. To see, let's group the elements of the set in pairs.
All the pairs will have the following type $(5k-1,\ 5k), \ \textnormal{where} \ k \in \mathbb{Z}^+$.
Then, we know that for for $5k-1$, $5k$ is in the set and for $5k$, $5k-1$ is in the set
which makes the set shifty.
\\
\item[(e)]
Write the negation of the statement “S is shifty” without using the word “not.”\\

There exists $x \in S$ such that $x-1 \notin S$ and $x+1 \notin S$.
\\
\item[(f)]
Does every non-empty shifty set contain an even integer? Why or why not?\\

Yes. In order for the set $S$ to be shifty, for every element (positive integer) $n$, either $n - 1$
or $n + 1$ must be in the set. If $n$ is odd, then $n-1$ and $n+1$ are even and if $n$ is even, then
$n-1$ and $n+1$ are odd. Thus, for a set to be shifty, it must contain an even integer.

\item[2.]
If $\sum a_n$ is a convergent infinite series, then $\lim_{n \to \infty} a_n = 0$.
\begin{enumerate}
\item[(a)]
If $\lim_{n \to \infty} a_n \neq 0$, then $\sum a_n$ is not a convergent series.
\item[(b)]
If $\lim_{n \to \infty} a_n = 0$, then $\sum a_n$ is a convergent series.\\
It's true.
\end{enumerate}

\item[6.]
If $\sum |a_n|$ is convergent, then $\sum a_n$ converges.
\begin{enumerate}
\item[(a)]
If $\sum a_n$ is not convergent, then $\sum |a_n|$ does not converge.
\item[(b)]
If $\sum |a_n|$ is not convergent, then $\sum a_n$ does not converge.\\
It's true
\end{enumerate}

\item[8.]
If $f$ has a local (relative) maximum at the real number $a$ or a
local (relative) minimum at $a$, then $f^{'}(a) = 0$ or $f^{'}(a)$ does not exist.
\begin{enumerate}
\item[(a)]
If $f^{'}(a)$ exists and $f^{'}(a) \neq 0$, then $f$ does not have a local
(relative) maximum at the real number $a$ or a local (relative) minimum at $a$.
\item[(b)]
If $f^{'}(a) = 0$ or $f^{'}(a)$ does not exist, then $f$ has a local
(relative) maximum at the real number $a$ or a local (relative) minimum at $a$.\\
It's true.
\end{enumerate}

\item[12.]
To the ideal mathematician, what is a proof?

To the "ideal mathematician", the proof is the process which is comprised of 3 stages:
\begin{enumerate}
\item[1.]
Writing down the axioms of the theory in a formal language with a given list of symbols or alphabet.
\item[2.]
Writing down the hypothesis of the theorem in the same symbolism.
\item[3.]
Showing that it is possible to transform the hypothesis step by step, using the rules of logic, till the conclusion is reached.
\end{enumerate}

\item[13.]
The ideal mathematician regards it as obvious that an extraterrestrial intelligence
capable of intergalactic travel would recognize the binary expansion of $\pi$.
Do you believe this? Explain why.

Interesting question. Actually, I am not sure. If the question was about something not
involving $\pi$ (or such "universal" constants), it would be easier... It is \underline{possible} that they
simply cannot create something like circle (for some reasons which I am not sure of yet).
One possible thing might be some very very weird gravity or atmosphere or the way their
universe works (it might be that the universe is "anti-circle"). Secondly, "intelligence"
is a very ambiguous thing and the definition depends on a person. For some, it's knowing maths,
for some it's philosophy, for some it is everything altogether etc.
Hence, due to the limited information about the aliens and their environment/universe as well as
the vague "intelligence" definition, my answer is I DON'T KNOW.

\item[14]
How does this article make you feel about studying mathematics?

Good.
\end{enumerate}

\end{document}
