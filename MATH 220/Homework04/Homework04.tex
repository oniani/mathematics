\documentclass[12pt, a4paper]{article}                      % use "amsart" instead of "article" for AMSLaTeX format
\usepackage[a4paper,margin=1in]{geometry}                   % Adjust margins
\usepackage{amsmath,amssymb,listings,color,textcomp}        % Math packages: amsmath, amssymb, listings, color
\usepackage[makeroom]{cancel}

\title{\bf{Homework \textnumero 4}}
\author{Author: David Oniani
\\
\ \ \ Instructor: Tommy Occhipinti}
\date{September 13, 2018}

\usepackage{listings}
\usepackage{color}

\definecolor{dkgreen}{rgb}{0,0.6,0}
\definecolor{gray}{rgb}{0.5,0.5,0.5}
\definecolor{mauve}{rgb}{0.58,0,0.82}
\definecolor{backcolour}{rgb}{0.95,0.95,0.92}

\lstset{
backgroundcolor=\color{backcolour},
aboveskip=3mm,
belowskip=3mm,
showstringspaces=false,
columns=flexible,
basicstyle={\small\ttfamily},
numbers=left,
numberstyle=\normalsize\color{gray},
keywordstyle=\color{blue},
commentstyle=\color{dkgreen},
stringstyle=\color{mauve},
breaklines=true,
breakatwhitespace=true,
tabsize=4
}


\begin{document}
\maketitle

{\large Section 3.1}
\\

\begin{enumerate}
\item[15.]
Prove the following statement is false by providing a counterexample: If $n \in \mathbb{Z}^+$ is odd
and $n > 1$ then there exists a non-negative integer $i$ and a prime $p$ such that $n = 2^i + p$.
\\

\begin{quote}
This is the type of statement where "for every" part is hidden. In other words,
the statement could be translated in the following way: "For every $n > 1 \in \mathbb{Z}^+$
there exists a non-negative integer $i$ and a prime $p$ such that
$n = 2^i + p$". Notice that $n = 2$ is a counterexample. If $n = 2$, then $i$ has to be either 0 or 1 (since $i$ is non-negative and
if $i > 1$, $2^i > 2$ and the equality will not hold).
Hence, for $n = 2$, we have $i = 0$ or $i = 1$.
If $n = 2$ and $i = 0$, $2^i = 1$ and $p$ has to be 1 which is not prime.
If $n = 2$ and $i = 1$, $2^i = 2$ and $p$ has to be 0 which is also not a prime.
Thus, we found $n$ for which the statement is false which
proves that the initial statement is indeed false.
\begin{flushright}
\textit{Q.E.D.}
\end{flushright}
\end{quote}

\item[16.]
Prove the following statement is false by providing a counterexample:
If S and T are shifty sets (in the sense of a previous exercise), then $S \cap T$ is also a shifty set.
\begin{quote}
Definition of the shifty set: A subset $S$ of $\mathbb{Z}$ is called \textit{shifty} if for every $x \in S, \ x-1 \in S \ \textnormal{or} \ x + 1 \in S$.\\

Suppose $S$ and $T$ and shifty. Let $S = \{1,2,4,5\}$ and $T = \{2,3,6,7\}$.
Then $S \cap T = \{2\}$ and $1, 3 \notin S \cap T$ so we found an example for which
$S$ and $T$ are shifty, but $S \cap T$ is not.
\begin{flushright}
\textit{Q.E.D.}
\end{flushright}
\end{quote}

\item[17.]
Prove that if $x$ is odd, then $x^3$ is odd.

\begin{quote}
Suppose $x$ is odd. Then, by the definition of an odd number, we have:
$$x = 2k + 1, \ \textnormal{where} \ k \in \mathbb{Z}$$
Now, we can plug $2k + 1$ into $x^3$. We get:
$$
x = (2k + 1)^3 = 8k^3 + 12k^2 + 6k + 1 = 2 \times (4k^3 + 6k^2 + 3k) + 1
$$

Let's introduce a new variable $l$ and set it equal to $(4k^3 + 6k^2 + 3k)$.
Then we can rewrite $x$ as $x = 2l + 1$. Finally, we conclude that since
$(4k^3 + 6k^2 + 3k) \in \mathbb{Z}$, $l \in \mathbb{Z}$ and $2l + 1$ is odd which
means that $x$ is also odd.
\end{quote}

\item[18.]
Suppose that $m$ and $n$ are doubly even (in the sense of an earlier exercise):
\begin{quote}
Definition of the \textit{doubly even} integer: An integer $n$ is called doubly even if there exist even integers $x$ and $y$ such that $n = xy$.\\

\begin{itemize}
\item[a.]
Prove that $mn$ is doubly even.
\begin{quote}
Suppose $m$ and $n$ be doubly even integers, then we know that both of them are even (as they both are the multiplications of even integers).
Thus, according to the definition, $mn$ is doubly even since it is the multiplication of two even integers.
\end{quote}

\item[b.]
Prove that $m + n$ is doubly even
\begin{quote}
Suppose $m$ and $n$ be doubly even integers, then we know that both of them are even (as they both are the multiplications of even integers).
Let $m = 2a \times 2b$ and $n = 2c \times 2d$, where $a,b,c,d \in \mathbb{Z}$. Then $m + n = 4ab + 4cd = 2 \times (2ab + 2cd)$. Hence, we managed to represent
$m + n$ as the multiplication of two even integers ($2$ and $2ab + 2cd$) and this proves that $m + n$ is doubly even.
\end{quote}
\end{itemize}
\begin{flushright}
\textit{Q.E.D.}
\end{flushright}
\end{quote}

\item[19.]
Prove that if $m$ is even but not doubly even then $m + 2$ is doubly even.
\begin{quote}
Let's first analyze what it means to be even but not doubly even. As we know, an integer
is doubly even if and only if it can be represented as a multiplication of two even integers.
This means that doubly even integer is always divisible by 4. So what would be integers which are even
but not doubly even? Those are the integers which are divisible by 2 but not by 4 (e.g., 2,6,10 etc.).
Then we let $m = 4k + 2$, where $k \in \mathbb{Z}$. We get $m + 2 = 4k + 2 + 2 = 2 \times (2k + 2)$.
Hence, we represented $m + 2$ as a multiplication of two even integers which proves that $m + 2$ is
doubly even.
\begin{flushright}
\textit{Q.E.D.}
\end{flushright}
\end{quote}

\item[20.]
Prove or Disprove: if $A$ and $B$ are sets then there exists a set $C$ such that $A \cup B = A \cup C$.
\begin{quote}
It is right so let's prove it.
Suppose $C = B$, then we can rewrite $A \cup C$ as $A \cup B$ and we get $A \cup B = A \cup B$.
We know that $A \cup B$ is a set itself so let's set it equal to $D$. Finally, we get $D = D$ which
is true.
\end{quote}
\begin{flushright}
\textit{Q.E.D.}
\end{flushright}

\item[21.]
What does Fermat’s Last Theorem say? How many years passed, roughly, from when
Fermat originally wrote down his Last Theorem and when it was proved?
\begin{quote}
Fermat's Last Theorem implies that $x^y + y^n = z^n$ where $n > 2$ has no integer solutions.
It was proven after roughly 350 years.
\end{quote}

\item[22.]
Given that he very rarely wrote them down, Fermat clearly did not have the reverence
for proofs that ideal mathematician from our recent reading did. Based on what you
learned about Fermat from the video, why do you think that is?
\begin{quote}
Fermat was not really a mathematician, he was a judge with the hobby of solving math problems.
Generally speaking, mathematicians tend to be very respectful towards proofs (most of
them at the very least).
\end{quote}

\newpage
{\Large Bookwork}
\item[1.]
Prove: The sum of two even integers is an even integer.
\begin{quote}
Suppose $x, y \in \mathbb{Z}$ are even integers. Then, by the definition of even
numbers, we can say that $x = 2k$ and $y = 2l$ where $k, l \in \mathbb{Z}$.
Then we have: $x + y = 2k + 2l = 2 \times (k + l)$. Now, let $m = k + l$.
Finally, we get $x + y = 2k + 2l = 2 \times (k + l) = 2m$ where $m \in \mathbb{Z}$
and once again, by the definition of even numbers, $2m$ is even which makes $x + y$ even as well.
Thus, we proved that the sum of two even integers is an even integer.
\begin{flushright}
\textit{Q.E.D.}
\end{flushright}
\end{quote}

\item[5.]
Prove: if $a, \ b$, and $c$ are integers such that $a$ is a factor of $b$ and $b$ is a factor of $c$,
then $a$ is a factor of $c$.
\begin{quote}
Suppose $a, \ b$, and $c$ are integers such that $a$ is a factor of $b$ and $b$ is a factor of $c$.
Then, we can represent $b$ and $c$ as follows:
$$
b = ak \ \textnormal{where} \ k \in \mathbb{Z}
$$
$$
c = bl \ \textnormal{where} \ l \in \mathbb{Z}
$$
Now, let's substitute $b$ in the second equation. We get:
$$
c = bl = ak \times bl = akbl = a \times (kbl)
$$
Thus, we have $c = a \times (kbl)$ which shows that $a$ is a factor of $c$.
\begin{flushright}
\textit{Q.E.D.}
\end{flushright}
\end{quote}

\item[10.]
Prove: If $A \subseteq B$ and $A \subseteq C$, then $A \subseteq B \cap C$.
\begin{quote}
Suppose $A \subseteq B$ and $A \subseteq C$. We can then conclude that $B$ and $C$ share
at least those elements which are in $A$. Thus, $B \cap C \supseteq A$ from which we get
$A \subseteq B \cap C$.
\begin{flushright}
\textit{Q.E.D.}
\end{flushright}
\end{quote}

\item[19.]

\begin{itemize}
\item[(a)]
Prove that if $a$ is an integer of the form $3n + 1$ for some integer $n$, then 3 is a factor
of $a^2 - 1$.
\begin{quote}
Let $a$ be an integer of the form $3n + 1$. Then we can substitute $a$ with $3n + 1$ and we get:
$$
a^2 - 1 = (3n + 1)^2 - 1 = 9n^2 + 6n + 1 - 1 = 9n^2 + 6n = 3 \times (3n^2 + 2n)
$$
Finally, we have that $a^2 - 1 = 3 \times (3n^2 + 2n)$ which makes $a^2 - 1$ a multiple of 3.
Thus, if $a$ is an integer of the form $3n + 1$ for some integer $n$, then 3 is a factor
of $a^2 - 1$.
\begin{flushright}
\textit{Q.E.D.}
\end{flushright}
\end{quote}
\end{itemize}
\end{enumerate}

\end{document}
