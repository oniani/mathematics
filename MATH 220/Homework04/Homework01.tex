\documentclass[12pt, a4paper]{article}             % use "amsart" instead of "article" for AMSLaTeX format
\usepackage[a4paper,margin=1in]{geometry}          % Adjust margins
\usepackage{amsmath,amssymb,listings,color}        % Math packages: amsmath, amssymb, listings, color
\usepackage[makeroom]{cancel}

\title{\bf{Getting Started with \LaTeX}}
\author{Author: David Oniani
\\
\ \ \ Instructor: Tommy Occhipinti}
\date{August 29, 2018}

\usepackage{listings}
\usepackage{color}

\definecolor{dkgreen}{rgb}{0,0.6,0}
\definecolor{gray}{rgb}{0.5,0.5,0.5}
\definecolor{mauve}{rgb}{0.58,0,0.82}
\definecolor{backcolour}{rgb}{0.95,0.95,0.92}

\lstset{
backgroundcolor=\color{backcolour},
aboveskip=3mm,
belowskip=3mm,
showstringspaces=false,
columns=flexible,
basicstyle={\small\ttfamily},
numbers=left,
numberstyle=\normalsize\color{gray},
keywordstyle=\color{blue},
commentstyle=\color{dkgreen},
stringstyle=\color{mauve},
breaklines=true,
breakatwhitespace=true,
tabsize=4
}


\begin{document}
\maketitle


\begin{enumerate}
\item[15.]
Prove the following statement is false by providing a counterexample: If $n \in \mathbb{Z}^+$ is odd
and $n > 1$ then there exists a non-negative integer $i$ and a prime $p$ such that $n = 2^i + p$.
\\

\begin{quote}
This is the type of statement where "for every" part is hidden. In other words,
the statement could be translated in the following way: "For every $n > 1 \in \mathbb{Z}^+$
there exists a non-negative integer $i$ and a prime $p$ such that
$n = 2^i + p$".
\end{quote}

Thus, all we have to do is find an integer $n > 1$, $i \in \mathbb{Z}^+$, and $p \in \mathbb{P}$ where $\mathbb{P}$ is a set
of prime numbers. Now, notice that $n = 2$ is a counterexample. If $n = 2$, then $i$ has to be either 0 or 1 (since $i$ is non-negative and
if $i > 1$, $2^i > 2$ and the equality will not hold).
Hence, for $n = 2$, we have $i = 0$ or $i = 1$. Let's only consider the first example.
If $n = 2$ and $i = 0$, $2^i = 1$ and $p$ has to be 1 which is not prime. Thus, we found $n$ for which the statement is false which
proves that the initial statement is indeed false.
\begin{flushright}
\textit{Q.E.D.}
\end{flushright}

\item[16.]
Prove the following statement is false by providing a counterexample:
If S and T are shifty sets (in the sense of a previous exercise), then $S \cap T$ is also a shifty set.
\begin{quote}
Definition of the shifty set: A subset $S$ of $\mathbb{Z}$ is called \textit{shifty} if for every $x \in S, \ x-1 \in S \ \textnormal{or} \ x + 1 \in S$.\\

Suppose $S$ and $T$ and shifty. Let $S = \{1,2,4,5\}$ and $T = \{2,3,6,7\}$.
Then $S \cap T = \{2\}$ and $1, 3 \notin S \cap T$ so we found an example for which
$S$ and $T$ are shifty, but $S \cap T$ is not.
\begin{flushright}
\textit{Q.E.D.}
\end{flushright}
\end{quote}

\item[17.]
Prove that if $x$ is odd, then $x^3$ is odd.

\begin{quote}
Suppose $x$ is odd. Then, by the definition of an odd number, we have:
$$x = 2k + 1, \ \textnormal{where} \ k \in \mathbb{Z}$$
Now, we can plug $2k + 1$ into $x^3$. We get:
$$
x = (2k + 1)^3 = 8k^3 + 12k^2 + 6k + 1 = 2 \times (4k^3 + 6k^2 + 3k) + 1
$$

Let's introduce a new variable $l$ and set it equal to $(4k^3 + 6k^2 + 3k)$.
Then we can rewrite $x$ as $x = 2l + 1$. Finally, we conclude that since
$(4k^3 + 6k^2 + 3k) \in \mathbb{Z}$, $l \in \mathbb{Z}$ and $2l + 1$ is odd which
means that $x$ is also odd.
\end{quote}

\item[18.]
Suppose that $m$ and $n$ are doubly even (in the sense of an earlier exercise):
\begin{quote}
Definition of the \textit{doubly even} integer: An integer $n$ is called doubly even if there exist even integers $x$ and $y$ such that $n = xy$.\\

\begin{itemize}
\item[a.]
Prove that $mn$ is doubly even.
\begin{quote}
According to the definition, an integer $k$ is doubly even if there exist even integers $m$ and $n$ such that
$k = mn$. Then we can write that 
\end{quote}
\end{itemize}

\begin{flushright}
\textit{Q.E.D.}
\end{flushright}
\end{quote}

\end{enumerate}


\end{document}
