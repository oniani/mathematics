\documentclass[12pt, a4paper]{article}                      % use "amsart" instead of "article" for AMSLaTeX format
\usepackage[a4paper,margin=1in]{geometry}                   % Adjust margins
\usepackage{amsmath,amssymb,listings,color,textcomp,adjustbox,tikz,pgfplots}        % Math packages: amsmath, amssymb, listings, color
\usepgfplotslibrary{external,fillbetween}
\pgfplotsset{compat=1.14}
\usepackage[makeroom]{cancel}
\title{\bf{Homework \textnumero 8}}
\author{Author: David Oniani
\\
\ \ \ Instructor: Tommy Occhipinti}
\date{October 20, 2018}

\usepackage{listings}
\usepackage{color}

\newcommand{\natn}{\mathbb{N}}
\newcommand{\intz}{\mathbb{Z}}
\newcommand{\intzp}{\mathbb{Z^+}}
\newcommand{\intzn}{\mathbb{Z^-}}

\definecolor{dkgreen}{rgb}{0,0.6,0}
\definecolor{gray}{rgb}{0.5,0.5,0.5}
\definecolor{mauve}{rgb}{0.58,0,0.82}
\definecolor{backcolour}{rgb}{0.95,0.95,0.92}

\lstset{
backgroundcolor=\color{backcolour},
aboveskip=3mm,
belowskip=3mm,
showstringspaces=false,
columns=flexible,
basicstyle={\small\ttfamily},
numbers=left,
numberstyle=\normalsize\color{gray},
keywordstyle=\color{blue},
commentstyle=\color{dkgreen},
stringstyle=\color{mauve},
breaklines=true,
breakatwhitespace=true,
tabsize=4
}


\begin{document}
\maketitle

{\Large Additional Proof Practice}
\\\\
\begin{itemize}
\item[53.]
A subset $S$ of $\mathbb{Z}^{+}$ is called a $P_3$\textbf{-set} if there exists (not necessarily distinct) elements
$x, y, z \in S$ such that $x + y + z$ is prime.

\begin{itemize}
\item[(a)]
Give some examples of $P_3$-sets.
\begin{quote}
$\{1\}$ because 1 + 1 + 1 = 3 is a prime.\\
$\{2, 3\}$ because 2 + 3 = 5 is a prime.\\
$\{12, 25, 30\}$ because 12 + 25 + 30 = 67 is a prime.
\end{quote}

\item[]

\item[(b)]
Prove or Disprove: If $A$ is a $P_3$-set, and $A \subseteq B \subseteq \mathbb{Z}^+$, then $B$ is a $P_3$-set.
\begin{quote}
It's right so let's prove it.
Since we know that $A$ is a $P_3$-set, we know there exist elements (not necessarily distinct) $x,y$ and $z$ such that
$x + y + z$ is prime.\ Since $A \subseteq B$, we know that all elements of $A$ are also in $B$ meaning that $x,y$ and $z$
are in $B$ as well and there exist elements $x,y,z$ (which are also in $A$) such that $x + y + z$ is a prime.
\begin{flushright}
\textit{Q.E.D.}
\end{flushright}
\end{quote}

\item[]

\item[(c)]
Prove or Disprove: If $S$ is a $P_3$-set, then so is $S_{+3} := \{x + 3 \mid x \in S\}$.
\begin{quote}
It's false. Counterexample: Let $S = \{1\}$, then we know that $S$ is a $P_3$-set since $1 + 1 + 1 = 3$
is a prime. However, $S_{+3} = \{4\}$ and $4 + 4 + 4 = 12$ which is certainly composite.
\end{quote}

\item[]

\item[(d)]
Prove or Disprove: Every $P_3$ set contains a prime.
\begin{quote}
False. Let $S = \{1\}$, then 1 is not a prime but $1 + 1 + 1 = 3$ is a prime.
\end{quote}

\item[]

\item[(e)]
Prove or Disprove: The intersection of two $P_3$-sets is a $P_3$-set.
\begin{quote}
It's false. Let $A = \{1\}$, then $A$ is $P_3$-set since 1 + 1 + 1 = 3 is a prime.
Let $B = \{2, 3\}$, then $2 + 3 = 5$ is a prime thus $B$ is also $P_3$-set. $A \cap B = \emptyset$
which means that there are no elements $x,y,z$ such that $x + y + z$ is a prime and thus,
the intersection of two $P_3$-sets is not necessarily a $P_3$-set.
\end{quote} 

\item[]

\item[(f)]
Prove or Disprove: Every $P_3$-set contains an odd integer.
\begin{quote}
It's true. Suppose, for the sake of contradiction, that $S$ is a $P_3$-set and it does not contain any odd integers. Thus, it means
that all the elements of $S$ are even. Now, since all the elements are even, it means that no matter
what 3 elements $x, y$ and $z$ we take, their sum will always be even. On the other hand, the only even
prime we have is 2. But unfortunately, there are no three numbers $x, y, z \in \mathbb{Z}^+$ which sum up to 2. The best we can do is 1 + 1 + 1
which is 3 and is one more than 2. Thus, there is no way to get 2 and 
otherwise, we won't have 3 elements which sum up to the prime. Hence,
we have reached the contradiction and $S$ is a not a $P_3$-set.
\begin{flushright}
\textit{Q.E.D.}
\end{flushright}
\end{quote}

\item[]

\item[(g)]
Prove or Disprove: Every infinite subset $S$ of $\mathbb{Z}^+$ is a $P_3$-set.
\begin{quote}
It's false. Since we already proved that every $P_3$-set contains an odd integer,
we can take a set of all positive even integers which is a subset of $\mathbb{Z}^+$. Let's call this subset $E$. Then, we know that every element of the subset $E$ is even and sum of any 3 elements (not necessarily distinct) will also be even. However, once again, the only even integer which is a prime is 2 and we cannot get 2 by summing 3 integers which are greater than or equal to 2 (greater than equal because $E = \{2, 4, 6, 8, 10 ...\}$).
\end{quote}

\item[]

\item[(h)]
Prove or Disprove: If $S$ is a finite subset of $\mathbb{Z}^+$, then $\mathbb{Z}^+ - S$ is a $P_3$-set.
\begin{quote}
It's true. Let's prove it. Since $S$ is a finite set, we know that it cannot contain all the elements of $\mathbb{Z}^+$ because $\mathbb{Z}^+$ is infinite.\ We already proved that there are infinitely many primes.\ Then we can find a prime $p$ such that $p - 2 \notin S$.
then, we can have a set $L = \{1, p - 2\}$ which is a $P_3$-set because $1 + 1 + p - 2 = p$ is a prime.
\begin{flushright}
\textit{Q.E.D.}
\end{flushright}
\end{quote}
\end{itemize}

\item[]
\item[]

\item[54.]
\begin{itemize}
If a subset $S$ of $\mathbb{Z}^+$ is a $P_3$-set then the \textbf{core} of $S$ is the set
$$\mbox{core}(S) := \{s \in S \mid S - \{s\} \mbox{ is not a $P_3$-set}\}.$$

\item[]

\item[(a)]
What is the core of $S = \{2, 3, 6\}$?
\begin{quote}
The core of $S = \{2, 3, 6\}$ is core$(S) = \{2, 3\}$.
The reason is that if we take out $2$, we are left with 3 and 6 which are both multiples of 3 and any variations of their sums will never be a prime (3 + 3 + 3 is a not a prime, 3 + 6 + 3 is not a prime etc.).
If we take out 3, we are left with two even numbers, namely 2 and 6, and still we know that every $P_3$-set contains an odd integer thus, taking out 3 will leave us with non-$P_3$-set (any variations of the sums of the even integers will not be even; the only case is 2 but 2 + 2 + 2 = 6 is the best we can do). On the other hand, if we take out 6, $S$ will still be a $P_3$-set since 2 + 2 + 3 = 7 is a prime.
Thus, core$(S) = \{2, 3\}$.
\end{quote}

\item[]

\item[(b)]
Give an example of a $P_3$-set whose core is the empty set, or prove none exists.
\begin{quote}
Here is an example: let $S = \{1\}$, then $S$ is a $P_3$ set since 1 + 1 + 1 = 3 is a prime. However, if we take out 1, we have $S = \emptyset$
and there are no $x, y, z$ such that $x + y + z$ is a prime.
\end{quote}

\item[]

\item[(c)]
Give an example of a $P_3$-set whose core is infinite, or prove none exists.
\begin{quote}
There is no such $P_3$-set. Let's prove it. Suppose, for the sake of contradiction, there exists a $P_3$ set $S$ such that its core is infinite. Let the core be the set $C = \{c_0, c_1, c_2, c_3 \ ...\}$.
Then, we know that if we took out $c_0$, the set $S$ would not be $P_3$.
Then, since $c_0$ affected the outcome of whether $S$ is a $P_3$-set or not, it means that $c_0$ plays a role in $x + y + z$. Same goes if we took out $c_1$. The same for $c_2$, and the same for $c_4, c_5...$ etc.
However, since $c_0, c_1$ and $c_2$ play a role in the sum, we know that there are at most 3 different elements of the set in the sum as by the definition $x + y + z$ must be a prime. But here we see that infinitely many elements are in this sum and we reached a contradiction.
Thus, there are no $P_3$-set whose core is finite.
\end{quote}

\item[]

\item[(d)]
Prove or Disprove: If $S$ is a $P_3$-set then core$(S)$ is a $P_3$-set. 
\begin{quote}
It's false. Counterexample: let $S = \{2, 3, 5\}$. Then core$(S) = \{3\}$ since if we take out 2, 3 + 3 + 5 = 11 is still a prime and if we take out 5, 2 + 2 + 3 = 7 is still a prime. However, if we take out 3, all the possible sums are:\\
2 + 2 + 2 = 6 is not a prime\\
2 + 2 + 5 = 9 is not a prime\\
2 + 5 + 5 = 12 is not a prime\\
5 + 5 + 5 = 15 is not a prime\\

Thus, we end up with a set core$(S) = \{3\}$ which is not a $P_3$-set since 3 + 3 + 3 = 9 is the only sum we can get and it is not a prime.
\end{quote}

\item[]

\item[(e)]
Prove or Disprove: If $S$ and $T$ are $P_3$-sets then core$(S \cup T) \subseteq \mbox{ core}(S) \cap \mbox{ core}(T)$.
\begin{quote}
It's true, let's prove it. Suppose, for the sake of contradiction, that for some two $P_3$-sets $S_1$ and $S_2$,
$\mbox{ core}(S_1) \ \cap \mbox{core}(S_2) \subset  \mbox{core}(S_1 \cup S_2)$. Then we know that there exists $x \in$ core$(S_1 \cup S_2)$ such that $x \notin \mbox{core}(S_1) \ \cap \mbox{core}(S_2)$. Now, since $x \in \mbox{core}(S_1) \cap \mbox{ core}(S_2)$, it means that $x \in \mbox{core}(S_1)$ and $x \in \mbox{core}(S_2)$.
\end{quote} 

\item[]
\item[(f)]
Prove or Disprove: If $S$ and $T$ are $P_3$-sets with $S \subseteq T$ then we have $\mbox{core}(T) \subseteq \mbox{ core}(S)$.
\begin{quote}
Suppose $S_1$ and $S_2$ are two $P_3$-sets and $S_1 \subseteq S_2$. Then for all $x \in S_1$, $x \in S_2$.
If there are $k$ elements in core$(S_1)$, it means that core$(S_2)$ will 
\end{quote}
\end{itemize}

\item[]
\item[]

\item[55.]
A subset $S$ of $\mathbb{Z}$ is called threequaline if for every $x, y \in S$ one has $3 \mid (x - y)$.
\begin{itemize}
\item[(a)]
Prove or Disprove: Every subset of a threequaline set is threequaline.
\begin{quote}
It's false. Let, for the sake of contradiction, that $S$ is a threequaline set and also suppose that all the subsets of $S$ are threequaline. Then we know that an empty set is a subset of every set and $S$ is also a set thus, the emptyset is also a subset of $S$. However, an empty set has no elements and we cannot find $x, y$ such that $x - y$ is a multiple of 3 and we reached the contradiction.
\begin{flushright}
\textit{Q.E.D.}
\end{flushright}
\end{quote}

\item[]

\item[(b)]
Prove that if $S$ is threequaline than either every element of $S$ is divisible by 3 or none are.
\begin{quote}
Suppose, for the sake of contradiction, that $S$ is a threequaline set and there exists two elements $x, y$ such that $x$ is a multiple of 3 and $y$ is not a multiple of 3. Then $x - y$ will not be a multiple of 3 and we have reached the contradiction.
\begin{flushright}
\textit{Q.E.D.}
\end{flushright}
\end{quote}

\item[]

\item[(c)]
Prove that if $S$ is threequaline and $r$ and $t$ are integers, then the set $\{rx + t \mid x \in S\}$ is also threequaline.
\begin{quote}
Since we know that $S$ is a threequaline set, for every $x, y \in S$, $x - y$ is a multiple of 3. Suppose, $z_1, z_2$ are some elements of the set $S$. Then, we know that $z_1 - z_2$ is a multiple of 3. The new set will "transform" these elements into $rz_1 + t$ and $rz_2 + t$. On the other hand, $z_1 - z_2 = rz_1 + t - (rz_2 + t) = r(z_1 - z_2)$ which is a multiple of 3 since $z_1, z_2$ are the members of $S$ and $z_1 - z_2$ is a multiple of 3.
\begin{flushright}
\textit{Q.E.D.}
\end{flushright}
\end{quote}

\item[]

\item[(d)]
Prove that if $S$ and $T$ are threequaline and $S \ \cap \ T \neq \emptyset$ then $S \ \cup \ T$ is threequaline.
\begin{quote}
Suppose, for the sake of contradiction, that $S_1$ and $S_2$ are threequaline sets and $S_1 \ \cap \ S_2 \neq \emptyset$ and let's prove that
$S_1 \ \cup \ S_2$ is not a threequaline. Since $S_1 \cap S_2 \neq 0$, there exists element $x$, such that $x \in S_1$ and $x \in S_2$.
Let's consider the following cases:\\\\
Case I : $x$ is divisible by 3, thus $x = 3k$ where $k \in \mathbb{Z}$\\
Case II: $x$ gives remainder of 1 when divided by 3, thus $x = 3k + 1$ where $k \in \mathbb{Z}$\\
Case III: $x$ gives remainder of 2 when divided by 3, thus $x = 3k + 2$ where $k \in \mathbb{Z}$\\\\
In Case I, if $x = 3k$, then all the other elements of $S_1$ as well as $S_2$ must be of the type $3l$ where $l \in \mathbb{Z}$
and all the elements in the union of $S_1$ and $S_2$ will be the multiples of 3 which means that for all $i, j \in S_1 \cup S_2$,
$i - j$ is a multiple of 3. And we reached the contradiction.\\\\
In Case II, if $x = 3k + 1$, then all the other elements of $S_1$ as well as $S_2$ must be of the type $3l + 1$ where $l \in \mathbb{Z}$
and all the elements in the union of $S_1$ and $S_2$ will be the multiples of 3 plus 1 which means that for all $i, j \in S_1 \cup S_2$,
$i - j$ is a multiple of 3. And we reached the contradiction.\\\\
In Case II, if $x = 3k + 2$, then all the other elements of $S_1$ as well as $S_2$ must be of the type $3l + 2$ where $l \in \mathbb{Z}$
and all the elements in the union of $S_1$ and $S_2$ will be the multiples of 3 plus 2 which means that for all $i, j \in S_1 \cup S_2$,
$i - j$ is a multiple of 3. And we reached the contradiction.
\end{quote}
\end{itemize}

\item[]
\item[]

\item[56.]
A subset $S$ of $\mathbb{R}$ is called \textbf{crunched} if there exist integers $m, n \in \mathbb{Z}$ such that for all $x \in S$ we have $m < x < n$.\\\\
NOTE: When I mention lower bound or upper bound, I really mean the smallest element or the biggest element of the set.
\begin{itemize}
\item[]

\item[(a)]
Give some examples of sets that are and are not crunched.
\begin{quote}
$S_1 = \{1\}$ is crunched as for all $x \in S$, $0 < x < 2$ ($m = 0, n = 2$).\\
$S_2 = \{1, 2, 3\}$ is crunched as for all $x \in S$, $0 < x < 4$ ($m = 0, n = 4$).\\
$S_3 = \mathbb{Z^+}$ is not crunched as it has no bounds and we cannot find $m, n$ such that for all $x \in \mathbb{Z^+}, m< x < n$.\\
$S_4 = \{2,4,6 ...\}$ (a set of positive even numbers) is not crunched as it has no upper bound and we cannot find $n$ such that for all $x \in \mathbb{Z^+}, m< x < n$ (note: we can find $m$. $m$ can be any integer that is less than or equal to 1 but $n$ cannot be fixed).
\end{quote}

\item[]

\item[(b)]
Prove or Disprove: All crunched sets are finite.
\begin{quote}
That's false. Counterexample: let $S = \{1, 1/2, 1/4, ...\}$ (infinite geometric series), then we know that $S$ has an upper bound 1 and the lower bound which is 0. Then, we can say with the great certainty, that for all $x \in S$, $-10 < x < 10$ ($m = -10$, $n = 10$).
Thus, crunched sets are not necessarily finite and the initial claim is false.
\end{quote} 

\item[]

\item[(c)]
Prove or Disprove: All finite sets are crunched.
\begin{quote}
It's true. Let $S$ be a finite set. Then it must have a lower bound (the smallest element), let it be $k_1$ and the upper bound (the biggest element), let it be $k_2$. Then let $m = k_1 - 1$ and let $n = k_2 + 1$ and we have that for all $x \in S$, $m < x < n$.
\begin{flushright}
\textit{Q.E.D.}
\end{flushright}
\end{quote}

\item[]

\item[(d)]
Prove or Disprove: Every subset of a crunched set is crunched
\begin{quote}
It's true. Suppose $S$ is a crunched set. Then we know that for all $x \in S$, there exist $m, n$ such that $m < x < n$. Let $S_0$ be a subset of $S$. Then, since all the elements of $S_0$ are also in $S_1$, we know that all elements of $S_0$ are between $m$ and $n$ which makes $S_0$ crunched. Thus, every subset of a crunched set is crunched.
\begin{flushright}
\textit{Q.E.D.}
\end{flushright}
\end{quote}

\item[]

\item[(e)]
Prove or Disprove: The union of two crunched sets is crunched.
\begin{quote}
Suppose $S_1$ and $S_2$ are two crunched sets.
Then we know that for all $x_1 \in S_1$, $m_1 < x_1 < n_1$ and for all $x_2 \in S_2$, $m_2 < x_1 < n_2$. Then, for all $z$ in $S_1 \cup S_2, max(m_1, m_2) < z < max(n_1, n_2)$ which means that $S_1 \cup S_2$ is crunched.
\begin{flushright}
\textit{Q.E.D.}
\end{flushright}
\end{quote}
\end{itemize}

\item[]
\item[]

\item[57.]
We call a finite subset $S$ of $\mathbb{Z}$ \textbf{balanced} if
$|\mathbb{Z}^+ \cap S| = |\mathbb{Z}^- \cap S|.$ (Recall that $\mathbb{Z}^- = \{-1, -2, -3, ...\})$.
\\\\
NOTE: This statement really means that if $A$ is balanced, the number of positive elements in it equals
the number of negative elements in it.
\begin{itemize}
\item[]
\item[(a)]
Prove or Disprove: If $A$ is a balanced set then so is $A \cup \{0\}$.
\begin{quote}
It's true. Let's prove it.\\
Suppose $A$ is a balanced set, then we know that $|\mathbb{Z}^+ \cap A| = |\mathbb{Z}^- \cap A|$.
Now, since $0 \notin \mathbb{Z}^+$ and $0 \notin \mathbb{Z}^-$, it means that
$\mathbb{Z}^+ \cap A = \mathbb{Z}^+ \cap (A \cup \{0\})$ and
$\mathbb{Z}^- \cap A = \mathbb{Z}^- \cap (A \cup \{0\})$
and finally, $|\mathbb{Z}^+ \cap A \cup \{0\}| = |\mathbb{Z}^- \cap (A \cup \{0\})|$
Thus, if $A$ is a balanced set then so is $A \cup \{0\}$.
\begin{flushright}
\textit{Q.E.D.}
\end{flushright}
\end{quote}

\item[]

\item[(b)]
Prove or Disprove: The union of two balanced sets is balanced.
\begin{quote}
It's false. Counterexample: let $S_1 = \{-1, 1\}$ and $S_2 = \{-1, 5\}$.
Then $S_1$ is balanced since $\mathbb{Z}^+ \cap S_1 = \{1\}$ and $\mathbb{Z}^- \cap S_1 = \{-1\}$ which
means that $|\mathbb{Z}^+ \cap S_1| = |\mathbb{Z}^- \cap S_1| = 1$. $S_2$ is balanced too since
$\mathbb{Z}^+ \cap S_2 = \{5\}$ and $\mathbb{Z}^- \cap S_2 = \{-1\}$ which means that
$|\mathbb{Z}^+ \cap S_2| = |\mathbb{Z}^- \cap S_2| = 1$. On the other hand, set $S_1 \cup S_2$ is not balanced
since $S_1 \cup S_2 = \{-1, 1, 5\}$ and $|\mathbb{Z}^+ \cap (S_1 \cup S_2)| = 2$ while $|\mathbb{Z}^- \cap (S_1 \cup S_2)| = 1$.
\end{quote}

\item[]

\item[(c)]
Prove or Disprove: The intersection of two balanced sets is balanced. 
\begin{quote}
It's false. Counterexample: let $S_1 = \{-1, 1\}$ and $S_2 = \{-1, 5\}$.
Then $S_1$ is balanced since $\mathbb{Z}^+ \cap S_1 = \{1\}$ and $\mathbb{Z}^- \cap S_1 = \{-1\}$ which
means that $|\mathbb{Z}^+ \cap S_1| = |\mathbb{Z}^- \cap S_1| = 1$. $S_2$ is balanced too since
$\mathbb{Z}^+ \cap S_2 = \{5\}$ and $\mathbb{Z}^- \cap S_2 = \{-1\}$ which means that
$|\mathbb{Z}^+ \cap S_2| = |\mathbb{Z}^- \cap S_2| = 1$. On the other hand, $S_1 \cap S_2$ is not balanced
since $S_1 \cap S_2 = \{-1\}$ and $|\mathbb{Z}^+ \cap (S_1 \cap S_2)| = \emptyset$ while $|\mathbb{Z}^- \cap (S_1 \cap S_2)| = 1$.
\end{quote}

\item[]

\item[(d)]
Prove or Disprove: For every $n \in \mathbb{Z}^+$ there exists a balanced set $S$ with exactly $n$ elements.
\begin{quote}
It's true. Let's prove it by construction. Let $S$ be a set and for all $n \in \mathbb{Z}^+$, dump in some elements to make it balanced.
If $n$ is even, then we take $n/2$ elements that are positive
and $n/2$ elements that are negative which will give us a balanced set since $|\mathbb{Z}^+ \cap S| = |\mathbb{Z}^- \cap S|$.
If $n$ is odd, then we can throw in $0$ and then take $(n - 1)/2$ positive elements and $(n - 1)/2$ negative elements.
This will guarantee that $|\mathbb{Z}^+ \cap S| = |\mathbb{Z}^- \cap S|$ since there will be exactly same number of positive and negative
elements while 0 is neither in $\mathbb{Z}^+$, nor in $\mathbb{Z}^-$.
\begin{flushright}
\textit{Q.E.D.}
\end{flushright}
\end{quote}

\item[]

\item[(e)]
If $A$ is a subset of $\mathbb{Z}$ we denote by $\overline{A}$ the set $\{-a \ | \ a \in A\}$. Prove or Disprove:
For every finite subset of $A$ of $\mathbb{Z}$, the set $A \cup \overline{A}$ is balanced.
\begin{quote}
It's true. Let's prove it by cases.\\\\
Case I: suppose that $A$ is a subset of $\mathbb{Z}$ such that it does not contain 0.\\
Case II: suppose that $A$ is a subset of $\mathbb{Z}$ such that it does contain 0.
\\\\\\
Proof of Case I: If $A$ is a subset of $\mathbb{Z}$ such that it does not contain 0, we know that for all $x \in A$,
we have $-x \in \overline{A}$. This means that the number of negative elements in $A \cup \overline{A}$
will be equal to the number of positive elements in $A \cup \overline{A}$ and because of this,
$|\mathbb{Z}^+ \cap (A \cup \overline{A})| = |\mathbb{Z}^- \cap (A \cup \overline{A})|$.\\

Proof of Case II: If $A$ is a subset of $\mathbb{Z}$ such that it does contain 0, for all $x \in A$,
we have $-x \in \overline{A}$. This means that if we took out 0 out of $A \cup \overline{A}$, the number
of positive and negative elements in $A \cup \overline{A}$ would be equal. On the other hand, 0 plays no role
in determining whether $|\mathbb{Z}^+ \cap (A \cup \overline{A})| = |\mathbb{Z}^- \cap (A \cup \overline{A})|$
or not because $0 \notin \mathbb{Z}^+$ and $0 \notin \mathbb{Z}^-$. Thus, $|\mathbb{Z}^+ \cap (A \cup \overline{A})| = |\mathbb{Z}^- \cap (A \cup \overline{A})|$.
\begin{flushright}
\textit{Q.E.D.}
\end{flushright}
\end{quote}

\item[]

\item[(f)]
Prove or Disprove: If $A$ is balanced and $|A|$ is odd, then $0 \in A$.
\begin{quote}
It's true so let's prove it. Suppose, for the sake of contradiction, that $A$ is balanced
and $|A|$ is odd, but $0 \notin A$. Now, since $|A|$ is odd, it means that there is no way
to have the number of positive elements be equal to the number of negative elements and we've
reached a contradiction.
\begin{flushright}
\textit{Q.E.D.}
\end{flushright}
\end{quote}
\end{itemize}

\item[]
\item[]

\item[58.]
We call a subset $S$ of $\mathbb{R}$ \textbf{positively scattered} if for every $x \in S$
there exists $y \in \mathbb{R}$ such that $y > x$ and $S \cap (x,y] = \emptyset$. 
\begin{itemize}
\item[(a)]
Is $\mathbb{Z}^+$ positively scattered? 
\begin{quote}
It is. Let's prove it by construction. If we take some element $x \in \mathbb{Z}^+$, then we know that it is positive.
If $y > x$, we know that $y$ is also positive and let $y = x + 0.1$. Then, we know that interval $(x, x + 0.1]$
contains no positive integers and thus, $\mathbb{Z}^+ \cap (x, x + 01] = \emptyset$. Hence, for all $e \in \mathbb{Z}$, we can
construct $(e, e+0.1]$ (there are no positive integers in this interval) and finally, $\mathbb{Z}^+$ is indeed positively\
scattered.
\begin{flushright}
\textit{Q.E.D.}
\end{flushright}
\end{quote}

\item[]

\item[(b)]
Is $[2, 3]$ positively scattered? 
\begin{quote}
It is not. Suppose, for the sake of contradiction, that for all $x \in [2, 3]$, there
exists $y$ such that $y > x$ and $[2, 3] \cap (x, y] = \emptyset$. Now, since the condition
holds for all $x \in [2, 3]$ and $[2, 3]$ is a closed interval, it should hold for 2 as well.
Then, let $x = 2$. If the condition holds for $x = 2$, it means that we can find $y > x$ such
that $[2, 3] \cap (2, y] = \emptyset$. Since $y > x$, $y = 2 + t$ where $t > 0$. Thus,
we have $[2, 3] \cap (2, 2 + t] = \emptyset$, but this is impossible and to see why, let's
consider two cases:\\\\
Case I: $t \geq 1$\\
Case I: $0 < t <  1$\\\\
If $t \geq 1$, it means that $[2, 3] \cap (2, 2 + t] = (2, 3]$ and we have reach the contradiction.\\\\
If $0 < t < 1$, it means that $[2, 3] \cap (2, 2 + t] = (2, 2 + t]$ and interval $(2, 2 + t)$ is obviously infinite. Thus, we have, once again, reached the contradiction.\\\\
Thus, in all cases we've reached the contradiction and $[2, 3]$ is not positively scattered.
\begin{flushright}
\textit{Q.E.D.}
\end{flushright}
\end{quote}

\item[]

\item[(c)]
Is $\{\dfrac{1}{n} \ | \ n \in \mathbb{Z}^+\}$ positively scattered?
\begin{quote}
It is. Let's prove it by construction. Let $x \in \{\dfrac{1}{n} \ | \ n \in \mathbb{Z}^+\}$, then
we know that $x = \dfrac{1}{k}$ where $k \in \mathbb{Z}^+$. Now, let's take $y = \dfrac{1}{k + 0.5}$.
Then, we know that $\{\dfrac{1}{n} \ | \ n \in \mathbb{Z}^+\} \cap (\dfrac{1}{k}, \dfrac{1}{k + 0.5}] = \emptyset$
because $n$ is always an integer and there are no integer denominators in the interval $(k, k + 0.5]$.
\begin{flushright}
\textit{Q.E.D.}
\end{flushright}
\end{quote}

\item[]

\item[(d)]
Is $\{\dfrac{1}{n} \ | \ n \in \mathbb{Z}^+\} \cup \{0\}$ positively scattered? 
\begin{quote}
It is not. Suppose, for the sake of contradiction, that $\{\dfrac{1}{n} \ | \ n \in \mathbb{Z}^+\} \cup \{0\}$
is scattered. Then for all $x \in \{\dfrac{1}{n} \ | \ n \in \mathbb{Z}^+\} \cup \{0\}$, there exist
$y > x$, such that $(\{\dfrac{1}{n} \ | \ n \in \mathbb{Z}^+\} \cup \{0\}) \cap (x,y] = \emptyset$.
If the condition holds for all $x \in \{\dfrac{1}{n} \ | \ n \in \mathbb{Z}^+\} \cup \{0\}$, it should hold
for $x = 0$ too ($x = 0$ is the member of the set as well as the set contains element 0).\\
If $x = 0$, there must exist $y > 0$ such that $(\{\dfrac{1}{n} \ | \ n \in \mathbb{Z}^+\} \cup \{0\}) \cap (0,y] = \emptyset$.
This, however, is impossible since the interval $(0, y]$ is infinite and we can always find $n$ for which
$\dfrac{1}{n} \in (0, y]$. Thus, we've reached the contradiction.
\begin{flushright}
\textit{Q.E.D.}
\end{flushright}
\end{quote}

\item[]

\item[(e)]
Prove or Disprove: A subset of a positively scattered set is positively scattered.
\begin{quote}

\end{quote}
\end{itemize}



\end{itemize} 

\end{document}
