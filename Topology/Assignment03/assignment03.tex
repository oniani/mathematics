\documentclass[12pt, a4paper]{article}
\usepackage[margin=0.7in]{geometry}

\usepackage[none]{hyphenat}

\usepackage{bookmark}
\usepackage{adjustbox}
\usepackage{mathtools}
\usepackage{amsmath}
\usepackage{amssymb}
\usepackage{amsthm}
\usepackage{amsfonts}
\usepackage{algorithmic}

\usepackage{pgfplots}
\pgfplotsset{compat=1.16}
\usepackage{tikz}
\usetikzlibrary{arrows}
\usepackage{graphicx}
\usepackage{pst-solides3d}
\usepackage{xcolor}
\usepackage{hyperref}
\usepackage[cache=false]{minted}

\usepackage{soul}
\usepackage{cite}
\usepackage{textcomp}

\usepackage{wasysym}

% Sets and related operations
\newcommand{\nats}{\mathbb{N}} % Natural numbers
\newcommand{\pnats}{\mathbb{N}^+} % Positive natural numbers

\newcommand{\ints}{\mathbb{Z}} % Integers
\newcommand{\pints}{\mathbb{Z}^+} % Positive integers
\newcommand{\nints}{\mathbb{Z}^-} % Negative integers

\newcommand{\rats}{\mathbb{Q}} % Rational numbers
\newcommand{\prats}{\mathbb{Q}^+} % Positive rational numbers
\newcommand{\nrats}{\mathbb{Q}^-} % Negative rational numbers

\newcommand{\reals}{\mathbb{R}} % Real numbers
\newcommand{\preals}{\mathbb{R}^+} % Positive real numbers
\newcommand{\nreals}{\mathbb{R}^-} % Negative real numbers

\newcommand{\irrats}{\mathbb{I}} % Irrational numbers


\newcommand{\pset}{\mathcal{P}} % Powerset
\newcommand{\card}{\abs} % Cardinality
\newcommand{\topology}{\mathcal{T}} % Topology
\newcommand{\basis}{\mathcal{B}} % Basis

% Calligraphy
\newcommand\und[1]{\underline{\smash{#1}}}

% Operators
\DeclarePairedDelimiter\abs{\lvert}{\rvert}
\DeclarePairedDelimiter\ceil{\lceil}{\rceil}
\DeclarePairedDelimiter\floor{\lfloor}{\rfloor}

% Other
\newcommand{\rarr}{\rightarrow}
\newcommand{\larr}{\leftarrow}

% Setting stuff
\setlength{\parindent}{0pt}


% Title
\title{\rule{\paperwidth - 150pt}{1pt}\textbf{\\\textit{Topology}\\}\rule{\paperwidth - 150pt}{1pt}}

\author
{
Author: David Oniani\\
Instructor: Dr. Eric Westlund
}

\date{January 18, 2019}


\begin{document}
\maketitle

\center{\Large Assignment \textnumero{3}}

\begin{itemize}
\item[]
\item[]
{\large \textbf{Section 18}}
\vspace{0.3cm}

% Begin here!
\item[2.]
Suppose that $f : X \rarr Y$ is continuous. If $x$ is
a limit point of of the subset $A$ of $X$, is it necessarily
true that $f(x)$ is a limit point of $f(A)$?
\begin{quote}
It is not. Consider the constant continuous function $f : \reals \rarr \reals : x \mapsto 0$.
Then $0$ is the limit point of $A$, however, $f(0) = 0$ is not a limit point
of $f(A) = \{0\}$ since there is no neighborhood of $0$ that intersects $\{0\}$ at point other than $0$.
\end{quote}

\item[]
\item[]

\item[5.]
Show that the subspace $(a, b)$ of $\reals$ is homeomorphic with $(0, 1)$
and the subspace $[a, b]$ of $\reals$ is homeomorphic with $[0, 1]$.
\begin{quote}
Recall that a homeomorphism is a bijective and continuous function whose inverse is also continuous.
Therefore, all we have to do here is to find bijective and continuous function(s) which would map $(a, b)$ to $(0, 1)$
in the first case and $[a, b]$ to $[0, 1]$ in the second case.
\newline
\newline
Let's first show that the subspace $(a, b)$ of $\reals$ is homeomorphic with $(0, 1)$.\\
\vspace{0.1cm}
Consider the function $f : (a, b) \rarr (0, 1) : x \mapsto \dfrac{x - a}{b - a}$.\
\vspace{0.1cm}
Then notice that it is both injective and surjective hence is a bijection.
Besides, it is also a continuous function (it can be verified using epsilon-delta definition of continuity).\
The inverse of $f$ is a function $f^{-1} : (0, 1) \rarr x \mapsto (a, b) : (b - a)x + a$ which
is obviously bijective and also continuous (once again, can be verified using epsilon-delta definition of continuity).
Finally, we have that the subspace $(a, b)$ of $\reals$ is homeomorphic with $(0, 1)$.$\qed$
\newline
\newline
Now, let's show that the subspace $[a, b]$ of $\reals$ is homeomorphic with $[0, 1]$.
Let's take the exact same function $f$ but let's reconstruct it in the way that it maps $[a, b]$
to $[0, 1]$. We have, $f : [a, b] \rarr [0, 1] : x \mapsto \dfrac{x - a}{b - a}$.
Once again, this is a continuous bijective function whose inverse is also continuous
and therefore the subspace $[a, b]$ of $\reals$ is homeomorphic with $[0, 1]$.$\qed$
\end{quote}

\item[]
\item[]
\item[]
{\large \textbf{Section 19}}
\vspace{0.3cm}

\item[3.]
Prove theorem 19.3.
\begin{quote}
\und{\textbf{Theorem 19.3}}
\vspace{0.2cm}
\newline
"Let $A_\alpha$ be a subspace of $X_\alpha$, for each $\alpha \in J$.
Then $\prod A_\alpha$ is a subspace of $\prod X_\alpha$ if both products are given the box topology,
or if both products are given the product topology.``
\newline
\newline
asdsad
\end{quote}


\end{itemize}
\end{document}
