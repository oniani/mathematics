\documentclass[12pt, a4paper]{article}
\usepackage[margin=0.7in]{geometry}

\usepackage[none]{hyphenat}

\usepackage{bookmark}
\usepackage{adjustbox}
\usepackage{mathtools}
\usepackage{amsmath}
\usepackage{amssymb}
\usepackage{amsthm}
\usepackage{amsfonts}
\usepackage{algorithmic}

\usepackage{pgfplots}
\pgfplotsset{compat=1.16}
\usepackage{tikz}
\usetikzlibrary{arrows}
\usepackage{graphicx}
\usepackage{pst-solides3d}
\usepackage{xcolor}
\usepackage{hyperref}
\usepackage[cache=false]{minted}

\usepackage{soul}
\usepackage{cite}
\usepackage{textcomp}

\usepackage{wasysym}

% Sets and related operations
\newcommand{\nats}{\mathbb{N}}         % Natural numbers
\newcommand{\pnats}{\mathbb{N}^+}      % Positive natural numbers

\newcommand{\ints}{\mathbb{Z}}         % Integers
\newcommand{\pints}{\mathbb{Z}^+}      % Positive integers
\newcommand{\nints}{\mathbb{Z}^-}      % Negative integers

\newcommand{\rats}{\mathbb{Q}}         % Rational numbers
\newcommand{\prats}{\mathbb{Q}^+}      % Positive rational numbers
\newcommand{\nrats}{\mathbb{Q}^-}      % Negative rational numbers

\newcommand{\reals}{\mathbb{R}}        % Real numbers
\newcommand{\preals}{\mathbb{R}^+}     % Positive real numbers
\newcommand{\nreals}{\mathbb{R}^-}     % Negative real numbers

\newcommand{\irrats}{\mathbb{I}}       % Irrational numbers


\newcommand{\pset}{\mathcal{P}}        % Powerset
\newcommand{\card}{\abs}               % Cardinality
\newcommand{\topology}{\mathcal{T}}    % Topology
\newcommand{\basis}{\mathcal{B}}       % Basis
\newcommand{\oldemptyset}{\emptyset}   % Old empty set
\renewcommand{\emptyset}{\varnothing}  % New and nice empty set

% Calligraphy
\newcommand\und[1]{\underline{\smash{#1}}}

% Operators
\DeclarePairedDelimiter\abs{\lvert}{\rvert}
\DeclarePairedDelimiter\ceil{\lceil}{\rceil}
\DeclarePairedDelimiter\floor{\lfloor}{\rfloor}

% Other
\newcommand{\rarr}{\rightarrow}
\newcommand{\larr}{\leftarrow}

% Setting stuff
\setlength{\parindent}{0pt}


% Title
\title{\rule{\paperwidth - 150pt}{1pt}\textbf{\\\textit{Topology}\\}\rule{\paperwidth - 150pt}{1pt}}

\author
{
Author: David Oniani\\
Instructor: Dr. Eric Westlund
}

\date{January 22, 2019}


\begin{document}
\maketitle

\center{\Large Assignment \textnumero{5}}

\begin{itemize}
\item[]
\item[]
{\large \textbf{Section 31}}
\vspace{0.3cm}

% Begin here!
\item[1.]
Show that if $X$ is regular, every pair of points of $X$ have neighborhoods whose
closures are disjoint.
\begin{quote}
Since $X$ is regular, by the definition,
$\forall x, y \in X \ \exists U, V \mbox{ with } x \in U, y \in V \mbox{ and } U \cap V = \emptyset \mbox{ (with $U$ and $V$ being open sets)}$.
Now, recall that $X$ is regular if and only if given a point $x$ of $X$ and a neighborhood $U$ of $x$,
there is a neighborhood $V$ of $x$ such that $\bar{V} \subset U$ (\textbf{Lemma 31.1 (a)}). Then, according
to \textbf{Lemma 31.1 (a)}, $\exists U^\prime, V^\prime \mbox{ such that } U^\prime \subset U \mbox{ and } V^\prime \subset V$.
Now, because $U \cap V = \emptyset$, it follows that $\bar{U} \cap \bar{V} = \emptyset$.$\qed$
\end{quote}

\item[]
\item[]

\item[2.]
Show that if $X$ is normal, every pair of disjoint closed sets have neighborhoods whose
closures are disjoint.
\begin{quote}
Let $A$ and $B$ be disjoint closed sets. Then by the definition of normality, $\exists U, V$\\
$\mbox{such that } U \cap V = \emptyset, A \subset U, \mbox{ and } B \subset V \mbox{ (with $U$ and $V$ being open sets)}$.
Now, recall that $X$ normal if and only if given a closed set $A$ and an open set $U$ containing
$A$, there is an open set $V$ containing $A$ such that $\bar{V} \subset U$ (\textbf{Lemma 31.1 (b)}).
Then it follows from \textbf{Lemma 31.1 (b)} that $\exists U^\prime, V^\prime \mbox{ with }
A \subset U^\prime \mbox{ and } B \subset V^\prime \mbox{ such that }$
$\bar{U^\prime} \subset U \mbox{ and } \bar{V^\prime} \subset V$. Now, since $U \cap V = \emptyset$,
we get $\bar{U^\prime} \cap \bar{V^\prime} = \emptyset$.$\qed$
\end{quote}

\newpage

\item[3.]
Show that every order topology is regular.
\begin{quote}
Suppose that $X$ is an ordered set. Let $x \in X$ and let $U = (a, b)$
be the neighborhood of $x$. Also, let $A = (a, x)$ and $B = (x, b)$.
Now, according to the \textbf{Lemma 31.1 (a)}, $X$ is regular if and only if given a point $x$ of $X$ and a neighborhood
$U$ of $x$, there is a neighborhood $V$ of $X$ such that $\bar{V} \subset U$.
Then it follows that we have the following four cases:
\begin{itemize}
\item[1.]
If $u \in A$ and $v \in B$, then $x \in (u, v) \subset \overline{(u, v)} \subset [u, v] \subset (a, b)$.
\item[2.]
If $A = B = \emptyset$, then $(a, b) = \{x\}$ is both open and closed (since every order topology is Hausdorff)
\item[3.]
If $A = \emptyset$ and $v \in B$, then $x \in (a, v) \subset [x, v) \subset \overline{[x, v)} \subset [x, v] \subset (a, b)$.
\item[4.]
If $B = \emptyset$ and $u \in A$, then $x \in (u, b) \subset (u, x] \subset \overline{(u, x]} \subset [u, x] \subset (a, b)$.
\end{itemize}
Finally, we have considered all the cases and have exhaustively shown that a closed subspace of a normal space is normal.$\qed$
\end{quote}


\item[]
\item[]
\item[]
{\large \textbf{Section 31}}
\vspace{0.3cm}

\item[1.]
Show that a closed subspace of a normal space is normal.
\begin{quote}
Suppose that $Y$ is a closed subspace of the normal space $X$.
Now, recall that every simply ordered set is a Hausdorff space in the
order topology; The product of two Hausdorff spaces is a Hausdorff space;
A subspace of a Hausdorff space is a Hausdorff space. (\textbf{Theorem 17.11}).
Then according to the \textbf{Theorem 17.11}, $Y$ is Hausdorff.
Now let $A$ and $B$ be disjoint closed subspaces of $Y$. Since $A$ and $B$ are
closed in $X$, they can be separated in $X$ by open sets $U$ and $V$ (with $U \cap V = \emptyset$).
Then $U \cap Y$ and $V \cap Y$ are open sets in $Y$ separating $A$ and $B$.
Therefore, $Y$ is normal.$\qed$
\end{quote}

\end{itemize}
\end{document}
