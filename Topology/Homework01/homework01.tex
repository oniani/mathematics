\documentclass[12pt, a4paper]{article}
\usepackage[margin=0.5in]{geometry}

\usepackage{bookmark}
\usepackage{adjustbox}
\usepackage{mathtools}
\usepackage{amsmath}
\usepackage{amssymb}
\usepackage{amsthm}
\usepackage{amsfonts}
\usepackage{algorithmic}

\usepackage{pgfplots}
\pgfplotsset{compat=1.16}
\usepackage{tikz}
\usepackage{graphicx}
\usepackage{pst-solides3d}

\usepackage{xcolor}
\usepackage{hyperref}
\usepackage[cache=false]{minted}

\usepackage{soul}
\usepackage{cite}
\usepackage{textcomp}

% Sets and related operations
\newcommand{\nats}{\mathbb{N}} % Natural numbers
\newcommand{\pnats}{\mathbb{N}^+} % Positive natural numbers

\newcommand{\ints}{\mathbb{Z}} % Integers
\newcommand{\pints}{\mathbb{Z}^+} % Positive integers
\newcommand{\nints}{\mathbb{Z}^-} % Negative integers

\newcommand{\rats}{\mathbb{Q}} % Rational numbers
\newcommand{\prats}{\mathbb{Q}^+} % Positive rational numbers
\newcommand{\nrats}{\mathbb{Q}^-} % Negative rational numbers

\newcommand{\reals}{\mathbb{R}} % Real numbers
\newcommand{\preals}{\mathbb{R}^+} % Positive real numbers
\newcommand{\nreals}{\mathbb{R}^-} % Negative real numbers

\newcommand{\irrats}{\mathbb{I}} % Irrational numbers


\newcommand{\pset}{\mathcal{P}} % Powerset
\newcommand{\card}{\abs} % Cardinality

% Calligraphy
\newcommand\und[1]{\underline{\smash{#1}}}

% Operators
\DeclarePairedDelimiter\abs{\lvert}{\rvert}
\DeclarePairedDelimiter\ceil{\lceil}{\rceil}
\DeclarePairedDelimiter\floor{\lfloor}{\rfloor}

% Other
\newcommand{\rarr}{\rightarrow}
\newcommand{\larr}{\leftarrow}

% Setting stuff
\setlength{\parindent}{0pt}


% Title
\title{\rule{\paperwidth - 150pt}{1pt}\textbf{\\\textit{Topology}\\}\rule{\paperwidth - 150pt}{1pt}}

\author
{
\textit{Author: David Oniani}
\\
\textit{Instructor: Dr. Eric Westlund}
}

\date{\textit{January }7\textit{, }2019}


\begin{document}
\maketitle

\center{\Large Exercises}

\begin{itemize}
\vspace{0.3cm}

% Begin here!

\item[2.]
Let $f : A \rarr B$ and let $A_i \subset A$ and $B_i \subset B$
for $i = 0$ and $i = 1$. Show that if $f^{-1}$ preserves inclusions,
unions, intersections, and differences of sets:

\begin{itemize}
\item[(c)]
$f^{-1}(B_0 \cap B_1) = f^{-1}(B_0) \cap f^{-1}(B_1)$.
\vspace{0.25cm}

\begin{quote}
To prove that the set $f^{-1}(B_0 \cap B_1)$
is equal to the set $f^{-1}(B_0) \cap f^{-1}(B_1)$,
we have to show that $f^{-1}(B_0 \cap B_1) \subseteq f^{-1}(B_0) \cap f^{-1}(B_1)$
and $f^{-1}(B_0) \cap f^{-1}(B_1) \subseteq f^{-1}(B_0 \cap B_1)$.
\\
\vspace{0.5cm}
Case I: $f^{-1}(B_0 \cap B_1) \subseteq f^{-1}(B_0) \cap f^{-1}(B_1)$\\
\vspace{0.15cm}

\begin{quote}
Let $x \in B_0 \cap B_1$. Then $x \in B_0$ and $x \in B_1$. Besides,
$f(x) = A_p$ where $A_p \subseteq A$ (the author calls it the preimage).
Now, since $x \in B_0 \cap B_1$, its preimage is in $f^{-1}(B_0 \cap B_1)$.
On the other hand, as $x \in B_0$, its preimage lies in $f^{-1}(B_0)$ and as
$x \in B_1$, its preimage also lies in $f^{-1}(B_1)$. In other words, the preimage of $x$
lies in $f^{-1}(B_0) \cap f^{-1}(B_1)$. Therefore, $f^{-1}(B_0 \cap B_1) \subseteq f^{-1}(B_0) \cap f^{-1}(B_1)$.
\end{quote}

\vspace{0.5cm}
Case II: $f^{-1}(B_0) \cap f^{-1}(B_1) \subseteq f^{-1}(B_0 \cap B_1)$\\
\vspace{0.15cm}

\begin{quote}
Let $x_0 \in B_0$ and $x_1 \in B_1$. Then the preimages of $x_0$ and $x_1$ are in
$f^{-1}(B_0)$ and $f^{-1}(B_1)$ respectively. Thus, $x$ mapping to $f^{-1}(B_0) \cap f^{-1}(B_1)$
has the preimage that maps to both $f^{-1}(B_0)$ and $f^{-1}(B_1)$. In other words,
$x \in B_0$ and $x \in B_1$ which means that the preimage of $x$ also lies in $f^{-1}(B_0 \cap B_1)$.
Therefore, $f^{-1}(B_0) \cap f^{-1}(B_1) \subseteq f^{-1}(B_0 \cap B_1)$.
\end{quote}
\vspace{0.5cm}

We have now proven that $f^{-1}(B_0 \cap B_1) \subseteq f^{-1}(B_0) \cap f^{-1}(B_1)$
and $f^{-1}(B_0) \cap f^{-1}(B_1) \subseteq f^{-1}(B_0 \cap B_1)$ and thus, $f^{-1}(B_0 \cap B_1) = f^{-1}(B_0) \cap f^{-1}(B_1)$.

\begin{flushright}
\textit{Q.E.D.}
\end{flushright}

\end{quote}

\item[(g)]
$f(A_0 \cap A_1) \subset f(A_0) \cap f(A_1)$; show that inequality holds if $f$ is injective.
\vspace{0.25cm}

\begin{quote}
Let $x \in A_0 \cap A_1$. Then $x \in A_0$ and $x \in A_1$.
\vspace{0.5cm}

sadasdasd

\begin{flushright}
\textit{Q.E.D.}
\end{flushright}

\end{quote}

\end{itemize}

\end{itemize}
\end{document}
