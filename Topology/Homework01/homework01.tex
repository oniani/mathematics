\documentclass[12pt, a4paper]{article}
\usepackage[margin=0.7in]{geometry}

\usepackage{bookmark}
\usepackage{adjustbox}
\usepackage{mathtools}
\usepackage{amsmath}
\usepackage{amssymb}
\usepackage{amsthm}
\usepackage{amsfonts}
\usepackage{algorithmic}

\usepackage{pgfplots}
\pgfplotsset{compat=1.16}
\usepackage{tikz}
\usepackage{graphicx}
\usepackage{pst-solides3d}

\usepackage{xcolor}
\usepackage{hyperref}
\usepackage[cache=false]{minted}

\usepackage{soul}
\usepackage{cite}
\usepackage{textcomp}

% Sets and related operations
\newcommand{\nats}{\mathbb{N}} % Natural numbers
\newcommand{\pnats}{\mathbb{N}^+} % Positive natural numbers

\newcommand{\ints}{\mathbb{Z}} % Integers
\newcommand{\pints}{\mathbb{Z}^+} % Positive integers
\newcommand{\nints}{\mathbb{Z}^-} % Negative integers

\newcommand{\rats}{\mathbb{Q}} % Rational numbers
\newcommand{\prats}{\mathbb{Q}^+} % Positive rational numbers
\newcommand{\nrats}{\mathbb{Q}^-} % Negative rational numbers

\newcommand{\reals}{\mathbb{R}} % Real numbers
\newcommand{\preals}{\mathbb{R}^+} % Positive real numbers
\newcommand{\nreals}{\mathbb{R}^-} % Negative real numbers

\newcommand{\irrats}{\mathbb{I}} % Irrational numbers


\newcommand{\pset}{\mathcal{P}} % Powerset
\newcommand{\card}{\abs} % Cardinality

% Calligraphy
\newcommand\und[1]{\underline{\smash{#1}}}

% Operators
\DeclarePairedDelimiter\abs{\lvert}{\rvert}
\DeclarePairedDelimiter\ceil{\lceil}{\rceil}
\DeclarePairedDelimiter\floor{\lfloor}{\rfloor}

% Other
\newcommand{\rarr}{\rightarrow}
\newcommand{\larr}{\leftarrow}

% Setting stuff
\setlength{\parindent}{0pt}


% Title
\title{\rule{\paperwidth - 150pt}{1pt}\textbf{\\\textit{Topology}\\}\rule{\paperwidth - 150pt}{1pt}}

\author
{
\textit{Author: David Oniani}
\\
\textit{Instructor: Dr. Eric Westlund}
}

\date{\textit{January }7\textit{, }2019}


\begin{document}
\maketitle

\center{\Large Assignment \textnumero{1}}

\begin{itemize}
\item[]
{\large Chapter 2}
\vspace{0.3cm}

% Begin here!

\item[2.]
Let $f : A \rarr B$ and let $A_i \subset A$ and $B_i \subset B$
for $i = 0$ and $i = 1$. Show that if $f^{-1}$\\
preserves inclusions, unions, intersections, and differences of sets:

\begin{itemize}
\item[(c)]
$f^{-1}(B_0 \cap B_1) = f^{-1}(B_0) \cap f^{-1}(B_1)$.
\vspace{0.25cm}

\begin{quote}
To prove that the set $f^{-1}(B_0 \cap B_1)$
is equal to the set $f^{-1}(B_0) \cap f^{-1}(B_1)$,
we have to show that $f^{-1}(B_0 \cap B_1) \subset f^{-1}(B_0) \cap f^{-1}(B_1)$
and $f^{-1}(B_0) \cap f^{-1}(B_1) \subset f^{-1}(B_0 \cap B_1)$.
\\
\vspace{0.5cm}
Case I: $f^{-1}(B_0 \cap B_1) \subset f^{-1}(B_0) \cap f^{-1}(B_1)$\\
\vspace{0.15cm}

\begin{quote}
Let $x \in f^{-1}(B_0 \cap B_1)$. Then $f(x) \in B_0 \cap B_1$. Thus, $f(x) \in B_0$
and $f(x) \in B_1$. From this, we get that $x \in f^{-1}(B_0)$ and $x \in f^{-1}(B_1)$.
Therefore, $x \in f^{-1}(B_0) \cap f^{-1}(B_1)$. Hence, if $x \in f^{-1}(B_0 \cap B_1)$,
then $x \in f^{-1}(B_0) \cap f^{-1}(B_1)$ which means that $f^{-1}(B_0 \cap B_1) \subset f^{-1}(B_0) \cap f^{-1}(B_1)$.$\qed$
\end{quote}

\vspace{0.5cm}
Case II: $f^{-1}(B_0) \cap f^{-1}(B_1) \subset f^{-1}(B_0 \cap B_1)$\\
\vspace{0.15cm}

\begin{quote}
Let $x \in f^{-1}(B_0) \cap f^{-1}(B_1)$. Then $x \in f^{-1}(B_0)$ and $x \in f^{-1}(B_1)$.
Thus, $f(x) \in B_0$ and $f(x) \in B_1$. Finally, we have that $f(x) \in B_0 \cap B_1$ which
is equivalent to saying $x \in f^{-1}(B_0 \cap B_1)$. Hence, if $x \in f^{-1}(B_0) \cap f^{-1}(B_1)$,
then $x \in f^{-1}(B_0 \cap B_1)$ which means that $f^{-1}(B_0) \cap f^{-1}(B_1) \subset f^{-1}(B_0 \cap B_1)$.$\qed$
\end{quote}
\vspace{0.5cm}

We have now proven that $f^{-1}(B_0 \cap B_1) \subset f^{-1}(B_0) \cap f^{-1}(B_1)$
and $f^{-1}(B_0) \cap f^{-1}(B_1) \subset f^{-1}(B_0 \cap B_1)$ and thus, $f^{-1}(B_0 \cap B_1) = f^{-1}(B_0) \cap f^{-1}(B_1)$.$\qed$

\end{quote}

\newpage

\item[(g)]
$f(A_0 \cap A_1) \subset f(A_0) \cap f(A_1)$; show that inequality holds if $f$ is injective.
\vspace{0.25cm}

\begin{quote}
Let's first show that $f(A_0 \cap A_1) \subset f(A_0) \cap f(A_1)$ even if $f$ is not injective.\\
\medskip
Let $x \in f(A_0 \cap A_1)$. Then $\exists x^\prime \in A_0 \cap A_1$ such that $f(x^\prime) = x$.
Now, since $x^\prime \in A_0$ and $x^\prime \in A_1$, we get that $x \in f(A_0)$ and $x \in f(A_1)$
thus, $x \in f(A_0) \cap f(A_1)$.$\qed$\\
\bigskip
Now let's prove that $f(A_0 \cap A_1) = f(A_0) \cap f(A_1)$ if $f$ is injective.
We have already shown that independent of whether $f$ is injective or not, $f(A_0 \cap A_1) \subset f(A_0) \cap f(A_1)$.
Thus, we just have to show that $f(A_0) \cap f(A_1) \subset f(A_0 \cap A_1)$ if $f$ is injective.\\
\medskip
Let $x \in f(A_0) \cap f(A_1)$.\ Then $x \in f(A_0)$ and $x \in f(A_1)$.\
Besides, since $f$ is injective, there exists \textbf{\und{unique}} $x^\prime$ such that $f(x^\prime) = x$.\ Therefore, $x^\prime \in A_0$ and $x^\prime \in A_1$.
Finally, we get that $x \in f(A_0 \cap A_1)$.$\qed$
\end{quote}
\end{itemize}

\item[]

\item[5.]
In general, let us denote the $\textit{\textbf{identity function}}$ for a set $C$ by $i_C$. That is, define
$i_C : C \rarr C$ to be the function given by the rule $i_C(x) = x$ for all $x \in C$.
Given $f : A \rarr B$, we say that a function $g : B \rarr A$ is a $\textbf{\textit{left inverse}}$ for $f$
if $g \circ f = i_A$; and we say that $h : B \rarr A$ is a $\textit{\textbf{right inverse}}$ for $f$ if $f \circ h = i_B$.

\begin{itemize}
\item[(a)]
Show that if $f$ has a left inverse, $f$ is injective; and if $f$ has a right inverse, $f$ is surjective.

\begin{quote}
Let's first show that if $f$ has a left inverse, then $f$ is injective.\\
\medskip
Suppose, for the sake of contradiction, that $f : A \rarr B$ is function such that it has a left inverse
and that $f$ is not injective. Since $f$ is not injective, there exists $x_0, x_1 \in A$ such that $f(x_0) = f(x_1)$ and $x_0 \neq x_1$.\
Since $f$ has the left inverse, there exists $g : B \rarr A$ such that $g \circ f = i_A$.
Consider functions $(g \circ f)(x_0)$ and $(g \circ f)(x_1)$.\ These functions could be rewritten as
$g(f(x_0))$ and $g(f(x_1))$.\ Since $f(x_0) = f(x_1)$, we have that $g(f(x_0)) = g(f(x_1))$. Therefore,
we got that $i_A(x_0) = i_A(x_1)$ and thus, $x_0 = x_1$. At last, we have reached the contradiction since initially we assumed that $x_0 \neq x_1$.
Hence, if $f$ has a left inverse, then $f$ is injective.$\qed$
\\
\bigskip
Now let's show that if $f$ has a right inverse, then $f$ is surjective.\\
\medskip
Suppose $f : A \rarr B$ is a function such that it has a right inverse.\
Then there exists $h : B \rarr A$ such that $f \circ h = i_B$.
Note that $\forall x \in B$, we have $f(h(x)) = x$ and since $h(x) \in A$,
we effectively got that every $x \in B$ has something that maps to it.
In other words, $f$ is surjective.$\qed$
\end{quote}

\item[]

\item[(b)]
Give an example of a function that has a left inverse but no right inverse.
\begin{quote}
$f : \ints \rarr \reals : x \mapsto x$.
\end{quote}

\item[]

\item[(c)]
Give an example of a function that has a right inverse but no left inverse.
\begin{quote}
$f : \ints \rarr \pints \cup \{0\} : x \mapsto x^2$.
\end{quote}

\item[]

\item[(d)]
Can a function have more than one left inverse? More than one right inverse?
\begin{quote}
Yes, it can have more than one left inverse.\\
Consider the following functions:
\begin{align*}
f : \{1, 2, 3\} \rarr \{1, 2, 3\}\\
g : \{1, 2, 3\} \rarr \{1, 2, 3\}\\
g^{\prime} : \{1, 2, 3\} \rarr \{1, 2, 3\}\\
\end{align*}
\end{quote}

\item[]

\item[(e)]
Show that if $f$ has both a left inverse $g$ and a right inverse $h$, then $f$
is bijective and $g = h = f^{-1}$.
\begin{quote}
Solution to e.
\end{quote}

\end{itemize}

\item[]
\item[]
\item[]

{\large Chapter 3}
\vspace{0.3cm}

\item[10.]
\begin{itemize}
\item[(a)]
Show that the map $f : (-1, 1) \rarr \reals$ of Example 9 is order preserving.
\begin{quote}
We have to show that $f : (-1, 1) \rarr \reals : x \mapsto \dfrac{x}{1 - x^2}$ is
an order-preserving map. In other words, we have to prove that for arbitrary $x_0, x_1 \in (-1, 1)$,
if $x_1 > x_0$, then $f(x_1) > f(x_0)$. Suppose $x_0, x_1 \in (-1, 1)$ and $x_1 > x_0$.
Then we have:
$$f(x_1) - f(x_0) = \dfrac{x_1}{1 - x_1^2} - \dfrac{x_0}{1 - x_0^2} = \dfrac{x_1 - x_1x_0^2 - x_0 + x_0x_1^2}{(1 - x_1^2)(1 - x_0^2)} = \dfrac{(x_1 - x_0)(x_0x_1 + 1)}{(1 - x_1^2)(1 - x_0^2)}$$
\vspace{0.15cm}
Finally, we got that $f(x_1) - f(x_0) = \dfrac{x_1 - x_1x_0^2 - x_0 + x_0x_1^2}{(1 - x_1^2)(1 - x_0^2)} = \dfrac{(x_1 - x_0)(x_0x_1 + 1)}{(1 - x_1^2)(1 - x_0^2)}$
where $x_1 > x_0$. Notice that all the members of the fraction are positive.\\
$x_1 - x_0 > 0$ since $x_1 > x_0$, $x_0x_1 + 1 > 0$ as $x_0, x_1 \in (-1, 1)$, and $(1 - x_1^2)(1 - x_0^2)$ is positive
since, once again, $x_0, x_1 \in (-1, 1)$. Thus, we have effectively shown that if $x_0, x_1 \in (-1, 1)$ such that $x_1 > x_0$,
then $f(x_1) - f(x_0) > 0$. In others, if $x_0, x_1 \in (-1, 1)$ and $x_1 > x_0$, then $f(x_1) > f(x_0)$ and $f$ is indeed an order-preserving map.$\qed$
\end{quote}

\item[]

\item[(b)]
Show that the equation $g(y) = 2y / [1 + (1 + 4y^2)^{1/2}]$ defines a function
$g : \reals \rarr (-1, 1)$ that is both a left and a right inverse for $f$.

\begin{quote}
Let's first show that it is indeed the left inverse for the function $f$.\\
Consider the function composition $g \circ f$. This composition is possible
since the codomain of $f$ is the same as the domain of $g$. Now, let's do some high
school algebra:
\begin{align*}
(g \circ f)(x) = g(f(x)) = \dfrac{2f(x)}{1 + (1 + 4f(x)^2)^{\frac{1}{2}}}
= \dfrac{\dfrac{2x}{1 - x^2}}{1 + (1 + 4(\dfrac{x}{1 - x^2})^2)^{\frac{1}{2}}}\\[1em]
= \dfrac{\dfrac{2x}{1 - x^2}}{1 + (\dfrac{(x^2 + 1)^2}{(1 - x^2)^2})^{\frac{1}{2}}}
= \dfrac{\dfrac{2x}{1 - x^2}}{\dfrac{1 - x^2 + x^2 + 1}{1 - x^2}}
= \dfrac{\dfrac{2x}{1 - x^2}}{\dfrac{2}{1 - x^2}}
= x
\end{align*}
Hence, we've got that $(g \circ f)(x) = x = i_{(-1, 1)}$ thus, is the identity function for the open interval $(-1, 1)$.
This is sufficient show for the function $g$ to be the left inverse for the function $f$.$\qed$
\\
\vspace{0.4cm}
Now, let's show that $g$ is also the right inverse for $f$.
Consider, the function composition $f \circ g$. Once again, we will need
to do some high school math:
\begin{align*}
(f \circ g)(x) = f(g(x)) = \dfrac{g(x)}{1 - g(x)^2}
= \dfrac{\dfrac{2x}{1 + (1 + 4x^2)^\frac{1}{2}}}{1 - (\dfrac{2x}{1 + (1 + 4x^2)^\frac{1}{2}})^2}\\[1em]
= \dfrac{2x(1 + (1 + 4x^2)^\frac{1}{2})}{(1 + (1 + 4x^2)^\frac{1}{2})^2 - 4x^2}
= \dfrac{2x(1 + (1 + 4x^2)^\frac{1}{2})}{1 + 1 + 4x^2 + 2(1 + 4x^2)^\frac{1}{2} -4x^2}\\[1em]
= \dfrac{2x(1 + (1 + 4x^2)^\frac{1}{2})}{2(1 + (1 + 4x^2)^\frac{1}{2})}
= x
\end{align*}
Thus, we got that $(f \circ g)(x) = x = i_{\reals}$. This means that the function composition $f \circ g$
yields the identity function and this means that function $g$ is indeed the right inverse for the function $f$.$\qed$
\end{quote}
\end{itemize}

\item[]
\item[]
\item[]

{\large Chapter 6}
\vspace{0.3cm}

\item[2.]
Show that if $B$ is not finite and $B \subset A$, then $A$ is not finite.
\begin{quote}
Suppose, for the sake of contradiction, that $B$ is not finite, $B \subset A$, and $A$ is finite.
Then since $B \subset A$ and $A$ is finite, it must be the case that $B$ is finite too as it is
the portion of the set with finite elements. This, however, contradicts our initial assumption that $B$
is not finite thus, we have reached the contradiction and if $B$ is not finite and $B \subset A$, then $A$
is not finite.$\qed$
\end{quote}

\item[]

\item[3.]
Let $X$ be the two-element set $\{0, 1\}$. Find a bijective correspondence between $X^\omega$ and a proper
subset of itself.
\begin{quote}
Let $S = \{A \in X^\omega \ | \ A = (0, x_1, x_2, ...)\}$ and define $f : X^\omega \rarr S$ such that $f(x_1, x_2, x_3, ...) = (0, x_1, x_2, x_3 ...)$.
This is a function that shifts each element of the sequence one place to the right.
\end{quote}

\item[]

\item[7.]
If $A$ and $B$ are finite, show that the set of all functions $f : A \rarr B$ is finite.
\begin{quote}
Notice that there are $\abs{B}^{\abs{A}}$ different functions of the form $f : A \rarr B$.
Since $A$ and $B$ are finite, so is $\abs{B}^{\abs{A}}$. Hence, we got that the cardinality
of the set of all functions $f : A \rarr B$ is finite which implies that the set of all functions
$f : A \rarr B$ is also finite.$\qed$
\end{quote}

\item[]
\item[]
\item[]

{\large Chapter 7}
\vspace{0.3cm}

\item[1.]
Show that $\rats$ is countably infinite.

\begin{quote}
To show that the set of rational numbers is countable, it is sufficient to show that there is a way to enumerate (count down)
all of them. This is the classic visual proof of enumerability/countability of the set of rational numbers.\\
\vspace{0.15cm}
In order to prove the countability of the set of rational numbers, let's first construct a two-dimensional, infinite matrix.
We will construct it in the following way:
\begin{enumerate}
\item[1.]
The first column of the matrix is comprised of non-negative integers.
\item[2.]
The first row of the matrix is comprised of non-negative integers.
\item[3.]
The rest of the cells are pairs which are composed by dividing the cell value of the first row by the corresponding column (the division product with its opposite).
\end{enumerate}
The grid will look like this:
\\
\begin{table}[h!]
	\centering
	\begin{adjustbox}{max width=\textwidth}
	\resizebox{0.8\linewidth}{!}
	{
		\begin{tabular}{*{7}{|c}|}%%{|c|c|c|c|c|c|c|c|c|c|c|c|c|c|}
		\hline
		0 & 1 & 2 & 3 & 4 & 5 & etc\\ \hline
		1 & (1/1, -1/1) & (2/1, -2/1) & (3/1, -3/1) & (4/1, -4/1) & (5/1, -5/1) & etc\\ \hline
		2 & (1/2, -1/2) & (2/2, -2/2) & (3/2, -3/2) & (4/2, -4/2) & (5/2, -5/2) & etc\\ \hline
		3 & (1/3, -1/3) & (2/3, -2/3) & (3/3, -3/3) & (4/3, -4/3) & (5/3, -5/3) & etc\\ \hline
		4 & (1/3, -1/4) & (2/3, -2/4) & (3/3, -3/4) & (4/3, -4/4) & (5/3, -5/4) & etc\\ \hline
		5 & (1/5, -1/5) & (2/5, -2/5) & (3/5, -3/5) & (4/5, -4/5) & (5/5, -5/5) & etc\\ \hline
		etc & etc & etc & etc & etc & etc & etc\\
		\hline
	\end{tabular}
    }
	\end{adjustbox}
\end{table}
Now, our enumeration could proceed as follows: 0, 1/1, -1/1, 1/2, -1/2, 2/1, -2/1, 1/3, -1/3, 2/3, -2/3, 3/3, -3/3 etc.
Notice that the enumeration starts at the first entry of the first row and proceeds diagonally.
This will guarantee that all of the rational numbers will be listed (notice that we would not be able to list them if we proceeded horizontally or vertically).
Thus, the set of rational numbers is countable.$\qed$
\end{quote}

\item[]

\item[5.]
Determine, for each of the following sets, whether or not it is countable. Justify
your answers.
\begin{itemize}
\item[(a)]
Solution.

\item[(b)]
Solution.

\item[(c)]
Solution.

\item[(d)]
Solution.

\item[(e)]
Solution.

\item[(f)]
Solution.

\item[(g)]
Solution.

\item[(h)]
Solution.

\item[(i)]
Solution.

\item[(j)]
Solution.
\end{itemize}

\end{itemize}
\end{document}
