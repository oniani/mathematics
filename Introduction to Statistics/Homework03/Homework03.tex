\documentclass[11pt, a4paper]{article}
\usepackage[a4paper, margin=1in]{geometry}


\usepackage{adjustbox}
\usepackage{mathtools}
\usepackage{amsmath}
\usepackage{amssymb}
\usepackage{amsthm}

\usepackage{pgfplots}
\usepackage{listings}
\usepackage{color}
\usepackage{tikz}

\usepackage{textcomp}
\usepackage{soul}

\usepackage[hidelinks]{hyperref}
\pgfplotsset{width=7.5cm,compat=1.12}
\usepgfplotslibrary{fillbetween}
\usepackage[makeroom]{cancel}
\title{\bf{Homework \textnumero 1}}
\author{Author: David Oniani
\\
\ \ \ Instructor: Dr. Eric Westlund}
\date{February 12, 2019}

\usepackage{listings}
\usepackage{color}

%%%%%%%%%%%%%%% S E T S %%%%%%%%%%%%%%%
\newcommand{\nats}{\mathbb{N}}
\newcommand{\ints}{\mathbb{Z}}
\newcommand{\rats}{\mathbb{Q}}
\newcommand{\reals}{\mathbb{R}}
\newcommand{\irrats}{\mathbb{I}}

\newcommand{\pnats}{\mathbb{N}^+}
\newcommand{\pints}{\mathbb{Z}^+}
\newcommand{\prats}{\mathbb{Q}^+}
\newcommand{\preals}{\mathbb{R}^+}
\newcommand{\nreals}{\mathbb{R}^-}

\newcommand{\nints}{\mathbb{Z}^-}
\newcommand{\nrats}{\mathbb{Q}^-}
%%%%%%%%%%%%%%%%%%%%%%%%%%%%%%%%%%%%%%%

% Calligraphy
\newcommand\und[1]{\underline{\smash{#1}}}

% Operators
\DeclarePairedDelimiter\abs{\lvert}{\rvert}
\DeclarePairedDelimiter\ceil{\lceil}{\rceil}
\DeclarePairedDelimiter\floor{\lfloor}{\rfloor}

% Other
\newcommand{\rarr}{\rightarrow}

\definecolor{dkgreen}{rgb}{0,0.6,0}
\definecolor{gray}{rgb}{0.5,0.5,0.5}
\definecolor{mauve}{rgb}{0.58,0,0.82}
\definecolor{backcolour}{rgb}{0.95,0.95,0.92}

\lstset{
backgroundcolor=\color{backcolour},
aboveskip=3mm,
belowskip=3mm,
showstringspaces=false,
columns=flexible,
basicstyle={\small\ttfamily},
numbers=left,
numberstyle=\normalsize\color{gray},
keywordstyle=\color{blue},
commentstyle=\color{dkgreen},
stringstyle=\color{mauve},
breaklines=true,
breakatwhitespace=true,
tabsize=4
}


\begin{document}
\maketitle

\begin{itemize}
\item[3.4]
\begin{itemize}
\item[(a)]
$C$ is the mean and $B$ is the median (since the distribution is right-skewed).

\item[]

\item[(b)]
$B$ is both the mean and the median (since the distribution is symmetric).

\item[]

\item[(c)]
$A$ is the mean and $B$ is the median (since the distribution is left-skewed).
\end{itemize}

\item[]
\item[]

\item[3.6]
\begin{itemize}
\item[(a)]
The area for $99.7\%$ corresponds to the three standard deviations and therefore, the range for lengths
that cover almost all $(99.7\%)$ of this distribution is from $35.8 - 3 \times 2.1$
to $35.8 + 3 \times 2.1$. That is the range from $29.5$ to $42.1$.

\item[]

\item[(b)]
Notice that $33.7 = 35.8 - 2.1$. Therefore, the datapoint is located one deviation to the left from the center.
Hence, we got that $\dfrac{32}{2}\% = 16\%$ of women over $20$ have the arm length less than $33.7$cm.
\end{itemize}

\item[]
\item[]

\item[3.7]
\begin{itemize}
\item[(a)]
According to the $68 - 95 - 99.7$ rule, it will be between
$852 - 2 \times 82$ and $852 + 2 \times 82$. That is,
between $688$ and $1016$.

\item[]

\item[(b)]
According to the $68 - 95 - 99.7$ rule, it will be
$852 - 2 \times 82 = 688$ (this is since $95\%$ leaves
us with $2.5\%$ on both sides and we need the left one).
\end{itemize}

\item[]
\item[]

\item[3.8]
$z_{\text{Idonna}} = \dfrac{x - \mu}{\sigma} = \dfrac{670 - 514}{118} = 1.32$\\\\
$z_{\text{Jonathan}} = \dfrac{x - \mu}{\sigma} = \dfrac{26 - 20.9}{5.3} = 0.96$\\\\
Since $z_{\text{Idonna}} > z_{\text{Jonathan}} (1.32 > 0.96)$, it appears that Idonna did better.

\newpage

\item[3.10]


\end{itemize}

\end{document}
