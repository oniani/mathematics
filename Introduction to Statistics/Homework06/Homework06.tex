\documentclass[11pt, a4paper]{article}
\usepackage[a4paper, margin=1in]{geometry}


\usepackage{adjustbox}
\usepackage{mathtools}
\usepackage{amsmath}
\usepackage{amssymb}
\usepackage{amsthm}

\usepackage{pgfplots}
\usepackage{listings}
\usepackage{color}
\usepackage{tikz}
\usepackage{multirow}

\usepackage{textcomp}
\usepackage{soul}

\usepackage[hidelinks]{hyperref}
\pgfplotsset{width=7.5cm,compat=1.12}
\usepgfplotslibrary{fillbetween}
\usepackage[makeroom]{cancel}
\title{\bf{Homework \textnumero 6}}
\author{Author: David Oniani
\\
\ \ \ Instructor: Dr. Eric Westlund}
\date{February 24, 2019}

\usepackage{listings}
\usepackage{color}

%%%%%%%%%%%%%%% S E T S %%%%%%%%%%%%%%%
\newcommand{\nats}{\mathbb{N}}
\newcommand{\ints}{\mathbb{Z}}
\newcommand{\rats}{\mathbb{Q}}
\newcommand{\reals}{\mathbb{R}}
\newcommand{\irrats}{\mathbb{I}}

\newcommand{\pnats}{\mathbb{N}^+}
\newcommand{\pints}{\mathbb{Z}^+}
\newcommand{\prats}{\mathbb{Q}^+}
\newcommand{\preals}{\mathbb{R}^+}
\newcommand{\nreals}{\mathbb{R}^-}

\newcommand{\nints}{\mathbb{Z}^-}
\newcommand{\nrats}{\mathbb{Q}^-}
%%%%%%%%%%%%%%%%%%%%%%%%%%%%%%%%%%%%%%%

% Calligraphy
\newcommand\und[1]{\underline{\smash{#1}}}

% Operators
\DeclarePairedDelimiter\abs{\lvert}{\rvert}
\DeclarePairedDelimiter\ceil{\lceil}{\rceil}
\DeclarePairedDelimiter\floor{\lfloor}{\rfloor}

% Other
\newcommand{\rarr}{\rightarrow}

\definecolor{dkgreen}{rgb}{0,0.6,0}
\definecolor{gray}{rgb}{0.5,0.5,0.5}
\definecolor{mauve}{rgb}{0.58,0,0.82}
\definecolor{backcolour}{rgb}{0.95,0.95,0.92}

\lstset{
backgroundcolor=\color{backcolour},
aboveskip=3mm,
belowskip=3mm,
showstringspaces=false,
columns=flexible,
basicstyle={\small\ttfamily},
numbers=left,
numberstyle=\normalsize\color{gray},
keywordstyle=\color{blue},
commentstyle=\color{dkgreen},
stringstyle=\color{mauve},
breaklines=true,
breakatwhitespace=true,
tabsize=4
}


\begin{document}
\maketitle

\begin{itemize}
\item[6.1]
\begin{itemize}
\item[(a)]
It describes $736 + 450 + 193 + 205 + 144 + 80 = 1808$ people.\\
Of 1808 people, $736 + 450 + 193 = 1379$ played video games.

\item[]

\item[(b)]
As and Bs would be $\dfrac{736 + 205}{1808} \approx 0.5204$ which is $52.04\%$.\\\\
Cs would be $\dfrac{450 + 144}{1808} \approx 0.3285$ which is $32.85\%$.\\\\
Ds and Fs would be $\dfrac{193 + 80}{1808} \approx 0.1509$ which is $15.09\%$.
\end{itemize}

\item[]
\item[]

\item[6.3]
From the exercise 6.1, we know that there are 1379 players in total.
Then we have:\\\\
$\dfrac{736}{1379} \approx 0.5337 \text{ which is } 53.37\% \text{ got As or Bs}.$\\\\
$\dfrac{450}{1379} \approx 0.3263 \text{ which is } 32.63\% \text{ got Cs}.$\\\\
$\dfrac{193}{1379} \approx 0.1399 \text{ which is } 13.99\% \text{got Ds and Fs}$.

\item[]
\item[]

\item[6.5]
We just have to solve a system of linear equations shown below.
$$
\begin{cases}
a + b = 50\\
c + d = 50\\
a + c = 60\\
b + d = 40\\
\end{cases}
$$
Let's pick $a = 20$. Then from the first equation, we get $b = 50 - 20 = 30$.
From the third equation, we get $c = 60 - a = 60 - 20 = 40$. And from the fourth equation,
we get $d = 40 - b = 40 - 30 = 10$. At last, we got the solution which is a four-tuple
$(a = 20, b = 30, c = 40, d = 10)$.\\\\
We can now pick a different value for $a$. Say $a = 50$. Then From the first equation,
we get $b = 50 - 50 = 0$. From the third equation, we get $c = 60 - a = 60 - 50 = 10$.
And from the fourth equation we get $d = 40 - b = 40 - 0 = 40$. In this case, our solution
is a different four-tuple which is $(a = 50, b = 0, c = 10, d = 40)$\\\\
At last, we indeed have gotten two different sets which are $a = 20, b = 30, c = 40, d = 10$
and $a = 50, b = 0, c = 10, d = 40$.

\item[]
\item[]

\item[6.6]
\begin{itemize}
\item[(a)]
Alex Brailsford made $\dfrac{15 + 5}{15 + 15 + 5 + 15} \times 100\%= \dfrac{20}{50} \times 100\% = 40\%$ of all field goals.\\\\
Rickie Jackson made $\dfrac{30 + 65}{30 + 29 + 65 + 130} \times 100\%= \dfrac{95}{254} \times 100\% \approx 37.4\%$ of all field goals.

\item[]

\item[(b)]
Alex Brailsford made $\dfrac{15}{15 + 15} \times 100\%= \dfrac{15}{30} \times 100\% = 50\%$ of all two-point field goals.\\\\
Alex Brailsford made $\dfrac{5}{5 + 15} \times 100\%= \dfrac{5}{20} \times 100\% = 25\%$ of all two-point field goals.\\\\\\
Rickie Jackson made $\dfrac{30}{30 + 29} \times 100\%= \dfrac{30}{59} \times 100\% \approx 50.8\%$ of all two-point field goals.\\\\
Rickie Jackson made $\dfrac{65}{65 + 130} \times 100\%= \dfrac{65}{195} \times 100\% \approx 33.3\%$ of all two-point field goals.

\item[]

\item[(c)]
This is due to Simpson's paradox which implies that am association or comparison
that holds for all several groups can \und{reverse} direction when the data are
combined to form a single group.
\end{itemize}

\item[]
\item[]

\item[20.]
\textbf{\und{Marital Status}}\\
There would be $\dfrac{4938}{39648} \times 100\% \approx 12.45\%$ for singles.\\\\
There would be $\dfrac{28132}{39648} \times 100\% \approx 70.95\%$ for married.\\\\
There would be $\dfrac{5923}{39648} \times 100\% \approx 14.93\%$ for divorced.\\\\
There would be $\dfrac{655}{39648} \times 100\% \approx 1.65\%$ for widowed.\\\\
The percentages add up to $99.98\%$ and not to $100\%$ due to the roundoff errors.
\\\\
\textbf{\und{Income}}\\
There would be $\dfrac{1918}{39648} \times 100\% \approx 4.83\%$ for singles.\\\\
There would be $\dfrac{20932}{39648} \times 100\% \approx 52.79\%$ for married.\\\\
There would be $\dfrac{10750}{39648} \times 100\% \approx 27.11\%$ for divorced.\\\\
There would be $\dfrac{6048}{39648} \times 100\% \approx 15.25\%$ for widowed.\\\\
The percentages add up to $99.98\%$ and not to $100\%$ due to the roundoff errors.

\item[]
\item[]

\item[6.21]
Percent of single men with no income is $\dfrac{513}{4938} \times 100\% \approx 10.39\%$.\\
Percent of men with no income that are single is $\dfrac{513}{1918} \times 100\% \approx 26.75\%$.

\item[]
\item[]

\item[6.22]
$\dfrac{513}{4938} \times 100\% \approx 10.39\%$ for men with no income.\\\\
$\dfrac{3323}{4938} \times 100\% \approx 67.29\%$ for men with income in range $\$1 - \$49, 999$.\\\\
$\dfrac{814}{4938} \times 100\% \approx 16.48\%$ for men with income in range $\$50, 000 - \%99, 999$.\\\\
$\dfrac{288}{4938} \times 100\% \approx 5.83\%$ for men with income $\%100, 000$ and over.\\\\
$10.39 + 67.29 + 16.48 + 5.83 = 99.9\%$ and therefore, the percents do
add up to $100\%$ (up to roundoff error).

\item[]
\item[]

\item[6.26]
We must come up with two two-way tables of obese by early death
for smokers and non-smokers.
\begin{center}
        \begin{tabular}{ |c|c|c|c| }
        \hline
         & Obese & Non-obese\\
        \hline
        Early death & 134 & 274\\
        Non-early-death & 35 & 47\\
        & & \\
        \hline
        Smokers & Obese & Non-obese\\
        \hline
        Early death & 112 & 162\\
        Non-early-death & 22 & 26\\
        & & \\
        \hline
        Non-smokers & Obese & Non-obese\\
        \hline
        Early death & 22 & 112\\
        Non-early-death & 13 & 21\\
        \hline
    \end{tabular}
\end{center}

\newpage

\item[6.30]
We have to calculate the percent of each type of complication for each disease.\\\\
\textbf{\und{Gastric banding}}\\\\
$\dfrac{81}{5380} \times 100\% \approx 1.51\%$ is non-life-threatening.\\\\
$\dfrac{46}{5380} \times 100\% \approx 0.86\%$ is serious.\\\\
$\dfrac{5253}{5380} \times 100\% \approx 97.64\%$ is none.\\\\\\
\textbf{\und{Sleeve gastrectomy}}\\\\
$\dfrac{31}{854} \times 100\% \approx 3.63\%$ is non-life-threatening.\\\\
$\dfrac{19}{854} \times 100\% \approx 2.22\%$ is serious.\\\\
$\dfrac{804}{854} \times 100\% \approx 94.15\%$ is none.\\\\\\
\textbf{\und{Gastric bypass}}\\\\
$\dfrac{606}{9041} \times 100\% \approx 6.70\%$ is non-life-threatening.\\\\
$\dfrac{325}{854} \times 100\% \approx 3.59\%$ is serious.\\\\
$\dfrac{8110}{854} \times 100\% \approx 89.70\%$ is none.\\\\\\
Then from this data, it is easy to see that \textbf{\und{gastric bypass}} leads
to the most complications and \textbf{\und{gastric banding}} to the least complications
with \textbf{\und{sleeve gastrectomy}} being somewhere in-between in terms of complications.
\end{itemize}

\end{document}
