\documentclass[11pt, a4paper]{article}
\usepackage[a4paper, margin=1in]{geometry}

\usepackage{adjustbox}
\usepackage{mathtools}
\usepackage{amsmath}
\usepackage{amssymb}
\usepackage{amsthm}

\usepackage{pgfplots}
\usepackage{listings}
\usepackage{color}
\usepackage{tikz}

\usepackage{textcomp}
\usepackage{soul}

\usepackage[hidelinks]{hyperref}
\pgfplotsset{width=7.5cm,compat=1.12}
\usepgfplotslibrary{fillbetween}
\pgfplotsset{compat=1.8}
\usepgfplotslibrary{statistics}
\usepackage[makeroom]{cancel}
\title{\bf{Homework \textnumero 16}}
\author{Author: David Oniani
\\
\ \ \ Instructor: Dr. Eric Westlund}
\date{April 13, 2019}

\usepackage{listings}
\usepackage{color}

%%%%%%%%%%%%%%% S E T S %%%%%%%%%%%%%%%
\newcommand{\nats}{\mathbb{N}}
\newcommand{\ints}{\mathbb{Z}}
\newcommand{\rats}{\mathbb{Q}}
\newcommand{\reals}{\mathbb{R}}
\newcommand{\irrats}{\mathbb{I}}

\newcommand{\pnats}{\mathbb{N}^+}
\newcommand{\pints}{\mathbb{Z}^+}
\newcommand{\prats}{\mathbb{Q}^+}
\newcommand{\preals}{\mathbb{R}^+}
\newcommand{\nreals}{\mathbb{R}^-}

\newcommand{\nints}{\mathbb{Z}^-}
\newcommand{\nrats}{\mathbb{Q}^-}
%%%%%%%%%%%%%%%%%%%%%%%%%%%%%%%%%%%%%%%

% Calligraphy
\newcommand\und[1]{\underline{\smash{#1}}}

% Operators
\DeclarePairedDelimiter\abs{\lvert}{\rvert}
\DeclarePairedDelimiter\ceil{\lceil}{\rceil}
\DeclarePairedDelimiter\floor{\lfloor}{\rfloor}

% Other
\newcommand{\rarr}{\rightarrow}

\definecolor{dkgreen}{rgb}{0,0.6,0}
\definecolor{gray}{rgb}{0.5,0.5,0.5}
\definecolor{mauve}{rgb}{0.58,0,0.82}
\definecolor{backcolour}{rgb}{0.95,0.95,0.92}

\lstset{
backgroundcolor=\color{backcolour},
aboveskip=3mm,
belowskip=3mm,
showstringspaces=false,
columns=flexible,
basicstyle={\small\ttfamily},
numbers=left,
numberstyle=\normalsize\color{gray},
keywordstyle=\color{blue},
commentstyle=\color{dkgreen},
stringstyle=\color{mauve},
breaklines=true,
breakatwhitespace=true,
tabsize=4
}


\begin{document}
\maketitle
\begin{itemize}
\item[22.4]
There are two reasons why we can't use the large-sample
confidence interval to estimate the proportion $p$ in the
population who share these two risk factors. At first,
$\hat{p}$ is too close to 0. Secondly, people reached
might have provided wrong data.

\item[]
\item[]

\item[22.6]
We start with the \und{\textbf{PLAN}} part as the \und{\textbf{STATE}}
part is the description of the problem itself.\\\\
\und{\textbf{PLAN}}\\
We must find the $\hat{p}$ value and the $z^*$ value for the 90\% confidence
interval to find the confidence interval.
\\\\\\
\und{\textbf{SOLVE}}\\\\
$\hat{p} = \dfrac{1552}{4111} \approx 0.378$.
From Table C, we get that $z^* = 1.645$. Therefore, the confidence\\
interval is $0.378 \pm 1.64 \times \sqrt{\dfrac{0.378 * (1 - 0.378)}{4111}}$
which is  $0.378 \pm 0.012$. In other words, the confidence interval is between 0.366 and 0.39
\\\\\\
\und{\textbf{CONCLUDE}}\\
We are 90\% confident that the proportion of the weightlifting injuries
in this age group that were accidental is between 0.366 and 0.39.

\item[]
\item[]

\item[22.7]
\begin{itemize}
\item[(a)]
$\hat{p}= \dfrac{42}{165} \approx 0.255$.
Therefore, $\text{ME} = 1.96 \times \sqrt{\dfrac{0.255 (1 - 0.255)}{165}} \approx 0.0665$.

\item[]

\item[(b)]
To get a $\pm 3$ margin of error, we need $n$ to be at least $\bigg(\dfrac{1.96}{0.03}\bigg)^2 \times 0.255 \times (1 - 0.255) \approx 810.898$.
Therefore, $n = 810.898$ is the answer.
\end{itemize}

\item[]
\item[]

\item[22.9]
We start with the \und{\textbf{PLAN}} part as the \und{\textbf{STATE}}
part is the description of the problem itself.\\\\
\und{\textbf{PLAN}}\\
We test the null hypothesis $\text{H}_0$.
Our null hypothesis is is $\text{H}_0: p = 0.5$ and the
alternative hypothesis is $\text{H}_a: p \neq 0.5$.
\\\\\\
\und{\textbf{SOLVE}}\\\\
$\hat{p} = \dfrac{140}{250} = 0.56$.
$z = \dfrac{0.56 - 0.5}{\sqrt{\frac{0.5 \times (1 - 0.5)}{250}}} \approx 1.90$.
We then get that the corresponding $P$-value is $P = 0.0574$.
\\\\\\
\und{\textbf{CONCLUDE}}\\
Since $P = 0.0574 \approx 0.05$, we can say that there is some evidence, yet not strong,
that the proportion of times a Belgian euro coin spins heads is not 0.50.
However, since P > 0.05, we accept the null hypothesis $\text{H}_0$.

\item[]
\item[]

\item[22.10]
We start with the \und{\textbf{PLAN}} part as the \und{\textbf{STATE}}
part is the description of the problem itself.\\\\
\und{\textbf{PLAN}}\\
We test the null hypothesis $\text{H}_0$.
Our null hypothesis is is $\text{H}_0: p = 0.5$ and the
alternative hypothesis is $\text{H}_a: p > 0.5$.
\\\\\\
\und{\textbf{SOLVE}}\\\\
$\hat{p} = \dfrac{22}{32} = 0.6875$.
$z = \dfrac{0.6875 - 0.5}{\sqrt{\frac{0.5 \times (1 - 0.5)}{32}}} \approx 1.90 \approx 2.121$.
We then get that the corresponding $P$-value is $P = 0.017$.
\\\\\\
\und{\textbf{CONCLUDE}}\\
Since $P = 0.017 < 0.05$, we reject the null hypothesis $\text{H}_0$
and conclude that there is a strong evidence that the candidate
with the better face wins more than half the time.

\item[]
\item[]

\item[22.11]
\begin{itemize}
\item[(a)]
We cannot use the $z$ test for the proportion since
$10 \times 0.5 = 5 < 10$ and the number of trials is not
sufficiently large.

\item[]

\item[(b)]
We can use the $z$ test for the proportion if the sample
is derived using SRS.

\item[]

\item[(c)]
We cannot use the $z$ test for the proportion since
$200 \times (1 - 0.99) = 2 < 10$ and the number of trials is not
sufficiently large.
\end{itemize}

\newpage

\item[22.39]
\begin{itemize}
\item[(a)]
We start with the \und{\textbf{PLAN}} part as the \und{\textbf{STATE}}
part is the description of the problem itself.\\\\
\und{\textbf{PLAN}}\\
We test the null hypothesis $\text{H}_0$.
Our null hypothesis is is $\text{H}_0: p = 0.5$ and the
alternative hypothesis is $\text{H}_a: p \neq 0.5$.
\\\\\\
\und{\textbf{SOLVE}}\\\\
$\hat{p} = \dfrac{22}{32} = 0.6875$.
$z = \dfrac{0.6875 - 0.5}{\sqrt{\frac{0.5 \times (1 - 0.5)}{32}}} \approx 1.90 \approx 2.121$.
We then get that the corresponding $P$-value is $P = 2 \times 0.017 = 0.034$.
\\\\\\
\und{\textbf{CONCLUDE}}\\
Since $P = 0.034 < 0.05$, we reject the null hypothesis $\text{H}_0$
and conclude that there is a strong evidence that people are not equally
likely to choose either of the two positions when presented with two identical
wine samples in sequence.

\item[]

\item[(b)]
We do not know whether we indeed have the SRS (Simple Random Sample) of the people.
People who partook in the experiments might have a bias (response bias). Due to this
reason, generalization of our conclusions to all wine tasters will, most likely, not
yield accurate results.
\end{itemize}

\item[]
\item[]


\item[22.41]
$n = \bigg(\dfrac{z^*}{m}\bigg)^2 \times p^* \times (1 - p^*) = \bigg(\dfrac{1.96}{0.05}\bigg)^2 \times 0.6875 \times (1 - 0.6875) \approx 330.14$.\\\\
Hence, we would need at least 331 (we round \und{up}) wine tasters to estimate the proportion that would choose the first option to within 0.05
with 95\% confidence.
\end{itemize}
\end{document}
