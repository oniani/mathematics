\documentclass[11pt, a4paper]{article}
\usepackage[a4paper, margin=1in]{geometry}


\usepackage{adjustbox}
\usepackage{mathtools}
\usepackage{amsmath}
\usepackage{amssymb}
\usepackage{amsthm}

\usepackage{pgfplots}
\usepackage{listings}
\usepackage{color}
\usepackage{tikz}
\usepackage{multirow}

\usepackage{textcomp}
\usepackage{soul}

\usepackage[hidelinks]{hyperref}
\pgfplotsset{width=7.5cm,compat=1.12}
\usepgfplotslibrary{fillbetween}
\usepackage[makeroom]{cancel}
\title{\bf{Homework \textnumero 7}}
\author{Author: David Oniani
\\
\ \ \ Instructor: Dr. Eric Westlund}
\date{March 3, 2019}

\usepackage{listings}
\usepackage{color}

%%%%%%%%%%%%%%% S E T S %%%%%%%%%%%%%%%
\newcommand{\nats}{\mathbb{N}}
\newcommand{\ints}{\mathbb{Z}}
\newcommand{\rats}{\mathbb{Q}}
\newcommand{\reals}{\mathbb{R}}
\newcommand{\irrats}{\mathbb{I}}

\newcommand{\pnats}{\mathbb{N}^+}
\newcommand{\pints}{\mathbb{Z}^+}
\newcommand{\prats}{\mathbb{Q}^+}
\newcommand{\preals}{\mathbb{R}^+}
\newcommand{\nreals}{\mathbb{R}^-}

\newcommand{\nints}{\mathbb{Z}^-}
\newcommand{\nrats}{\mathbb{Q}^-}
%%%%%%%%%%%%%%%%%%%%%%%%%%%%%%%%%%%%%%%

% Calligraphy
\newcommand\und[1]{\underline{\smash{#1}}}

% Operators
\DeclarePairedDelimiter\abs{\lvert}{\rvert}
\DeclarePairedDelimiter\ceil{\lceil}{\rceil}
\DeclarePairedDelimiter\floor{\lfloor}{\rfloor}

% Other
\newcommand{\rarr}{\rightarrow}

\definecolor{dkgreen}{rgb}{0,0.6,0}
\definecolor{gray}{rgb}{0.5,0.5,0.5}
\definecolor{mauve}{rgb}{0.58,0,0.82}
\definecolor{backcolour}{rgb}{0.95,0.95,0.92}

\lstset{
backgroundcolor=\color{backcolour},
aboveskip=3mm,
belowskip=3mm,
showstringspaces=false,
columns=flexible,
basicstyle={\small\ttfamily},
numbers=left,
numberstyle=\normalsize\color{gray},
keywordstyle=\color{blue},
commentstyle=\color{dkgreen},
stringstyle=\color{mauve},
breaklines=true,
breakatwhitespace=true,
tabsize=4
}


\begin{document}
\maketitle

\begin{itemize}
\item[8.13]
\begin{itemize}
\item[(a)]
The sample size would be the number of people
who took the survey and therefore is $2379$.
Technically, we could draw the conclusion \und{but the conclusion would be inaccurate}
\und{there were a number of people that did not responsd meaning that there was the}\\
\und{nonresponse bias}.

\item[]

\item[(b)]
$\text{Rate of nonresponse} = \dfrac{100000 - 2379}{100000} \times 100\% \approx 97.62\%$.\\
\vspace{0.015cm}\\
Survey results would not be too credible as what usually happens is that people
with the strongest opinion respond.

\item[]

\item[(c)]
The survey has only 2379 participants in lieu of 100000.
\end{itemize}

\item[]
\item[]

\item[8.14]
Question A is slanted toward more negative response on gays in the military
since the question mentions that ``\textit{Federal law currently \und{prohibits} openly gay men and women from serving in the military.}''
Prohibition means NO CHOICE which makes the question more ``negative.''\\\\
Question B, on the other hand, has no negative introductions and asks the question
right away resulting in less negative responses (on average).

\item[]
\item[]

\item[8.16]
\begin{itemize}
\item[(a)]
Example question $\to$ ``How often do you play mobile games?.''\\\\
People who have a cell phone would, on average, respond with a number greater than 0
while those who have landline phone only would respond, on average, with a number 0.
In other words, people who have a cell phone only are a lot more likely to play
games that are available on cell phone only than those who have landline phones only.

\item[]

\item[(b)]
It would be biased as the question itself is constructed in a way
that people with cell phones would show higher results as opposed to
those with landline phones.

\item[]

\item[(c)]
It is of the utmost importance to include both a landline sample and
a cell phone sample since otherwise, we would lose a big part of population
from the sample.\\\\
The more telephone lines there are, the greater the probability that someone picks
up the phone and therefore, greater the probability that the household is included in the sample.
\end{itemize}

\item[]
\item[]

\item[8.37]
\begin{itemize}
\item[(a)]
$P(\text{Man}) = \dfrac{4}{40} = 0.1$.\\
$P(\text{Woman}) = \dfrac{3}{30} = 0.1$.\\

\item[]

\item[(b)]
Result in the part (a) of the exercise is a strong evidence that the probability
of any person in the party being interviewed is the same. Seven men and one woman
is almost having 100\% men sample which is certainly against the results we obtained
in part (a) of the exercise (it is virtually impossible to get a sample with only men
since everyone has an equal chance to be interviewed). Therefore, this is not an SRS
of people from the party.
\end{itemize}

\item[]
\item[]

\item[8.43]
\begin{itemize}
\item[(a)]
Since $\dfrac{200}{5} = 40$, we first choose the number randomly from first 40 names.
\vspace{0.02cm}\\
Let this number be 23. Then we proceed by adding
40 to the number until we reach 200. Finally, we get the systematic
random sample 23, 63, 103, 143, 183.

\item[]

\item[(b)]
Systematic random sampling ensures that each individual has $\dfrac{1}{40} = 0.025$
probability of being selected and is similar to SRS in this way. On the other hand,
This is not exactly SRS since it contains only 1 out of first 40 names, 1 out of next 40
names, etc. SRS, on the other hand, could contain all 5 out of the second 40 names or
3 out of first 40 names and 2 out of the third 40 names, etc. (this is not possible in
systematic random sampling).
\end{itemize}

\item[]
\item[]

\item[8.45]
\begin{itemize}
\item[(a)]
Households that do not have telephones, households whose
numbers are not listed, and households that only have cell phones
would be omitted.\\\\
\textit{Households that do not have telephones} -- there will probably be more poor people in this category.\\\\
\textit{Households whose telephone numbers are not listed} -- people who do not wish to have their numbers published (be publicly available)\\\\
\textit{Households that only have cell phones} -- people who do not want to have landline phones

\item[]

\item[(b)]
The households whose telephone numbers are not listed will be included in the sampling frame.
\end{itemize}

\item[]
\item[]

\item[8.47]
\begin{itemize}
\item[(a)]
The wording is clear. The way the question is stated makes it slanted toward the high
positive response.

\item[]

\item[(b)]
The wording is clear. The way the question is stated makes it slanted toward the high
positive response (answering ``yes'').

\item[]

\item[(c)]
``Externality'' is an economic term. Most people would not be able to understand the question.
Therefore, the wording is unclear. Responses will vary yet, responses similar to ``What?'', ``Umm....yes?''
would be most frequent (as people would be confused).
\end{itemize}

\item[]
\item[]

\item[8.48]
\begin{itemize}
\item[(a)]
``Due to the data revolution and widespread adoption of internet technologies,
it has become extremely easy to get data of any kind. Applied math and data
science are one of the few jobs that are in-demand right now. Should our university
approve a proposal for the new ``Data Science'' program?''\\\\
Most people would answer with ``YES'' since the introduction to this question
is clearly pro-Data-Science.

\item[]

\item[(b)]
``Have you ever stolen a fruit from the neighbor's garden?''\\\\
A lot of people will not be truthful since it would make them
look bad (in fact, in some countries and regions, it is considered to be a crime).
\end{itemize}

\end{itemize}

\end{document}
