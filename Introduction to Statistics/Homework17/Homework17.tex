\documentclass[11pt, a4paper]{article}
\usepackage[a4paper, margin=1in]{geometry}

\usepackage{adjustbox}
\usepackage{mathtools}
\usepackage{amsmath}
\usepackage{amssymb}
\usepackage{amsthm}

\usepackage{pgfplots}
\usepackage{listings}
\usepackage{color}
\usepackage{tikz}

\usepackage{textcomp}
\usepackage{soul}

\usepackage[hidelinks]{hyperref}
\pgfplotsset{width=7.5cm,compat=1.12}
\usepgfplotslibrary{fillbetween}
\pgfplotsset{compat=1.8}
\usepgfplotslibrary{statistics}
\usepackage[makeroom]{cancel}
\title{\bf{Homework \textnumero 17}}
\author{Author: David Oniani
\\
\ \ \ Instructor: Dr. Eric Westlund}
\date{April 14, 2019}

\usepackage{listings}
\usepackage{color}

%%%%%%%%%%%%%%% S E T S %%%%%%%%%%%%%%%
\newcommand{\nats}{\mathbb{N}}
\newcommand{\ints}{\mathbb{Z}}
\newcommand{\rats}{\mathbb{Q}}
\newcommand{\reals}{\mathbb{R}}
\newcommand{\irrats}{\mathbb{I}}

\newcommand{\pnats}{\mathbb{N}^+}
\newcommand{\pints}{\mathbb{Z}^+}
\newcommand{\prats}{\mathbb{Q}^+}
\newcommand{\preals}{\mathbb{R}^+}
\newcommand{\nreals}{\mathbb{R}^-}

\newcommand{\nints}{\mathbb{Z}^-}
\newcommand{\nrats}{\mathbb{Q}^-}
%%%%%%%%%%%%%%%%%%%%%%%%%%%%%%%%%%%%%%%

% Calligraphy
\newcommand\und[1]{\underline{\smash{#1}}}

% Operators
\DeclarePairedDelimiter\abs{\lvert}{\rvert}
\DeclarePairedDelimiter\ceil{\lceil}{\rceil}
\DeclarePairedDelimiter\floor{\lfloor}{\rfloor}

% Other
\newcommand{\rarr}{\rightarrow}

\definecolor{dkgreen}{rgb}{0,0.6,0}
\definecolor{gray}{rgb}{0.5,0.5,0.5}
\definecolor{mauve}{rgb}{0.58,0,0.82}
\definecolor{backcolour}{rgb}{0.95,0.95,0.92}

\lstset{
backgroundcolor=\color{backcolour},
aboveskip=3mm,
belowskip=3mm,
showstringspaces=false,
columns=flexible,
basicstyle={\small\ttfamily},
numbers=left,
numberstyle=\normalsize\color{gray},
keywordstyle=\color{blue},
commentstyle=\color{dkgreen},
stringstyle=\color{mauve},
breaklines=true,
breakatwhitespace=true,
tabsize=4
}


\begin{document}
\maketitle
\begin{itemize}
\item[23.1]
We start with the \und{\textbf{PLAN}} part as the \und{\textbf{STATE}}
part is the description of the problem itself.\\\\
\und{\textbf{PLAN}}\\
Let $p_\text{F}$ be the proportion of females who have used internet to search for health
information and $p_\text{M}$ be the the proportion of males who have used internet to
search for health. Then we must find a 95\% confidence interval for the difference
in these proportions.
\\\\\\
\und{\textbf{SOLVE}}\\\\
We have, $\hat{p_\text{F}} = \dfrac{811}{1308} \approx 0.6200$ and $\hat{p_\text{M}} = \dfrac{520}{1084} \approx 0.4797$.
Then the standard error (SE) is
$$\text{SE} = \sqrt{\dfrac{0.62 \times (1 - 0.62)}{1308} + \dfrac{0.4797 \times (1 - 0.4797)}{1084}} \approx 0.0203.$$
Now, from table $C$, we know that $z^*$ value for 95\% is 1.96 and the corresponding confidence interval is
$(0.6200 - 0.4797) \pm 1.96 \times 0.0203$ which is (approximately) the confidence interval $0.1403 \pm 0.0398$.
This confidence interval is approximately from 10\% to 18\% (or more precisely, from 10.05\% to 18.01\%).
\\\\\\
\und{\textbf{CONCLUDE}}\\
We can be 95\% confident that from 10\% to 18\% more women have looked for health information on the internet (than men).

\item[]
\item[]

\item[23.2]
We start with the \und{\textbf{PLAN}} part as the \und{\textbf{STATE}}
part is the description of the problem itself.\\\\
\und{\textbf{PLAN}}\\
Let $p_\text{N}$ be the proportion of ninth-graders who answered ``yes'' on the question whether they smoke marijuana or not.
Let $p_\text{T}$ be the proportion of twelfth-graders who answered ``yes'' on the question whether they smoke marijuana or not.
Then we must find a 99\% confidence interval for the difference in these proportions.
\\\\\\
\und{\textbf{SOLVE}}\\\\
We have, $\hat{p_\text{N}} = \dfrac{620}{3500} \approx 0.1771$ and $\hat{p_\text{T}} = \dfrac{969}{3497} \approx 0.2771$.
Then the standard error (SE) is
$$\text{SE} = \sqrt{\dfrac{0.1771 \times (1 - 1771)}{3500} + \dfrac{0.2771 \times (1 - 0.2771)}{3497}} \approx 0.01.$$
Now, from table $C$, we know that $z^*$ value for 99\% is 2.576 and the corresponding confidence interval is
$(0.1771 - 0.2771) \pm 2.576 \times 0.01$ which is the confidence interval $-0.1 \pm 0.02576$.
This confidence interval is approximately from -12.6\% to -7.4\% (or more precisely, from -12.576\% to -7.424\%).
\\\\\\
\und{\textbf{CONCLUDE}}\\
We can be 99\% confident that the difference between the proportions of all ninth-graders and twelfth graders
who smoked marijuana at least once in the 30 days prior to the survey is between -12.6\% and -7.4\%.

\item[]
\item[]

\item[23.4]
\begin{itemize}
\item[(a)]
This is an observational study. It is an observational study
as people were not assigned to a city.

\item[]

\item[(b)]
We start with the \und{\textbf{PLAN}} part as the \und{\textbf{STATE}}
part is the description of the problem itself.\\\\
\und{\textbf{PLAN}}\\
Let $p_\text{N}$ be the proportion of the belted drivers in New York.
Let $p_\text{B}$ be the proportion of the belted drivers in Boston.
Then we must find a 95\% confidence interval for the difference in these proportions.
\\\\\\
\und{\textbf{SOLVE}}\\\\
We have, $\hat{p_\text{N}} = \dfrac{183}{220} \approx 0.832$ and $\hat{p_\text{B}} = \dfrac{68}{117} \approx 0.581$.
Then the standard error (SE) is
$$\text{SE} = \sqrt{\dfrac{0.832 \times (1 - 0.832)}{220} + \dfrac{0.581 \times (1 - 0.581)}{117}} \approx 0.052.$$
Now, from table $C$, we know that $z^*$ value for 95\% is 1.96 and the corresponding confidence interval is
$(0.832 - 0.581) \pm 1.96 \times 0.52$ which is the confidence interval $0.251 \pm 0.102$.
This confidence interval is from 14.9\% to 35.3\%.
\\\\\\
\und{\textbf{CONCLUDE}}\\
According to our results, we have a good evidence that a smaller proportion of female drivers wear seat belts in Boston than in New York.
\end{itemize}

\item[]
\item[]

\item[23.5]
We start with the \und{\textbf{PLAN}} part as the \und{\textbf{STATE}}
part is the description of the problem itself.\\\\
\und{\textbf{PLAN}}\\
Let $p_\text{SK}$ be the proportion of the injured skiers who wear helmets.
Let $p_\text{SN}$ be the proportion of the injured snowboarders who wear helmets.
Then our null hypothesis is $\text{H}_0: p_\text{SK} = p_\text{SN}$ and the alternative
hypothesis is $\text{H}_\text{a}: p_\text{SK} < p_\text{SN}$.
\\\\\\
\und{\textbf{SOLVE}}\\\\
$p_\text{SK} = \dfrac{96}{578} \approx 0.1661$.
$p_\text{SN} = \dfrac{656}{2992} \approx 0.1661$.
$\hat{p} = \dfrac{96 + 656}{578 + 2992} \approx 0.2106$.
Then, the standard error is
$$\text{SE} = \sqrt{0.2106 \times (1 - 0.2106) \times \bigg(\frac{1}{578} + \frac{1}{2992}\bigg)} \approx 0.01853.$$
Now, $z = \dfrac{0.1661 - 0.2193}{0.01851} = -2.87$ and therefore, $P = 0.0021$.
\\\\\\
\und{\textbf{CONCLUDE}}\\
Since $P = 0.0021 < 0.01$, we have a strong evidence that skiers and snowboarders with head injuries are less liekly
to use helmets than those without head injuries.

\item[]
\item[]

\item[23.18]
\begin{itemize}
\item[(a)]
Let $p_\text{S}$ be the proportion of subjects experiencing the primary outcome for the sibutramine group.
Let $p_\text{P}$ be the proportion of subjects experiencing the primary outcome for the placebo group.
Then, we have that
$$p_\text{S} = \dfrac{561}{4906} \approx 0.1143.$$
$$p_\text{P} = \dfrac{490}{4898} \approx 0.1000.$$

\item[]

\item[(b)]
It is appropriate to use the large-sample confidence interval
for comparing the proportions of sibutramine and placebo subjects
who experienced the primary outcome. This is since both the number
of successes and the number of failures are greater than or equal
to 10.

\item[]

\item[(c)]
For 95\%, our $z$-value is 1.96.
Thus, the confidence interval is
$$(0.1143 - 0.1000) \pm 1.96 \times \sqrt{\dfrac{0.1143 \times (1 - 0.1143)}{4906} + \dfrac{0.1000 \times (1 - 0.1000)}{4898}}.$$
which is approximately a confidence interval from $0.0021$ to $0.0265$.
\end{itemize}

\newpage

\item[23.20]
\begin{itemize}
\item[(a)]
We start with the \und{\textbf{PLAN}} part as the \und{\textbf{STATE}}
part is the description of the problem itself.\\\\
\und{\textbf{STATE THE HYPOTHESIS}}\\
Let $p_\text{T}$ be the proportion of treatment group.
Let $p_\text{P}$ be the proportion of the placebo group.
Then our null hypothesis is $\text{H}_0: p_\text{T} = p_\text{P}$ and the alternative
hypothesis is $\text{H}_\text{a}: p_\text{T} \neq p_\text{P}$.
\\\\\\
\und{\textbf{FIND THE TEST STATISTIC}}\\\\
$p_\text{T} = \dfrac{561}{4906} \approx 0.1143$.
$p_\text{P} = \dfrac{490}{4898} \approx 0.1000$.
$p = \dfrac{561 + 490}{4906 + 4898} \approx 0.1072$.
Then, the test statistic is
$$z = \dfrac{0.1143 - 0.1000}{\sqrt{0.1072 \times (1 - 0.1072)} \times \sqrt{\dfrac{1}{4906} + \dfrac{1}{4898}}} \approx 2.29.$$
\\\\\\
\und{\textbf{FIND THE P-VALUE}}\\
Now, using Table A, we get that our $P$-value is $P = 2 \times 0.011 = 0.022$.
\\\\\\
\und{\textbf{STATE THE CONCLUSION}}\\
Since $P = 0.022 < 0.05$, we reject the null hypothesis $\text{H}_0$
and support the claim of the difference in proportions.

\item[]

\item[(b)]
This is important since the improvements could possible be due to the ``placebo effect''
and if no placebo groups is present, we do not know for sure whether the treatment is
effective or not.
\end{itemize}

\item[]
\item[]

\item[20.35]
\begin{itemize}
\item[(a)]
This is an experiment since the subjects were assigned to the groups.

\item[]

\item[(b)]
Let $p_\text{HL+}$ be the proportion for the HL+ group.
Let $p_\text{C}$ be the proportion for the control group.
Then our null hypothesis is $\text{H}_0: p_\text{HL+} = p_\text{C}$ and the alternative
hypothesis is $\text{H}_\text{a}: p_\text{HL+} < p_\text{C}$.\\\\
Now, $\hat{p_\text{HL+}} = \dfrac{49}{49 + 67} \approx 0.4224$, $\hat{p_\text{C}} = \dfrac{49}{49 + 47} \approx 0.5104$, and $\hat{p} = \dfrac{49 + 49}{116 + 96} \approx 0.4623$.\\\\
From these values, we get that the standard error is
$$\text{SE} = \sqrt{0.4623 \times (1 - 0.4623) \times (\dfrac{1}{116} + \dfrac{1}{96})} \approx 0.0688.$$
Then $z = \dfrac{0.4224 - 0.5104}{0.0688} \approx -1.28$ and $P = 0.1003$.
At last, since $P = 0.1003 > 0.05$ we cannot reject the null hypothesis $\text{H}_0$
and therefore, we accept the null hypothesis $\text{H}_0$. In other words, there
is a weak/little evidence that to support the claim that the proportion of HL+ users
with a rhinovirus infection is less than the proportion of non-HL+ users with a rhinovirus infection.
\end{itemize}

\end{itemize}
\end{document}
